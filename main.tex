\documentclass[twocolumn,twocolappendix,tighten,dvipsnames,linenumbers]{aastex63}
\usepackage{eht}

\NewPageAfterKeywords

% Inline comments
\newcommand\note[1]{{\color{OliveGreen}[note: #1]}}
\newcommand\ckc[1]{{\color{MidnightBlue}[ckc: #1]}}
\newcommand\hyp[1]{{\color{Salmon}[HYP: #1]}}
\newcommand\monika[1]{{\color{Orange}[MM: #1]}}

%==============================================================================
\begin{document}

\title{First Sagittarius A* Event Horizon Telescope Results. V.\\
  Constraining the Astrophysical Properties of the Galactic Center Black Hole}
\shorttitle{First \sgra EHT Results on Astrophysics}

\author{Event Horizon Telescope Collaboration et al.}
\shortauthors{EHT Collaboration et al.}

\submitjournal{ApJL}

\received{\today}
\revised{\today}
%\accepted{\today}

\begin{abstract}
  We present theory interpretations of the accretion flow around the
  supermassive black hole Sagittarius A* (\sgra) using the Event
  Horizon Telescope (EHT) observation at 1.3\,mm collected during the
  2017 April 5--11 observing campaign.

  \color{BrickRed}

  [This is EHT's \sgra paper V.
  We are at the position to turn this from a sketchpad to an actual
  draft of the paper.
  The original draft is backed up in the \texttt{2018-08-01/}
  directory on Overleaf.

  For adding references, simply cite the ADS key, e.g.,
  \texttt{\textbackslash citep\{2019ApJ...875L...5E\}}.
  The bash script \texttt{tools/adsbib.sh} would automatically pull
  the full BibTeX database from ADS.
  See \texttt{Makefile} to learn how to use the scripts in
  \texttt{tools/}.]
\end{abstract}

\keywords{galaxies: individual: \sgra -- Galaxy: center -- black hole
  physics -- techniques: high angular resolution -- techniques: image
  processing -- techniques: interferometric}

\tableofcontents

\clearpage

%==============================================================================
\section{Introduction}

The scinetific goals and legacies of this paper may include:
\begin{itemize}
\item Describe a standard simulation library that is used in EHT's
  first \sgra papers.
  E.g., the ``GRMHD'' test set for imaging and calibration sets for
  MCFE and gravity.
\item Provide the first estimate on uncertainties in theoretical
  modelings by comparing models from different realizations/groups in
  similar parameter space.
\item Use the large number of models to extract theory-backed and
  statistical significant trends.
  Example includes edge-on images are more asymmetric.
\item Gather a standard observation data set that \emph{all future}
  theoretical studies should include.
\item Provide the first theoretical/numerical interpretation of the
  EHT \sgra VLBI observation, with helps from historical data.
  This should include ``scoring'' of the standard model sets, a few
  best bet models.
\item Summarize what models do not work and explain why.
\item Outlook into future theoretical research directions, i.e.,
  importance of variability and viscosity.
\end{itemize}

\clearpage

%==============================================================================
\section{Astrophysical Models}

Many of them have the common assumption that \sgra is a black hole
described by the Kerr metric, with mass $\mbh$ and dimensionless spin
$\abh$ within the range $-1 < \abh < 1$.
Here $\abh \equiv Jc/G\mbh^2$, where $J$, $G$, and $c$ are the black
hole angular momentum, gravitational constant, and speed of light,
respectively.
We follow the convention of \citetalias{2019ApJ...875L...5E} that
$\abh < 0$ indicate anti-aligned angular momentum between the
accretion flow and that of the black hole.
\note{Describe other details of the current standard \sgra mode, e.g.,
  thick SANE or MAD accretion disks without tilted, etc.
  Describe main controversy in theory/model, e.g., disk vs jet.}

...

We adopted a weighted mean of \citet{2019Sci...365..664D} and
\citet{2019A&A...625L..10G} and fix the mass of \sgra to
\begin{align}
  \mbh &= (4.136 \pm 0.014) \times 10^6 \msun,\\
  D    &= (8.142 \pm 0.023) \kpc.
\end{align}
\note{The \sgra parameter memo indicates that there's no errors in the
  $M/D$ measurements.
  The errors for $M$ and $D$ don't seem consistent.
  Need to check with the parameter definition working group.}

...

%------------------------------------------------------------------------------
\subsection{One-Zone Model and Estimate}

Using the above mass and distance, the implied characteristic length
is $G\mbh/c^2 = 6.1 \times 10^{11}\cm$, characteristic time is
$G\mbh/c^3 = 20.4 \sec$, angular scale is $G\mbh/(c^2 D) = 5.03\uas$.
The radius of the shadow is $26.1\uas$.

Then
$ \nu L_\nu
= 4 \pi D^2 \nu F_\nu
= 1.8 \times 10^{34} (D/8127 \pc)^2 \times\allowbreak
  (\nu/230 \GHz)(F_\nu/{\rm Jy}) \erg \sec^{-1}$.
The Eddington luminosity for \sgra is
$ L_\mathrm{Edd}
= 4\pi G\mbh c/\kappa_{es}
= 5.2 \times 10^{44}\allowbreak\erg\sec^{-1}$.
The Eddington accretion rate is
$ \dot\mbh_\mathrm{Edd}
\equiv L_\mathrm{Edd}/(0.1 c^2)
= 5.8 \times 10^{24} \gm \sec^{-1}
= 0.09 \msun \yr^{-1}$.
The Eddington ratio is
$ L_\mathrm{bol}/L_\mathrm{Edd}
= 1.9 \times 10^{-10} (L_\mathrm{bol}/10^{35})$.

Next we consider a one zone model for \sgra.
The model consists of a single temperature sphere with magnetic field
oriented at $\pi/3$ to the line of sight.
Assumed flux is $2\Jy$, which is the average flux for \sgra measured
by ALMA during the 2017 campaign.
The sphere radius is
$5 G\mbh/c^2$.
The ratio of ion to electron temperature
$R \equiv T_i/T_e = 3$.
Assumed plasma
$\beta \equiv P_{gas}/(B^2/(8\pi)) = 1$,
electron temperature
$\Theta_e \equiv k T_e/(m_e c^2) = 10$.

The one zone model flux density is
\begin{align}
  F_\nu = \int I_\nu \, d\Omega =
  \frac{2 k T_e \nu^2}{c^2 D^2} \int dx dy\,\left[1-\exp(-\tau_\nu(x,y)\right]
\end{align}
where $d\Omega$ is a differential solid angle, $x$ and $y$ are
coordinates on the source plane in the same units as $D$, and
$\tau_\nu(x,y) = 2 \alpha_\nu (G\mbh/c^2) \sqrt{X_0^2 - X^2}$,
where $X$ is the offset on the sky from the center of the sphere in
units of
$GM/c^2$, and $X_0 = 5 G\mbh/c^2$.
Finally, $\alpha_\nu$ is the thermal synchrotron absorption
coefficient (units: $\cm^{-1}$).
The density is adjusted until the flux density is $2\Jy$.

For a pure hydrogen model, resulting parameters are: $n_e = 1.1 \times
10^6$, $B = 30\,\mathrm{G}$, $\tau_I = 0.48$, $\tau_Q = 0.35 $,
$\tau_V = 0.27$, Thomson depth $\tau_e = 5 G\mbh/c^2 \sigma_T = 2.2
\times 10^{-6}$, Compton $y = \tau_e 16 \Theta_e^2 = 3.6 \times
10^{-3}$, synchrotron cooling time is $t_{cool} = 3.1 \times 10^4\sec
= 1.5 \times 10^3 G\mbh/c^3$.

There is reason to believe that, if \sgra is fed by stellar winds as
in \citet{2019MNRAS.482L.123R}, the inflowing plasma is almost pure
helium.
In this case the results change only slightly; the cooling time drops
slightly because the rest-mass density (and therefore field strength)
per electron increases.

%------------------------------------------------------------------------------
\subsection{Analytical Accretion Models}

\subsection{Numerical Models of the Inner Accretion Flows}

\subsection{Scattering Models}

\clearpage

%==============================================================================
\section{Observational Constraints}

\subsection{Standard Measurement and Assumptions}

\subsection{EHT Observations}
\subsubsection{230\,GHz Light Curve Variability}
\subsubsection{230\,GHz VLBI Null Locations}
\subsubsection{230\,GHz VLBI Pre-Image Size}
\subsubsection{230\,GHz VLBI Ring Size}
\subsubsection{230\,GHz VLBI Ring Thickness}
\subsubsection{230\,GHz VLBI Ring Asymmetry}

\subsection{Non-EHT Observations}
\subsubsection{86\,GHz Flux}
\subsubsection{86\,GHz Image Size}
\subsubsection{NIR (Non-Overproduction) Constraints}
\subsubsection{X-ray (Non-Overproduction) Constraints}

\clearpage

%==============================================================================
\section{Comparisons}

\subsection{EHT Constraints}

\subsection{Non-EHT Constraints}

\subsection{Discussions}

\subsection{Caveats and Limitations}

\clearpage

%==============================================================================
\section{Conclusions}

\subsection{MAD vs SANE}

\subsection{Jet vs Disk}

\subsection{Electron Temperature}

\subsection{Inclination}

\subsection{Black Hole Spin}

\clearpage

%==============================================================================
\facility{EHT} \software{\ehtim, \difmap, \smili, \dmc, \themis, \foci}

%==============================================================================
\appendix

\section{Pass/Fail Tables}

\clearpage

%==============================================================================
\section{Numerical Details}

\subsection{GRMHD Resolution and Convergence}

\subsection{Image Resolution and Field of View}

\subsection{Samping SED}

\subsection{Fast Light vs Light-Speed Light}

\clearpage

%==============================================================================
\bibliographystyle{yahapj}
\bibliography{main,EHTCPapers}

%==============================================================================
\end{document}
