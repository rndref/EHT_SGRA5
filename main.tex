\documentclass[twocolumn,twocolappendix,tighten,dvipsnames,linenumbers]{aastex63}
\usepackage{eht}

\usepackage{savesym}
\savesymbol{tablenum}
\usepackage[group-separator={,}]{siunitx}
\restoresymbol{SIX}{tablenum}

\newcommand\subsubsubsection[1]{\paragraph{#1}}

% Inline comments
%\renewcommand\comment[3]{{\color{#1}[#2 #3]}}
%\renewcommand\comment[3]{}

\newcommand\note[1]{{\comment{OliveGreen}{note:}{#1}}}
\newcommand\ckc[1]{{\comment{MidnightBlue}{ckc:}{#1}}}
\newcommand\hyp[1]{{\comment{Salmon}{HYP:}{#1}}}
\newcommand\monika[1]{{\comment{Orange}{MM:}{#1}}}
\newcommand\cfg[1]{{\comment{Red}{CFG:}{#1}}}
\newcommand\cmf[1]{{\comment{ForestGreen}{CMF:}{#1}}}
\newcommand\mw[1]{{\comment{teal}{MW:}{#1}}}
\newcommand\kc[1]{{\comment{red}{KC:}{#1}}}
\newcommand\jd[1]{{\comment{ForestGreen}{JD:}{#1}}}
\newcommand\br[1]{{\comment{magenta}{BR:}{#1}}}
\newcommand\dl[1]{{\comment{purple}{DL:}{#1}}}
\newcommand\op[1]{{\comment{lime}{OP:}{#1}}}
\newcommand\aco[1]{{\comment{cyan}{AlejandroCruzOsorio:}{#1}}}
\newcommand\aeb[1]{{\comment{red}{AEB:}{#1}}}
\newcommand\bg[1]{{\comment{ForestGreen}{BG:}{#1}}}
\newcommand\bp[1]{{\comment{Lavender}{BP:}{#1}}}
\newcommand\gw[1]{{\comment{blue}{GW:}{#1}}}
\newcommand\vd[1]{{\comment{OrangeRed}{VD:}{#1}}}
\newcommand\michi[1]{{\comment{Brown}{Michi:}{#1}}}
\newcommand\smr[1]{{\comment{JungleGreen}{SMR:}{#1}}}
\newcommand\ajr[1]{{\comment{Red}{AJR:}{#1}}}

% some paper-V local macros
\newcommand{\aprilvi}{April~6\xspace}
\newcommand{\aprilvii}{April~7\xspace}
\newcommand{\aprilxi}{April~11\xspace}

\newcommand{\rhigh}{\Rh}
\newcommand{\vam}{VA morphology\xspace}
\newcommand{\mring}{m-ring\xspace}
\newcommand{\mrings}{m-rings\xspace}
\newcommand{\Mring}{M-ring\xspace}

%==============================================================================
\begin{document}

\title{First Sagittarius A* Event Horizon Telescope Results. V.\\
  Testing Astrophysical Models of the Galactic Center Black Hole}
\shorttitle{EHT \sgra Astrophysical Interpretation}

%\collaboration{The Event Horizon Telescope Collaboration}
%\input{newGAL_ApJ}
\author{Event Horizon Telescope Collaboration et al.}
\shortauthors{EHT Collaboration et al.}

\submitjournal{ApJL}

\received{\today}
\revised{\today}
%\accepted{\today}

%==============================================================================
\begin{abstract}
  In this paper, we provide a first physical interpretation for the Event Horizon Telescope (EHT)'s 2017 observations of \sgra.
  Our main approach is to compare resolved EHT data at $230\GHz$ and unresolved non-EHT observations from radio to X-ray wavelengths to predictions from a library of models based on time-dependent general relativistic magnetohydrodynamics (GRMHD) simulations, including aligned, tilted, and stellar wind-fed simulations; radiative transfer is performed assuming both thermal and non-thermal electron distribution functions. We test the models against 11 constraints drawn from EHT 230\GHz data and observations at 86\GHz, 2.2\um, and in the X-ray.
  All models fail at least one constraint.
  Light curve variability provides a particularly severe constraint, failing nearly all strongly magnetized (MAD) models and a large fraction of weakly magnetized (SANE) models.
  A number of models fail only the variability constraints.  We identify a promising cluster of these models, which are MAD and have inclination $i \le 30\degree$.
  They  have accretion rate $(5.2$--$9.5)\times10^{-9}\msun\yr^{-1}$, bolometric luminosity $(6.8$--$9.2)\times10^{35}\ergps$, and outflow power $(1.3$--$4.8)\times10^{38}\ergps$.
  We also find that: all models with $i \ge 70\degree$ fail at least two constraints, as do all models with equal ion and electron temperature;  exploratory, non-thermal model sets tend to have higher 2.2\um flux density; the population of cold electrons is limited by X-ray constraints due to the risk of bremsstrahlung overproduction.
  Finally we discuss physical and numerical limitations of the models, highlighting the possible importance of kinetic effects and duration of the simulations.
\end{abstract}

\keywords{galaxies: individual: \sgra -- Galaxy: center -- black hole
  physics -- techniques: high angular resolution -- techniques: image
  processing -- techniques: interferometric}

%------------------------------------------------------------------------------
\tableofcontents

\vspace{18pt}

%==============================================================================
\section{Introduction}
\label{sec:intro}

The center of the Milky Way contains a massive compact object that is likely a supermassive black hole \citep{2019Sci...365..664D, 2019A&A...625L..10G}.
The putative black hole is surrounded by hot plasma that is visible across 17 decades in electromagnetic frequency.
Hereafter, we will use \sgra to refer to the supermassive black hole candidate and the hot plasma.

\sgra is one of the most studied objects on the sky, both observationally and theoretically.
A key  characteristic of the \sgra system is its extremely low overall luminosity with respect to the Eddington limit.
The low luminosity suggests that matter falls onto \sgra's central object in the form of a radiatively inefficient/advection dominated accretion flow (RIAF/ADAF, as proposed by \citealt{1977ApJ...214..840I,1994ApJ...428L..13N, 1995ApJ...444..231N, 1995ApJ...452..710N, 1996A&AS..120C.287N, 1998ApJ...492..554N,2014ARA&A..52..529Y}) rather than in the form of a radiatively efficient thin disk \citep{1973A&A....24..337S}.
Since the nearly flat radio spectrum of \sgra is similar to radio spectra observed in jets from Active Galactic Nuclei, it has also been suggested that the majority of the \sgra emission could be produced by a jet launched by an accreting black hole rather than matter falling through the black hole event horizon \citep{1993A&A...278L...1F, 2000A&A...362..113F}.
% cfg 2/5: removed as TMI
% The jet scenario in \sgra is further supported by time lags observed between emission measured at multiple radio frequencies \citep{2021arXiv210713402B}.

%\sgra's radio and millimeter wavelength emission is produced by the  synchrotron process, so source models require a description of the plasma and magnetic field configuration.
Models of magnetized RIAFs/ADAFs have been constructed using semi-analytic prescriptions \citep[e.g.,][]{1995Natur.374..623N,2000ApJ...541..234O, 2009ApJ...697...45B,2011ApJ...735..110B} and using time-dependent General Relativistic Magnetohydrodynamics (GRMHD) simulations \citep[e.g.,][]{2000ApJ...528..462H, 2003ApJ...589..458D, 2003ApJ...589..444G, 2007CQGra..24S.235G, 2012ApJS..201....9F, 2014ApJ...796...22F, 2016ApJS..225...22W, 2017ApJS..231...17A, 2018JPhCS1031a2008O, Olivares2019, 2019ApJS..243...26P, Liska2019}.
Semi-analytic RIAF/ADAF models typically do not include relativistic jets or outflows, but those are naturally produced in GRMHD simulation and contribute to the observed emission.
GRMHD simulations also naturally produce variability, which is observed in \sgra at multiple wavelengths.

GRMHD simulations of ADAFs show that ADAF-like inflows are not unique.
In particular two dramatically different modes are observed, depending on the magnetic flux interior to the black hole equator: the standard and normal evolution (SANE) mode, in which the midplane magnetic field pressure is less than the gas pressure and magnetic fields are turbulent; and the magnetically arrested disk (MAD) mode, in which magnetic fields are strong and organized and can even disrupt accretion.
An outstanding question about \sgra is whether the flow is in MAD or SANE mode, or possibly in a third mode that results from wind-fed accretion \citep{2020ApJ...896L...6R}.

The energy distribution of electrons in the emitting plasma is also not known.
Because emission is driven by the synchrotron process, this is critical in determining the observational appearance of the source.
In particular the energy per electron may increase with latitude in the flow, leading to a jet or outflow that outshines an equatorial inflow.

The question of whether emission is dominated by an inflow or outflow is intimately tied to the problem of what drives an outflow, if there is one.
In GRMHD simulations of black hole accretion the strength of the outflow depends sensitively on the black hole spin (e.g., \citealt{M87PaperV}, hereafter \citetalias{M87PaperV}).
At large spin GRMHD simulations produce powerful jets driven by extraction of black hole spin energy via the \citet{1977MNRAS.179..433B} process.
A spatially resolved study of \sgra may thus also constrain the black hole spin and provide direct evidence for black hole energy extraction.

Previously published GRMHD models of \sgra generically predict source sizes at millimeter wavelengths consistent with observational data \citep[e.g.,][]{2008Natur.455...78D, 2009ApJ...706..497M, 2009ApJ...703L.142D,2010ApJ...717.1092D};
the radio spectral shape is similar to jet emission \citep[e.g.,][]{2013A&A...559L...3M, 2017MNRAS.467.3604R}, and the source linear polarization requires strongly magnetized flow or nonthermal electrons \citep{2015Sci...350.1242J, 2017ApJ...837..180G, 2020MNRAS.494.4168D}.
% CFG 2/5: propose we remove the rest of this paragraph
% Near event horizon variability can be produced by a turbulence \citep{2020MNRAS.497.4999D}, lensing events, magnetic flux tubes \citep{2015ApJ...812..103C}, or magnetic flux eruptions associated with MAD flow \citep{2011MNRAS.418L..79T}.
%The 86\GHz data seem to require nonthermal electrons \citep{2018MNRAS.478.5209C, 2019ApJ...871...30I}.

A major difficulty in determining the nature of \sgra radio emission is caused by the interstellar scattering screen that distorts our view of the Galactic Center up to $\lambda \sim 1\mm$ wavelengths
\citep[see][and references therein]{2018arXiv180501242P, 2018ApJ...865..104J,2019ApJ...871...30I}.
The Event Horizon Telescope (EHT) is a very-long-baseline interferometric (VLBI) experiment operating at $230\GHz$ or wavelength $\lambda = 1.3\mm$ (see \citealt{M87PaperII}, hereafter \citetalias{M87PaperII}, for an introduction to the instrument).
EHT operates at high enough frequency to penetrate the scattering screen, with angular resolution sufficient to directly image structures in the immediate vicinity of the black hole event horizon.

In April~2017 the EHT observed \sgra (among other sources, including the core of the M87 galaxy, see \citealt{M87PaperI}, hereafter \citetalias{M87PaperI}) and produced the first ever horizon scale images of the source.
We report the results of these observations in \citet{PaperII}, hereafter \citetalias{PaperII} and \citet{PaperIII}, hereafter \citetalias{PaperIII}, characterize the basic properties of the emission visible in the EHT images in \citealt{PaperIV}, hereafter \citetalias{PaperIV}, and discuss implications for tests of general relativity in \citealt{PaperVI}, hereafter \citetalias{PaperVI}.
The main goal of this paper \citepalias{PaperV} is to provide the first comprehensive physical interpretation of the EHT~2017 \sgra datasets.

This paper is structured as follows.
Section~\ref{sec:models} describes our main assumptions, a one-zone source models, and a standard simulation and synthetic image library used to model near-horizon emission from \sgra.
Our model library assumes that general relativity is valid and the spacetime around \sgra is described by the Kerr metric  \citep{1963PhRvL..11..237K}.
A discussion of \sgra observations in the context of alternative theories of gravity can be found in  \citet{PaperVI}, hereafter \citetalias{PaperVI}.
Our model library is based on time-dependent GRMHD simulations that, combined with general relativistic radiative transfer models, result in images and broadband spectra of the models.
The library of simulated images was used in \citetalias{PaperIII} and \citetalias{PaperIV}, to validate the \sgra EHT imaging and parameter estimation algorithms.
In Section~\ref{sec:observations}, we describe the  observational constraints that are used in the present work to test theoretical models of \sgra.
These data comprise a subset of EHT~2017 observations and other non-EHT historical or other data.
In Section~\ref{sec:comparisons}, we describe model scoring procedures and use our model library to infer physical properties of \sgra system.
We discuss model limitations, results in the context of previous studies and outlook for future \sgra theoretical research directions in Section~\ref{sec:discussions}.
Finally, we conclude in Section~\ref{sec:conclusions}.

This paper is supplemented with several appendices.
In Appendix~\ref{app:numerical},  discusses numerical details of our simulations.
In Appendix~\ref{app:variability}, discusses the impact of physical  and numerical effects on the model variability.
In Appendix~\ref{app:tables}, summarizes the results of applying constraints to our fiducial models in an extended set of figures. 


\section{Astrophysical Models}
\label{sec:models}

%==============================================================================
\subsection{Basic Assumptions}
\label{sec:basic}

%\cfg{why do we have error bars, if these are basically the mean of the ucla and mpe values?}

% this is now synchronized with numbers in paper 6, but check
In this paper, we assume the mass of and distance to \sgra are
\begin{align}
  \mbh &= 4.14  \times 10^6 \msun, \label{eq:mass} \\
  D    &= 8.127 \kpc,              \label{eq:dist}
\end{align}
which are approximately the mean of values reported by \citet{2019Sci...365..664D} and \citet{2019A&A...625L..10G}.

Throughout the paper we also assume that \sgra is a black hole and is described by a Kerr spacetime.
The black hole dimensionless spin, $\abh \equiv Jc/G\mbh^2$, is a free parameter with $-1 < \abh < 1$.
Here, $J$, $G$, and $c$ are the black hole angular momentum, gravitational constant, and speed of light, respectively.
Following \citetalias{M87PaperV},
$\abh > 0$ indicates the angular momentum of the accretion flow and black hole are parallel (the accretion flow is ``prograde'') and
$\abh < 0$ indicates the angular momentum of the accretion flow and black hole are antiparallel (``retrograde'').

Using the above mass and distance, the implied
characteristic length $r_\mathrm{g}         \equiv G\mbh/ c^2    \simeq 6.1\times10^{11}\cm$,
characteristic time   $t_\mathrm{g}         \equiv G\mbh/ c^3    \simeq 20.4\sec$, and
angular scale         $\vartheta_\mathrm{g} \equiv G\mbh/(c^2 D) \simeq 5.03\uas$.
The expected diameter of the black hole shadow is $2\sqrt{27} G\mbh/(c^2 D) = (52.3 \pm 2.08)\uas$,
where errorbars enclose uncertainty in the black hole spin and viewing angle \citep[see, e.g.,][]{2013ApJ...777...13C, 2020ApJ...896....7M}.

If the emitting plasma is ionized hydrogen (electron-proton plasma), then the Eddington luminosity is
$L_\mathrm{Edd} = 4\pi G\mbh c m_p/\sigma_\mathrm{T} = 5.2 \times 10^{44}\ergps$.
The corresponding Eddington accretion rate is
$\dot\mbh_\mathrm{Edd} \equiv L_\mathrm{Edd}/(0.1 c^2) = 5.8 \times 10^{24} \gm \rm{s}^{-1} = 0.09 \msun \yr^{-1}$,
where the nominal efficiency is 10\% and the Eddington ratio:
$L_\mathrm{bol}/L_\mathrm{Edd} = 1.9 \times 10^{-10} (L_\mathrm{bol} /10^{35})$,
where $L_\mathrm{bol}$ is in unit of $\ergps$.
In a quiescent, non-flaring state, the bolometric luminosity of \sgra is $L_\mathrm{bol} \sim 10^{35}\ergps$, resulting in an extremely small Eddington ratio.
In what follows, we will assume that the radiative cooling of plasma around the black hole can be neglected and that model emission can be calculated as post-processing.

%==============================================================================
\subsection{Estimate from One-Zone Model}

The development of complex models is guided by simple estimates.
Following \citetalias{M87PaperV}, we first consider an one-zone model for \sgra.
These results follow one-zone models developed in the literature over many decades \note{(refs)}.

Our one-zone model is a uniform plasma sphere of radius $r = 5\rg$ with uniform magnetic field oriented at $\pi/3$ to the line-of-sight.
The magnetic field is given by $n_i k T_i + n_e k T_e = \beta B^2/(8\pi)$, where $T_i$ denotes ion temperature, $T_e$ denotes electron temperature, B denotes magnetic field strength, $\beta=1$, and $T_i = 3 T_e$.
We also assume $\Theta_e \equiv  k T_e / m_e c^2 = 10$.

Using the thermal emissivity for synchrotron radiation $j_\nu$ \citep[e.g.,][]{2011ApJ...737...21L} and assuming optically thin emission, the flux density is given by $F_\nu = (4/3)\pi r^3 j_\nu D^{-2} 10^{23}\,\mathrm{Jy}$.  Setting $F_\nu = 2.4\,\mathrm{Jy}$ (the average measured by ALMA during the 2017 campaign) yields a nonlinear equation for plasma electron density, $n_e$ with solution
\begin{align}
  n_e &\simeq 1.1\times10^6\cm^{-3},\\
  B   &\simeq 30\,\mathrm{G}
  \label{eq:onezone}
\end{align}
(the synchrotron optical depth $\tau_\mathrm{sync} = r j_\nu/B_\nu \simeq 0.4$, so the optically thin approximation is marginal).
These values are consistent with $n_e$ and $B$ of a similar one-zone model fitted to archival \sgra millimeter spectrum as reported in \citet{2019ApJ...881L...2B}.
%MM: fit of similar one zone model to Terahertz spectrum from ALMA infers ne=2-5x10^6 cm^-3, B=10-50 Gauss, T_e=1-3x10^11 K => Thetae=16-50
% CFG: I've placed a mathematica script implementing the one zone model in eht.astro.illinois.edu://bd4/eht/paperV/OneZoneThin.ma

The one-zone model has optical depth to Compton scattering $\tau_e \sim \sigma_T n_e r \simeq 2\times10^{-5}$ and thus a small Compton parameter: $y = 16 \Theta_e^2 \max(\tau_e,\tau_e^2) \simeq 3\times10^{-2}$.
Synchrotron cooling therefore dominates Compton cooling.

The synchrotron cooling timescale is $t_\mathrm{cool} \equiv u/\Lambda$ where $u_e = 3 n_e k T_e$ is the electron internal energy and $\Lambda \simeq 5.4 B^2 e^4 n_e \Theta_e^2 /(c^3 m_e^2)$ is the synchrotron cooling rate from thermal population of electrons with $\Theta_e \gtrsim 1$ (for details see Appendix~A in \citealt{2011ApJ...735....9M}; finite optical depth reduces $\Lambda$).
The cooling time in the one-zone model is therefore $t_\mathrm{cool}=2.3 \times 10^4\sec \simeq 1.1 \times 10^3 \tg$ which is longer than the characteristic inflow time.
Simple estimates suggest that the radiative cooling can be neglected in the plasma models \citep[more detailed calculations confirm this estimate][]{2012MNRAS.426.1928D}.\footnote{Notice that if \sgra is fed by stellar winds then the inflowing plasma may be mainly helium \citep{2019MNRAS.482L.123R}; this changes the one-zone model only slightly. Helium accretion is discussed in detail in Wong+2022.}

The one-zone model implies that the accretion flow is at high temperature and low density and is therefore collisionless, in the sense that the mean free path to Coulomb scattering is large compared to $\rg$.
At $\Theta_e \sim 1$, for example, the Coulomb scattering cross section is comparable to the Thomson cross section, and the mean free path therefore $\sim \tau_e^{-1} \rg$.
The electron-ion Coulomb scattering timescale is also long, and the electrons and ions are therefore poorly coupled.
This motivates consideration of
\emph{i})~a two-temperature model for the plasma where electrons are cooler than the ions \citep{1976ApJ...204..187S,1977ApJ...214..840I, 1982Natur.295...17R} and
\emph{ii})~nonthermal (unrelaxed) electron distribution functions.
As discussed in \citet{2000ApJ...541..234O}, \citet{2009ApJ...701..521C}, and \citet{2014A&A...570A...7M} \citep[see also more recent work by][]{2018A&A...612A..34D,2021arXiv211102518F, 2021NatAs.tmp..218C, 2021arXiv211203933E}, both effects may change the predicted properties of \sgra.
%\br{my only remaining comment is that I think that the sentence ``As demonstrated in \citet{2000ApJ...541..234O} and \citet{2014A&A...570A...7M} (see also more recent work by \citealt{2018A&A...612A..34D} and references therein) both effects may change the predicted properties of \sgra.'' should be cut, or extended with non-EHT refs on how other groups actually probe those collisionless effects dynamically}
%\ckc{In principle, we could numerically check all of the assumptions in the one zone model for consistency.  Probably leave until after PC submission.}
%\monika{i think that the purpose of this section is to make a rough estimates on the back of the envelope so that everyone can repeat them, numerical work is always more difficult to repeat}
%\ckc{Yes but these also usually physical interesting quantities.  At least we should compute the values that paper~I wants.}

%==============================================================================
\subsection{Numerical Models of the Inner Accretion Flow}

\begin{deluxetable*}{cccccccc}
  \tabletypesize{\footnotesize}
  \renewcommand{\arraystretch}{1.1}
  %
  \tablehead{                     &
    \colhead{Spacetime}           &
    \multicolumn{2}{c}{Fluid}     &
    \multicolumn{3}{c}{Numerical} &
    \colhead{Note} \\
    \colhead{Setup/Code}                  &
    \colhead{$\abh$}                     &
    \colhead{Mode}                       &
    \colhead{$\Gamma_\mathrm{ad}$}       &
    \colhead{\!\!\!\!\!\!$t_\mathrm{final}$ [$\rg$]} &
    \colhead{Size [$\rg$]\!\!\!\!\!\!}               &
    \colhead{Resolution}                 &
    \colhead{Reference}
  }
  \startdata
  \begin{tabular}{@{}c@{}} standard / \\ \kharma \end{tabular} & 0,$\pm1/2$,$\pm15/16$                 & MAD/SANE     & $4/3$      & 30,000  & 1000     & [288x128x128]     & \!\!\!\!\!\!\!\!\!
  \begin{tabular}{@{}c@{}c@{}c@{}} This work\\\citet{kharma_2022}\\\citet{Wong_2022} \\ \citet{Dhruv_2022}\end{tabular}\\
% \citet{Wong_2022, Dhruv_2022} \\
  \begin{tabular}{@{}c@{}} standard / \\ \bhac \end{tabular}   & 0,$\pm1/2$,$\pm15/16$                 & MAD/SANE     & $4/3$      & 30,000  & 3333     & [512x192x192]     & This work \\
  \begin{tabular}{@{}c@{}} standard / \\ \hamr \end{tabular}   & 0,$\pm1/2$,$\pm15/16$                 & MAD/SANE     & $13/9,5/3$ & 35,000  & 1000/200 & [348/240×192×192] & This work \\
  \begin{tabular}{@{}c@{}} standard / \\ \koral \end{tabular}  & \!\!\!\!\!\!\!\!\!
  \begin{tabular}{@{}c@{}c@{}}   0,$\pm0.3$,$\pm0.5$\\$\pm0.7$,$\pm0.9$ \end{tabular}
  \!\!\!\!\!\!\!\!\! & MAD          & $13/9$     & 101,000 & 100,000  & [288x192x144]     & \citet{2021arXiv210812380N} \\
  \begin{tabular}{@{}c@{}} tilted / \\ \hamr \end{tabular}     & $15/16$     & IN-SANE      & 5/3        & 105,000 & 100,000  & [448x144x240]     & \begin{tabular}{@{}c@{}} \citet{Liska2018} \\ \citet{Chatterjee2020}\end{tabular} \\
  \begin{tabular}{@{}c@{}} wind-fed / \\ \athenapp \end{tabular} & 0           & MAD$\times2$ & 5/3        & 20,000  & 2,400    & [356x128x128]     & \begin{tabular}{@{}c@{}} \cite{2016ApJS..225...22W} \\ \citet{2020ApJ...896L...6R} \end{tabular}
  \enddata
  %
  %\tablenotetext{$*$}{Non-standard model.}
  \caption{Summary of GRMHD simulations in the EHT \sgra GRMHD model library.
    The first four entries are standard \sgra simulations.
    The last two entries are the tilted accretion model and two realizations of the Wind Accretion models which differ in stellar wind magnetization.}
  \label{tab:GRMHDmodels}
\end{deluxetable*}

The one-zone model is too simple for comparison with the rich set of observations available for \sgra.
Going beyond the one zone model, analytic spherical accretion models \citep[e.g.,][]{2019ApJ...885L..33N, 2021arXiv211102178B} incorporating relativistic  gravity and analytic disk-like (RIAF) accretion models in the Kerr metric incorporating rotation and departures from spherical symmetry \citep[e.g.,][]{2009ApJ...697...45B, 2009ApJ...706..960H,2018ApJ...863..148P} can be used for parameter surveys (see Section~\ref{sec:future}).
% I think the previous sentence says both analysis spherical and disk can be used for blab blab blab...  Minor edit to retain that meaning.
They do not, however, self-consistently capture fluctuations in the flow---that requires either a statistical model \citep{2021ApJ...906...39L} or a time-dependent numerical integration.
\ckc{May be also say they don't self-consistently capture turbulence for transport.}
Here we use numerical simulations and adopt an ideal GRMHD model for the flow\footnote{Limitations of the ideal GRMHD model are discussed in Section~\ref{sec:limits}.}; use simple parameterized models to assign an electron distribution function, and solve the radiative transfer equation along geodesics to produce simulated images.

%------------------------------------------------------------------------------
\subsubsection{Plasma Flow Model}

We model the plasma flow around \sgra using ideal, non-radiative GRMHD.
The gravitational field is given by the Kerr metric, with mass from Equation~(\ref{eq:mass}) and with black hole spin $\abh$ a free parameter \citep[see e.g.,][]{2003ApJ...589..444G, 2005ApJ...635..723A, 2007A&A...473...11D}.

We integrate the GRMHD equations in three spatial dimensions using multiple algorithms:
\kharma   \citep{2021JOSS....6.3336P},
\bhac     \citep{2017ComAC...4....1P},
\hamr     \citep{Liska2018},
\koral    \citep{2013MNRAS.429.3533S}, and
\athenapp \citep{2016ApJS..225...22W};
see \citet{2019ApJS..243...26P} and \citet{Olivares_et_al} for comparisons of GRMHD codes.
All simulations assume a constant adiabatic index $\Gamma_\mathrm{ad}$.

The \emph{standard} initial conditions for the GRMHD integrations are a hydrodynamic, constant-angular-momentum equilibrium, the Fishbone-Moncrief torus \citep{1976ApJ...207..962F}
%\aco{\citep[see also][]{2002MNRAS.334..383F}} \monika{why do we need to have this second citation? is that essential?} \br{I thought Font shows another hydro torus equilibrium in this paper? We don't use that one, so I would not cite that paper} \ckc{all modelers: do any of you use the Font initial condition?  If not I will comment out this reference.}\aco{There are two independent hydro solutions, we have both of them. With same initial condition for magnetic field on top of hydro-solution}\br{is there a reason for using different initial conditions? I tried both FD and FM torus myself and the FD torus evolution is somewhat different. It would be better to start from the same IC to compare late states of the models to images. Which models exactly do use the (different) initial conditions of the Font-Daigne torus? I thought in paper V for M87 we used only the FM torus. Then the first sentence in this paragraph above is also wrong, because we are not using the FM torus exclusively.}\monika{exactly which model in the library starts from FD torus? please answer before midnight.} % \aco{Yes, I think is essential to cite, is true that we not using FD but we also have it in BHAC. Please be free to remove it. But definitively don't have same initial condition in all GRMHD codes.}
% /monika{11dec: we dont want to elaborate on details like that here, this section is intended to describe only common features of all simulations, other details can be described in section 4.]
in a prograde ($\abh > 0$) or retrograde ($\abh < 0$) orbit.
In all models except the \hamr-Tilted models, the torus orbital angular momentum is either parallel or antiparallel to the black hole spin. The torus is parameterized by an inner radius $R_{\rm in}$, typically $12~\rg$, and the pressure maximum radius $R_\mathrm{max}$, typically $24~\rg$.
%\aco{We should specify the torus size? or should we refer to the reader to SANE and MAD code comparison. We have different initial data between codes, not only in the size but also in adiabatic index in the equation of state.}
%\aco{ with magnetization $\beta \equiv p_{\rm gas}/p_{\rm mag} =100$}
%\monika{11dec: we dont want to elaborate on details like that here, this section is intended to describe only common features of all simulations, other details can be described in section 4, incase there are differences in the results that may be due to initial conditions.}
The torus is seeded with a weak, poloidal magnetic field.

The torus initial conditions are motivated by the notion that the near-horizon flow, where most of the emission is generated (\citetalias{M87PaperV}) relaxes to a statistically steady state that is nearly independent of the flow at larger radius.  This notion is challenged in the stellar wind-fed models of \cite{2020ApJ...896L...6R}, which we include in our study.

All simulations are run in horizon penetrating coordinates that derive from Kerr-Schild coordinates, which are regular on the horizon.
Most are run in a variant of spherical polar coordinates.
The \emph{standard} boundary conditions are outflow at the inner boundary, located inside the event horizon, outflow at the outer boundary, located at $r \gtrsim 1000~\rg$, and reflecting boundary conditions at the poles.
Standard simulations are evolved to $t_\mathrm{final} = 30,000~\tg$.

Once the evolution has started, a combination of instabilities including the magnetorotational instability \citep[MRI][]{1992ApJ...400..610B} drives the torus to a turbulent, fluctuating state.
Letting $P_\mathrm{gas}$ be the gas pressure and $P_\mathrm{mag} \equiv B^2 / (8\pi)$ be the magnetic pressure, the standard accretion flow models can be divided by latitude into three zones:
\emph{i})~equatorial inflow,
\emph{ii})~a mid-latitude disk wind/corona with  $\beta  \equiv P_\mathrm{gas} / P_\mathrm{mag} \sim 1$, and
\emph{iii})~a polar funnel/relativistic jet with $\sigma \equiv B^2/4\pi \rho c^2 \gg 1$.

It is well established \citepalias[see, e.g.,][and references therein]{M87PaperV, M87PaperVIII} that the magnetic flux through the event horizon divides the outcomes into two categories: the magnetically arrested disk (MAD) state \citep[e.g.,][]{1974Ap&SS..28...45B, 2003ApJ...592.1042I, 2003PASJ...55L..69N} in which the magnetic flux near the horizon saturates and significantly affects the dynamics of the flow, and the Standard and Normal Evolution (SANE) state \citep[e.g.,][]{2003ApJ...589..444G, 2003ApJ...599.1238D, 2012MNRAS.426.3241N}.
The relative importance of magnetic flux can be described by $\phi \equiv \Phi_{\mathrm{BH}} (\dot{M} r_\mathrm{g}^2 c)^{-1/2}$, where $\Phi_{\rm BH}$ is the magnetic flux interior to the black hole equator and $\dot{M}$ is the mass accretion rate through the horizon.
MAD models have $\phi \sim \phi_{\rm crit} \sim 60$.\footnote{In the Lorentz-Heaviside units commonly used in GRMHD simulations $\phi_\mathrm{crit}$ is smaller by a factor of $(4\pi)^{1/2} \simeq 3.545$.}
%\aco{Should we cite the work-in-progress by Narayan about spinup-spindown for MAD models where magnetic flux it can be written as a function of BH spin? I've verified the same tendency for BHAC.}
%\monika{not necessary, this is not established and should be first published we want to limit in prep as much as possible}
In MAD models, magnetic flux accretes onto the hole until $\phi \gtrsim \phi_\mathrm{crit}$, then magnetic flux is expelled from the hole and escapes through the inflowing plasma.
SANE models have $\phi < \phi_c$, and in most of our standard models have $\phi \sim 1$.

We also consider two \emph{non-standard} GRMHD simulations: strongly magnetized non-MAD tilted torus simulations \citep{Liska2018, Chatterjee2020} and a model in which \sgra is fed directly by winds from stars in its vicinity \citep{2020ApJ...896L...6R}.
The self-consistent wind feeding simulations result in a mode of accretion that is similar to MAD but typically has lower mean angular momentum and is less well organized.
The wind-fed models have $\abh = 0$.

The GRMHD model library is summarized in Table~\ref{tab:GRMHDmodels}.
In Figure~\ref{fig:GRMHD} we show a few examples of standard and non-standard GRMHD runs.
The models vary in numerical method and in numerical resolution. We present more information on the numerical methods and models in Appendices~\ref{app:numerical} and \ref{app:variability}.

\begin{figure*}
  \centering
  \includegraphics[width=0.425\textwidth]{figures/sane_3D_final.png}\hspace{1.5pt}%
  \includegraphics[width=0.425\textwidth]{figures/mad_3D_final.png}\\
  \includegraphics[width=0.425\textwidth]{figures/tilted_3D_final.png}\hspace{1.5pt}%
  \includegraphics[width=0.425\textwidth]{figures/ressler_3D_final.png}
  \caption{3-D overview of selected GRMHD simulations of \sgra in our library.
    The color marks constant dimensionless density surfaces and lines follow magnetic field lines.
    Two top panels show standard accretion models: SANE (left panel, $\abh=0.94$) and MAD (right panel, $\abh=0.5$).
    Two bottom panels show non-standard accretion models: tilted disk (left panel, $\abh=0.94$ and tilt angle of $60\degree$) and Wind Accretion (right panel, $\abh=0$).
    In case of spinning black holes, the spin is aligned with z-axis.}
  \label{fig:GRMHD}
\end{figure*}

One critical feature of the GRMHD models that is important for the interpretation of our results is the temperature profile.  Figure \ref{fig:grmhd_temp} shows the time- and azimuth- averaged profiles of the midplane {\em ion} temperature in a set of aligned GRMHD models.  The temperature profiles exhibit strong trends with spin and magnetic state (MAD or SANE) that drives many of the trends seen in the radiative models: MAD models are factor of several hotter than than SANE models and both MAD and SANE become hotter as $\abh$ increases.

\begin{figure*}
  \centering
  \includegraphics[width=0.9\textwidth]{figures/grmhd_temp.png}
  \caption{Time- and azimuth- averaged profiles of midplane gas temperature in aligned GRMHD models.  Evidently MAD models are hotter than SANE, and both MADs and SANEs grow hotter as the black hole spin $\abh$ increases.  The hottest models are $\abh = 0.94$ MAD models.}
  \label{fig:grmhd_temp}
\end{figure*}

%------------------------------------------------------------------------------
\subsubsection{Radiative Transfer Model}

Simulated images are generated from the GRMHD model in a radiative transfer step.
Given the total internal energy output by the GRMHD model, the transfer step requires: %(1) CK: I prefer using i), ii), iii) in text because (1) etc are for equations.
\emph{i})~a model for the electron temperature and electron distribution function (hereafter eDF);
\emph{ii})~an assignment of a density scale to the GRMHD model;
\emph{iii})~a numerical radiative transfer step performed after the GRMHD model, assuming that the plasma evolution is unaffected by radiation.

%..............................................................................
\subsubsubsection{Electron Distribution Function}
\label{sec:eDF}

In \emph{thermal} models electron energies are distributed into the Maxwell-J{\"u}ttner distribution function:
\begin{align}
  \frac{1}{n_e}\frac{dn_e}{d\gamma} = \frac{\gamma^2 \sqrt{1-1/\gamma^2}} {\Theta_e K_2(1/\Theta_e)} \exp\left(-\frac{\gamma}{\Theta_e}\right);
\end{align}
where $K_2$ is the modified Bessel function of the second kind and $\gamma$ is the Lorentz factor of an electron. Recall $\Theta_e = k_b T_e/(m_e c^2)$, which is determined by the ion-electron temperature ratio $R \equiv T_i/T_e$:
\begin{align}
  T_e=\frac{2 m_p u}{3 k_B \rho (2+R)}.
\end{align}
Here $u$ and $\rho$ are the internal energy density and rest-mass density from the GRMHD simulation.
Thermal models are motivated by the idea that wave-particle scattering drives partial relaxation of the eDF, even though Coulomb scattering is ineffective.

Plasma heating models suggest that the partitioning of dissipation between ions and electrons depends on the local magnetic properties \citep[e.g.,][]{2010MNRAS.409L.104H, Kawazura771}.  This motivates a prescription in which the temperature ratio is a function of the plasma $\beta \equiv P_\mathrm{gas}/P_\mathrm{mag}$ \citep{2015ApJ...799....1C}.
We adopt the same model as \citetalias{M87PaperV} and \citetalias{M87PaperVIII}, where $R$ is a smooth function adopted from \cite{2016A&A...586A..38M}:
\begin{equation}
  R = \frac{T_i}{T_e} = \Rh \frac{b^2}{b^2+1} + \Rl \frac{1}{b^2+1}
  \label{eq:rhigh_prescription}
\end{equation}
where $b \equiv \beta/\beta_\mathrm{crit}$.
This model has three free parameters: $\beta_\mathrm{crit}$, $\Rl$, $\Rh$.  Here we take $\Rl = 1$ and $\beta_\mathrm{crit} = 1$, but allow $\Rh$ to vary from 1 to 160. \cmf{This parameterised electron temperature prescription is well matched to electron temperature distribution obtained by two-temperature GRMHD simulations including electron heating prescription \citep{Mizuno2021}.}

In \emph{non-thermal} models, the eDF has a power-law tail extending to high energy.
We explore two implementations in this paper:
\emph{i}) a power-law distribution function
\begin{align}
  \frac{1}{n_e} \frac{d n_e}{d\gamma} &=
  \frac{p-1}{\gamma_{\min}^{1-p} - \gamma_{\vphantom{i}\max}^{1-p}}
  \gamma^{-p},
  \label{eq:nonthermaleDF}
\end{align}
which has power-law index $p$ and upper and lower limits $\gamma_{\min}$ and $\gamma_{\vphantom{i}\max}$; and
\emph{ii}) a so-called $\kappa$ distribution function, inspired by observations of the solar wind and by results of collisionless plasma simulations \citep[e.g.,][and references therein]{2015JPlPh..81e3201K}
\begin{align}
  \frac{1}{n_e} \frac{d n_e}{d\gamma} =
  \gamma \sqrt{\gamma^2-1} \left(1+\frac{\gamma+1}{\kappa w}\right)^{-(\kappa+1)},
\end{align}
which has width parameter $w$ and power-law index parameter $\kappa$.

Evidently, any eDF assignment scheme is an approximation since the eDF depends in general on both local conditions and particle histories.  Notice that we also assume the eDF is isotropic and neglect electron-positron pairs (see Section~\ref{sec:pair} for a discussion of potential impact of pair plasma for \sgra).

Once the eDF is specified, the radiative transfer coefficients (emissivities, absorptivities, and rotativities) can be readily calculated; see \cite{2021ApJ...921...17M} for a recent summary.

% cfg 11 dec: decided this is not essential.
%\kc{Perhaps add a plot of the different eDFs to display the variety of models considered.}

%..............................................................................
\subsubsubsection{Model Scaling}

With the exception of the special stellar wind-fed simulations, the GRMHD models considered in this work contain a characteristic speed, $c$, but are otherwise scale-free; they set $GM = c = 1$.
Physical scales are assigned during the radiative transfer step.
The black hole mass $\mbh$ fixes the length unit $\rg$ and time unit $\tg$.
Because the GRMHD models are not self-gravitating, one is free to set a density scale, or equivalently the accretion rate $\dot{M}$ or plasma mass scale $\Munit$ \citep[see, e.g.,][for a full discussion]{Wong_2022}.

% cfg, 11 dec 21: resolved this by reference to the PATOKA paper.  We really don't need all this anywhere else in the paper.
%The density in cgs units $\rho_\mathrm{cgs}$ is obtained from the density in simulation units $\rho_\mathrm{sim}$ via
%\begin{align}
%  \rho_\mathrm{cgs} = \rho_\mathrm{sim} \Munit_\mathrm{cgs} {(\rg)}_\mathrm{cgs}^{-3}.
%\end{align}
%Similarly, the energy density and magnetic fields in cgs units are set by:
%\begin{align}
%  u_\mathrm{cgs} &= u_\mathrm{sim} \Munit_\mathrm{cgs} {(\rg)}_\mathrm{cgs}^{-3} c_\mathrm{cgs}^2\\
%  B_\mathrm{cgs} &= (4\pi)^{1/2} B_\mathrm{sim} \Munit_\mathrm{cgs}^{1/2} (\rg)_\mathrm{cgs}^{-3/2} c_\mathrm{cgs};
%\end{align}
%the factor of $(4\pi)^{1/2}$ comes from converting Heaviside-Lorentz units (used in GRMHD simulations for efficiency) to cgs.
%\ckc{I found this notation confusion.  For example, $c$ is dimensional, although it's not necessary in cgs.  So I'm not sure what $c_\mathrm{cgs}$ really means here.  Is the dimensionless value of $c$ in cgs?  I'm probably too picky...}\monika{we just want to stress that c should be in cm/s}

The plasma mass scale parameter $\Munit$ controls the plasma emissivity and the plasma optical depth and thus the source brightness.  We adjust $\Munit$ iteratively until the time-averaged 230\GHz flux densities of the models are within a few percent of the $2.4\,\mathrm{Jy}$ mean observed during the 2017 campaign (see next section).  The reader should keep in mind that in this work model parameters are always varied at constant time-averaged millimeter flux density.

%Notice that the flux density is a nonlinear function of $\Munit$ because the accretion flow transits between the optical thick and thin regimes around 230\GHz.
%This makes the interpretations of the model trends in Section~\ref{sec:trends} less obvious.

%..............................................................................
\subsubsubsection{Radiative Transfer Calculation}

\begin{figure*}
  \centering
  \includegraphics[width=\textwidth]{figures/sample_imgs.pdf}
  \caption{Sample 86\GHz (top row) and 230\GHz (bottom row) images from the simulation library.  Left column: thermal SANE; middle column: thermal MAD; right column: nonthermal variable $\kappa$ MAD.  The models were chosen to show the variety of possible images and do necessarily not not pass all constraints.
    % cfg 15 dec: edited remainder out
    %On one hand, most of the 230\GHz images are in an optically thin regime, making the sharp photon ring visibile (bottom left and middle panels). On the other hand, most of the 86\GHz images are in an optically thick regime, and the accretion disk would block the sharp photon ring (top middle and right panels). In general, face-on images like the ones in the left column tend to be too big in size, and the edge-on images in the right column tend to be too asymmetric.
    }
  \label{fig:sample_imgs}
\end{figure*}

\begin{figure*}
  \centering
  \includegraphics[width=\textwidth]{figures/sample_seds.pdf}
  \caption{Sample SEDs from the models shown in
    Figure~\ref{fig:sample_imgs}:
    (\emph{left})~thermal SANE,
    (\emph{middle})~thermal MAD, and
    (\emph{right})~non-thermal variable $\kappa$ MAD.  Black dots and arrows show measured flux density at $86\GHz$ and $230\GHz$ and $2.2\um$ and x-ray quiescent upper limits. The black line shows the total SED; colored lines show contributions from direct synchrotron emission, Compton scattering, and bremsstrahlung (see legend).
    % cfg 15 dec: edited out the rest
    %The left panel shows a interesting special case in our simulation library, where the spectral index at 230\GHz is negative.  This indicates the 230\GHz image is completely optically thin and the 86\GHz would over produce the flux.  The NIR has a flat spectral index and is dominated by inverse Compton, instead of direct synchrotron radiation. These properties occur only for $\Rh = 1$ SANE models.     For $\Rh > 1$, the direct synchrotron peak shifts to the right, broadens its width, and dominates the NIR. For SANE, the spetral index for x-ray is almost always positive and dominated by thermal bremsstrahlung by (relatively) cool electrons. The middle and right panels show the more typical MAD SEDs, where 86\GHz, 230\GHz, and NIR all come from direct synchrotron, and x-ray is dominated by inverse Compton.
    }
  \label{fig:sample_seds}
\end{figure*}

Given an eDF, density scale $\Munit$, and radiative transfer coefficients based on local properties of plasma, the emergent radiation is obtained by integrating the radiative transfer equation.
We use two classes of numerical methods: ray tracing to generate synthetic images, and Monte Carlo to generate spectral energy distributions (SEDs).
Further detail on numerical methods is given Appendix~\ref{app:radtrans}.
Comparisons of numerical methods \citep{2020ApJ...897..148G, Prather_et_al_2022} show that differences between radiative transfer schemes do not contribute significantly to the error budget.
% cfg 11 dec: done
%\ckc{Ben, please fill in paper title etc for the polarized radiative transfer comparison paper in refs.bib.}

The models are imaged at 86\GHz, 230\GHz\footnote{The mean frequency for  EHT observations is closer to 228\GHz.} and 2.2\um (near infrared, herafter NIR).
Direct imaging includes synchrotron and bremsstrahlung \citep[both ion-electron and electron-electron; see][for a recent review]{2020ApJ...898...50Y}.
The SED is averaged over a narrow range in inclination and over all azimuthal angles.
The SED includes synchrotron, bremsstrahlung, \emph{and} Compton scattering.
NIR emission is usually dominated by synchrotron, but we find that occasionally NIR synchrotron is so weak that Compton scattering overtakes.
The x-ray can be dominated by bremsstrahlung or Compton scattering, mainly depending on the electron temperature.
In Figures~\ref{fig:sample_imgs} and \ref{fig:sample_seds}, we present an illustrative set of model images and multiwavelength SEDs from our library.

All radiative transfer models used here set the emissivity and inverse-Compton scattering cross-sections to $0$ at $\sigma \equiv B^2/(8\pi\rho c^2) > \sigma_\mathrm{cut} = 1$.  This cutoff is motivated by the notion that the numerical solution is unreliable at $\sigma > \sigma_\mathrm{cut}$.  In particular numerical diffusion into the low density region (at the ``funnel wall'') makes the density too large, and truncation error in integration of the total energy equation produces large fractional errors in temperature.

%==============================================================================
\subsection{Summary of \sgra Model Library}

%You can check which data is available here:
%https://docs.google.com/spreadsheets/d/1gw9ichvvYGHLFsZl2wlxqu-O03qEULrwcw3Wixd8BhQ/edit#gid=930351969
%not sure that gives complete answer, but it will help.

A summary of all radiative transfer calculations is given in Table~\ref{tab:radiativemodels}. The entire image library
contains $6$ model sets; thousands of points in model parameter space and
%$364$ models (with different spins and electron prescriptions);
%$1,872,409 @230GHz and 86 GHz$
$\sim 1.8$M images in each of 86~GHz, 230~GHz, and $\sim0.5$M images in $2.2\mu$m;
%$1,233,100+60,000 SEDs$
$\sim1.3$M SEDs.  The images and SEDs occupy about $50$TB.

%\clearpage
%\pagebreak
%\movetabledown=3cm % recommended in AASTeX docs to center table on page
%\begin{rotatetable}
\begin{deluxetable*}{ccccccccccc}
\tabletypesize{\footnotesize}
\renewcommand{\arraystretch}{1.1}
\tablehead{
  \colhead{Model}              &%
  \colhead{$\Rl$}              &%
  \colhead{$\Rh$}              &%
  \colhead{$\beta_{\rm crit}$} &%
  \colhead{$p$}           &%
% \colhead{$\gamma_{\rm min/max}$} &%
  \colhead{$\kappa$}           &%
  \colhead{$i^\circ$}          &%
  \colhead{$\nu$}              &%
  \colhead{MWL SED}            &%
  \colhead{$\Delta t$}         &%
  \colhead{$\#frames$}     \\
  & & & & & & & \colhead{[GHz]} & & \colhead{[1000 M]} & [@230 GHz]}
\startdata
\multicolumn{11}{c}{\bf Thermal models}\\
\kharma        & 1 & [1,10,40,160] & 1 & - & - & [10,30,...,170] & [86,230] & Yes & [15,20) & 360,000 \\
\kharma        & 1 & [1,10,40,160] & 1 & - & - & [10,30,...,170] & [86,230] & Yes & [20,25) & 360,000 \\
\kharma        & 1 & [1,10,40,160] & 1 & - & - & [10,30,...,170] & [86,230] & Yes & [25,30) & 360,000 \\
\bhac          & 1 & [1,10,40,160] & 1 & - & - & [10,30,...,90]  & [86,230] & Yes & [10,15) & 100,000 \\
\bhac          & 1 & [1,10,40,160] & 1 & - & - & [10,30,...,90]  & [86,230] & Yes & [20,25) & 100,000 \\
\bhac          & 1 & [1,10,40,160] & 1 & - & - & [10,30,...,90]  & [86,230] & Yes & [25,30) & 100,000 \\
\hamr          & 1 & [1,40,160]    & 1 & - & - & [10,50,90]      & [86,230] & Yes & [30,35) &  45,000 \\
\koral         & 1 & [20]          & 1 & - & - & [10,30,...,170] & [86,230] & No  & [5,100) & 107,507 \\
\hamr Tilted   & 1 & [1,40,160] & 1 & - & - & [10,50,90] & [86,230] & Yes & [100-103) & 8,100 \\
Wind Accretion & 1 & [13,28]    & 1 & - & - & N/A        & [86,230] & No  & 10        & 1,802 \\
\hline
\multicolumn{11}{c}{\bf Critical $\beta$ model} \\
\kharma & 1 & [1,40,160] & 1 & 4 & - & [10,50,90] & [86,230] & No & [30-35) & 45,000 \\
\hline
\multicolumn{11}{c}{\bf Thermal + non-thermal power-law models} \\
\hamr & 1 & [1,40,160] & 1 & 4 & - & [10,50,90] & [86,230] & No & [30-35) & 45,000 \\
\hline
\multicolumn{11}{c}{\bf Thermal + non-thermal $\kappa$ models} \\
\bhac & 1 & [1,10,40,160]    & 1 & - & 5 & [10,30,...,90]  & [86,230] & No & [25-30) & 20,000 \\
\bhac & 1 & [1,10,40,160]    & 1 & - & 3.5 ($\epsilon_0=0.05$) & [10,30,...,90]  & [86,230] & No & [25-30) & 20,000 \\
\bhac & 1 & [1,10,40,160]    & 1 & - & 3.5 ($\epsilon_0=0.10$) & [10,30,...,90]  & [86,230] & No & [25-30) & 20,000 \\
\bhac & 1 & [1,10,40,160]    & 1 & - & 3.5 ($\epsilon_0=0.20$) & [10,30,...,90]  & [86,230] & No & [25-30) & 20,000 \\
\bhac & 1 & [1,10,40,80,160] & 1 & - & variable $\kappa=\kappa(\beta,\sigma)$ & [10,30,...,90] & [86,230] & No  & [25-30) & 125,000 \\
\hamr & 1 & [1,10,40,160]    & 1 & - & variable $\kappa=\kappa(\beta,\sigma)$ & [10,30,...,90] & [86,230] & Yes & [30-35) & 100,000 \\
\enddata
\caption{Summary of emission simulations in \sgra EHT model library. For the Wind Accretion model the viewing angle is set by the boundary conditions used in the model and $\Rh$ is set so the model matches the observed 230\,GHz flux.  The two reported values correspond to two models with different stellar wind magnetizations.
%\monika{last time (Dec 11) I checked google table BHAC did not have MULTIWAVELENGTH spectra, so why did you change No to Yes? \cmf{Hi Monica, we have SEDs for thermal but not for non-thermal, I think I included this in one of the google docs \url{https://docs.google.com/spreadsheets/d/11ZP_jz0ub-SLPwXft4x_9byDS2KBsP1KEIt9xn77LoU/edit#gid=0} (row 12, maybe the confusion is due to row 20) }}
%\monika{why is wind model inclination N/A? can anyone explain?}
%\gw{I didn't write N/A, but as I recall from Sean, the wind models *have* an absolute orientation with respect to Earth because of how their boundary condition is set up, so it wouldn't make sense to vary inclination. Also their bhspin = 0.}\monika{Ah, ok , so there is an inclination but we dont know it. can you find out ? maybe I can also change the Fig 1 accordingly (although that would require some thinking)!}\gw{sorry not sure what you mean by ``there is an inclination''. there is an inclination that we plug into ipole, but that inclination is with respect to the coordinate system, not anything particularly ``physical'' (because the black hole has zero spin). the coordinate system is oriented such that images are computed at inc->180 (inc as defined in the ipole sense)}\monika{ok, this makes no sense since from our point of view the "boundary conditions" for this model are fixed...we should just find best rhigh for our viewing angle..}
%The cadence of KHARMA=5M, BHAC=10M for thermal and nonthermal variable kappa, 50M for nonthermal variable efficiency models, and HAMR=10M for thermal, nonthermal powerlaw and variable kappa.
%\gw{Rhigh for the wind accretion models is either 28 or 13, but these are for two separate simulations (so it's not like there's one simulation where Rhigh has been varied); no SED from grmonty-like codes, but there is ``SED''-like information from imaging at different frequencies}\monika{how many 230 ghz frames we have for these two wind models and within what times??}
}
\label{tab:radiativemodels}
\end{deluxetable*}
%\end{rotatetable}

%\pagebreak
%\clearpage
% %..............................................................................
% \subsubsection{Thermal Models}
%
% ...
%
% \paragraph{Illinois Models}
%
% % Please fill in basic information of the models in the following list.
% % Please add more details if necessary. A full paragraph description
% % of the model is welcome, but not required at this point.
% \begin{itemize}[noitemsep]
% \item $a_\mathrm{spin}$: 0, $\pm1/2$, $\pm15/16$
% \item Magnetic Flux: MAD, SANE
% \item Adiabatic Index $\Gamma$: 4/3
% \item Time $t_\mathrm{final}$: 30,000$M$
% \item $\rho_0$: 3 different density normalization chosen for each parameter set for $t \in [15,000, 20,000), [20,000, 25,000), [25,000, 30,000)$.
% \item $\Rh$: 1, 10, 40, 160
% \item Inclination $i$: 10$^\circ$, 30$^\circ$, 50$^\circ$, ..., 170$^\circ$
% \item Resolution:
% \item Initial conditions:
% \item Reference: this work
% \item Status: w4 and w5 all done; w3 in progress
% \end{itemize}
%
% \paragraph{Frankfurt Models}
%
% % Please fill in basic information of the models in the following list.
% % Please add more details if necessary. A full paragraph description
% % of the model is welcome, but not required at this point.
% \begin{itemize}[noitemsep]
% \item $a_\mathrm{spin}$: 0, $\pm1/2$, $\pm15/16$
% \item Magnetic Flux: MAD, SANE
% \item Adiabatic Index $\Gamma$: 4/3
% \item Time $t_\mathrm{final}$: 30000
% \item $\rho_0$: 3 different density normalizations chosen for each parameter set for $t \in [10,000, 15,000), [20,000, 25,000), [25,000, 30,000)$
% \item $\Rh$: 1, 2.5, 5, 10, 40, 160
% \item Inclination $i$: 10$^\circ$, 30$^\circ$, 50$^\circ$,..., 90$^\circ$
% \item Resolution:
% \item Initial conditions:
% \item Reference: this work
% \item Status: all done except for SANE a=-15o16
% \end{itemize}
%
% \paragraph{HAMR Models}
%
% % Please fill in basic information of the models in the following list.
% % Please add more details if necessary. A full paragraph description
% % of the model is welcome, but not required at this point.
% \begin{itemize}[noitemsep]
% \item $a_\mathrm{spin}$: 0, $\pm1/2$, $\pm15/16$
% \item Magnetic Flux: MAD, SANE
% \item Adiabatic Index $\Gamma$: 13/9, 5/3
% \item Time $t_\mathrm{final}$: $35,000M$
% \item $\rho_0$: 1 density normalization for $[30,000-35,000)M$
% \item $\Rh$: 1, 40, 160
% \item Inclination $i$: 10$^\circ$, 50$^\circ$, 90$^\circ$
% \item Resolution: $348\times 192\times 192$, $240\times 192\times 192$
% \item Initial conditions: FM: $r_{\rm in}=6, 20M$; $r_{\rm pmax}=12, 41M$
% \item Grid outer radius: $1000M$, $200M$
% \item Reference: this work
% \item Status: GRMHD simulations done
% \end{itemize}
%
% %..............................................................................
% \subsubsection{Non-thermal (power-law )Models}
%
% ...
%
% \paragraph{Frankfurt Models}
%
% % Please fill in basic information of the models in the following list.
% % Please add more details if necessary. A full paragraph description
% % of the model is welcome, but not required at this point.
% \begin{itemize}[noitemsep]
% \item $a_\mathrm{spin}$: 0, $\pm1/2$, $\pm15/16$
% \item Magnetic Flux: MAD, SANE
% \item Adiabatic Index $\Gamma$:
% \item Time $t_\mathrm{final}$:
% \item $\rho_0$:
% \item Power law fraction $f$:
% \item Power law index $p$:
% \item Inclination $i$:
% \item Reference:
% \item Status: no power-law model so far
% \end{itemize}

% \paragraph{HAMR Models}

% % Please fill in basic information of the models in the following list.
% % Please add more details if necessary. A full paragraph description
% % of the model is welcome, but not required at this point.
% \begin{itemize}[noitemsep]
% \item $a_\mathrm{spin}$: 0, $\pm1/2$, $\pm15/16$
% \item Magnetic Flux: MAD, SANE
% \item Adiabatic Index $\Gamma$: 13/9, 5/3
% \item Time $t_\mathrm{final}$: $35,000M$
% \item $\rho_0$: 1 density normalization for $[30,000-35,000)M$
% \item $\Rh$: 1, 40, 160
% \item Inclination $i$: 10$^\circ$, 50$^\circ$, 90$^\circ$
% \item Resolution: $348\times 192\times 192$, $240\times 192\times 192$
% \item Initial conditions: FM: $r_{\rm in}=6, 20M$; $r_{\rm pmax}=12, 41M$
% \item Grid outer radius: $1000M$, $200M$
% \item Reference: this work
% \item Status: GRMHD simulations same as for thermal models
% \end{itemize}
%
% %..............................................................................
% \subsubsection{Non-thermal ($\kappa$) Models}
%
% ...
%
% % Please fill in basic information of the models in the following list.
% % Please add more details if necessary. A full paragraph description
% % of the model is welcome, but not required at this point.
% \begin{itemize}[noitemsep]
% \item $a_\mathrm{spin}$: 0, $\pm1/2$, $\pm15/16$
% \item Magnetic Flux: MAD, SANE
% \item Adiabatic Index $\Gamma$:
% \item Time $t_\mathrm{final}$:
% \item $\rho_0$:
% \item $\kappa$:
% \item Inclination $i$:
% \item Reference:
% \item Status:
% \end{itemize}
%
% \paragraph{Frankfurt Models}
%
% % Please fill in basic information of the models in the following list.
% % Please add more details if necessary. A full paragraph description
% % of the model is welcome, but not required at this point.
% \begin{itemize}[noitemsep]
% \item $a_\mathrm{spin}$: 0, $\pm1/2$, $\pm15/16$
% \item Magnetic Flux: MAD, SANE
% \item Adiabatic Index $\Gamma$: 4/3
% \item Time $t_\mathrm{final}$: 30000
% \item $\rho_0$: individual normalisation for each kappa model; only  for $t \in [25,000, 30,000)$
% \item $\kappa$: variable $\kappa(\beta, \sigma)$, fixed $\kappa=3.5$ with $\epsilon=\epsilon_{0} f(\beta,\sigma)$ for $\epsilon_{0}=0.05,0.10,0.20$
% \item $\Rh$: 1, 2.5, 5, 10, 40, 160
% \item Inclination $i$: 10$^\circ$, 30$^\circ$, 50$^\circ$,..., 90$^\circ$
% \item Reference: this work
% \item Status: in production
% \end{itemize}
%
% \paragraph{HAMR Models}
%
% % Please fill in basic information of the models in the following list.
% % Please add more details if necessary. A full paragraph description
% % of the model is welcome, but not required at this point.
% \begin{itemize}[noitemsep]
% \item $a_\mathrm{spin}$: 0, $\pm1/2$, $\pm15/16$
% \item Magnetic Flux: MAD, SANE
% \item Adiabatic Index $\Gamma$: 13/9, 5/3
% \item Time $t_\mathrm{final}$: $35,000M$
% \item $\rho_0$: 1 density normalization for $[30,000-35,000)M$
% \item $\kappa$: variable $\kappa (\beta, \sigma)$
% \item $\Rh$: 1, 40, 160
% \item Inclination $i$: 10$^\circ$, 50$^\circ$, 90$^\circ$
% \item Resolution: $348\times 192\times 192$, $240\times 192\times 192$
% \item Initial conditions: FM: $r_{\rm in}=6, 20M$; $r_{\rm pmax}=12, 41M$
% \item Grid outer radius: $1000M$, $200M$
% \item Reference: this work
% \item Status:
% \end{itemize}
%
% %..............................................................................
% \subsubsection{Critical $\beta$ Models}
%
% % Please fill in basic information of the models in the following list.
% % Please add more details if necessary. A full paragraph description
% % of the model is welcome, but not required at this point.
% \begin{itemize}[noitemsep]
% \item $a_\mathrm{spin}$:
% \item Magnetic Flux: MAD, SANE
% \item Adiabatic Index $\Gamma$:
% \item Time $t_\mathrm{final}$:
% \item $\rho_0$:
% \item Power law fraction $f$:
% \item Power law index $p$:
% \item Inclination $i$:
% \item Reference:
% \item Status:
% \end{itemize}
%
% %..............................................................................
% \subsubsection{Stellar Wind Accretion Models}
%
% % Please fill in basic information of the models in the following list.
% % Please add more details if necessary. A full paragraph description
% % of the model is welcome, but not required at this point.
% \begin{itemize}[noitemsep]
% \item $a_\mathrm{spin}$: 0
% \item Magnetic Flux: MAD, SANE
% \item Adiabatic Index $\Gamma$:
% \item Time $t_\mathrm{final}$:
% \item $\rho_0$:
% \item Power law fraction $f$:
% \item Power law index $p$:
% \item Inclination $i$:
% \item Reference:
% \item Status:
% \end{itemize}
%
% %..............................................................................
% \subsubsection{Koral Long MAD Models}
%
% % Please fill in basic information of the models in the following list.
% % Please add more details if necessary. A full paragraph description
% % of the model is welcome, but not required at this point.
% \begin{itemize}[noitemsep]
% \item $a_\mathrm{spin}$: 0, $\pm0.3$, $\pm0.5$, $\pm0.7$, $\pm0.9$
% \item Magnetic Flux: MAD
% \item Adiabatic Index $\Gamma$:
% \item Time $t_\mathrm{final}$: 100,000$M$
% \item $\rho_0$:
% \item Power law fraction $f$:
% \item Power law index $p$:
% \item Inclination $i$:
% \item Reference:
% \item Status:
% \end{itemize}
%
% %..............................................................................
% \subsubsection{Tilted Models}
%
% % Please fill in basic information of the models in the following list.
% % Please add more details if necessary. A full paragraph description
% % of the model is welcome, but not required at this point.
% \begin{itemize}[noitemsep]
% \item $a_\mathrm{spin}$: $+15/16$
% \item Magnetic Flux: INSANE
% \item Adiabatic Index $\Gamma$: 5/3
% \item Time $t_\mathrm{final}$: $>100,000M$
% \item $\rho_0$: 1 density normalization for $[100,000-103,000)M$
% \item $\Rh$: 1, 40, 160
% \item Inclination $i$: 10$^\circ$, 50$^\circ$, 90$^\circ$
% \item Resolution: $448\times 144\times 240$,
% \item Initial conditions: FM: $r_{\rm in}=12.5M$; $r_{\rm pmax}=25M$
% \item Grid outer radius: $100,000M$
% \item Reference: Chatterjee+20, Liska+18
% \end{itemize}


\section{Observational Constraints}\label{sec:observations}

\sgra is one of the most observed objects in the sky.
We have data obtained with a slew of telescopes, across 5 decades in time and more than 17 decades in frequency. We need to select a manageable subset of this data for comparison with the models. In doing so we have attempted to select (1) approximately uncorrelated constraints, so that each can test a distinct aspect of the model; (2) constraints based on data that can be simulated with the models; (3) constraints based on EHT 2017 1.3mm VLBI data or based on photons produced within or close to the 1.3mm emission region that are contemporaneous or near-contemporaneous.

%==============================================================================
% \subsubsection{Scattering Models}

%==============================================================================
\subsection{EHT Observational Constraints}

\ckc{ck's first pass}
\mw{I think it would be good to mention the EHT array composition (name the telescopes)}
\cfg{much of this material could be incorporated by reference to paperII}

\begin{figure*}
  \centering
  %\includegraphics{}
  [altex: (left) visibility amplitude vs baseline for the day(s) that
    this study use, overplotted by the visibility amplitude from a
    fiducial model.
    Similar to paper~II, figure~7.
    Visually mark null location constraint, pre-imaging size (i.e.,
    second moment).
    (right) SED from Gurther overplotted by the SEDs from a fiducial
    model.
    Visually mark the SED constraints.]
  \caption{(\emph{left}) Measured correlated flux densities of \sgra
    on [DAY X] from the HOPS pipeline overplotted with a fiducial
    GRMHD+GRRT model.
    Details on the data can be found in paper~II, section~5.
    A description of the fiducial model is in
    section~\ref{sec:models}.
    (\emph{right}) \ckc{Q: do we want to show only EHT observation and
      referencing to other papers for non-EHT constraints?
      Or should we have representative figures for all measurements?}}
  \label{fig:visibility}
\end{figure*}

The EHT observed \sgra at 1.3 mm during the 2017 April 5--11 observing campaign and obtained horizon scale complex visibilities.  Figure~\ref{fig:visibility} shows the visibility amplitudes on April 6 and 7 from  the HOPS pipeline, overplotted with visibility amplitudes derived from a model.  Evidently there are nulls near $\sim 5\mathrm{G}\lambda$ and $7\mathrm{G}\lambda$, suggesting a ringlike structure.

We test the models using interferometric data in three ways.  First, we compare the location of the first null and the visibility amplitude of long baselines to the model (``null constraint'').  Second, we compare an estimate of the source size (``second moment constraint'') from short baselines to the model.  Finally, we follow a variant of the procedure used in Paper~\ref{PaperIV} and compare fits for the diameter, width, and asymmetry of an m-ring to the model.  

%------------------------------------------------------------------------------
\subsubsection{230\,GHz VLBI Null Locations}

\ckc{ck's first pass}

One way we characterize the GRMHD simulations carried out for
different black hole and flow parameters and assess their
compatibility with the \sgra\ data involves an analysis in the visibility
domain.
Figure~\ref{fig:visibility} shows two example snapshots and the
corresponding visibility amplitudes (VA) as a function of baseline
length for a vertical and horizontal image cross sections from a
simulation.
Even though the locations and depths of the minima in the visibility
amplitudes are primarily set by the image size, which is set by the
black hole mass, this figure shows that they can exhibit significant
variability from snapshot to snapshot because of the multitude of
structures that originate in the turbulent flow (Medeiros et
al. 2018).
In particular, the minima tend to move to larger baselines as the ring
thickness temporarily changes in response to, e.g., the appearance of
a flux tube or a similar structure.
In addition, images that have a higher degree of azimuthal asymmetry
show pronounced differences in their vertical and horizontal cross
sections.

The degree of VA variability is different from model to model and also
depends on some of the global characteristics of the flow.
For example, the overall electron density in the disk plays a role by
its effect on the ring thickness: thicker rings show more change in
the visibility minima between snapshots than thinner ones (see also
Satapathy et al. 2021 for the effect on closure phases).
Because of this, while no model is expected to resemble the observed
visibility amplitude data 100\% of the time, it is nevertheless
possible as well as discriminating to require an agreement between the
locations of the minima in each simulation snapshot and the visibility
amplitude data from \sgra\ a reasonable fraction of the time.

To carry out the comparison with the data, we focus on the VA observed
on 2017 April 7, \#3599 because that night has the best u-v coverage
near the minima.
The first visibility minima in both the N-S and E-W directions occur
between $2.5-3.5$\;G$\lambda$, as we show in
Figure~\ref{fig:cmp_null}.
We define compliance for a snapshot by requiring that the VA obtained
for {\it either} the horizontal {\it or} the vertical cross section of
a snapshot have a minimum in this range of baseline lengths.

A second feature of the visibility amplitudes that can help discern
between models is the behavior of the VA at long baselines.
April 7 data also show that the amplitudes have declined to $<6\%$ of
the zero baseline flux at baselines between $6-8$\;G$\lambda$ along
all orientations.
This is characteristic of ring-like images with a relatively high
degree of azimuthal symmetry.
Images that are more asymmetric, on the other hand, lead to
significantly higher amplitudes at baselines much larger than the
first minimum, or no minima at all, as the lower panels in
Figure~\ref{fig:cmp_VA} illustrate.
As a result, selecting models based on how frequently they produce
snapshots with such large-baseline power helps identify models that
are in accordance with the data.

One consideration when comparing models to data at long baselines is
the effect of interstellar scattering.
Diffractive scattering has the effect of convolving the image with a
smooth kernel and can reduce the amplitudes to $\sim 70\%$ of their
intrinsic value in the $6-8$\;G$\lambda$ range (REF).
Refractive scattering, on the other hand, introduces a noise at these
baselines of the order of $0.5-3\%$, depending on the particular
characteristics of the scattering screen toward \sgra\ (REF).
To account for both of these effects, we choose a $6\%$ upper limit to
visibility amplitudes from a model snapshot when defining compliance,
as we show in Figure~\ref{fig:cmp_null}.

We assign a compliance fraction to a simulation based on the fraction
of snapshots that pass both criteria we described above.
We will discuss the results of this comparison in the next section,
along with the other model scoring criteria.

%------------------------------------------------------------------------------
\subsubsection{230\,GHz VLBI Pre-Image Size}

\ckc{ck's first pass}

The second moment of an EHT source corresponds, in the $uv$ domain, to
the curvature of the visibility amplitude near zero baseline length.
That is, the 2nd moment tensor
\begin{equation}
    \sigma_{ij} \equiv \int \, d^2x\, x_i x_j I/\int \, d^2x \, I = (2\pi)^2 \left(\partial_i \partial_j \tilde{I}\right)/\tilde{I}
\end{equation}
where $I$ is the intensity, $\tilde{I}$ its Fourier transform, $i =
x,y$ on the left and $i = u,v$ on the right, and the terms on the
right are evaluated at $u = v = 0$.
A pair of visibility amplitudes ($|\tilde{I}|$) on short baselines or
zero can therefore be used to estimate the second moments of the
image.

This procedure is used in EHT3 to set an upper limit of $95\mu$as on
the scattered source size and lower limit of $38\mu$as.

NOTE: data is descattered.

%------------------------------------------------------------------------------
\subsubsection{230\,GHz M-Ring Fitting}

EHT4 fits an ``m-ring'' source model to the 7 April data.  The simplified m-ring source-plane model used here is a $\delta$ function in radius with diameter $d$ multiplied by a truncated (for $m > 3$) Fourier series, convolved with a Gaussian of with $w$.  In addition the model contains a centered Gaussian component to absorb large scale emission and emission interior to the ring.

The simplified m-ring model has 10 parameters; 3 are commonly well constrained and physically interpretable and are therefore used here: the m-ring diameter $d$, the m-ring width $w$ (FWHM of the convolved Gaussian), and the $m=1$ relative amplitude $\beta_1$.

For the comparison dataset we selected 10 120s scans spread approximately uniformly through the best-times region on 7 April.  The scans were selected to .  Detections are obtained on LMT baselines for 7 of the scans.  Only 10 scans were selected due to computational cost.   No changes in model selection were observed if any one scan was removed from the comparison.  The data were de-scattered, that is, the visibility amplitudes were divided by the refractive scattering kernel.  Maximum likelihood m-ring parameters were found for each scan.  

We then read in a series of model images, generate synthetic data from the models, and fit m-rings to the synthetic data.  This produces a distribution of m-ring parameters for each model.   

The synthetic data is generated as follows.  The image $I(x,y)$ is fourier transformed to complex visibilities $V(u,v)$, and then sampled on baselines $i$ drawn from the comparison scan, $V_i \equiv V(u_i,v_i)$.  Normally distributed thermal noise $\delta V_{th,i}$ based on telescope performance during the scan is added, and multiplicative, normally distributed noise $N$ is added to (very crudely) model gain corrections: $\tilde{V}_i = V_i (1 + \epsilon N) + \delta V_{th,i}$; the synthetic data used here has $\epsilon = 0.05$, but no significant changes in fit parameters were observed for $\epsilon = 0.02$.  We then fit to the resulting visibility amplitudes $|\tilde{V}_i|$ and closure phases arg$(\tilde{V}_i \tilde{V}_i \tilde{V}_j \tilde{V}_k^*$, where $ijk$ form a triangle in the uv plane.  



NOTE: data is descattered.

%------------------------------------------------------------------------------
\subsubsection{230\,GHz VLBI Ring Thickness}

%------------------------------------------------------------------------------
\subsubsection{230\,GHz VLBI Ring Asymmetry}

%==============================================================================
\subsection{Non-EHT Observations Constraints}

%------------------------------------------------------------------------------
\subsubsection{86\,GHz Flux}

%------------------------------------------------------------------------------
\subsubsection{86\,GHz Image Size}

%------------------------------------------------------------------------------
\subsubsection{NIR (Non-Overproduction) Constraints}

%------------------------------------------------------------------------------
\subsubsection{X-ray (Non-Overproduction) Constraints}

%==============================================================================
\subsection{Time Dependence}

\ckc{we can model "long-term climate" well but not "long-term weather".  Hence we only compare the slowly varying quantities such as mean 2nd moments (climate) but not modulation index (depends on weather).  TODO: look up correct statistical terminology in weather/climate modeling and build that this into this paper.}

%------------------------------------------------------------------------------
\subsubsection{EHT (Baseline variability)}

\ckc{I think Boris needs to write this...}

%------------------------------------------------------------------------------
\subsubsection{ALMA Light curves}
\mw{I edited this a bit}
\\
ALMA and SMA produced \sgra light curves at 230GHz as a byproduct of the
2017 EHT VLBI observing campaign. The complete set of light curves is presented and analyzed in \cite{wielgus2021}. We have chosen to compare the models to the 7 April 2017 ALMA light using
the 3 hour {\em modulation index} $M_3$, where $M_T \equiv
\sigma/\mu$, $\sigma$ is the standard deviation measured over some
interval $T$, and $\mu$ is the mean measured over the same interval.

We use $M_T$ because it is: easy to describe; easy to compute;
commonly used in the literature (in the X-ray astronomy literature it
is ``rms \%''); and closely related to the structure function, since
the expectation value for $\sigma^2$ is given by an integral over the
structure function (see Lee et al. 2021).
We use $T = 3$ hours because: 3 hours is similar to the correlation
time for $F_{230}$ in most of the models; 3 hours is similar to the
characteristic timescales measured in damped random walk fits to the
ALMA lightcurve \citep[see Table 10 of][]{wielgus2021}; 3 hours is the
longest timescale for which we can consistently estimate the mean and
variance of the distribution of M$_3$ from the models.
For a damped random walk process one can show that $M_3$ is very weakly
correlated over successive 3 hour intervals (Lee et al. 2021).

The constraint comes from $M_T$ measured over 3 maximally spaced intervals
in the 7 April 2017 ALMA light curve, where $M_3 = 0.024, 0.051,
0.047$ (Wielgus et al. 2021). These values are consistent with being drawn from the distribution estimated from historical non-EHT 2005-2017 light curves.
%curves. \mw{maybe sth like "from the estimated distribution of the historical 2005-2017 lightcurves"?}

%==============================================================================
\subsection{Future Constraints}

\monika{this section should be moved to the end of the paper}
\ckc{Agree}
integrated polarization,

resolved polarization

fits to more sophisticated models such as RIAF analytic models,

closure phase variability


\section{Model Comparison}\label{sec:comparisons}

We now apply the observational constraints from Section \ref{sec:observations} to the models described in Section \ref{sec:constraints}.

%==============================================================================
\subsection{Thermal, Aligned Models}

We begin with a set of ``standard'' models with aligned (prograde or retrograde) GRMHD simulations, thermal eDFs, and electron temperature assigned according to the $\Rh$ model, as in \citetalias{M87PaperV}.  This includes the Illinois, Frankfurt, and HAMR model sets (see Table \ref{tab:radiativemodels}).

%------------------------------------------------------------------------------
\subsubsection{EHT Constraints}

How do the models fare when compared to the EHT data alone, using the null location, size, and m-ring fitting constraints?  The null location test is informative and tends to rule out edge-on models.  The pre-image size constraint is simple but uninformative: only two face-on, $\Rh = 1$ models fail the test.   The m-ring fitting is noisy but highly informative.  Many thermal model distributions look like the data, but edge-on models are strongly disfavored.

%..............................................................................
\subsubsubsection{Null Location}

\note{CK to summarize}

% Statement about consistency between overlapping model sets.

\begin{figure*}
  \centering
  %\includegraphics[width=0.5\textwidth]{figures/SANE_snapshot.pdf}%
  %\includegraphics[width=0.5\textwidth]{figures/MAD_snapshot.pdf}\\
  \includegraphics[width=0.5\textwidth]{figures/SANE_va.pdf}%
  \includegraphics[width=0.5\textwidth]{figures/MAD_va.pdf}
  \caption{The left panels show two snapshots from a GRMHD simulation
    with SANE field configuration and black hole spin $a=0.5$ and the
    right panels the corresponding visibility amplitudes for a
    horizontal and a vertical cross section through the images.
    The snapshot in the top row obeys both selection criteria: the
    minima are in the 2.5-3.5 G$\lambda$ range and the amplitude in
    the $6-8$G$\lambda$ is below 6\%.
    The image in the bottom row, on the other hand, is an example that
    has no minimum in one cross section and too much power at long
    baselines, due to the asymmetry introduced by a transient bright
    structure in the flow.
    \ckc{The above plot looks odd because it's just one of the snapshot.  Do we want to show the mean?  The range?}}
  \label{fig:cmp_VA}
\end{figure*}

%\begin{figure}
%  \centering
%  \includegraphics[width=0.5\textwidth]{figures/va_compare.pdf}
%  \caption{Visibility amplitude as a function of baseline length
%    observed on 2017 April 7.
%    The pink band marks the location of the first minima in the
%    visibility amplitude along different orientations.
%    The horizontal red line marks our conservative upper limit for the
%    observed visibility amplitude between $6-8$G$\lambda$.}
%  \label{fig:cmp_null}
%\end{figure}

\begin{figure}
  \centering
  \includegraphics[width=\columnwidth]{./figures/Null_loc_Constraints.png}
  \caption{Null Location Constraint\ckc{Texts/labels in pizza plots too small to read.}}
  \label{fig:cmp_ozel}
\end{figure}

The null location constraint rules out edge-on MAD models at positive spin and a few large $\Rh$ SANE models.

The models that fail tend to...

Null location constraint passes 77\% of models.

%..............................................................................
\subsubsubsection{Second Moment}

\begin{figure}
  \centering
  \includegraphics[width=\columnwidth]{./figures/230GHz_size_Constraints.png}
  \caption{2nd moment plots}
  \label{fig:cmp_2nd_moment}
\end{figure}

The second moment constraint passes 99\% of models, that is, the models are all crudely the right size.  The models that fail are retrograde, face-on, SANE models with $\Rh = 1$. These models have extended rings of emission with angular extent that is large compared to the critical impact parameter.

%..............................................................................
\subsubsubsection{M-ring Fits}
\lae{sec:mring}

The m-ring fits pass 94\%, 65\%, and 36\% of models for the ring asymmetry, diameter, and width respectively.

The asymmetry parameter is typically not very well constrained.  The models that fail are almost all high inclination models with positive spin.  The failing models have asymmetries that are large and detectable because Doppler boosting concentrates emission in an equatorial spot on the approaching side of the disk.

\begin{figure}
  \centering
  \includegraphics[width=\columnwidth]{./figures/Mring_f1_Constraints.png}
  \caption{m-ring asymmetry}
  \label{fig:cmp_m-ring_asymm}
\end{figure}

The ring diameter is typically much better constrained than the asymmetry parameter, and it also varies systematically from model to model.  A much larger fraction of models therefore fails the ring diameter test.

Most of the models that fail are low inclination models with large ring widths that are too large.  For example, the 
In other cases the fit distribution fails because it is too broad, with too little mass close to the center of the distribution.  In still other cases (positive spin, low inclination SANE models) the bulk of the model fits are too small compared to the data.

\begin{figure}
  \centering
  \includegraphics[width=\columnwidth]{./figures/Mring_d_Constraints.png}
  \caption{M-Ring diameter.}
  \label{fig:cmp_m-ring_diam}
\end{figure}

The m-ring width is typically the best constrained and therefore the most constraining.  Although the closure phases depend on the m-ring width as well, it is easy to see how the visibility amplitudes are affected by the m-ring width: the width controls the long-baseline amplitudes.  All models that fail have m-ring widths that are too small.  This includes all but 6 MAD models at $\abh \le 0$ and all MAD models at $i = 90\degree$.

\begin{figure}
  \centering
  \includegraphics[width=\columnwidth]{./figures/Mring_w_Constraints.png}
  \caption{m-ring widths}
  \label{fig:cmp_m-ring_width}
\end{figure}

\subsubsubsection{EHT Constraint Summary}

Constraints based on EHT data are summarized in Figure \ref{fig:all_EHT_constraints}.  The cuts favor spin $> 0$ models.
We are left with $31/100$ SANE models and $20/100$ MAD models.

\begin{figure}
  \centering
    \includegraphics[width=\columnwidth]{./figures/Interferometric_Constraints.png}
  \caption{Combined EHT constraints (logical {\em and}) including the second moment, null location, and m-ring fit constraints.}
  \label{fig:all_EHT_constraints}
\end{figure}


\subsubsection{Non-EHT Constraints}

Here we compare constraints from 86GHz, NIR, and X-ray observations.  Most or all of the emission in these bands is believed to originate in the compact source from plasma that is close to or overlaps the plasma the 230GHz-emitting plasma observed by EHT.

%..............................................................................
\subsubsubsection{NIR Median Flux}

\note{Michi}

NIR photons are produced by synchrotron process from photons at the high energy end of the distribution function.  For a one-zone model with $B = 30$G, the  critical frequency $\nu_{crit} \simeq \gamma^2 e B/(m_e c) \simeq 80$GHz and emission at $2.2\mu$m is therefore produced by electrons with $\gamma \simeq 10^3$, compared to a mean Lorentz factor of $30$ for plasma with $\Theta_e = 10$.  NIR flux density will therefore be sensitive to $\Theta_e$ and therefore to $\Rh$.

Interestingly, we find that some models are synchrotron-weak and Compton-strong in the NIR.  \note{Discussion of which models fall in this category}

Models that pass the NIR flux limit are shown in Figure \ref{fig:cmp_2um_flux}.

All but 6/100 of the SANE models pass; the exceptions are high spin, high inclination models where Doppler boosting increases the NIR flux from the bright spot on the approaching side of the disk.

Approximately XX\% of the MAD models fail the NIR test, including all but 1 model at $\Rh = 1$ and $10$.

\begin{figure}
  \centering
  \includegraphics[width=\columnwidth]{./figures/2um_flux_Constraints.png}
  \caption{NIR flux limit}
  \label{fig:cmp_2um_flux}
\end{figure}

%..............................................................................
\subsubsubsection{X-ray Luminosity}

\note{Michi}

Most thermal models produce X-ray emission through Compton upscattering of thermal synchroton photons.  Typically the X-ray band lies in the first Compton bump, although that is sensitive to the temperature of the electrons doing the upscattering.  In the first Compton bump $\nu L_\nu$ is proportional to the y-parameter $y \sim 16 \Theta_e^2 \tau_e$ where $\tau_e$ is a characteristic electron-scattering optical depth and $\Theta_e$ is a typical electron temperature.

In many large $\Rh$ SANE models, however, X-ray emission is dominated by bremsstrahlung.  Bremsstrahlung comes from the midplane where the density is largest, at larger radius than the synchrotron and Compton-upscattered X-ray emission.  It dominates in high accretion rate models (this turns out to mean large $\Rh$ models; see \S 5) where $\Theta_e < 1$, and is significant only when the midplane is cool and $r: \Theta_e = 1 < YY$, i.e. at large $\Rh$.  The resurgence of bremsstrahlung in cool disks occurs because at $\Theta_e < 1$, $j_\nu \propto n^2 \Theta_e^{-1/2}$.  When the disk is cool and dense the latter is large.

The set of models that passes the X-ray cut is shown in Figure \ref{fig:cmp_xray_flux}.

Many large $\Rh$ SANE models fail the X-ray test: all but 3/25 at $\Rh = 160$ and all but 6/25 at $\Rh = 40$.  These models all fail due to excess bremsstrahlung.

Many of the MAD models that fail have large absolute spin and low $\Rh$.  These models are Compton-dominated.  The midplane temperature is highest at low $\Rh$.  Since the midplane contributes most of the electron scattering optical depth, low $\Rh$ models have the largest $y$ parameter.

\begin{figure}
  \centering
  \includegraphics[width=\columnwidth]{./figures/Xray_flux_Constraints.png}
  \caption{X-ray flux limits}
  \label{fig:cmp_xray_flux}
\end{figure}

%..............................................................................
\subsubsubsection{86 GHz Median Flux}

In a naive picture \sgra's millimeter flux is produced at a photosphere that decreases in size as frequency increases.  Because of the marginal optical depth at $1.3$mm ($\sim 0.3$ in the one-zone model) and the complicated source structure (the optical depth varies across the image; the $\tau = 2/3$ surface is non-spherical, folded, and not even simply connected) this picture does not hold exactly.  Nevertheless 86GHz photons are typically produced at larger radius than 230GHz photons, and the 86GHz source size is therefore larger than the 230GHz source size; see Ricarte et al. 2021 for a discussion.

The 86GHz/230GHz color therefore probes the optical depth and radial structure of the source plasma.  Figure \ref{fig:cmp_86ghz_flux} shows the result of requiring that the 86GHz flux match the observed flux 3 days before the beginning of the EHT 2017 campaign.

Many $\Rh = 1$ models, both MAD and SANE, fail the $86$GHz flux density test: 23/25 SANE and 9/25 MAD.  These models overproduce $86$GHz emission.
There are also a substantial set of SANE models, 19/100 in all, that underproduce $86$GHz emission.  These models have larger $\Rh$.

\begin{figure}
  \centering
  \includegraphics[width=\columnwidth]{./figures/86GHz_flux_Constraints.png}
  \caption{86GHz median flux limits}
  \label{fig:cmp_86ghz_flux}
\end{figure}

\note{Michi}

%..............................................................................
\subsubsubsection{86 GHz Major Axis}

As for the $86$GHz flux, the $86$GHz size is sensitive to optical depth as a function of radius in the source plasma.

The models that pass and fail are shown in Figure \ref{fig:cmp_86ghz_size}.

Many face-on models fail because they are too small (purple in the figure), while a few other high inclination models fail because they are too large.  Overall only about $50\%$ of models pass, making this one of the tightest constraints.

The physical picture for 86GHz source size is complicated, as is the extraction of the constraint itself from observations.  We note that (1) two different values for the 86GHz intrinsic source size have been reported in the literature; (2) scattering is $7$ times stronger at $86$GHz than at $230$GHz; (3) scattering must be subtracted accurately to obtain the intrinsic source size; (4) the error bars for the 86GHz source size are narrow and this is the principal strength of the constraint.

\begin{figure}
  \centering
  \includegraphics[width=\columnwidth]{./figures/86GHz_size_Constraints.png}
  \caption{86GHz size}
  \label{fig:cmp_86ghz_size}
\end{figure}

%..............................................................................
\subsubsubsection{Summary of Non-EHT constraints}

Applying only non-EHT constraints, we are left with the 9/100 SANE models and 25/100 MAD models shown in Figure \ref{fig:non_eht_cuts}.

The surviving models include a set of SANE models at intermediate $\Rh$ and modest inclination, as well as MAD models at large $\Rh \ge 40$.  No $\Rh = 1$ models survive the non-EHT cuts.

\begin{figure}
  \centering
  \includegraphics[width=\columnwidth]{./figures/Non_Interferometric_Constraints.png}
  \caption{Combined non-EHT constraints}
  \label{fig:non_eht_cuts}
\end{figure}

%------------------------------------------------------------------------------
\subsubsection{Variability}

Variability is central to interpretation of Sgr A*: the small black hole size means that observations considered here are taken over intervals when the source is expected to vary significantly.  This distinguishes Sgr A* from M87*, where the source is expected to vary only over timescales that are long compared to a single track.

Variability is a strong constraint on the models.  Although models differ in their degree of variability, both in an integrated sense and on 4 $G\lambda$ baselines, only a very small fraction of models are as quiet as the data.  For the light curve variability, this remains true whether we use data from 2017 April 7, all days from the 2017 observing campaign, or from historical monitoring of Sgr A*.   In general, SANE models are quieter than MAD models, and (less strongly) face-on models are quieter than edge-on models.

If we were to apply the variability constraints directly to the models, there would be 12/200 successful models left using 1\% cuts (65/200 for the ALMA constraint and 12/200 for the visibility amplitude constraint, with all models that pass the latter also passing the former)\dl{numbers and overlap between constraints subject to change}.  One interpretation of this result is that the surviving models are the correct description of the source (although we would expect some misclassification of models as consistent or inconsistent when using 1\% cuts on such a large model set).  Another interpretation is that there is a missing physical ingredient in the models, see Section \ref{sec:discussions} for a discussion.

%..............................................................................
\subsubsubsection{ALMA Light Curve}

The distributions of 3 hour modulation index (rms \%) across all SANE models, across all MAD models, and across the historical dataset are shown in Figure \ref{fig:cmp_ALMA_var}, along with individual distributions for the models with the lowest and highest MI for SANEs and MADs. Although individually, some models may pass (particularly SANE models), the distributions for the SANEs and the MADs are noticeably offset from the data, with the MADs in particular being more variable. As can be seen, even the quietest MAD model lies above the historical distribution. 

If we compare each individual model to the three segments from the 7 April 2017 ALMA observation using a 2-sample KS test, we find that 56 SANE models and 9 MAD models pass with $p > 1\%$. 

If we instead compare the models to the full historical distribution (40 3h MIs in all, including the ALMA MIs), we find that 53 SANE models and no MAD models pass with $p > 1\%$. This is more stringent than the comparison with just 7 April, since the historical distribution has more samples and thus disparate models can be eliminated with higher confidence. 

%If we compare each individual model to the five segments from the EHT 2017 campaign using a 2-sample KS test, we find that 80 (75) SANE and 21 (19) MAD models pass with $p > 1\%$. 

\begin{figure}
  \centering
  \includegraphics[width=\columnwidth]{./figures/mi_dist.pdf}
  \caption{Distributions of MI for Illinois thermal models (black), compared to distributions from historical observations (gray) and the values from the 2017 ALMA light curve. The latter are shown as vertical lines. The distributions for models with the lowest (blue) and highest (red) average MI for SANEs and MADs are also shown. }
  \label{fig:cmp_ALMA_var}
\end{figure}

%..............................................................................
\subsubsubsection{4 $G\lambda$ Visibility Amplitude Variability}

The distribution of model $4G\lambda$ lightcurve-normalized PSDs are shown in Figure \ref{fig:cmp_VLBI_var}.  The best fit PSD from the 2017 EHT campaign (excluding April 11) is shown as a solid vertical line, with the other vertical lines showing percentiles in the posterior.  Evidently the observations are quieter than both SANEs and MADs as a group.

The PSD estimates for the models are broken up into $5000\tg$ windows for each model. To compare the models to the observation, we take the mean value across all windows and assume the width of the distribution is of $\log_{10} a_{4G\lambda} \pm 0.1$. A model is considered passing if this estimated distribution overlaps with the median observed value. \note{refer GRMHD variability paper appendix} \dl{passing criterion subject to change}

With this approach only 6\% of the models pass, and all SANE. While the quietest models tend to be SANEs, the general distribution across all MADs is not as offset from the SANEs as in the MI distributions. 

\begin{figure}
  \centering
    \includegraphics[width=\columnwidth]{./figures/va_dist.pdf}
  \caption{Distributions of $PSD(4G\lambda)$ for thermal models, compared to the observed distribution from the 2017 EHT campaign.
  \dl{HAMR distribution will be updated later. }}
  \label{fig:cmp_VLBI_var}
\end{figure}

At a 1\% cut, the models that pass the PSD constraint also all pass the MI constraint. It should be noted that this is partially because these constraints are weakly correlated \dl{also partially because we're imposing constraints differently between VLBI and light curve, and the light curve constraint is more lenient} None of the 12 models with acceptable variability pass the other tests.

%------------------------------------------------------------------------------

%------------------------------------------------------------------------------
\subsubsection{Summary of Constraints}

If we set aside variability constraints but use all the remaining constraints we are left with the models shown in Figure \ref{fig:all_cuts}.  Only 1/100 SANE models and 8/100 MAD models survive. The passing models cluster amongst low inclination ($i \le 50$deg) MAD models with $\Rh > 10$.

It is remarkable both that so many of the models look like the EHT data, lending confidence to the models, and that EHT data alone are capable of distinguishing between models in the model set, with only $N = 6$ antennas.  Future experiments with more antennas will contain much more information and provide even tighter constraints.

All $\Rh = 1$ models have been eliminated, most by multiple constraints, and we conclude that models with equal ion and electron temperatures are unacceptable.

All models with $i > 50$deg are eliminated, most by multiple constraints, and we conclude that high inclination models are disfavored.  In the SED edge-on models have a clear signature derived from Doppler boosting, with increased NIR and X-ray flux density.  In EHT constraints many edge-on models have a clear signature derived from low m-ring widths, strong asymmetry (although only for a few models), and failed null location constraint.

For thermal model sets both EHT and non-EHT constraints individually eliminate many models, but together the eliminate all but 5\%.  The success of the models for EHT constraints (apart from variability) supports the use of the models for applying non-EHT constraints and highlights the importance of contemporaneous monitoring of \sgra at other wavelengths.

\begin{figure}
  \centering
  \includegraphics[width=\columnwidth]{./figures/All_Constraints.png}
  \caption{Combined EHT and non-EHT constraints}
  \label{fig:all_cuts}
\end{figure}

Variablity, were it included in the constraint map, would eliminate all thermal models.  Although a few models pass the variability constraints alone this does not mean that we should regard them as favored, since we are likely to find some false positives using a $1\%$ cut in a set of $200$ models.

%------------------------------------------------------------------------------
\subsubsection{Inter-Model Comparison for thermal models}

\cmf{First version}
As can be seen in Table \ref{tab:GRMHDmodels} the thermal models have been calculated for an identical parameter space from two different codes, namely iHARM and BHAC for the GRMHD simulations and iPOLE and BHOSS for the GRRT calculations. This allows us for the first time to perform an in depth comparison between the different numerical methods used in this work in addition to the EHTC code comparison projects \citep{2019ApJS..243...26P,2020ApJ...897..148G}.

\begin{figure}
  \centering
  \includegraphics[width=\columnwidth]{./figures/BHAC_iharm_correlation}
  \caption{Correlation between BHAC and iHARM models based on model constraints}
  \label{fig:modelcorrelation}
\end{figure}

In Figure \ref{fig:modelcorrelation} we show the correlation between the thermal iHARM and BHAC models for various constraints.
\newline The top row shows from left to right the 230\,GHz flux density, the 230\,GHz modulation index, MI, computed for a time window of 3 hours, and the 230\,GHz image size obtained from image moments. Since we normalise the 230\,GHz images to an average flux of 2.4\,Jy within a time window of 5000\,M (corresponds to 28.5 h for SgrA a mass of $4.14\times 10^6\,\msun$), the scatter around this values is small. The deviation from an ideal correlation reflects the precision and number of GRMHD snapshots included during normalisation procedure.
\newline The correlation in the 3\,hour modulation index spreads over $\Delta \rm{MI}=0.75$ which reflects the intra-code ( e.g., MAD vs. SANE accretion and plasma models) and inter-code (BHAC vs. iHARM) differences. Despite these differences the models show a strong correlation throughout the investigated models and parameter space.
\newline We found a strong correlation between models and codes for the image size computed from image moments, i.e. second moments analysis. \\

The middle row presents the correlation plots for the 86\,GHz flux density (left), the 86\,GHz image size using second moments (middle) and the NIR flux (right). The 86\,GHz flux and 86\,GHz image size exhibit a shift toward larger values for the BHAC models. This difference can be explained by the larger field of view used for the BHAC models at 86\,GHz during the radiative transfer calculations. Thus, more extended structure and therefore a larger total flux is included in the BHAC models. This affects mainly models with large inclinations $i\geq70^\circ$ and jet dominated emission models ($\rm{R}_{\rm high}\geq 40$).
\newline The NIR fluxes show a tight correlation over four orders of magnitude and systematically larger flux for the BHAC models for low NIR fluxes ($\log_{10}(NIR)<-7$). These fluxes are far below the NIR constraints of $\sim 1mJy$, and therefore they do not affect the passing or failing of the models. In the thermal models the NIR flux is generated from the tail of the electron distribution function and thus very sensitive to the electron temperature. Thus, small differences in the distribution and value of the electron temperature between the two codes explain the observed de-correlation at very low NIR flux.  \\

The correlation between the models for the m-ring parameters is presented in the third row of Fig.~\ref{fig:modelcorrelation}. The correlation of the diameter of the m-ring is plotted in the left panel. The spread covers nearly the same extent as the 230\,GHz image size (top row, right panel) however the scatter in the correlation is larger.
The same is true for the width of the m-ring (middle panel in the last row of Fig.~\ref{fig:modelcorrelation}). Compared to the diameter and width of the m-ring, the asymmetry of the m-ring is less correlated (right panel). Notice, that the horizontal and vertical line in the asymmetries occurs since the parameter hits the boundary of the allowed range.

%\newline
The smaller correlation of the m-ring parameters as compared to the other parameters presented in Fig.~\ref{fig:modelcorrelation} can be
 understood by the different nature of the constraint i.e. image plane vs. Fourier space with sparse u-v coverage. For example, the m-ring fit fails for high inclination and large $\Rh$ values which leads to a large scatter in parameters for these models. %\\
%\newline
Nevertheless, the correlation between the two different codes and various models is astonishing and provides a robust basis for applied analysis and conclusions drawn from our model/image library.

%------------------------------------------------------------------------------
\subsubsection{Inter-Model Comparison}

\note{Doosoo, Koushik to write here about HAMR thermal models.  Define a subsection, describe the results.}

\note{Angelo and Richard to write here about critical beta models.  Define a subsection, describe the results.}

The $R$-$\beta$ model has been the default parametric model to span the vast uncertainties in emitting particle thermodynamics in EHT's computational pipelines. They are compatible with the well-motivated turbulent plasma heating ADAF models of \cite{1999ApJ...520..248Q} in asymptotic behavior of electron-to-total heating ratio as a function of plasma $\beta$. However, there is a vast function space of alternate interpolations with different behavior over various $\beta$ ranges.

The main theoretical motivation for the Critical Beta Model of \cite{2020MNRAS.493.1404A} is that when the transition between electron heating domination and proton heating domination is smoothed (controlled by increasing exponent parameter $\beta_c$), the 230 GHz emitting region profile tends to be less coronally dominated and more compact and asymmetric. This is a consequence, when fixing synthetic image flux, of higher critical beta values  shifting the locus of electrons dominating the emission profile towards a relativistically colder, higher density equatorial inflow.

Another motivation is that there are preliminary indications that the exponential electron cooling in the Critical Beta Model suppressed the SANE bremsstrahlung spectral component allowing more to pass the X-ray constraint.

%==============================================================================
\subsection{Nonthermal Aligned Models}

\note{Koushik to write here about powerlaw nonthermal HAMR models.  Define a subsection, describe the results and how they differ from the thermal results.][Maybe add Tomohisa's models here as well.] [by including power-law, we see this and that change (only include significant changes from the thermal models)}

\subsubsubsection{Constant power-law models $p=4.0$}




\note{Razi to write here about variable kappa models.  Define a subsection, describe the results and how they differ from the thermal results.}

\note{Christian to write here about constant kappa models.  Define a subsection, describe the results and how they differ from the thermal results.}\cmf{done}

%------------------------------------------------------------------------------
\subsubsection{Constant kappa models $\kappa=3.5$ with variable efficiency, $\varepsilon(\sigma,\beta)$}

In order to investigate the fraction on thermal to non-thermal particles we combine a thermal electron distribution function with a kappa electron distribution with $\kappa=3.5$. The value of $\kappa=3.5$ is motivated from the spectral slope in the NIR during a quiescent state \cmf{add reference here}. In addition to the fixed kappa value we assume that the fraction between thermal and non-thermal particles depends on the local plasma properties, e.g, the magnetisation, $\sigma$, and the plasma beta parameter, $\beta_{\rm p}$. Given this assumption we can write the total emissivity as:
\begin{equation}
j_{\nu,\rm{tot}}=\left[1-\epsilon(\varepsilon,\beta, \sigma)\right] j_{\nu,\rm{thermal}} + \epsilon(\varepsilon,\beta, \sigma) j_{\nu, \kappa},
\label{eq:kappaeff}
\end{equation}
where the efficiency, $\epsilon(\varepsilon,\beta,\sigma)$, is given by:
\begin{equation}
    \epsilon(\varepsilon,\beta,\sigma)=\varepsilon\bigg[1 - \exp\left(-\beta_{\rm p}^{-2}\right)\bigg]
    \left[1-\exp\left(\frac{-\sigma^2}{\sigma_{\rm min}^2}\right)\right].
    \label{eq:efficiencybetasigma}
\end{equation}
Throughout this work we fix $\sigma_{\rm min}=0.01$ and vary the base efficiency, $\varepsilon$, between 0.05, 0.1 and 0.2. For each base efficiency we generate a set of models spanning the same parameter space as the thermal models (see Table \ref{tab:radiativemodels} for details). For each model we iterate the mass accretion rate to obtain an average flux of 2.4\,Jy at 230\,GHz across a time interval of 5000\,M. In order to explore a several values of $\varepsilon$ we increased the cadence of the radiative transfer to 50\,M. This allows us to keep the numerical costs for this parameter sweep low while still being within the correlation time of the GRMHD simulations ($t_{\rm corr}\approx 50-100\,M$) \cmf{ do we have reference for this? So far this result is not published, maybe Boris paper?}. An example for the distribution of the efficiency can be seen in the right half of  Fig. \ref{fig:varepsilon}. The efficiency quickly approaches $\epsilon=0$ within the disk region while within the jet the efficiency reaches the defined base-efficiency. Thus, togehter with the used cut of in sigma, $\sigma_{\rm cut}=1$ the non-thermal particles are mainly located in jet sheath.

\begin{figure}
  \centering
    \includegraphics[width=\columnwidth]{./figures/GRMHDphiavera0.94sigmaeta.pdf}
  \caption{Time and azimuthal averaged distribution of the magnetization, $\sigma$ (left half) and the efficiency, $\epsilon(\varepsilon,\beta,\sigma)$ using $\varepsilon=0.2 $ for a BHAC MAD GRMHD simulation with spin $a_{\star}=0.94$. The solid gray line corresponds to $\sigma=1$ and the dashed line indicates out-flowing plasma via the Bernoulli parameter ($-h u_{t}>1.02$).}
  \label{fig:varepsilon}
\end{figure}

In the following we will elaborate the impact of adding non-thermal particles via the kappa electron distribution with fixed kappa value ($\kappa=3.5$) and there different base efficiencies $\varepsilon=$0.05, 0.1 and 0.2 on the observational constraints listed in Sect. \ref{}.

%..............................................................................
\subsubsubsection{230\,GHz VLBI pre-image size}

The addition of non-thermal particles alters the 230\,GHz VLBI pre-image sizes hardly for the MAD models and shows a minor effect on the SANE models. In the SANE case only models with $\Rh>40$ exhibit a minor increase in the source size where we see a monotonic increase of the source size with inclination. This effect can be understood if we consider that the bulk of the emission in all cases consider here is still produce by the thermal electron distribution, their temperature is given by Eq. \ref{} and an increase in $\Rh$ suppresses the emission from particles in the disk (by decreasing the electron temperature) and thus enhances the emission from jet.
Given that most the non-thermal particles are located in the sheath of the jet their impact on the source size is largest if bulk of the thermal emission is also produced there. In addition the emissivity of thermal synchrotron radiation decreases as $j_{\nu}\propto \exp{\left(-\nu^{1/3}
\right)}$ while for the kappa distribution as $j_{\nu}\propto \nu^{-(\kappa -2)/2}$. This implies that for the same electron temperature the non-thermal flux is compared to a thermal one higher and thus leads to a more extended jet structure for the models including non-thermal particles.
\newline In the MAD case, independent of the choice of $\Rh$ the emission is mostly produced in the disk region (see EHT paper V and Fig 8 in Wong et al. 2021 for 3D rendering). Increasing $\Rh$ will not push the emission region into the jet where the non-thermal particles are located and thus their contribution to the total emission structure is negligible.

%..............................................................................
\subsubsubsection{86\,GHz flux}

Since the GMVA+ALMA observations at 86\,GHz \cmf{ref to Issaoun paper} probe a larger field of view as the 230\,GHz EHT observations, we increased the field of view for the 86\,GHz to 800\,$\rm{\mu as}$ during the radiative transfer calculations. As mentioned earlier the non-thermal particles are mainly located in the jet sheath and thus the increased field of view ensures that no extend flux is missing during the comparison with the 86\,GHz observations.
\newline The 86\,GHz for both SANE and MAD models flux is not affected by the addition of non-thermal particles. In case of the SANE models the edges of the 86\,GHz flux distribution are slightly shifted in the case of $\Rh>40$. However, including non-thermal particles even with the highest base efficiency $\varepsilon=0.2$ does not change the scoring of a model, i.e., a thermal-only model which full-fills the 86\,GHz constrain is still accepted if non-thermal particles are included. This behaviour can be explained by the fact that the bulk of the emission in both accretion models is generated in with a few gravitational radii. Since the non-thermal particles are mainly located in the jet sheath and thei ratio between non-thermal to thermal particles is at most 20\% the contribution of the non-thermal particles to the 86\,GHz flux can be neglected.

%..............................................................................
\subsubsubsection{86\,GHz image size}

The behaviour of the 86\,GHz image size is very similar to the above described 230\,GHz image size: There is no change in image size for the MAD models and only a minor increase in the SANE models for $\Rh>40$. The physical reasons for this behaviour follows the same arguments as in the 230\,GHz VLBI pre-image size.

%..............................................................................
\subsubsubsection{NIR constraints}
As expected, the addition of non-thermal particles via the kappa electron distribution function with variable efficiency has a large influence on the NIR flux for all models independent of the accretion model and the $\Rh$ value. In Fig.~\ref{fig:NIR_kappaepsilon} we compare the distribution of the NIR fluxes for thermal and kappa eDF for a SANE accretion model.

\begin{figure}
  \centering
  \includegraphics[width=\columnwidth]{./figures/SANE_NIR_standard.pdf}
  \caption{NIR constraints for SANE accretion models for thermal and non-thermal electron distribution function with $\kappa=3.4$ fixed and $\epsilon\left(\sigma,\beta,\varepsilon\right)$. The red violin plots correspond to a thermal eDF, blue, orange and green indicate kappa eDF with $\varepsilon=$0.05,0.1 and 0.2.}
  \label{fig:NIR_kappaepsilon}
\end{figure}

As can be seen in the figure, including non-thermal particles changes the distribution of the NIR fluxes significantly. Except for the $\Rh=1$ models the addition of non-thermal particles even with the lowest base efficiency used in this analysis $\left( \varepsilon=0.05\right)$ leads to a over-production of NIR fluxes. In the case of the MAD models all models over-produce the NIR flux for $\varepsilon=0.05$.

The NIR fluxes are produced by particles in the tail of the distribution function and are therefore are sensitive to slope of the tail. The emissivity for a thermal distribution decreases exponentially $\left(j_{\nu}\propto\exp(-\nu^{1/3})\right)$ while for the kappa distribution it behaves like $j_{\nu}\propto \nu^{-(\kappa-2)/2}$. Thus even at low base efficiencies, $\varepsilon$, the the flux from the kappa distribution in the NIR $\left(\nu\sim 10^{14}\,\rm{Hz}\right)$ is larger than for the thermal eDF.

%..............................................................................
\subsubsubsection{ALMA Light curves}

For the comparison with the ALMA light curves we compute the modulation index for a 3\,hour time window and across the entire simulated time window of 28\,hours (5000\,GM/c$^3$). Similar to the 86\,GHz flux the 230\,GHz flux is mainly produced within a few gravitational radii and thus not affected by the addition of non-thermal particles using Eq. \ref{eq:kappaeff}. As in the previous constraints, the MAD accretion models are insensitive to the addition of non-thermal particles whereas the SANE models show some increased modulation index for $\Rh>40$. However, the distributions for thermal and non-thermal eDF are still largely overlapping.

%..............................................................................
\subsubsubsection{m-ring}

The m-ring fitting is applied to the 230\,GHz images. As mentioned earlier the flux and size of the 230\,GHz images are not affected by the inclusion of non-thermal particles with fixed kappa and variable efficiency. Thus, we do not expect any changes in the distribution of the m-ring parameters.  Applying m-ring fitting to non-thermal models and a detailed comparison with the thermal results confirmed our initial assumption. \cmf{based on the results of Kotaro, who run the m-ring on non-thermal models}

%------------------------------------------------------------------------------
\subsubsection{Variable kappa model $\kappa(\sigma,\beta)$}

The non-thermal electron distribution function given in Equation \eqref{eq:nonthermaleDF} can be written in a general and self-consistent way by using a power-law slope dependent of the magnetisation $\sigma $ and plasma-$\beta$ from GRMHD simulations, $\kappa(\sigma, \beta)$, assuming that non-thermal particle acceleration due to magnetic reconnection following particle-in-cell small scale simulation from \citep{2018ApJ...862...80B}

\begin{align}
\kappa=2.8 +0.7\sigma^{-1/2} + 3.7\sigma^{-0.19}\tanh{(23.4\sigma^{0.26} \beta)}\label{eq:kappa} \\
w= \frac{ \kappa -3 }{\kappa} \Theta_{\rm e} + \frac{\varepsilon}{2}\left[1+\tanh(r-r_{\rm inj})\right]\, \frac{ \kappa -3 }{6 \kappa} \frac{m_{\rm p}}{m_{\rm e}} \sigma \label{eq:w}, 
\end{align}

where the width function contain thermal temperature and non-thermal contributions, second part defined by the magnetization and non-thermal particle acceleration efficiency \citep{2019A&A...632A...2D,2021NatAs.tmp..218C}. Although efficiency can be computed from first principles in PIC simulations
for convenience we only considered $\varepsilon=0$ and injection radius $r_{\rm inj}=10r_{g}$ following previous studies. Large heating efficiency $\varepsilon>0$ has large impact on NIR emission, the averaged flux increase two orders of magnitude when efficiency goes from 0 to 1, this implies that all passing models will overshoot NIR constraint, on the other hand, at 86GHz the flux increase around two times as well as the image size \citep{2021arXiv211102518F}. The emission, $j_{\nu}$, and absorption coefficients, $\alpha_{\nu}$, computed numerically can be found for the  interval $3< \kappa\leq 8$ in  \cite{2016ApJ...822...34P}. In order to compute the emission from regions where $\kappa$ is out of the validity range we use the thermal model, and neglect the emission in the jet spine, $\sigma > 1$ (see Figure \ref{fig:varepsilon}). 
\aco{Starting first draft}

%..............................................................................
\subsubsubsection{86\,GHz flux}

...
%..............................................................................
\subsubsubsection{86\,GHz image size}

...

\note{Razi, Angelo, Richard?}
\ckc{Write subsections only if they are different from the thermal models.}

%------------------------------------------------------------------------------
\subsubsection{Summary}

The effect of adding non-thermal particles via a kappa electron distribution function with fixed $\kappa=3.4$ values and variable efficiency via Eq.~\ref{eq:efficiencybetasigma} can be summarised in the following way:
\begin{itemize}
    \item The 86\,GHz and 230\,GHz constraints are hardly effected by the addition of non-thermal particles.
    \item Even the lowest a base efficiency considered, $\varepsilon=0.05$, leads to an over-production of NIR flux. \cmf{indication that we need very localised regions of non-thermal particle productions and no "global" approach?}
\end{itemize}

\note{Alejandro to write here about Frankfurt nonthermal models.  Define a subsection, describe the results and how they differ from the thermal results.}

\note{Tomohisa to write here about UWABAMI nonthermal power-law models}

\ckc{Write subsections only if they are different from the thermal models.}

%==============================================================================
\subsection{Tilted Models}

\note{Koushik to write here.  Describe the results and how they differ from aligned thermal results.}
\ckc{Write subsections only if they are different from the thermal models.}

%==============================================================================
\subsection{Stellar Wind Fed Models}

\note{Angelo and George to write here.  Describe the results and how they differ from aligned thermal results.}
\ckc{Write subsections only if they are different from the thermal models.}

%==============================================================================
\subsection{Best Bet Models}

\begin{figure*}
    \centering
    \note{altex: a grid figure showing the $\sim$ 4 best bet models.  Each model in a column.  Top two rows are the 230GHz images and 86GHz images, last row is SEDs.}
    \caption{Best bet models.  Each column corresponds to one best bet model; top row shows a 230GHz image \note{average or selected image; or use GIF?}; middle row shows an 86GHz images; bottom row shows SEDs.}
    \label{fig:my_label}
\end{figure*}

\note{Michi to provide first draft}

We now consider combined constraints, excluding variability, under the hypothesis that (1) there is a missing physical ingredient in the models that would lower variability, but (2) that missing ingredient would not vitiate the models entirely.  Models are eliminated if they fail any one constraint.

Figure showing combined constraints for thermal models.

Figure showing combined constraints for nonthermal models.

We then inspected the remaining models and selected four best-bet models that approximately span the space of successful models.  \note{Written characterization of remaining models; possibly 3 or 5}


\section{Discussions}\label{sec:discussions}

%==============================================================================

\subsection{MAD, SANE, and Self-Consistent Wind Feeding}

%==============================================================================

\subsection{Electron Distribution Function}

[Koushik to provide first draft]

Strong constraint on abundance of cold electrons from bremss.   

%==============================================================================
\subsection{Inclination}

%==============================================================================
\subsection{Position Angle}

%==============================================================================
\subsection{Black Hole Spin}

%==============================================================================
\subsection{Jet Power}

%==============================================================================
\subsection{Efficiency}

\subsection{Caveats and Limitations}

%==============================================================================


\section{Conclusions}\label{sec:conclusions}

\note{to be written last}
\ckc{ck's first pass}

We have carried out and presented the most extensive comparison
between numerical models and observations of \sgra to date.
By using multiple simulation pipelines (cite PATOKA, BHAC+BOSS, ...),
we have demostrated that, given a set of boundary conditions,
numerical simulations are remarkablely predictive in averaged image
properties.
This ability allows us to significantly reduce the uncertainty in the
plasma properties and infer the spacetime properties of \sgra.

EHT's VLBI data is most constraining on the ring size and width but
shows a supriringly low level of asymmetry in the image.
These constraints roughly reduce the dimension of the parameter space
by two, carving out a region that prefers edge-on retrogade or low
spin SANE models, or face-on prograde models.
Together with multi-wavelength observations, we conclude that \sgra is
most likely a low spin prograde $\abh \sim 0.5$ MAD with mid
inclination $20\degree \lesssim i \lesssim 60\degree$.
For thermal models, \sgra is likely to have low electron temperate
with $\Rh \gtrsim 40$.
\ckc{I think it will be useful to translate $\Rh$ to actual electron
  temperature.
  I think someone good at handling GRMHD needs to do it, maybe George?
  Hector?}

% Future: polarization
While there are also isolated pockets in the parameter space that pass
all constraints, however, even with this extensive study, it is
difficult to draw strong conclusion about these regions.
We expect the EHT's future results, especially on polarimetry, will
address these ambiguity and further narrow down \sgra's possible parameter
space.

Although we are successful in predicting \sgra's average image
properties, the variability is a lot more challenging.
...
\ckc{Missing plasma physics... etc}\br{maybe also mention order(s) of magnitude higher resolutions might be needed to capture ``variability''}

Predictions: what would ngEHT see?


\begin{acknowledgments}

This paper is dedicated to the memory of John F. Hawley, whose pioneering work on black hole accretion flows made this paper possible.  We are grateful to an anonymous referee whose comments significantly improved this paper.

\end{acknowledgments}

%------------------------------------------------------------------------------
\facility{EHT, the Global mm VLBI Array, Atacama Large Millimeter Array, Chandra X-ray Observatory;
Frontera supercomputer (Texas Advanced Computing Center), Puma Supercomputer (Arizona), Open Science Grid, CyVerse}

\software{\bhac, \bhoss, \difmap \citep{1997ASPC..125...77S}, \dmc, \ehtim \citep{2019zndo...2614016C}, \foci,
  \grmonty, \hallmark, \hamr, \ipole, \kharma, \koral, \mockservation,
  \smili, \themis, \vida, VisIt, numpy \citep{vanderwalt2011}, scipy \citep{oliphant2007}, matplotlib \citep{hunter2007}}

%------------------------------------------------------------------------------
\appendix

\section{Numerical Methods}\label{app:numerical}

%==============================================================================
\subsection{GRMHD Consistency and Convergence}\label{app:resolution_study}

\note{Hector to provide first draft.}

\note{Brief general discussion of Porth et al. and Olivares et al.}

\note{Comparison of GRMHD output from Illinois/Frankfurt/HAMR}

\note{Comparison of GRMHD output from Illinois at multiple resolutions}

%==============================================================================
\subsection{Radiative Transfer Consistency and Convergence}
\label{app:radtrans}

\note{Ben to provide first draft.}

Brief discussion of Gold et al., Prather et al.

Field of View

Resolution

%==============================================================================
\subsection{Spectral Energy Distribution Consistency and Convergence}

\begin{figure*}
    \centering
    %\includegraphics{}
    \note{altex: plot GRRT flux vs SED flux at different wavelenghts.  Demostrates that we have good agreements for 230GHz.  Demostrate that NIR may require monty carlo calculations due to inverse Compton.  Demostrate some 86GHz images require larger FoV but they are ruled out anyway and would not affect the results.}
    \caption{Comparing GRRT flux from monte carlo calculations.  The three columns are 86GHz, 230GHz, and NIR, respectively.  GRRT is only used to spot check x-ray and does not have a corresponding scatter plot.}
    \label{fig:sed_vv}
\end{figure*}

%==============================================================================
\subsection{Cross-Validations}

\begin{figure*}
    \centering
    %\includegraphics{}
    \note{altex: scatter plot between, e.g., Illinois and Frankfurt models for the different measurable.}
    \caption{Comparing model predictions from different modeling pipelines.  ...}
    \label{fig:xv}
\end{figure*}

\section{Variability Checks}\label{app:variability}

\monika{things that are missing or we leave them out: initial conditions}

Majority of the simulations considered in this work fails to recover the \sgra observed variability in total flux.
In this Appendix we discuss and dismiss several possible causes for mismatch of $\mi{3}$ metric in models and data.

%%%%%%%%%%%%%%%%%%%%%%%%%%%%%%%%%%%%%%%%%%%%%%%%%%
%% notes from people who drafted the section

% \bp{Seems like (1) and (2) here are (or will be) discussed more thoroughly in the Numerical Methods appendix.  Given both seem to conclude that all models are consistent, and thus that changing these parameters would not reduce variability, maybe we can refer there instead of re-writing?}
%\monika{11Dec:addressed below}
%==============================================================================
% 1. GRMHD resolution.

% discussion of varying resolution [Ben to spelunk to find a 448 down set of images]

% Discussion of high resolution HAMR model.

% mapping between 230 GHz variability and accretion rate variability

%\monika{11Dec:addressed in B1}

%==============================================================================
%2. Image resolution. 

%\moinka{11Dec:addressed in B1}

%==============================================================================
%3. Model duration is not sufficiently long.

%KORAL model analysis.

%==============================================================================
%4. $\sigma_{cut}$ is overproducing variability.
% MM 11 Dec: when sigma_cut>sigma_crit  physics is unreliable and this is mentioned in the model section now. maybe not appropirate to discss here
%==============================================================================
%5. Initial conditions are not a good model.
%Comparison of Monika model initial conditions

%==============================================================================
% 6. Cooling filters the light curve. [Ben]
% \monika{11Dec:moved at the end: merged into subsection C.2}
%==============================================================================
%7. Electron distribution function model is inadequate.

%a.  Discussion of models with varying electron heating models.  Discussion of Jason's eheating models.   Forward reference to Diaz et al.  [Vedant]

%b. Discussion of models with $\Rl = 10$. [Vedant]
% \monika{11Dec:in C2}

%%%%%%%%%%%%%%%%%%%%%%%%%%%%%%%%%%%%%%%%%%%%%%%%%%

\subsection{Simulations Duration}\label{app:narayan}

\begin{figure}
  \centering
  \includegraphics[width=\columnwidth]{./figures/Koral_vs_IL_MI.png}
  \caption{Distribution of \mi{3} from the full \koral model set, from the first half, from the second half, and from the comparable \kharma thermal models.}
  \label{fig:koral_MI}
\end{figure}  

Our standard models of accretion are typically evolved for $\sim 30,000 \tg$. In Figure~\ref{fig:koral_MI} we show...

insert short description of David plot with m3 for \koral vs \kharma

%==============================================================================
\subsection{Effects of $\Rl$ and radiative cooling}
\monika{i am still editing this}

For systems with sub-Eddington accretion rates, $\Dot{M}\ll\Dot{M}_{Edd}$, such as SgrA$^{*}$, radiative processes can be neglected during fluid evolution (\citealt{2012MNRAS.426.1928D, 10.1093/mnras/stw3116, Ryan_2017}) and the plasma can be considered Coulomb collisionless (\citealt{Mahadevan_1997, 10.1093/mnras/stw3116, Ryan_2017}). However, the uncertainties with electron heating and advection, and a limited understanding of the funnel warrant a discussion of cooler electrons in regions of high magnetization, and in particular, its effect on the $230GHz$ lightcurve variability. 

The $\Rh$ prescription (Equation \ref{eq:rhigh_prescription}) has three free parameters: $\Rh$, $\Rl$ and $\beta_{crit}$. During post-processing, the $\Rh$ parameter is generally varied while $\Rl$ and $\beta_{crit}$ is set to unity. In this section we investigate the effect of varying $\Rl$ on the 3 hour modulation index, $M_{3}$.

Paticle-in-cell (PIC) simulations modelling turbulent dissipation or dissipation associated with magnetic reconnection suggest preferential heating of the electrons in regions of low plasma $\beta$ (\citealt{2010MNRAS.409L.104H, Rowan_2017, 10.1093/mnras/stx2530, Rowan_2019, Kawazura771, PhysRevX.10.041050, kawazura2021energy}). The $\Rl$ parameter dictates the electron temperature in these regions, that is, in the funnel. Figure \ref{fig:rlow_comparison} shows the effect of $\Rl$ on image morphology.

\begin{figure*}
\centering
\includegraphics[width=0.95\textwidth]{figures/rlow_comparison_rhigh160.png}
\caption{Comparison of images for the same fluid snapshot with varying $\Rl$. The density scale $\mathcal{M}$, and FOV were increased to accentuate the differences between the images. The total emission in the funnel decreases with increasing $\Rl$.}
\label{fig:rlow_comparison}
\end{figure*}



We vary $\Rl$ for a select set of the best bet Illinois/Thermal models and plot the distribution of the 3 hour modulation index $M_{3}$ in Figure \ref{fig:mi_rlow}.

\begin{figure*}
\centering
\includegraphics[width=0.95\textwidth]{figures/mi_rlow_select_models.png}
\caption{Modulation index computed over 3 hour intervals $M_{3}$, for a subset of the Illinois/Thermal models. For this analysis, we considered the (25k-30k)$GM/c^{3}$ time interval.}
\label{fig:mi_rlow}
\end{figure*}

Although the minimum value of the distribution decreases when $\Rl$ is increased, there is no clear trend for the mean of the distribution. In addition, the average $M_{3}$ still does not match observational values.


%\subsection{Radiative cooling}\label{app:cooling}

We also check if excess flux variability comes from an overestimate of electron temperatures in areas where radiative cooling of the electrons may become important. We can estimate the possible effects of this change by imposing a limit on electron temperatures based on the local dynamical time, reflecting the assumption that higher-temperature electrons would cool before crossing the event horizon.

This limit comes from comparing two timescales: the synchrotron cooling rate at high temperature,
\begin{align}
    \tau_\mathrm{cool} \approx \frac{3 m_e^3 c^5}{16 B^2 e^4 \Theta_e}
\end{align}
%dependent on the electron temperature $\Theta_e$, expressed in units of the rest mass energy, and the local field strength $B$ in Gauss. 
and the the dynamical time
\begin{align}
    \tau_\mathrm{dyn} \approx \left( r \sin{\theta} \right) ^{3/2} \frac{G M_\mathrm{BH}}{c^3}.
\end{align}

A set of images was produced which put a ceiling on the electron temperature $\Theta_e$, by requiring that the $\tau_\mathrm{cool} > N \cdot \tau_\mathrm{dyn}$ -- that is, assuming that electrons likely to cool on timescales shorter than $N$ dynamical times will do so. Lightcurves made using different values of this ceiling are shown in Figure~\ref{fig:ceiling_lc1} and the corresponding $\mi{3}$ index is shown in Figure~\ref{fig:ceiling_lc2}.

\begin{figure*}
    \centering
    \includegraphics[width=0.95\textwidth]{figures/ctcut_lightcurves.pdf}
    \caption{Lightcurves when applying a ceiling on electron heating based on an estimate of the local electron cooling time, compared against the local dynamical time. Different values represent ceilings at different numbers of dynamical times.}
    \label{fig:ceiling_lc1}
\end{figure*}

\begin{figure*}
    \centering
    \includegraphics[width=0.95\textwidth]{figures/ctcut_effects.pdf}
    \caption{Effects of the electron temperature ceiling on lightcurve and variability in sample models.}
    \label{fig:ceiling_lc2}
\end{figure*}


%==============================================================================
%8. The flow is actually made of helium. [George]

%brief discussion, forward reference to Wong+.

% now C4



%==============================================================================
%9. $\gamma = 5/3$ rather than $4/3$.   [Vedant]  When $\gamma = 4/3$ the compression ratio across the shock is larger.

\subsection{Effects of self-consistent electron heating}

The thermal models considered in Section~\ref{sec:models} assign a local electron temperature as a post-processing step based on local magnetic field strength, parameterized by plasma $\beta$ (Equation \ref{eq:rhigh_prescription}).

\citealt{10.1093/mnras/stv2084} provided a formulation to model electron thermodynamics during the fluid evolution. Numerical dissipation at grid scale sources entropy generation and is used to heat the electrons based on a microphysical, sub-grid heating prescription. Local fluid and electromagnetic variables are used to compute the electron entropy which along with the ideal gas equation of state, can be converted into a temperature ($\Theta_{e}$). This approach allows computing the electron temperature at each timestep of the simulation, and not during post-processing, as it is done in the $\Rh$ and Critical-$\beta$ prescriptions.

We consider three sub-grid heating models that prescribe the partition of dissipated energy into electrons and ions. \citealt{2010MNRAS.409L.104H} computed the ratio of ion-to-electron heating due to dissipation of Alfv\'enic turbulent cascade, while \citealt{10.1093/mnras/stx2530} and \citealt{Rowan_2017} considered magnetic reconnection as the source of energy dissipation at sub-grid scales. These studies provide an approximate fitting formula for the ion-to-electron heating rate ($Q_{i}/Q_{e}$) based on local ion-to-electron temperature ratio ($T_{i}/T_{e}$) and local magnetic field strength -- parameterized by $\sigma$ or plasma $\beta$.

The GRMHD simulations considered here are a subset of the simulations analyzed in \citealt{2020MNRAS.494.4168D}. These include MAD and SANE accretion flows at spins, $a_{*}=0,+1/2,+15/16$. We compute the 3 hour modulation index $M_{3}$, over the time interval (5k-10k)$GM/c^{3}$. The average $M_{3}$ values are comparable to similar $\Rh$ models, with SANE reconnection models exhibiting a reduced variability as compared to the corresponding turbulent heating models. However, the average $M_{3}$ for all the models is greater than the $M_{3}$ measured from the ALMA lightcurve on three days. 

\subsection{Effects of fluid adiabatic index, $\Gamma_\mathrm{ad}$}

We expect the ions and electrons in hot accretion flows to be thermally decoupled and the resulting plasma to be two-temperature (\citealt{1976ApJ...204..187S, Quataert_1998, 10.1093/mnras/stw3116, Ryan_2018}). The electrons in such flows are relativistic and can be modelled as a fluid with an adiabatic index $\Gamma_{e}=4/3$, while the ions are nonrelativistic and possess an adiabatic index, $\Gamma_{i}=5/3$.

The adiabatic index of the fluid assumes a value between $\Gamma_{e}$ and $\Gamma_{i}$ dictated by the thermodynamics of the ions and electrons (cf. Figure 4 in \citealt{10.1093/mnras/stw3116}). Since we do not model electron thermodynamics and ignore radiative effects during our fluid simulations, we consider a constant value $\Gamma_{ad}$, and set it to 4/3 for the \textit{standard} set of simulations, ie. $\Gamma_{ad}=\Gamma_{e}$. While this may be the case in the funnel, where the electrons are the hottest and highly relativistic; the fluid adiabatic index value away from the poles can be higher than the relativistic value.

We look at the interplay between fluid adiabatic index and lightcurve variability by evaluating $M_{3}$ for thermal, GRMHD models with a higher fluid adiabatic index. This includes MAD models with $\Gamma_{ad}=13/9$ (see Section~\ref{app:narayan} and  \citealt{2021arXiv210812380N}) and SANE models with $\Gamma_{ad}=5/3$. The models exhibit lightcurve variability similar to the \textit{standard} library and have an average $M_{3}$ that is greater than the value obtained from the ALMA lightcurve. \kc{maybe discuss hamr thermal models here too}

\subsection{Effects of plasma composition}\label{app:helium}

\monika{a few sentences and short reference to forthcoming paper are missing... }


\section{Pass/Fail Plots and Tables}\label{app:tables}

%\pagebreak
%\movetabledown=3cm
%\movetableright=-7cm

In this appendix we present the full set of constraint results in graphical and tabular form.  The tables are available in tabular form.

The pass-fail tables are presented below.  To assist in interpreting them a set of plots showing the pass-fail status of individual models in the standard Illinois model set are provided first.   

\begin{figure*}
  \centering
  \includegraphics[width=0.8\textwidth]{./figures/230GHz_size_Constraints.png}
  \caption{2nd Moment Constraint}
  \label{fig:230GHz_size_pizza}
\end{figure*}
\begin{figure*}
  \centering
  \includegraphics[width=0.8\textwidth]{./figures/Null_loc_Constraints.png}
  \caption{Null Location Constraint}
  \label{fig:null_pizza}
\end{figure*}
\begin{figure*}
  \centering
  \includegraphics[width=0.8\textwidth]{./figures/Mring_d_Constraints.png}
  \caption{M-ring Diameter Constraints}
  \label{fig:mring_diam_pizza}
\end{figure*}
\begin{figure*}
  \centering
  \includegraphics[width=0.8\textwidth]{./figures/Mring_w_Constraints.png}
  \caption{M-ring Width Constraints}
  \label{fig:mring_width_pizza}
\end{figure*}
\begin{figure*}
  \centering
  \includegraphics[width=0.8\textwidth]{./figures/Mring_f1_Constraints.png}
  \caption{M-ring Asymmetry Constraints}
  \label{fig:mring_asymm_pizza}
\end{figure*}
\begin{figure*}
  \centering
  \includegraphics[width=0.8\textwidth]{./figures/Interferometric_Constraints.png}
  \caption{Combined EHT Constraints}
  \label{fig:eht_comb_pizza}
\end{figure*}

Then the non-EHT constraints.

\begin{figure*}
  \centering
  \includegraphics[width=0.8\textwidth]{./figures/86GHz_flux_Constraints.png}
  \caption{86GHz Flux Constraints}
  \label{fig:86GHz_flux_pizza}
\end{figure*}

\begin{figure*}
  \centering
  \includegraphics[width=0.8\textwidth]{./figures/86GHz_size_Constraints.png}
  \caption{86GHz Size Constraints}
  \label{fig:86GHz_size_pizza}
\end{figure*}
\begin{figure*}
  \centering
  \includegraphics[width=0.8\textwidth]{./figures/2um_flux_Constraints.png}
  \caption{$2.2\mu$m Flux Constraints}
  \label{fig:2um_flux_pizza}
\end{figure*}
\begin{figure*}
  \centering
  \includegraphics[width=0.8\textwidth]{./figures/Xray_flux_Constraints.png}
  \caption{X-Ray Luminosity Constraints}
  \label{fig:xray_pizza}
\end{figure*}
\begin{figure*}
  \centering
  \includegraphics[width=0.8\textwidth]{./figures/Non_Interferometric_Constraints.png}
  \caption{Combined non-EHT Constraints}
  \label{fig:noneht_pizza}
\end{figure*}

Then the full set of combined constraints, without variability.

\begin{figure*}
  \centering
  \includegraphics[width=0.8\textwidth]{./figures/All_Constraints.png}
  \caption{Combined Constraints}
  \label{fig:all_pizza}
\end{figure*}

\clearpage

We then present the table for all the model sets.

\begin{longrotatetable}
\startlongtable
\begin{deluxetable*}{cccc|cccc|c|ccccc|c|c|cc|cccc}
\tabletypesize{\scriptsize}
\tablecaption{Pass/Fail Table, Illinois Thermal Models}
\label{tab:illinoisPF}
\tablehead{ \colhead{M/S}  &  %
\colhead{Spin}  &  %
\colhead{$i$}  &  %
\colhead{$\Rh$}  &  %
\colhead{$F_{86}$}  &  %
\colhead{$\lambda_{maj,86}$}  &  %
\colhead{$F_{2\mu{\rm m}}$}  &  %
\colhead{$L_X$}  &  %
\colhead{non-EHT}  &  %
%\colhead{\begin{tabular}{{non-}c{EHT}}\end{tabular}} & %
%\small\bfseries\begin{tabular}{@{}c@{}}#1\end{tabular}
\colhead{$\lambda_{230}$}  &  %
\colhead{Nulls}  &  %
\colhead{Ring D}  &  %
\colhead{Ring W}  &  %
\colhead{Ring A}  &  %
\colhead{EHT}  &  %
\colhead{All}  &  %
\colhead{MI} & %
\colhead{4G$\lambda$} & %
\colhead{$\dot{M}/\dot{M}_{Edd}$}  &  %
\colhead{$L_{bol}/(\dot{M} c^{2})$}  &  %
\colhead{$P_{out}$(cgs)}  &  %
\colhead{$P_{out}/(\dot{M} c^2)$}}
\startdata
S & -0.94 & 10.0 & 1.0 & Fail & Fail & Pass & Pass & Fail & Fail & Pass & Pass & Fail & Pass & Fail & Fail & Pass & Pass &1.8$\times10^{-7}$ & 1.3$\times10^{-4}$ & 2.2$\times10^{36}$ & 2.5$\times10^{-3}$\\
S & -0.94 & 10.0 & 10.0 & Pass & Pass & Pass & Fail & Fail & Pass & Pass & Fail & Fail & Pass & Fail & Fail & Fail & Fail &2.1$\times10^{-6}$ & 3.3$\times10^{-5}$ & 2.7$\times10^{37}$ & 2.5$\times10^{-3}$\\
S & -0.94 & 10.0 & 40.0 & Pass & Fail & Pass & Fail & Fail & Pass & Pass & Fail & Fail & Pass & Fail & Fail & Fail & Pass &7.8$\times10^{-6}$ & 2.4$\times10^{-5}$ & 9.9$\times10^{37}$ & 2.5$\times10^{-3}$\\
S & -0.94 & 10.0 & 160.0 & Pass & Fail & Pass & Fail & Fail & Pass & Pass & Fail & Fail & Pass & Fail & Fail & Fail & Pass &1.4$\times10^{-5}$ & 1.2$\times10^{-5}$ & 1.7$\times10^{38}$ & 2.5$\times10^{-3}$\\
S & -0.94 & 30.0 & 1.0 & Fail & Pass & Pass & Pass & Fail & Pass & Pass & Pass & Fail & Pass & Fail & Fail & Fail & Pass &1.7$\times10^{-7}$ & 1.3$\times10^{-4}$ & 2.1$\times10^{36}$ & 2.5$\times10^{-3}$\\
S & -0.94 & 30.0 & 10.0 & Pass & Pass & Pass & Fail & Fail & Pass & Pass & Pass & Fail & Pass & Fail & Fail & Fail & Fail &2.0$\times10^{-6}$ & 3.2$\times10^{-5}$ & 2.6$\times10^{37}$ & 2.5$\times10^{-3}$\\
S & -0.94 & 30.0 & 40.0 & Pass & Pass & Pass & Fail & Fail & Pass & Pass & Fail & Pass & Pass & Fail & Fail & Fail & Pass &8.1$\times10^{-6}$ & 2.5$\times10^{-5}$ & 1.0$\times10^{38}$ & 2.5$\times10^{-3}$\\
S & -0.94 & 30.0 & 160.0 & Pass & Pass & Pass & Fail & Fail & Pass & Pass & Fail & Fail & Pass & Fail & Fail & Fail & Pass &1.5$\times10^{-5}$ & 1.3$\times10^{-5}$ & 1.9$\times10^{38}$ & 2.5$\times10^{-3}$\\
S & -0.94 & 50.0 & 1.0 & Fail & Pass & Pass & Pass & Fail & Pass & Pass & Pass & Fail & Pass & Fail & Fail & Fail & Fail &1.6$\times10^{-7}$ & 1.2$\times10^{-4}$ & 2.1$\times10^{36}$ & 2.5$\times10^{-3}$\\
S & -0.94 & 50.0 & 10.0 & Pass & Pass & Pass & Fail & Fail & Pass & Pass & Pass & Fail & Pass & Fail & Fail & Fail & Fail &2.0$\times10^{-6}$ & 3.2$\times10^{-5}$ & 2.5$\times10^{37}$ & 2.5$\times10^{-3}$\\
S & -0.94 & 50.0 & 40.0 & Pass & Fail & Fail & Fail & Fail & Pass & Pass & Fail & Fail & Pass & Fail & Fail & Fail & Pass &8.5$\times10^{-6}$ & 2.6$\times10^{-5}$ & 1.1$\times10^{38}$ & 2.5$\times10^{-3}$\\
S & -0.94 & 50.0 & 160.0 & Pass & Fail & Pass & Fail & Fail & Pass & Pass & Pass & Fail & Pass & Fail & Fail & Fail & Fail &1.6$\times10^{-5}$ & 1.4$\times10^{-5}$ & 2.0$\times10^{38}$ & 2.5$\times10^{-3}$\\
S & -0.94 & 70.0 & 1.0 & Fail & Pass & Pass & Pass & Fail & Pass & Pass & Pass & Fail & Pass & Fail & Fail & Fail & Fail &1.7$\times10^{-7}$ & 1.3$\times10^{-4}$ & 2.2$\times10^{36}$ & 2.5$\times10^{-3}$\\
S & -0.94 & 70.0 & 10.0 & Pass & Pass & Pass & Fail & Fail & Pass & Pass & Fail & Fail & Pass & Fail & Fail & Fail & Fail &2.2$\times10^{-6}$ & 3.5$\times10^{-5}$ & 2.8$\times10^{37}$ & 2.5$\times10^{-3}$\\
S & -0.94 & 70.0 & 40.0 & Pass & Fail & Fail & Fail & Fail & Pass & Pass & Pass & Fail & Pass & Fail & Fail & Fail & Fail &8.8$\times10^{-6}$ & 2.7$\times10^{-5}$ & 1.1$\times10^{38}$ & 2.5$\times10^{-3}$\\
S & -0.94 & 70.0 & 160.0 & Pass & Fail & Pass & Fail & Fail & Pass & Pass & Pass & Fail & Pass & Fail & Fail & Fail & Fail &1.7$\times10^{-5}$ & 1.5$\times10^{-5}$ & 2.1$\times10^{38}$ & 2.5$\times10^{-3}$\\
S & -0.94 & 90.0 & 1.0 & Fail & Pass & Pass & Pass & Fail & Pass & Pass & Pass & Pass & Pass & Pass & Fail & Fail & Pass &1.7$\times10^{-7}$ & 1.3$\times10^{-4}$ & 2.2$\times10^{36}$ & 2.5$\times10^{-3}$\\
S & -0.94 & 90.0 & 10.0 & Fail & Pass & Pass & Fail & Fail & Pass & Pass & Fail & Fail & Pass & Fail & Fail & Fail & Pass &2.3$\times10^{-6}$ & 3.6$\times10^{-5}$ & 2.9$\times10^{37}$ & 2.5$\times10^{-3}$\\
S & -0.94 & 90.0 & 40.0 & Pass & Fail & Fail & Fail & Fail & Pass & Pass & Pass & Pass & Pass & Pass & Fail & Fail & Fail &9.2$\times10^{-6}$ & 2.8$\times10^{-5}$ & 1.2$\times10^{38}$ & 2.5$\times10^{-3}$\\
S & -0.94 & 90.0 & 160.0 & Pass & Fail & Pass & Fail & Fail & Pass & Pass & Pass & Fail & Pass & Fail & Fail & Fail & Fail &1.8$\times10^{-5}$ & 1.6$\times10^{-5}$ & 2.3$\times10^{38}$ & 2.5$\times10^{-3}$\\
S & -0.5 & 10.0 & 1.0 & Fail & Pass & Pass & Pass & Fail & Fail & Pass & Pass & Fail & Pass & Fail & Fail & Pass & Pass &1.1$\times10^{-7}$ & 2.3$\times10^{-4}$ & 1.9$\times10^{35}$ & 3.5$\times10^{-4}$\\
S & -0.5 & 10.0 & 10.0 & Pass & Pass & Pass & Fail & Fail & Pass & Pass & Fail & Fail & Pass & Fail & Fail & Fail & Fail &1.2$\times10^{-6}$ & 3.7$\times10^{-5}$ & 2.1$\times10^{36}$ & 3.5$\times10^{-4}$\\
S & -0.5 & 10.0 & 40.0 & Pass & Fail & Pass & Fail & Fail & Pass & Pass & Fail & Pass & Pass & Fail & Fail & Pass & Pass &6.9$\times10^{-6}$ & 2.8$\times10^{-5}$ & 1.2$\times10^{37}$ & 3.5$\times10^{-4}$\\
S & -0.5 & 10.0 & 160.0 & Pass & Fail & Pass & Fail & Fail & Pass & Pass & Fail & Pass & Pass & Fail & Fail & Pass & Pass &1.1$\times10^{-5}$ & 1.3$\times10^{-5}$ & 1.9$\times10^{37}$ & 3.5$\times10^{-4}$\\
S & -0.5 & 30.0 & 1.0 & Fail & Pass & Pass & Pass & Fail & Pass & Pass & Pass & Fail & Pass & Fail & Fail & Fail & Pass &1.0$\times10^{-7}$ & 2.2$\times10^{-4}$ & 1.8$\times10^{35}$ & 3.5$\times10^{-4}$\\
S & -0.5 & 30.0 & 10.0 & Pass & Pass & Pass & Fail & Fail & Pass & Pass & Pass & Fail & Pass & Fail & Fail & Fail & Fail &1.1$\times10^{-6}$ & 3.5$\times10^{-5}$ & 2.0$\times10^{36}$ & 3.5$\times10^{-4}$\\
S & -0.5 & 30.0 & 40.0 & Pass & Pass & Pass & Fail & Fail & Pass & Pass & Fail & Pass & Pass & Fail & Fail & Pass & Pass &6.9$\times10^{-6}$ & 2.9$\times10^{-5}$ & 1.2$\times10^{37}$ & 3.5$\times10^{-4}$\\
S & -0.5 & 30.0 & 160.0 & Pass & Pass & Pass & Fail & Fail & Pass & Pass & Fail & Fail & Pass & Fail & Fail & Pass & Pass &1.1$\times10^{-5}$ & 1.3$\times10^{-5}$ & 1.9$\times10^{37}$ & 3.5$\times10^{-4}$\\
S & -0.5 & 50.0 & 1.0 & Fail & Pass & Pass & Pass & Fail & Pass & Pass & Pass & Fail & Pass & Fail & Fail & Fail & Pass &9.8$\times10^{-8}$ & 2.1$\times10^{-4}$ & 1.7$\times10^{35}$ & 3.5$\times10^{-4}$\\
S & -0.5 & 50.0 & 10.0 & Pass & Pass & Pass & Fail & Fail & Pass & Pass & Pass & Fail & Pass & Fail & Fail & Fail & Fail &1.1$\times10^{-6}$ & 3.5$\times10^{-5}$ & 2.0$\times10^{36}$ & 3.5$\times10^{-4}$\\
S & -0.5 & 50.0 & 40.0 & Pass & Fail & Pass & Fail & Fail & Pass & Pass & Fail & Fail & Pass & Fail & Fail & Pass & Pass &6.6$\times10^{-6}$ & 2.7$\times10^{-5}$ & 1.1$\times10^{37}$ & 3.5$\times10^{-4}$\\
S & -0.5 & 50.0 & 160.0 & Pass & Pass & Pass & Fail & Fail & Pass & Fail & Fail & Fail & Pass & Fail & Fail & Pass & Pass &1.1$\times10^{-5}$ & 1.2$\times10^{-5}$ & 1.9$\times10^{37}$ & 3.5$\times10^{-4}$\\
S & -0.5 & 70.0 & 1.0 & Fail & Pass & Pass & Pass & Fail & Pass & Pass & Pass & Pass & Pass & Pass & Fail & Fail & Fail &10.0$\times10^{-8}$ & 2.1$\times10^{-4}$ & 1.8$\times10^{35}$ & 3.5$\times10^{-4}$\\
S & -0.5 & 70.0 & 10.0 & Fail & Pass & Pass & Fail & Fail & Pass & Pass & Pass & Fail & Pass & Fail & Fail & Fail & Fail &1.2$\times10^{-6}$ & 3.7$\times10^{-5}$ & 2.1$\times10^{36}$ & 3.5$\times10^{-4}$\\
S & -0.5 & 70.0 & 40.0 & Pass & Fail & Pass & Fail & Fail & Pass & Pass & Pass & Pass & Pass & Pass & Fail & Pass & Pass &6.4$\times10^{-6}$ & 2.6$\times10^{-5}$ & 1.1$\times10^{37}$ & 3.5$\times10^{-4}$\\
S & -0.5 & 70.0 & 160.0 & Pass & Fail & Pass & Fail & Fail & Pass & Fail & Pass & Fail & Pass & Fail & Fail & Pass & Fail &1.1$\times10^{-5}$ & 1.2$\times10^{-5}$ & 1.9$\times10^{37}$ & 3.5$\times10^{-4}$\\
S & -0.5 & 90.0 & 1.0 & Fail & Fail & Pass & Pass & Fail & Pass & Pass & Pass & Pass & Pass & Pass & Fail & Fail & Pass &1.0$\times10^{-7}$ & 2.2$\times10^{-4}$ & 1.8$\times10^{35}$ & 3.5$\times10^{-4}$\\
S & -0.5 & 90.0 & 10.0 & Fail & Pass & Pass & Fail & Fail & Pass & Pass & Pass & Fail & Pass & Fail & Fail & Fail & Pass &1.3$\times10^{-6}$ & 4.0$\times10^{-5}$ & 2.3$\times10^{36}$ & 3.5$\times10^{-4}$\\
S & -0.5 & 90.0 & 40.0 & Pass & Fail & Pass & Fail & Fail & Pass & Fail & Pass & Pass & Pass & Fail & Fail & Fail & Fail &6.4$\times10^{-6}$ & 2.6$\times10^{-5}$ & 1.1$\times10^{37}$ & 3.5$\times10^{-4}$\\
S & -0.5 & 90.0 & 160.0 & Pass & Fail & Pass & Fail & Fail & Pass & Fail & Pass & Fail & Pass & Fail & Fail & Pass & Fail &1.1$\times10^{-5}$ & 1.2$\times10^{-5}$ & 1.9$\times10^{37}$ & 3.5$\times10^{-4}$\\
S & 0.0 & 10.0 & 1.0 & Fail & Pass & Pass & Pass & Fail & Fail & Pass & Pass & Fail & Pass & Fail & Fail & Pass & Pass &5.8$\times10^{-8}$ & 4.5$\times10^{-4}$ & 1.2$\times10^{35}$ & 4.0$\times10^{-4}$\\
S & 0.0 & 10.0 & 10.0 & Pass & Pass & Pass & Pass & Pass & Pass & Pass & Fail & Fail & Pass & Fail & Fail & Fail & Fail &5.1$\times10^{-7}$ & 6.6$\times10^{-5}$ & 1.1$\times10^{36}$ & 4.0$\times10^{-4}$\\
S & 0.0 & 10.0 & 40.0 & Fail & Fail & Pass & Fail & Fail & Pass & Fail & Fail & Pass & Pass & Fail & Fail & Fail & Pass &2.7$\times10^{-6}$ & 3.9$\times10^{-5}$ & 5.6$\times10^{36}$ & 4.0$\times10^{-4}$\\
S & 0.0 & 10.0 & 160.0 & Pass & Fail & Pass & Fail & Fail & Pass & Fail & Fail & Pass & Pass & Fail & Fail & Fail & Pass &4.6$\times10^{-6}$ & 1.7$\times10^{-5}$ & 8.9$\times10^{36}$ & 4.0$\times10^{-4}$\\
S & 0.0 & 30.0 & 1.0 & Fail & Pass & Pass & Pass & Fail & Pass & Pass & Pass & Fail & Pass & Fail & Fail & Fail & Pass &5.5$\times10^{-8}$ & 4.3$\times10^{-4}$ & 1.1$\times10^{35}$ & 4.0$\times10^{-4}$\\
S & 0.0 & 30.0 & 10.0 & Pass & Fail & Pass & Pass & Fail & Pass & Pass & Pass & Fail & Pass & Fail & Fail & Fail & Fail &4.8$\times10^{-7}$ & 6.3$\times10^{-5}$ & 9.9$\times10^{35}$ & 4.0$\times10^{-4}$\\
S & 0.0 & 30.0 & 40.0 & Fail & Fail & Pass & Fail & Fail & Pass & Pass & Fail & Pass & Pass & Fail & Fail & Fail & Pass &2.6$\times10^{-6}$ & 3.8$\times10^{-5}$ & 5.4$\times10^{36}$ & 4.0$\times10^{-4}$\\
S & 0.0 & 30.0 & 160.0 & Pass & Fail & Pass & Fail & Fail & Pass & Pass & Fail & Fail & Pass & Fail & Fail & Fail & Pass &4.3$\times10^{-6}$ & 1.6$\times10^{-5}$ & 8.4$\times10^{36}$ & 4.0$\times10^{-4}$\\
S & 0.0 & 50.0 & 1.0 & Fail & Fail & Pass & Pass & Fail & Pass & Pass & Pass & Fail & Pass & Fail & Fail & Fail & Pass &5.3$\times10^{-8}$ & 4.2$\times10^{-4}$ & 1.1$\times10^{35}$ & 4.0$\times10^{-4}$\\
S & 0.0 & 50.0 & 10.0 & Pass & Fail & Pass & Pass & Fail & Pass & Pass & Pass & Pass & Pass & Pass & Fail & Fail & Fail &4.7$\times10^{-7}$ & 6.1$\times10^{-5}$ & 9.7$\times10^{35}$ & 4.0$\times10^{-4}$\\
S & 0.0 & 50.0 & 40.0 & Pass & Pass & Pass & Fail & Fail & Pass & Pass & Pass & Pass & Pass & Pass & Fail & Fail & Fail &2.7$\times10^{-6}$ & 4.0$\times10^{-5}$ & 5.7$\times10^{36}$ & 4.0$\times10^{-4}$\\
S & 0.0 & 50.0 & 160.0 & Pass & Pass & Pass & Fail & Fail & Pass & Pass & Pass & Fail & Pass & Fail & Fail & Pass & Pass &4.7$\times10^{-6}$ & 1.7$\times10^{-5}$ & 9.3$\times10^{36}$ & 4.0$\times10^{-4}$\\
S & 0.0 & 70.0 & 1.0 & Fail & Fail & Pass & Pass & Fail & Pass & Pass & Pass & Pass & Pass & Pass & Fail & Fail & Pass &5.3$\times10^{-8}$ & 4.2$\times10^{-4}$ & 1.1$\times10^{35}$ & 4.0$\times10^{-4}$\\
S & 0.0 & 70.0 & 10.0 & Fail & Fail & Pass & Pass & Fail & Pass & Pass & Pass & Pass & Pass & Pass & Fail & Fail & Fail &5.0$\times10^{-7}$ & 6.5$\times10^{-5}$ & 1.0$\times10^{36}$ & 4.0$\times10^{-4}$\\
S & 0.0 & 70.0 & 40.0 & Pass & Fail & Pass & Fail & Fail & Pass & Pass & Pass & Pass & Pass & Pass & Fail & Fail & Fail &2.8$\times10^{-6}$ & 4.1$\times10^{-5}$ & 5.9$\times10^{36}$ & 4.0$\times10^{-4}$\\
S & 0.0 & 70.0 & 160.0 & Pass & Fail & Pass & Fail & Fail & Pass & Fail & Pass & Fail & Pass & Fail & Fail & Pass & Pass &4.8$\times10^{-6}$ & 1.8$\times10^{-5}$ & 9.7$\times10^{36}$ & 4.0$\times10^{-4}$\\
S & 0.0 & 90.0 & 1.0 & Fail & Fail & Pass & Pass & Fail & Pass & Pass & Pass & Fail & Pass & Fail & Fail & Fail & Pass &5.3$\times10^{-8}$ & 4.2$\times10^{-4}$ & 1.1$\times10^{35}$ & 4.0$\times10^{-4}$\\
S & 0.0 & 90.0 & 10.0 & Fail & Fail & Pass & Pass & Fail & Pass & Pass & Pass & Pass & Pass & Pass & Fail & Fail & Fail &5.2$\times10^{-7}$ & 6.8$\times10^{-5}$ & 1.1$\times10^{36}$ & 4.0$\times10^{-4}$\\
S & 0.0 & 90.0 & 40.0 & Pass & Fail & Pass & Fail & Fail & Pass & Pass & Pass & Pass & Pass & Pass & Fail & Fail & Fail &2.8$\times10^{-6}$ & 4.0$\times10^{-5}$ & 5.9$\times10^{36}$ & 4.0$\times10^{-4}$\\
S & 0.0 & 90.0 & 160.0 & Pass & Fail & Pass & Fail & Fail & Pass & Fail & Pass & Pass & Pass & Fail & Fail & Pass & Pass &4.9$\times10^{-6}$ & 1.8$\times10^{-5}$ & 9.9$\times10^{36}$ & 4.0$\times10^{-4}$\\
S & 0.5 & 10.0 & 1.0 & Fail & Pass & Pass & Pass & Fail & Pass & Pass & Pass & Fail & Pass & Fail & Fail & Pass & Pass &2.6$\times10^{-8}$ & 1.1$\times10^{-3}$ & 1.1$\times10^{35}$ & 8.4$\times10^{-4}$\\
S & 0.5 & 10.0 & 10.0 & Pass & Fail & Pass & Pass & Fail & Pass & Pass & Fail & Fail & Pass & Fail & Fail & Fail & Pass &2.1$\times10^{-7}$ & 1.6$\times10^{-4}$ & 8.9$\times10^{35}$ & 8.4$\times10^{-4}$\\
S & 0.5 & 10.0 & 40.0 & Fail & Fail & Pass & Fail & Fail & Pass & Pass & Pass & Pass & Pass & Pass & Fail & Pass & Pass &2.5$\times10^{-6}$ & 9.8$\times10^{-5}$ & 9.5$\times10^{36}$ & 8.4$\times10^{-4}$\\
S & 0.5 & 10.0 & 160.0 & Pass & Fail & Pass & Fail & Fail & Pass & Pass & Fail & Pass & Pass & Fail & Fail & Fail & Pass &5.7$\times10^{-6}$ & 3.7$\times10^{-5}$ & 2.2$\times10^{37}$ & 8.4$\times10^{-4}$\\
S & 0.5 & 30.0 & 1.0 & Fail & Pass & Pass & Pass & Fail & Pass & Pass & Pass & Fail & Pass & Fail & Fail & Pass & Pass &2.5$\times10^{-8}$ & 1.1$\times10^{-3}$ & 1.1$\times10^{35}$ & 8.4$\times10^{-4}$\\
S & 0.5 & 30.0 & 10.0 & Pass & Fail & Pass & Pass & Fail & Pass & Pass & Pass & Pass & Pass & Pass & Fail & Fail & Pass &2.0$\times10^{-7}$ & 1.5$\times10^{-4}$ & 8.4$\times10^{35}$ & 8.4$\times10^{-4}$\\
S & 0.5 & 30.0 & 40.0 & Fail & Pass & Pass & Fail & Fail & Pass & Pass & Pass & Pass & Pass & Pass & Fail & Pass & Pass &2.7$\times10^{-6}$ & 1.1$\times10^{-4}$ & 1.0$\times10^{37}$ & 8.4$\times10^{-4}$\\
S & 0.5 & 30.0 & 160.0 & Pass & Pass & Pass & Fail & Fail & Pass & Pass & Fail & Pass & Pass & Fail & Fail & Fail & Pass &5.7$\times10^{-6}$ & 3.9$\times10^{-5}$ & 2.3$\times10^{37}$ & 8.4$\times10^{-4}$\\
S & 0.5 & 50.0 & 1.0 & Fail & Fail & Pass & Pass & Fail & Pass & Pass & Pass & Pass & Pass & Pass & Fail & Fail & Pass &2.4$\times10^{-8}$ & 1.1$\times10^{-3}$ & 1.0$\times10^{35}$ & 8.4$\times10^{-4}$\\
S & 0.5 & 50.0 & 10.0 & Pass & Fail & Pass & Pass & Fail & Pass & Pass & Pass & Pass & Pass & Pass & Fail & Fail & Pass &1.9$\times10^{-7}$ & 1.5$\times10^{-4}$ & 8.2$\times10^{35}$ & 8.4$\times10^{-4}$\\
S & 0.5 & 50.0 & 40.0 & Fail & Pass & Pass & Fail & Fail & Pass & Pass & Pass & Pass & Pass & Pass & Fail & Pass & Pass &2.9$\times10^{-6}$ & 1.2$\times10^{-4}$ & 1.1$\times10^{37}$ & 8.4$\times10^{-4}$\\
S & 0.5 & 50.0 & 160.0 & Pass & Pass & Pass & Fail & Fail & Pass & Pass & Pass & Fail & Pass & Fail & Fail & Fail & Fail &6.2$\times10^{-6}$ & 4.2$\times10^{-5}$ & 2.5$\times10^{37}$ & 8.4$\times10^{-4}$\\
S & 0.5 & 70.0 & 1.0 & Fail & Fail & Pass & Pass & Fail & Pass & Pass & Pass & Pass & Pass & Pass & Fail & Fail & Pass &2.4$\times10^{-8}$ & 1.0$\times10^{-3}$ & 1.0$\times10^{35}$ & 8.4$\times10^{-4}$\\
S & 0.5 & 70.0 & 10.0 & Fail & Fail & Pass & Pass & Fail & Pass & Pass & Pass & Pass & Pass & Pass & Fail & Pass & Pass &2.0$\times10^{-7}$ & 1.5$\times10^{-4}$ & 8.6$\times10^{35}$ & 8.4$\times10^{-4}$\\
S & 0.5 & 70.0 & 40.0 & Fail & Pass & Fail & Fail & Fail & Pass & Pass & Pass & Pass & Pass & Pass & Fail & Pass & Pass &3.0$\times10^{-6}$ & 1.3$\times10^{-4}$ & 1.2$\times10^{37}$ & 8.4$\times10^{-4}$\\
S & 0.5 & 70.0 & 160.0 & Pass & Fail & Pass & Fail & Fail & Pass & Fail & Pass & Fail & Pass & Fail & Fail & Fail & Fail &6.2$\times10^{-6}$ & 4.1$\times10^{-5}$ & 2.5$\times10^{37}$ & 8.4$\times10^{-4}$\\
S & 0.5 & 90.0 & 1.0 & Fail & Fail & Pass & Pass & Fail & Pass & Fail & Pass & Pass & Pass & Fail & Fail & Fail & Pass &2.4$\times10^{-8}$ & 1.0$\times10^{-3}$ & 9.9$\times10^{34}$ & 8.4$\times10^{-4}$\\
S & 0.5 & 90.0 & 10.0 & Fail & Fail & Pass & Pass & Fail & Pass & Pass & Pass & Pass & Pass & Pass & Fail & Pass & Pass &2.2$\times10^{-7}$ & 1.7$\times10^{-4}$ & 9.4$\times10^{35}$ & 8.4$\times10^{-4}$\\
S & 0.5 & 90.0 & 40.0 & Fail & Pass & Fail & Fail & Fail & Pass & Pass & Pass & Pass & Pass & Pass & Fail & Pass & Fail &2.9$\times10^{-6}$ & 1.2$\times10^{-4}$ & 1.2$\times10^{37}$ & 8.4$\times10^{-4}$\\
S & 0.5 & 90.0 & 160.0 & Pass & Fail & Pass & Fail & Fail & Pass & Fail & Pass & Fail & Pass & Fail & Fail & Fail & Fail &6.1$\times10^{-6}$ & 4.0$\times10^{-5}$ & 2.4$\times10^{37}$ & 8.4$\times10^{-4}$\\
S & 0.94 & 10.0 & 1.0 & Fail & Fail & Pass & Pass & Fail & Pass & Pass & Fail & Fail & Pass & Fail & Fail & Pass & Pass &8.8$\times10^{-9}$ & 8.0$\times10^{-3}$ & 2.0$\times10^{36}$ & 4.3$\times10^{-2}$\\
S & 0.94 & 10.0 & 10.0 & Pass & Fail & Pass & Pass & Fail & Pass & Pass & Pass & Pass & Pass & Pass & Fail & Fail & Pass &4.7$\times10^{-8}$ & 1.7$\times10^{-3}$ & 1.0$\times10^{37}$ & 4.3$\times10^{-2}$\\
S & 0.94 & 10.0 & 40.0 & Pass & Fail & Pass & Pass & Fail & Pass & Pass & Pass & Pass & Pass & Pass & Fail & Pass & Pass &3.4$\times10^{-7}$ & 9.7$\times10^{-4}$ & 6.9$\times10^{37}$ & 4.3$\times10^{-2}$\\
S & 0.94 & 10.0 & 160.0 & Pass & Fail & Pass & Pass & Fail & Pass & Pass & Pass & Pass & Pass & Pass & Fail & Fail & Pass &6.9$\times10^{-7}$ & 3.7$\times10^{-4}$ & 1.4$\times10^{38}$ & 4.3$\times10^{-2}$\\
S & 0.94 & 30.0 & 1.0 & Fail & Fail & Pass & Pass & Fail & Pass & Pass & Pass & Pass & Pass & Pass & Fail & Pass & Pass &8.3$\times10^{-9}$ & 7.5$\times10^{-3}$ & 1.9$\times10^{36}$ & 4.3$\times10^{-2}$\\
S & 0.94 & 30.0 & 10.0 & Pass & Fail & Pass & Pass & Fail & Pass & Pass & Pass & Pass & Pass & Pass & Fail & Fail & Pass &4.4$\times10^{-8}$ & 1.6$\times10^{-3}$ & 9.7$\times10^{36}$ & 4.3$\times10^{-2}$\\
S & 0.94 & 30.0 & 40.0 & Pass & Fail & Pass & Pass & Fail & Pass & Pass & Pass & Pass & Pass & Pass & Fail & Pass & Pass &3.5$\times10^{-7}$ & 1.0$\times10^{-3}$ & 7.2$\times10^{37}$ & 4.3$\times10^{-2}$\\
S & 0.94 & 30.0 & 160.0 & Pass & Fail & Pass & Pass & Fail & Pass & Pass & Pass & Pass & Pass & Pass & Fail & Fail & Pass &7.0$\times10^{-7}$ & 3.7$\times10^{-4}$ & 1.4$\times10^{38}$ & 4.3$\times10^{-2}$\\
S & 0.94 & 50.0 & 1.0 & Pass & Fail & Pass & Pass & Fail & Pass & Pass & Pass & Pass & Pass & Pass & Fail & Pass & Pass &7.7$\times10^{-9}$ & 7.0$\times10^{-3}$ & 1.7$\times10^{36}$ & 4.3$\times10^{-2}$\\
S & 0.94 & 50.0 & 10.0 & Pass & Fail & Pass & Pass & Fail & Pass & Pass & Pass & Pass & Fail & Fail & Fail & Fail & Fail &3.9$\times10^{-8}$ & 1.5$\times10^{-3}$ & 8.7$\times10^{36}$ & 4.3$\times10^{-2}$\\
S & 0.94 & 50.0 & 40.0 & Pass & Fail & Fail & Fail & Fail & Pass & Pass & Pass & Pass & Pass & Pass & Fail & Pass & Pass &3.7$\times10^{-7}$ & 1.0$\times10^{-3}$ & 7.5$\times10^{37}$ & 4.3$\times10^{-2}$\\
S & 0.94 & 50.0 & 160.0 & Pass & Fail & Fail & Fail & Fail & Pass & Pass & Pass & Pass & Pass & Pass & Fail & Pass & Fail &7.9$\times10^{-7}$ & 4.2$\times10^{-4}$ & 1.6$\times10^{38}$ & 4.3$\times10^{-2}$\\
S & 0.94 & 70.0 & 1.0 & Pass & Fail & Pass & Fail & Fail & Pass & Fail & Pass & Pass & Pass & Fail & Fail & Pass & Pass &7.5$\times10^{-9}$ & 6.8$\times10^{-3}$ & 1.7$\times10^{36}$ & 4.3$\times10^{-2}$\\
S & 0.94 & 70.0 & 10.0 & Fail & Fail & Pass & Pass & Fail & Pass & Fail & Pass & Pass & Fail & Fail & Fail & Pass & Fail &3.9$\times10^{-8}$ & 1.5$\times10^{-3}$ & 8.6$\times10^{36}$ & 4.3$\times10^{-2}$\\
S & 0.94 & 70.0 & 40.0 & Pass & Fail & Fail & Fail & Fail & Pass & Pass & Pass & Pass & Pass & Pass & Fail & Pass & Fail &3.8$\times10^{-7}$ & 1.1$\times10^{-3}$ & 7.8$\times10^{37}$ & 4.3$\times10^{-2}$\\
S & 0.94 & 70.0 & 160.0 & Pass & Fail & Fail & Fail & Fail & Pass & Fail & Pass & Fail & Pass & Fail & Fail & Pass & Pass &8.5$\times10^{-7}$ & 4.4$\times10^{-4}$ & 1.7$\times10^{38}$ & 4.3$\times10^{-2}$\\
S & 0.94 & 90.0 & 1.0 & Pass & Fail & Pass & Fail & Fail & Pass & Fail & Pass & Fail & Pass & Fail & Fail & Pass & Pass &7.4$\times10^{-9}$ & 6.6$\times10^{-3}$ & 1.7$\times10^{36}$ & 4.3$\times10^{-2}$\\
S & 0.94 & 90.0 & 10.0 & Fail & Fail & Pass & Pass & Fail & Pass & Fail & Pass & Fail & Fail & Fail & Fail & Pass & Fail &4.1$\times10^{-8}$ & 1.5$\times10^{-3}$ & 9.0$\times10^{36}$ & 4.3$\times10^{-2}$\\
S & 0.94 & 90.0 & 40.0 & Fail & Fail & Fail & Fail & Fail & Pass & Fail & Pass & Fail & Pass & Fail & Fail & Pass & Fail &3.9$\times10^{-7}$ & 1.1$\times10^{-3}$ & 7.9$\times10^{37}$ & 4.3$\times10^{-2}$\\
S & 0.94 & 90.0 & 160.0 & Pass & Fail & Fail & Fail & Fail & Pass & Fail & Pass & Fail & Pass & Fail & Fail & Pass & Pass &8.6$\times10^{-7}$ & 4.5$\times10^{-4}$ & 1.7$\times10^{38}$ & 4.3$\times10^{-2}$\\
M & -0.94 & 10.0 & 1.0 & Fail & Pass & Fail & Fail & Fail & Pass & Pass & Fail & Fail & Pass & Fail & Fail & Fail & Pass &5.5$\times10^{-8}$ & 1.4$\times10^{-2}$ & 1.1$\times10^{38}$ & 3.8$\times10^{-1}$\\
M & -0.94 & 10.0 & 10.0 & Pass & Pass & Fail & Fail & Fail & Pass & Pass & Pass & Fail & Pass & Fail & Fail & Fail & Pass &8.9$\times10^{-8}$ & 4.6$\times10^{-3}$ & 1.8$\times10^{38}$ & 3.8$\times10^{-1}$\\
M & -0.94 & 10.0 & 40.0 & Pass & Pass & Fail & Pass & Fail & Pass & Pass & Pass & Fail & Pass & Fail & Fail & Fail & Pass &1.3$\times10^{-7}$ & 1.9$\times10^{-3}$ & 2.6$\times10^{38}$ & 3.8$\times10^{-1}$\\
M & -0.94 & 10.0 & 160.0 & Pass & Pass & Pass & Pass & Pass & Pass & Pass & Pass & Pass & Pass & Pass & Pass & Fail & Fail &2.2$\times10^{-7}$ & 6.9$\times10^{-4}$ & 4.3$\times10^{38}$ & 3.8$\times10^{-1}$\\
M & -0.94 & 30.0 & 1.0 & Fail & Pass & Fail & Fail & Fail & Pass & Pass & Fail & Fail & Pass & Fail & Fail & Fail & Pass &5.3$\times10^{-8}$ & 1.4$\times10^{-2}$ & 1.1$\times10^{38}$ & 3.8$\times10^{-1}$\\
M & -0.94 & 30.0 & 10.0 & Pass & Pass & Fail & Fail & Fail & Pass & Pass & Pass & Fail & Pass & Fail & Fail & Fail & Fail &8.5$\times10^{-8}$ & 4.5$\times10^{-3}$ & 1.7$\times10^{38}$ & 3.8$\times10^{-1}$\\
M & -0.94 & 30.0 & 40.0 & Pass & Pass & Fail & Pass & Fail & Pass & Pass & Pass & Fail & Pass & Fail & Fail & Fail & Fail &1.2$\times10^{-7}$ & 1.8$\times10^{-3}$ & 2.4$\times10^{38}$ & 3.8$\times10^{-1}$\\
M & -0.94 & 30.0 & 160.0 & Pass & Pass & Pass & Pass & Pass & Pass & Pass & Pass & Fail & Pass & Fail & Fail & Fail & Fail &2.1$\times10^{-7}$ & 6.7$\times10^{-4}$ & 4.1$\times10^{38}$ & 3.8$\times10^{-1}$\\
M & -0.94 & 50.0 & 1.0 & Fail & Pass & Fail & Fail & Fail & Pass & Pass & Fail & Fail & Pass & Fail & Fail & Fail & Pass &5.0$\times10^{-8}$ & 1.3$\times10^{-2}$ & 9.9$\times10^{37}$ & 3.8$\times10^{-1}$\\
M & -0.94 & 50.0 & 10.0 & Fail & Pass & Fail & Fail & Fail & Pass & Pass & Pass & Fail & Pass & Fail & Fail & Fail & Fail &7.8$\times10^{-8}$ & 4.1$\times10^{-3}$ & 1.5$\times10^{38}$ & 3.8$\times10^{-1}$\\
M & -0.94 & 50.0 & 40.0 & Pass & Pass & Fail & Fail & Fail & Pass & Pass & Pass & Fail & Pass & Fail & Fail & Fail & Fail &1.1$\times10^{-7}$ & 1.7$\times10^{-3}$ & 2.3$\times10^{38}$ & 3.8$\times10^{-1}$\\
M & -0.94 & 50.0 & 160.0 & Pass & Pass & Pass & Pass & Pass & Pass & Pass & Pass & Fail & Pass & Fail & Fail & Fail & Fail &1.9$\times10^{-7}$ & 6.2$\times10^{-4}$ & 3.8$\times10^{38}$ & 3.8$\times10^{-1}$\\
M & -0.94 & 70.0 & 1.0 & Fail & Pass & Fail & Fail & Fail & Pass & Pass & Fail & Fail & Pass & Fail & Fail & Fail & Fail &4.6$\times10^{-8}$ & 1.1$\times10^{-2}$ & 9.1$\times10^{37}$ & 3.8$\times10^{-1}$\\
M & -0.94 & 70.0 & 10.0 & Pass & Pass & Fail & Fail & Fail & Pass & Pass & Pass & Fail & Pass & Fail & Fail & Fail & Fail &7.1$\times10^{-8}$ & 3.7$\times10^{-3}$ & 1.4$\times10^{38}$ & 3.8$\times10^{-1}$\\
M & -0.94 & 70.0 & 40.0 & Pass & Pass & Fail & Fail & Fail & Pass & Pass & Pass & Fail & Pass & Fail & Fail & Fail & Fail &1.0$\times10^{-7}$ & 1.5$\times10^{-3}$ & 2.1$\times10^{38}$ & 3.8$\times10^{-1}$\\
M & -0.94 & 70.0 & 160.0 & Pass & Pass & Fail & Pass & Fail & Pass & Pass & Pass & Fail & Pass & Fail & Fail & Fail & Fail &1.8$\times10^{-7}$ & 5.6$\times10^{-4}$ & 3.5$\times10^{38}$ & 3.8$\times10^{-1}$\\
M & -0.94 & 90.0 & 1.0 & Pass & Pass & Fail & Fail & Fail & Pass & Pass & Fail & Fail & Pass & Fail & Fail & Fail & Fail &4.4$\times10^{-8}$ & 1.1$\times10^{-2}$ & 8.6$\times10^{37}$ & 3.8$\times10^{-1}$\\
M & -0.94 & 90.0 & 10.0 & Pass & Pass & Fail & Fail & Fail & Pass & Pass & Pass & Fail & Pass & Fail & Fail & Fail & Fail &6.7$\times10^{-8}$ & 3.5$\times10^{-3}$ & 1.3$\times10^{38}$ & 3.8$\times10^{-1}$\\
M & -0.94 & 90.0 & 40.0 & Pass & Pass & Fail & Fail & Fail & Pass & Pass & Pass & Fail & Pass & Fail & Fail & Fail & Fail &10.0$\times10^{-8}$ & 1.5$\times10^{-3}$ & 2.0$\times10^{38}$ & 3.8$\times10^{-1}$\\
M & -0.94 & 90.0 & 160.0 & Pass & Pass & Fail & Pass & Fail & Pass & Pass & Pass & Fail & Pass & Fail & Fail & Fail & Fail &1.7$\times10^{-7}$ & 5.4$\times10^{-4}$ & 3.3$\times10^{38}$ & 3.8$\times10^{-1}$\\
M & -0.5 & 10.0 & 1.0 & Pass & Pass & Fail & Fail & Fail & Pass & Pass & Fail & Fail & Pass & Fail & Fail & Fail & Pass &4.2$\times10^{-8}$ & 4.9$\times10^{-3}$ & 2.7$\times10^{37}$ & 1.3$\times10^{-1}$\\
M & -0.5 & 10.0 & 10.0 & Pass & Pass & Fail & Fail & Fail & Pass & Pass & Fail & Fail & Pass & Fail & Fail & Fail & Pass &7.0$\times10^{-8}$ & 2.5$\times10^{-3}$ & 4.6$\times10^{37}$ & 1.3$\times10^{-1}$\\
M & -0.5 & 10.0 & 40.0 & Pass & Pass & Pass & Pass & Pass & Pass & Pass & Fail & Fail & Pass & Fail & Fail & Fail & Pass &1.0$\times10^{-7}$ & 1.4$\times10^{-3}$ & 6.7$\times10^{37}$ & 1.3$\times10^{-1}$\\
M & -0.5 & 10.0 & 160.0 & Pass & Fail & Pass & Pass & Fail & Pass & Pass & Fail & Fail & Pass & Fail & Fail & Fail & Pass &1.7$\times10^{-7}$ & 6.6$\times10^{-4}$ & 1.1$\times10^{38}$ & 1.3$\times10^{-1}$\\
M & -0.5 & 30.0 & 1.0 & Pass & Pass & Fail & Fail & Fail & Pass & Pass & Fail & Fail & Pass & Fail & Fail & Fail & Pass &4.1$\times10^{-8}$ & 4.8$\times10^{-3}$ & 2.7$\times10^{37}$ & 1.3$\times10^{-1}$\\
M & -0.5 & 30.0 & 10.0 & Pass & Pass & Fail & Fail & Fail & Pass & Pass & Fail & Fail & Pass & Fail & Fail & Fail & Pass &6.7$\times10^{-8}$ & 2.4$\times10^{-3}$ & 4.4$\times10^{37}$ & 1.3$\times10^{-1}$\\
M & -0.5 & 30.0 & 40.0 & Pass & Pass & Pass & Pass & Pass & Pass & Pass & Fail & Fail & Pass & Fail & Fail & Fail & Pass &9.9$\times10^{-8}$ & 1.4$\times10^{-3}$ & 6.4$\times10^{37}$ & 1.3$\times10^{-1}$\\
M & -0.5 & 30.0 & 160.0 & Pass & Pass & Pass & Pass & Pass & Pass & Pass & Fail & Fail & Pass & Fail & Fail & Fail & Pass &1.6$\times10^{-7}$ & 6.4$\times10^{-4}$ & 1.1$\times10^{38}$ & 1.3$\times10^{-1}$\\
M & -0.5 & 50.0 & 1.0 & Pass & Pass & Fail & Fail & Fail & Pass & Pass & Fail & Fail & Pass & Fail & Fail & Fail & Pass &3.9$\times10^{-8}$ & 4.5$\times10^{-3}$ & 2.5$\times10^{37}$ & 1.3$\times10^{-1}$\\
M & -0.5 & 50.0 & 10.0 & Pass & Pass & Fail & Fail & Fail & Pass & Pass & Pass & Fail & Pass & Fail & Fail & Fail & Pass &6.2$\times10^{-8}$ & 2.2$\times10^{-3}$ & 4.1$\times10^{37}$ & 1.3$\times10^{-1}$\\
M & -0.5 & 50.0 & 40.0 & Pass & Pass & Fail & Pass & Fail & Pass & Pass & Fail & Fail & Pass & Fail & Fail & Fail & Pass &9.2$\times10^{-8}$ & 1.3$\times10^{-3}$ & 6.0$\times10^{37}$ & 1.3$\times10^{-1}$\\
M & -0.5 & 50.0 & 160.0 & Pass & Pass & Pass & Pass & Pass & Pass & Pass & Fail & Fail & Pass & Fail & Fail & Fail & Fail &1.5$\times10^{-7}$ & 6.0$\times10^{-4}$ & 9.9$\times10^{37}$ & 1.3$\times10^{-1}$\\
M & -0.5 & 70.0 & 1.0 & Pass & Pass & Fail & Fail & Fail & Pass & Pass & Fail & Fail & Pass & Fail & Fail & Fail & Pass &3.6$\times10^{-8}$ & 4.2$\times10^{-3}$ & 2.4$\times10^{37}$ & 1.3$\times10^{-1}$\\
M & -0.5 & 70.0 & 10.0 & Pass & Pass & Fail & Fail & Fail & Pass & Pass & Fail & Fail & Pass & Fail & Fail & Fail & Fail &5.7$\times10^{-8}$ & 2.0$\times10^{-3}$ & 3.7$\times10^{37}$ & 1.3$\times10^{-1}$\\
M & -0.5 & 70.0 & 40.0 & Pass & Pass & Fail & Pass & Fail & Pass & Pass & Fail & Fail & Pass & Fail & Fail & Fail & Pass &8.5$\times10^{-8}$ & 1.2$\times10^{-3}$ & 5.6$\times10^{37}$ & 1.3$\times10^{-1}$\\
M & -0.5 & 70.0 & 160.0 & Pass & Pass & Pass & Pass & Pass & Pass & Pass & Fail & Fail & Pass & Fail & Fail & Fail & Fail &1.4$\times10^{-7}$ & 5.6$\times10^{-4}$ & 9.2$\times10^{37}$ & 1.3$\times10^{-1}$\\
M & -0.5 & 90.0 & 1.0 & Pass & Pass & Fail & Fail & Fail & Pass & Fail & Fail & Fail & Pass & Fail & Fail & Fail & Pass &3.4$\times10^{-8}$ & 3.9$\times10^{-3}$ & 2.2$\times10^{37}$ & 1.3$\times10^{-1}$\\
M & -0.5 & 90.0 & 10.0 & Pass & Pass & Fail & Fail & Fail & Pass & Fail & Fail & Fail & Pass & Fail & Fail & Fail & Fail &5.3$\times10^{-8}$ & 1.9$\times10^{-3}$ & 3.5$\times10^{37}$ & 1.3$\times10^{-1}$\\
M & -0.5 & 90.0 & 40.0 & Pass & Pass & Fail & Pass & Fail & Pass & Pass & Fail & Fail & Pass & Fail & Fail & Fail & Fail &8.1$\times10^{-8}$ & 1.1$\times10^{-3}$ & 5.3$\times10^{37}$ & 1.3$\times10^{-1}$\\
M & -0.5 & 90.0 & 160.0 & Pass & Pass & Pass & Pass & Pass & Pass & Pass & Fail & Fail & Pass & Fail & Fail & Fail & Fail &1.4$\times10^{-7}$ & 5.3$\times10^{-4}$ & 8.8$\times10^{37}$ & 1.3$\times10^{-1}$\\
M & 0.0 & 10.0 & 1.0 & Fail & Pass & Fail & Fail & Fail & Pass & Pass & Fail & Fail & Pass & Fail & Fail & Fail & Pass &3.1$\times10^{-8}$ & 5.5$\times10^{-3}$ & 3.7$\times10^{36}$ & 2.3$\times10^{-2}$\\
M & 0.0 & 10.0 & 10.0 & Pass & Fail & Pass & Pass & Fail & Pass & Pass & Fail & Fail & Pass & Fail & Fail & Fail & Pass &5.4$\times10^{-8}$ & 2.7$\times10^{-3}$ & 6.3$\times10^{36}$ & 2.3$\times10^{-2}$\\
M & 0.0 & 10.0 & 40.0 & Pass & Fail & Pass & Pass & Fail & Pass & Pass & Fail & Fail & Pass & Fail & Fail & Fail & Pass &8.1$\times10^{-8}$ & 1.7$\times10^{-3}$ & 9.5$\times10^{36}$ & 2.3$\times10^{-2}$\\
M & 0.0 & 10.0 & 160.0 & Pass & Fail & Pass & Pass & Fail & Pass & Pass & Fail & Fail & Pass & Fail & Fail & Fail & Pass &1.3$\times10^{-7}$ & 8.5$\times10^{-4}$ & 1.5$\times10^{37}$ & 2.3$\times10^{-2}$\\
M & 0.0 & 30.0 & 1.0 & Fail & Pass & Fail & Fail & Fail & Pass & Pass & Fail & Fail & Pass & Fail & Fail & Fail & Pass &3.1$\times10^{-8}$ & 5.4$\times10^{-3}$ & 3.6$\times10^{36}$ & 2.3$\times10^{-2}$\\
M & 0.0 & 30.0 & 10.0 & Pass & Pass & Pass & Pass & Pass & Pass & Pass & Fail & Fail & Pass & Fail & Fail & Fail & Pass &5.2$\times10^{-8}$ & 2.7$\times10^{-3}$ & 6.2$\times10^{36}$ & 2.3$\times10^{-2}$\\
M & 0.0 & 30.0 & 40.0 & Pass & Pass & Pass & Pass & Pass & Pass & Pass & Fail & Fail & Pass & Fail & Fail & Fail & Pass &7.8$\times10^{-8}$ & 1.6$\times10^{-3}$ & 9.2$\times10^{36}$ & 2.3$\times10^{-2}$\\
M & 0.0 & 30.0 & 160.0 & Pass & Pass & Pass & Pass & Pass & Pass & Pass & Fail & Fail & Pass & Fail & Fail & Fail & Pass &1.2$\times10^{-7}$ & 8.2$\times10^{-4}$ & 1.4$\times10^{37}$ & 2.3$\times10^{-2}$\\
M & 0.0 & 50.0 & 1.0 & Fail & Pass & Fail & Fail & Fail & Pass & Pass & Fail & Pass & Pass & Fail & Fail & Fail & Pass &3.0$\times10^{-8}$ & 5.2$\times10^{-3}$ & 3.5$\times10^{36}$ & 2.3$\times10^{-2}$\\
M & 0.0 & 50.0 & 10.0 & Pass & Pass & Fail & Pass & Fail & Pass & Pass & Pass & Pass & Pass & Pass & Fail & Fail & Pass &4.8$\times10^{-8}$ & 2.5$\times10^{-3}$ & 5.7$\times10^{36}$ & 2.3$\times10^{-2}$\\
M & 0.0 & 50.0 & 40.0 & Pass & Pass & Pass & Pass & Pass & Pass & Pass & Pass & Fail & Pass & Fail & Fail & Fail & Pass &7.3$\times10^{-8}$ & 1.5$\times10^{-3}$ & 8.6$\times10^{36}$ & 2.3$\times10^{-2}$\\
M & 0.0 & 50.0 & 160.0 & Pass & Pass & Pass & Pass & Pass & Pass & Pass & Pass & Fail & Pass & Fail & Fail & Fail & Pass &1.1$\times10^{-7}$ & 7.6$\times10^{-4}$ & 1.3$\times10^{37}$ & 2.3$\times10^{-2}$\\
M & 0.0 & 70.0 & 1.0 & Fail & Pass & Fail & Fail & Fail & Pass & Pass & Pass & Fail & Pass & Fail & Fail & Fail & Pass &2.9$\times10^{-8}$ & 5.0$\times10^{-3}$ & 3.4$\times10^{36}$ & 2.3$\times10^{-2}$\\
M & 0.0 & 70.0 & 10.0 & Pass & Pass & Fail & Fail & Fail & Pass & Pass & Pass & Fail & Pass & Fail & Fail & Fail & Pass &4.6$\times10^{-8}$ & 2.3$\times10^{-3}$ & 5.4$\times10^{36}$ & 2.3$\times10^{-2}$\\
M & 0.0 & 70.0 & 40.0 & Pass & Pass & Fail & Pass & Fail & Pass & Pass & Pass & Fail & Pass & Fail & Fail & Fail & Pass &6.7$\times10^{-8}$ & 1.4$\times10^{-3}$ & 7.9$\times10^{36}$ & 2.3$\times10^{-2}$\\
M & 0.0 & 70.0 & 160.0 & Pass & Pass & Pass & Pass & Pass & Pass & Pass & Fail & Fail & Pass & Fail & Fail & Fail & Pass &1.0$\times10^{-7}$ & 7.0$\times10^{-4}$ & 1.2$\times10^{37}$ & 2.3$\times10^{-2}$\\
M & 0.0 & 90.0 & 1.0 & Pass & Pass & Fail & Fail & Fail & Pass & Fail & Fail & Fail & Pass & Fail & Fail & Fail & Pass &2.6$\times10^{-8}$ & 4.5$\times10^{-3}$ & 3.1$\times10^{36}$ & 2.3$\times10^{-2}$\\
M & 0.0 & 90.0 & 10.0 & Pass & Pass & Fail & Fail & Fail & Pass & Fail & Pass & Fail & Fail & Fail & Fail & Fail & Pass &4.3$\times10^{-8}$ & 2.2$\times10^{-3}$ & 5.1$\times10^{36}$ & 2.3$\times10^{-2}$\\
M & 0.0 & 90.0 & 40.0 & Pass & Pass & Fail & Pass & Fail & Pass & Fail & Pass & Fail & Fail & Fail & Fail & Fail & Pass &6.3$\times10^{-8}$ & 1.3$\times10^{-3}$ & 7.4$\times10^{36}$ & 2.3$\times10^{-2}$\\
M & 0.0 & 90.0 & 160.0 & Pass & Fail & Pass & Pass & Fail & Pass & Fail & Fail & Fail & Pass & Fail & Fail & Fail & Pass &1.0$\times10^{-7}$ & 6.8$\times10^{-4}$ & 1.2$\times10^{37}$ & 2.3$\times10^{-2}$\\
M & 0.5 & 10.0 & 1.0 & Fail & Pass & Fail & Fail & Fail & Pass & Pass & Fail & Fail & Pass & Fail & Fail & Fail & Pass &2.4$\times10^{-8}$ & 9.6$\times10^{-3}$ & 3.0$\times10^{37}$ & 2.5$\times10^{-1}$\\
M & 0.5 & 10.0 & 10.0 & Pass & Fail & Fail & Pass & Fail & Pass & Pass & Fail & Fail & Pass & Fail & Fail & Fail & Pass &4.0$\times10^{-8}$ & 4.6$\times10^{-3}$ & 5.2$\times10^{37}$ & 2.5$\times10^{-1}$\\
M & 0.5 & 10.0 & 40.0 & Pass & Pass & Pass & Pass & Pass & Pass & Pass & Pass & Fail & Pass & Fail & Fail & Fail & Pass &6.5$\times10^{-8}$ & 2.7$\times10^{-3}$ & 8.3$\times10^{37}$ & 2.5$\times10^{-1}$\\
M & 0.5 & 10.0 & 160.0 & Pass & Pass & Pass & Pass & Pass & Pass & Pass & Pass & Fail & Pass & Fail & Fail & Fail & Pass &1.1$\times10^{-7}$ & 1.3$\times10^{-3}$ & 1.4$\times10^{38}$ & 2.5$\times10^{-1}$\\
M & 0.5 & 30.0 & 1.0 & Pass & Fail & Fail & Fail & Fail & Pass & Pass & Fail & Fail & Pass & Fail & Fail & Fail & Pass &2.3$\times10^{-8}$ & 9.3$\times10^{-3}$ & 2.9$\times10^{37}$ & 2.5$\times10^{-1}$\\
M & 0.5 & 30.0 & 10.0 & Pass & Pass & Fail & Pass & Fail & Pass & Pass & Pass & Pass & Pass & Pass & Fail & Fail & Pass &3.9$\times10^{-8}$ & 4.4$\times10^{-3}$ & 5.0$\times10^{37}$ & 2.5$\times10^{-1}$\\
M & 0.5 & 30.0 & 40.0 & Pass & Pass & Pass & Pass & Pass & Pass & Pass & Pass & Pass & Pass & Pass & Pass & Fail & Pass &6.3$\times10^{-8}$ & 2.6$\times10^{-3}$ & 8.0$\times10^{37}$ & 2.5$\times10^{-1}$\\
M & 0.5 & 30.0 & 160.0 & Pass & Pass & Pass & Pass & Pass & Pass & Pass & Pass & Pass & Pass & Pass & Pass & Fail & Pass &1.0$\times10^{-7}$ & 1.3$\times10^{-3}$ & 1.3$\times10^{38}$ & 2.5$\times10^{-1}$\\
M & 0.5 & 50.0 & 1.0 & Fail & Fail & Fail & Fail & Fail & Pass & Pass & Fail & Pass & Pass & Fail & Fail & Fail & Pass &2.2$\times10^{-8}$ & 9.0$\times10^{-3}$ & 2.8$\times10^{37}$ & 2.5$\times10^{-1}$\\
M & 0.5 & 50.0 & 10.0 & Pass & Pass & Fail & Pass & Fail & Pass & Pass & Pass & Pass & Fail & Fail & Fail & Fail & Fail &3.7$\times10^{-8}$ & 4.2$\times10^{-3}$ & 4.8$\times10^{37}$ & 2.5$\times10^{-1}$\\
M & 0.5 & 50.0 & 40.0 & Pass & Pass & Fail & Pass & Fail & Pass & Pass & Pass & Pass & Fail & Fail & Fail & Fail & Fail &6.0$\times10^{-8}$ & 2.5$\times10^{-3}$ & 7.7$\times10^{37}$ & 2.5$\times10^{-1}$\\
M & 0.5 & 50.0 & 160.0 & Pass & Fail & Pass & Pass & Fail & Pass & Pass & Pass & Pass & Pass & Pass & Fail & Fail & Fail &1.0$\times10^{-7}$ & 1.2$\times10^{-3}$ & 1.3$\times10^{38}$ & 2.5$\times10^{-1}$\\
M & 0.5 & 70.0 & 1.0 & Pass & Fail & Fail & Fail & Fail & Pass & Pass & Pass & Fail & Pass & Fail & Fail & Fail & Pass &2.1$\times10^{-8}$ & 8.7$\times10^{-3}$ & 2.7$\times10^{37}$ & 2.5$\times10^{-1}$\\
M & 0.5 & 70.0 & 10.0 & Pass & Pass & Fail & Fail & Fail & Pass & Pass & Pass & Fail & Pass & Fail & Fail & Fail & Fail &3.5$\times10^{-8}$ & 4.0$\times10^{-3}$ & 4.5$\times10^{37}$ & 2.5$\times10^{-1}$\\
M & 0.5 & 70.0 & 40.0 & Pass & Pass & Fail & Pass & Fail & Pass & Pass & Pass & Fail & Pass & Fail & Fail & Fail & Fail &5.6$\times10^{-8}$ & 2.3$\times10^{-3}$ & 7.1$\times10^{37}$ & 2.5$\times10^{-1}$\\
M & 0.5 & 70.0 & 160.0 & Pass & Pass & Pass & Pass & Pass & Pass & Pass & Pass & Fail & Pass & Fail & Fail & Fail & Fail &9.6$\times10^{-8}$ & 1.2$\times10^{-3}$ & 1.2$\times10^{38}$ & 2.5$\times10^{-1}$\\
M & 0.5 & 90.0 & 1.0 & Pass & Fail & Fail & Fail & Fail & Pass & Fail & Pass & Fail & Pass & Fail & Fail & Fail & Fail &2.0$\times10^{-8}$ & 7.9$\times10^{-3}$ & 2.5$\times10^{37}$ & 2.5$\times10^{-1}$\\
M & 0.5 & 90.0 & 10.0 & Pass & Fail & Fail & Fail & Fail & Pass & Fail & Pass & Fail & Fail & Fail & Fail & Fail & Fail &3.3$\times10^{-8}$ & 3.8$\times10^{-3}$ & 4.2$\times10^{37}$ & 2.5$\times10^{-1}$\\
M & 0.5 & 90.0 & 40.0 & Pass & Pass & Fail & Pass & Fail & Pass & Fail & Pass & Fail & Fail & Fail & Fail & Fail & Fail &5.4$\times10^{-8}$ & 2.3$\times10^{-3}$ & 6.9$\times10^{37}$ & 2.5$\times10^{-1}$\\
M & 0.5 & 90.0 & 160.0 & Pass & Pass & Fail & Pass & Fail & Pass & Pass & Pass & Fail & Pass & Fail & Fail & Fail & Fail &9.3$\times10^{-8}$ & 1.1$\times10^{-3}$ & 1.2$\times10^{38}$ & 2.5$\times10^{-1}$\\
M & 0.94 & 10.0 & 1.0 & Pass & Pass & Fail & Fail & Fail & Pass & Pass & Fail & Fail & Pass & Fail & Fail & Fail & Pass &1.7$\times10^{-8}$ & 7.4$\times10^{-2}$ & 1.4$\times10^{38}$ & 1.6\\
M & 0.94 & 10.0 & 10.0 & Pass & Pass & Fail & Fail & Fail & Pass & Pass & Pass & Pass & Pass & Pass & Fail & Fail & Pass &2.5$\times10^{-8}$ & 2.0$\times10^{-2}$ & 2.0$\times10^{38}$ & 1.6\\
M & 0.94 & 10.0 & 40.0 & Pass & Pass & Fail & Pass & Fail & Pass & Pass & Pass & Pass & Pass & Pass & Fail & Fail & Pass &3.6$\times10^{-8}$ & 8.5$\times10^{-3}$ & 3.0$\times10^{38}$ & 1.6\\
M & 0.94 & 10.0 & 160.0 & Pass & Pass & Pass & Pass & Pass & Pass & Pass & Pass & Pass & Pass & Pass & Pass & Fail & Pass &5.8$\times10^{-8}$ & 3.1$\times10^{-3}$ & 4.8$\times10^{38}$ & 1.6\\
M & 0.94 & 30.0 & 1.0 & Pass & Pass & Fail & Fail & Fail & Pass & Pass & Fail & Pass & Pass & Fail & Fail & Fail & Pass &1.6$\times10^{-8}$ & 7.1$\times10^{-2}$ & 1.3$\times10^{38}$ & 1.6\\
M & 0.94 & 30.0 & 10.0 & Pass & Pass & Fail & Fail & Fail & Pass & Pass & Pass & Pass & Pass & Pass & Fail & Fail & Pass &2.4$\times10^{-8}$ & 2.0$\times10^{-2}$ & 2.0$\times10^{38}$ & 1.6\\
M & 0.94 & 30.0 & 40.0 & Pass & Pass & Fail & Pass & Fail & Pass & Pass & Pass & Pass & Pass & Pass & Fail & Fail & Pass &3.5$\times10^{-8}$ & 8.2$\times10^{-3}$ & 2.9$\times10^{38}$ & 1.6\\
M & 0.94 & 30.0 & 160.0 & Pass & Pass & Pass & Pass & Pass & Pass & Pass & Pass & Pass & Pass & Pass & Pass & Fail & Pass &5.7$\times10^{-8}$ & 3.0$\times10^{-3}$ & 4.7$\times10^{38}$ & 1.6\\
M & 0.94 & 50.0 & 1.0 & Pass & Pass & Fail & Fail & Fail & Pass & Pass & Pass & Pass & Pass & Pass & Fail & Fail & Pass &1.6$\times10^{-8}$ & 6.8$\times10^{-2}$ & 1.3$\times10^{38}$ & 1.6\\
M & 0.94 & 50.0 & 10.0 & Pass & Pass & Fail & Fail & Fail & Pass & Pass & Pass & Pass & Pass & Pass & Fail & Fail & Fail &2.3$\times10^{-8}$ & 1.9$\times10^{-2}$ & 1.9$\times10^{38}$ & 1.6\\
M & 0.94 & 50.0 & 40.0 & Pass & Pass & Fail & Fail & Fail & Pass & Pass & Pass & Pass & Pass & Pass & Fail & Fail & Fail &3.4$\times10^{-8}$ & 7.9$\times10^{-3}$ & 2.8$\times10^{38}$ & 1.6\\
M & 0.94 & 50.0 & 160.0 & Pass & Pass & Pass & Pass & Pass & Pass & Pass & Pass & Pass & Pass & Pass & Pass & Fail & Fail &5.5$\times10^{-8}$ & 2.9$\times10^{-3}$ & 4.5$\times10^{38}$ & 1.6\\
M & 0.94 & 70.0 & 1.0 & Pass & Pass & Fail & Fail & Fail & Pass & Pass & Pass & Fail & Pass & Fail & Fail & Fail & Fail &1.5$\times10^{-8}$ & 6.4$\times10^{-2}$ & 1.2$\times10^{38}$ & 1.6\\
M & 0.94 & 70.0 & 10.0 & Pass & Pass & Fail & Fail & Fail & Pass & Pass & Pass & Fail & Pass & Fail & Fail & Fail & Fail &2.2$\times10^{-8}$ & 1.8$\times10^{-2}$ & 1.8$\times10^{38}$ & 1.6\\
M & 0.94 & 70.0 & 40.0 & Pass & Pass & Fail & Fail & Fail & Pass & Pass & Pass & Fail & Pass & Fail & Fail & Fail & Fail &3.2$\times10^{-8}$ & 7.6$\times10^{-3}$ & 2.7$\times10^{38}$ & 1.6\\
M & 0.94 & 70.0 & 160.0 & Pass & Pass & Fail & Pass & Fail & Pass & Pass & Pass & Fail & Pass & Fail & Fail & Fail & Fail &5.5$\times10^{-8}$ & 2.9$\times10^{-3}$ & 4.5$\times10^{38}$ & 1.6\\
M & 0.94 & 90.0 & 1.0 & Pass & Pass & Fail & Fail & Fail & Pass & Fail & Pass & Fail & Fail & Fail & Fail & Fail & Fail &1.4$\times10^{-8}$ & 6.1$\times10^{-2}$ & 1.2$\times10^{38}$ & 1.6\\
M & 0.94 & 90.0 & 10.0 & Pass & Pass & Fail & Fail & Fail & Pass & Fail & Pass & Fail & Fail & Fail & Fail & Fail & Fail &2.0$\times10^{-8}$ & 1.7$\times10^{-2}$ & 1.7$\times10^{38}$ & 1.6\\
M & 0.94 & 90.0 & 40.0 & Pass & Pass & Fail & Fail & Fail & Pass & Fail & Pass & Fail & Fail & Fail & Fail & Fail & Fail &3.1$\times10^{-8}$ & 7.2$\times10^{-3}$ & 2.5$\times10^{38}$ & 1.6\\
M & 0.94 & 90.0 & 160.0 & Pass & Pass & Fail & Pass & Fail & Pass & Fail & Pass & Fail & Fail & Fail & Fail & Fail & Fail &5.4$\times10^{-8}$ & 2.8$\times10^{-3}$ & 4.4$\times10^{38}$ & 1.6\\
\enddata
\end{deluxetable*}
\end{longrotatetable}
\begin{deluxetable*}{ccc|ccc|c|ccccc|c|c|cc}
\tabletypesize{\scriptsize}
\tablecaption{Pass/Fail Table, Critical Beta Models }\label{tab:betacritPF}
\tablehead{ \colhead{M/S}  &  %
\colhead{Spin}  &  %
\colhead{$i$}  &  %
\colhead{$F_{86}$}  &  %
\colhead{$\lambda_{maj,86}$}  &  %
\colhead{$F_{2\mu{\rm m}}$}  &  %
\colhead{non-EHT}  &  %
\colhead{$\lambda_{230}$}  &  %
\colhead{Nulls}  &  %
\colhead{Ring D}  &  %
\colhead{Ring W}  &  %
\colhead{Ring A}  &  %
\colhead{EHT}  &  %
\colhead{All}  &  %
\colhead{MI} & %
\colhead{4G$\lambda$} & %
}
\startdata
S & -0.94 & 10.0 & Fail & Fail & Pass & Fail & Pass & Pass & Fail & Pass & Pass & Fail & Fail & Fail & Fail\\
S & -0.94 & 50.0 & Fail & Pass & Pass & Fail & Pass & Pass & Fail & Fail & Pass & Fail & Fail & Fail & Fail\\
S & -0.94 & 90.0 & Fail & Pass & Pass & Fail & Pass & Pass & Pass & Fail & Pass & Fail & Fail & Fail & Fail\\
S & -0.5 & 10.0 & Fail & Pass & Pass & Fail & Pass & Pass & Fail & Fail & Pass & Fail & Fail & Fail & Fail\\
S & -0.5 & 50.0 & Fail & Pass & Pass & Fail & Pass & Pass & Fail & Fail & Pass & Fail & Fail & Fail & Fail\\
S & -0.5 & 90.0 & Fail & Pass & Pass & Fail & Pass & Pass & Pass & Fail & Pass & Fail & Fail & Fail & Fail\\
S & 0.0 & 10.0 & Fail & Fail & Pass & Fail & Pass & Pass & Fail & Fail & Pass & Fail & Fail & Fail & Fail\\
S & 0.0 & 50.0 & Fail & Fail & Pass & Fail & Pass & Pass & Fail & Pass & Pass & Fail & Fail & Fail & Fail\\
S & 0.0 & 90.0 & Fail & Pass & Pass & Fail & Pass & Pass & Pass & Fail & Pass & Fail & Fail & Pass & Fail\\
S & 0.5 & 10.0 & Fail & Fail & Pass & Fail & Pass & Pass & Fail & Pass & Pass & Fail & Fail & Fail & Fail\\
S & 0.5 & 50.0 & Fail & Fail & Pass & Fail & Pass & Pass & Pass & Pass & Pass & Pass & Fail & Fail & Fail\\
S & 0.5 & 90.0 & Fail & Fail & Pass & Fail & Pass & Pass & Fail & Pass & Pass & Fail & Fail & Pass & Fail\\
S & 0.94 & 10.0 & Pass & Fail & Pass & Fail & Pass & Pass & Pass & Pass & Pass & Pass & Fail & Pass & Pass\\
S & 0.94 & 50.0 & Fail & Fail & Pass & Fail & Pass & Pass & Pass & Pass & Pass & Pass & Fail & Pass & Fail\\
S & 0.94 & 90.0 & Fail & Fail & Fail & Fail & Pass & Pass & Pass & Fail & Pass & Fail & Fail & Pass & Fail\\
M & -0.94 & 10.0 & Pass & Fail & Fail & Fail & Pass & Pass & Fail & Fail & Pass & Fail & Fail & Fail & Pass\\
M & -0.94 & 50.0 & Pass & Pass & Fail & Fail & Pass & Pass & Fail & Fail & Pass & Fail & Fail & Fail & Fail\\
M & -0.94 & 90.0 & Pass & Pass & Fail & Fail & Pass & Pass & Pass & Fail & Pass & Fail & Fail & Fail & Fail\\
M & -0.5 & 10.0 & Pass & Fail & Fail & Fail & Pass & Pass & Fail & Fail & Pass & Fail & Fail & Pass & Pass\\
M & -0.5 & 50.0 & Pass & Pass & Fail & Fail & Pass & Pass & Fail & Fail & Pass & Fail & Fail & Pass & Fail\\
M & -0.5 & 90.0 & Pass & Fail & Fail & Fail & Pass & Fail & Pass & Fail & Pass & Fail & Fail & Fail & Fail\\
M & 0.0 & 10.0 & Pass & Fail & Fail & Fail & Pass & Pass & Fail & Fail & Pass & Fail & Fail & Fail & Pass\\
M & 0.0 & 50.0 & Pass & Fail & Fail & Fail & Pass & Pass & Fail & Pass & Pass & Fail & Fail & Fail & Pass\\
M & 0.0 & 90.0 & Pass & Pass & Fail & Fail & Pass & Fail & Fail & Fail & Pass & Fail & Fail & Fail & Pass\\
M & 0.5 & 10.0 & Pass & Fail & Fail & Fail & Pass & Pass & Fail & Pass & Pass & Fail & Fail & Fail & Pass\\
M & 0.5 & 50.0 & Pass & Fail & Fail & Fail & Pass & Pass & Pass & Pass & Pass & Pass & Fail & Fail & Fail\\
M & 0.5 & 90.0 & Pass & Fail & Fail & Fail & Pass & Fail & Pass & Fail & Pass & Fail & Fail & Fail & Fail\\
M & 0.94 & 10.0 & Pass & Fail & Fail & Fail & Pass & Pass & Pass & Pass & Pass & Pass & Fail & Fail & Pass\\
M & 0.94 & 50.0 & Pass & Fail & Fail & Fail & Pass & Pass & Pass & Pass & Pass & Pass & Fail & Fail & Fail\\
M & 0.94 & 90.0 & Pass & Fail & Fail & Fail & Pass & Fail & Pass & Fail & Fail & Fail & Fail & Fail & Fail\\
\enddata
\end{deluxetable*}

\begin{longrotatetable}
\startlongtable
\begin{deluxetable*}{cccc|ccc|c|ccccc|c|c|cc|cccccc}
\tabletypesize{\scriptsize}
\tablecaption{Pass/Fail Table, Frankfurt Thermal Models}
\label{tab:frankfurtPF}
\tablehead{ \colhead{M/S}  &  %
\colhead{Spin}  &  %
\colhead{$i$}  &  %
\colhead{$\Rh$}  &  %
\colhead{$F_{86}$}  &  %
\colhead{$\lambda_{maj,86}$}  &  %
\colhead{$F_{2\mu{\rm m}}$}  &  %
\colhead{non-EHT}  &  %
\colhead{$\lambda_{230}$}  &  %
\colhead{Nulls}  &  %
\colhead{Ring D}  &  %
\colhead{Ring W}  &  %
\colhead{Ring A}  &  %
\colhead{EHT}  &  %
\colhead{All}  &  %
\colhead{M$_3$} & %
\colhead{4G$\lambda$} & %
\colhead{$\dot{M}/\dot{M}_{Edd}$}  &  %
\colhead{$P_{out}$(cgs)}  &  %
\colhead{$P_{out}/(\dot{M} c^2)$}}
\startdata
S & -0.94 & 10.0 & 1.0 & Fail & Fail & Pass & Fail & Fail & Pass & Pass & Fail & Pass & Fail & Fail & Pass & Pass &8.9$\times10^{-8}$ & 1.9$\times10^{36}$ & 4.2$\times10^{-3}$\\
S & -0.94 & 10.0 & 10.0 & Pass & Pass & Pass & Pass & Pass & Pass & Fail & Fail & Pass & Fail & Fail & Fail & Fail &1.2$\times10^{-6}$ & 2.6$\times10^{37}$ & 4.2$\times10^{-3}$\\
S & -0.94 & 10.0 & 40.0 & Pass & Pass & Pass & Pass & Pass & Pass & Fail & Fail & Pass & Fail & Fail & Fail & Pass &2.1$\times10^{-6}$ & 4.7$\times10^{37}$ & 4.2$\times10^{-3}$\\
S & -0.94 & 10.0 & 160.0 & Pass & Pass & Pass & Pass & Pass & Pass & Fail & Fail & Pass & Fail & Fail & Fail & Pass &3.2$\times10^{-6}$ & 6.9$\times10^{37}$ & 4.2$\times10^{-3}$\\
S & -0.94 & 30.0 & 1.0 & Fail & Pass & Pass & Fail & Pass & Pass & Pass & Fail & Pass & Fail & Fail & Pass & Pass &8.6$\times10^{-8}$ & 1.9$\times10^{36}$ & 4.2$\times10^{-3}$\\
S & -0.94 & 30.0 & 10.0 & Pass & Pass & Pass & Pass & Pass & Pass & Fail & Fail & Pass & Fail & Fail & Fail & Fail &1.2$\times10^{-6}$ & 2.6$\times10^{37}$ & 4.2$\times10^{-3}$\\
S & -0.94 & 30.0 & 40.0 & Pass & Pass & Pass & Pass & Pass & Pass & Pass & Fail & Pass & Fail & Fail & Fail & Pass &2.2$\times10^{-6}$ & 4.8$\times10^{37}$ & 4.2$\times10^{-3}$\\
S & -0.94 & 30.0 & 160.0 & Pass & Pass & Pass & Pass & Pass & Pass & Fail & Fail & Pass & Fail & Fail & Fail & Fail &3.3$\times10^{-6}$ & 7.3$\times10^{37}$ & 4.2$\times10^{-3}$\\
S & -0.94 & 50.0 & 1.0 & Fail & Pass & Pass & Fail & Pass & Pass & Pass & Fail & Pass & Fail & Fail & Pass & Pass &8.3$\times10^{-8}$ & 1.8$\times10^{36}$ & 4.2$\times10^{-3}$\\
S & -0.94 & 50.0 & 10.0 & Pass & Pass & Fail & Fail & Pass & Pass & Fail & Fail & Pass & Fail & Fail & Fail & Fail &1.1$\times10^{-6}$ & 2.4$\times10^{37}$ & 4.2$\times10^{-3}$\\
S & -0.94 & 50.0 & 40.0 & Pass & Pass & Fail & Fail & Pass & Pass & Fail & Fail & Pass & Fail & Fail & Fail & Fail &2.0$\times10^{-6}$ & 4.4$\times10^{37}$ & 4.2$\times10^{-3}$\\
S & -0.94 & 50.0 & 160.0 & Pass & Pass & Pass & Pass & Pass & Pass & Pass & Fail & Pass & Fail & Fail & Fail & Fail &3.1$\times10^{-6}$ & 6.8$\times10^{37}$ & 4.2$\times10^{-3}$\\
S & -0.94 & 70.0 & 1.0 & Fail & Pass & Pass & Fail & Pass & Pass & Fail & Fail & Pass & Fail & Fail & Fail & Pass &8.3$\times10^{-8}$ & 1.8$\times10^{36}$ & 4.2$\times10^{-3}$\\
S & -0.94 & 70.0 & 10.0 & Pass & Pass & Fail & Fail & Pass & Pass & Pass & Fail & Pass & Fail & Fail & Fail & Fail &1.0$\times10^{-6}$ & 2.3$\times10^{37}$ & 4.2$\times10^{-3}$\\
S & -0.94 & 70.0 & 40.0 & Pass & Pass & Fail & Fail & Pass & Pass & Pass & Fail & Pass & Fail & Fail & Fail & Fail &2.0$\times10^{-6}$ & 4.3$\times10^{37}$ & 4.2$\times10^{-3}$\\
S & -0.94 & 70.0 & 160.0 & Pass & Pass & Pass & Pass & Pass & Pass & Pass & Fail & Pass & Fail & Fail & Fail & Fail &3.2$\times10^{-6}$ & 6.9$\times10^{37}$ & 4.2$\times10^{-3}$\\
S & -0.94 & 90.0 & 1.0 & Fail & Pass & Pass & Fail & Pass & Pass & Fail & Fail & Pass & Fail & Fail & Fail & Pass &8.0$\times10^{-8}$ & 1.7$\times10^{36}$ & 4.2$\times10^{-3}$\\
S & -0.94 & 90.0 & 10.0 & Pass & Pass & Fail & Fail & Pass & Pass & Pass & Fail & Pass & Fail & Fail & Fail & Fail &1.0$\times10^{-6}$ & 2.3$\times10^{37}$ & 4.2$\times10^{-3}$\\
S & -0.94 & 90.0 & 40.0 & Pass & Fail & Fail & Fail & Pass & Pass & Pass & Fail & Pass & Fail & Fail & Fail & Fail &2.0$\times10^{-6}$ & 4.4$\times10^{37}$ & 4.2$\times10^{-3}$\\
S & -0.94 & 90.0 & 160.0 & Pass & Pass & Pass & Pass & Pass & Fail & Pass & Fail & Pass & Fail & Fail & Fail & Fail &3.3$\times10^{-6}$ & 7.2$\times10^{37}$ & 4.2$\times10^{-3}$\\
S & -0.5 & 10.0 & 1.0 & Fail & Fail & Pass & Fail & Fail & Pass & Pass & Fail & Pass & Fail & Fail & Pass & Pass &5.7$\times10^{-8}$ & 3.5$\times10^{35}$ & 1.2$\times10^{-3}$\\
S & -0.5 & 10.0 & 10.0 & Pass & Pass & Pass & Pass & Pass & Pass & Fail & Fail & Pass & Fail & Fail & Fail & Pass &1.7$\times10^{-6}$ & 1.1$\times10^{37}$ & 1.2$\times10^{-3}$\\
S & -0.5 & 10.0 & 40.0 & Pass & Pass & Pass & Pass & Pass & Pass & Fail & Fail & Pass & Fail & Fail & Pass & Pass &3.6$\times10^{-6}$ & 2.2$\times10^{37}$ & 1.2$\times10^{-3}$\\
S & -0.5 & 10.0 & 160.0 & Pass & Fail & Pass & Fail & Pass & Pass & Fail & Fail & Pass & Fail & Fail & Pass & Pass &5.5$\times10^{-6}$ & 3.4$\times10^{37}$ & 1.2$\times10^{-3}$\\
S & -0.5 & 30.0 & 1.0 & Fail & Pass & Pass & Fail & Pass & Pass & Pass & Fail & Pass & Fail & Fail & Pass & Pass &5.6$\times10^{-8}$ & 3.4$\times10^{35}$ & 1.2$\times10^{-3}$\\
S & -0.5 & 30.0 & 10.0 & Pass & Pass & Pass & Pass & Pass & Pass & Fail & Fail & Pass & Fail & Fail & Fail & Pass &1.7$\times10^{-6}$ & 1.1$\times10^{37}$ & 1.2$\times10^{-3}$\\
S & -0.5 & 30.0 & 40.0 & Pass & Pass & Pass & Pass & Pass & Pass & Fail & Pass & Pass & Fail & Fail & Pass & Pass &3.7$\times10^{-6}$ & 2.3$\times10^{37}$ & 1.2$\times10^{-3}$\\
S & -0.5 & 30.0 & 160.0 & Pass & Pass & Pass & Pass & Pass & Pass & Fail & Fail & Pass & Fail & Fail & Pass & Pass &5.7$\times10^{-6}$ & 3.5$\times10^{37}$ & 1.2$\times10^{-3}$\\
S & -0.5 & 50.0 & 1.0 & Fail & Pass & Pass & Fail & Pass & Pass & Pass & Fail & Pass & Fail & Fail & Pass & Pass &5.5$\times10^{-8}$ & 3.4$\times10^{35}$ & 1.2$\times10^{-3}$\\
S & -0.5 & 50.0 & 10.0 & Pass & Pass & Pass & Pass & Pass & Pass & Fail & Fail & Pass & Fail & Fail & Fail & Pass &1.6$\times10^{-6}$ & 1.0$\times10^{37}$ & 1.2$\times10^{-3}$\\
S & -0.5 & 50.0 & 40.0 & Pass & Fail & Pass & Fail & Pass & Pass & Fail & Fail & Pass & Fail & Fail & Pass & Pass &3.5$\times10^{-6}$ & 2.2$\times10^{37}$ & 1.2$\times10^{-3}$\\
S & -0.5 & 50.0 & 160.0 & Pass & Fail & Pass & Fail & Pass & Fail & Fail & Fail & Pass & Fail & Fail & Pass & Pass &5.5$\times10^{-6}$ & 3.4$\times10^{37}$ & 1.2$\times10^{-3}$\\
S & -0.5 & 70.0 & 1.0 & Fail & Fail & Pass & Fail & Pass & Pass & Pass & Pass & Pass & Pass & Fail & Pass & Pass &5.5$\times10^{-8}$ & 3.4$\times10^{35}$ & 1.2$\times10^{-3}$\\
S & -0.5 & 70.0 & 10.0 & Pass & Fail & Pass & Fail & Pass & Pass & Fail & Fail & Pass & Fail & Fail & Fail & Pass &1.7$\times10^{-6}$ & 1.0$\times10^{37}$ & 1.2$\times10^{-3}$\\
S & -0.5 & 70.0 & 40.0 & Pass & Fail & Pass & Fail & Pass & Pass & Pass & Fail & Pass & Fail & Fail & Pass & Pass &3.6$\times10^{-6}$ & 2.2$\times10^{37}$ & 1.2$\times10^{-3}$\\
S & -0.5 & 70.0 & 160.0 & Pass & Fail & Pass & Fail & Pass & Fail & Pass & Fail & Pass & Fail & Fail & Pass & Pass &5.6$\times10^{-6}$ & 3.5$\times10^{37}$ & 1.2$\times10^{-3}$\\
S & -0.5 & 90.0 & 1.0 & Fail & Fail & Pass & Fail & Pass & Pass & Pass & Fail & Pass & Fail & Fail & Fail & Pass &5.4$\times10^{-8}$ & 3.4$\times10^{35}$ & 1.2$\times10^{-3}$\\
S & -0.5 & 90.0 & 10.0 & Pass & Fail & Pass & Fail & Pass & Pass & Fail & Fail & Pass & Fail & Fail & Fail & Pass &1.7$\times10^{-6}$ & 1.0$\times10^{37}$ & 1.2$\times10^{-3}$\\
S & -0.5 & 90.0 & 40.0 & Fail & Fail & Pass & Fail & Pass & Pass & Pass & Fail & Pass & Fail & Fail & Pass & Pass &3.7$\times10^{-6}$ & 2.3$\times10^{37}$ & 1.2$\times10^{-3}$\\
S & -0.5 & 90.0 & 160.0 & Pass & Fail & Pass & Fail & Pass & Pass & Pass & Fail & Pass & Fail & Fail & Pass & Pass &6.0$\times10^{-6}$ & 3.7$\times10^{37}$ & 1.2$\times10^{-3}$\\
S & 0.0 & 10.0 & 1.0 & Fail & Pass & Pass & Fail & Fail & Pass & Pass & Fail & Pass & Fail & Fail & Pass & Pass &3.6$\times10^{-8}$ & 8.5$\times10^{34}$ & 4.5$\times10^{-4}$\\
S & 0.0 & 10.0 & 10.0 & Pass & Fail & Pass & Fail & Pass & Pass & Fail & Fail & Pass & Fail & Fail & Fail & Fail &1.0$\times10^{-6}$ & 2.5$\times10^{36}$ & 4.5$\times10^{-4}$\\
S & 0.0 & 10.0 & 40.0 & Pass & Fail & Pass & Fail & Pass & Pass & Fail & Fail & Pass & Fail & Fail & Pass & Pass &3.4$\times10^{-6}$ & 8.0$\times10^{36}$ & 4.5$\times10^{-4}$\\
S & 0.0 & 10.0 & 160.0 & Pass & Fail & Pass & Fail & Pass & Pass & Fail & Fail & Pass & Fail & Fail & Pass & Pass &5.6$\times10^{-6}$ & 1.3$\times10^{37}$ & 4.5$\times10^{-4}$\\
S & 0.0 & 30.0 & 1.0 & Fail & Pass & Pass & Fail & Pass & Pass & Pass & Fail & Pass & Fail & Fail & Pass & Pass &3.5$\times10^{-8}$ & 8.2$\times10^{34}$ & 4.5$\times10^{-4}$\\
S & 0.0 & 30.0 & 10.0 & Fail & Fail & Pass & Fail & Pass & Pass & Fail & Pass & Pass & Fail & Fail & Fail & Fail &1.0$\times10^{-6}$ & 2.4$\times10^{36}$ & 4.5$\times10^{-4}$\\
S & 0.0 & 30.0 & 40.0 & Pass & Pass & Pass & Pass & Pass & Pass & Fail & Pass & Pass & Fail & Fail & Pass & Pass &3.3$\times10^{-6}$ & 7.6$\times10^{36}$ & 4.5$\times10^{-4}$\\
S & 0.0 & 30.0 & 160.0 & Pass & Fail & Pass & Fail & Pass & Pass & Fail & Fail & Pass & Fail & Fail & Pass & Pass &5.3$\times10^{-6}$ & 1.2$\times10^{37}$ & 4.5$\times10^{-4}$\\
S & 0.0 & 50.0 & 1.0 & Fail & Fail & Pass & Fail & Pass & Fail & Pass & Fail & Pass & Fail & Fail & Pass & Pass &3.4$\times10^{-8}$ & 8.1$\times10^{34}$ & 4.5$\times10^{-4}$\\
S & 0.0 & 50.0 & 10.0 & Fail & Pass & Pass & Fail & Pass & Pass & Pass & Pass & Pass & Pass & Fail & Fail & Pass &1.0$\times10^{-6}$ & 2.5$\times10^{36}$ & 4.5$\times10^{-4}$\\
S & 0.0 & 50.0 & 40.0 & Pass & Fail & Pass & Fail & Pass & Pass & Fail & Fail & Pass & Fail & Fail & Pass & Pass &3.3$\times10^{-6}$ & 7.7$\times10^{36}$ & 4.5$\times10^{-4}$\\
S & 0.0 & 50.0 & 160.0 & Pass & Fail & Pass & Fail & Pass & Pass & Fail & Fail & Pass & Fail & Fail & Pass & Pass &5.3$\times10^{-6}$ & 1.2$\times10^{37}$ & 4.5$\times10^{-4}$\\
S & 0.0 & 70.0 & 1.0 & Fail & Fail & Pass & Fail & Pass & Pass & Pass & Pass & Pass & Pass & Fail & Pass & Pass &3.5$\times10^{-8}$ & 8.1$\times10^{34}$ & 4.5$\times10^{-4}$\\
S & 0.0 & 70.0 & 10.0 & Fail & Fail & Pass & Fail & Pass & Pass & Pass & Fail & Pass & Fail & Fail & Fail & Pass &1.1$\times10^{-6}$ & 2.6$\times10^{36}$ & 4.5$\times10^{-4}$\\
S & 0.0 & 70.0 & 40.0 & Pass & Fail & Pass & Fail & Pass & Pass & Pass & Fail & Pass & Fail & Fail & Pass & Pass &3.5$\times10^{-6}$ & 8.2$\times10^{36}$ & 4.5$\times10^{-4}$\\
S & 0.0 & 70.0 & 160.0 & Fail & Fail & Pass & Fail & Pass & Pass & Pass & Fail & Pass & Fail & Fail & Pass & Pass &5.8$\times10^{-6}$ & 1.4$\times10^{37}$ & 4.5$\times10^{-4}$\\
S & 0.0 & 90.0 & 1.0 & Fail & Fail & Pass & Fail & Pass & Fail & Pass & Fail & Pass & Fail & Fail & Fail & Pass &3.4$\times10^{-8}$ & 8.0$\times10^{34}$ & 4.5$\times10^{-4}$\\
S & 0.0 & 90.0 & 10.0 & Fail & Fail & Pass & Fail & Pass & Pass & Pass & Fail & Pass & Fail & Fail & Fail & Pass &1.2$\times10^{-6}$ & 2.7$\times10^{36}$ & 4.5$\times10^{-4}$\\
S & 0.0 & 90.0 & 40.0 & Fail & Fail & Pass & Fail & Pass & Pass & Pass & Fail & Pass & Fail & Fail & Pass & Pass &3.6$\times10^{-6}$ & 8.5$\times10^{36}$ & 4.5$\times10^{-4}$\\
S & 0.0 & 90.0 & 160.0 & Fail & Fail & Pass & Fail & Pass & Pass & Pass & Fail & Pass & Fail & Fail & Pass & Pass &6.0$\times10^{-6}$ & 1.4$\times10^{37}$ & 4.5$\times10^{-4}$\\
S & 0.5 & 10.0 & 1.0 & Fail & Pass & Pass & Fail & Pass & Pass & Pass & Fail & Pass & Fail & Fail & Pass & Pass &2.0$\times10^{-8}$ & 4.4$\times10^{35}$ & 4.2$\times10^{-3}$\\
S & 0.5 & 10.0 & 10.0 & Pass & Fail & Pass & Fail & Pass & Pass & Pass & Pass & Pass & Pass & Fail & Fail & Fail &2.9$\times10^{-7}$ & 6.1$\times10^{36}$ & 4.2$\times10^{-3}$\\
S & 0.5 & 10.0 & 40.0 & Pass & Pass & Pass & Pass & Pass & Pass & Pass & Pass & Pass & Pass & Pass & Fail & Pass &7.8$\times10^{-7}$ & 1.7$\times10^{37}$ & 4.2$\times10^{-3}$\\
S & 0.5 & 10.0 & 160.0 & Pass & Pass & Pass & Pass & Pass & Pass & Pass & Pass & Pass & Pass & Pass & Fail & Pass &1.5$\times10^{-6}$ & 3.2$\times10^{37}$ & 4.2$\times10^{-3}$\\
S & 0.5 & 30.0 & 1.0 & Fail & Fail & Pass & Fail & Pass & Pass & Pass & Fail & Pass & Fail & Fail & Pass & Pass &2.0$\times10^{-8}$ & 4.3$\times10^{35}$ & 4.2$\times10^{-3}$\\
S & 0.5 & 30.0 & 10.0 & Pass & Fail & Pass & Fail & Pass & Pass & Pass & Pass & Pass & Pass & Fail & Fail & Fail &2.7$\times10^{-7}$ & 5.9$\times10^{36}$ & 4.2$\times10^{-3}$\\
S & 0.5 & 30.0 & 40.0 & Pass & Pass & Pass & Pass & Pass & Pass & Pass & Pass & Pass & Pass & Pass & Fail & Pass &7.6$\times10^{-7}$ & 1.6$\times10^{37}$ & 4.2$\times10^{-3}$\\
S & 0.5 & 30.0 & 160.0 & Pass & Pass & Pass & Pass & Pass & Pass & Pass & Pass & Pass & Pass & Pass & Fail & Pass &1.5$\times10^{-6}$ & 3.2$\times10^{37}$ & 4.2$\times10^{-3}$\\
S & 0.5 & 50.0 & 1.0 & Fail & Fail & Pass & Fail & Pass & Pass & Pass & Fail & Pass & Fail & Fail & Pass & Pass &1.9$\times10^{-8}$ & 4.2$\times10^{35}$ & 4.2$\times10^{-3}$\\
S & 0.5 & 50.0 & 10.0 & Fail & Fail & Pass & Fail & Pass & Pass & Pass & Pass & Pass & Pass & Fail & Fail & Fail &2.6$\times10^{-7}$ & 5.6$\times10^{36}$ & 4.2$\times10^{-3}$\\
S & 0.5 & 50.0 & 40.0 & Pass & Pass & Pass & Pass & Pass & Pass & Pass & Fail & Pass & Fail & Fail & Fail & Fail &7.4$\times10^{-7}$ & 1.6$\times10^{37}$ & 4.2$\times10^{-3}$\\
S & 0.5 & 50.0 & 160.0 & Pass & Fail & Pass & Fail & Pass & Pass & Pass & Fail & Pass & Fail & Fail & Fail & Fail &1.5$\times10^{-6}$ & 3.2$\times10^{37}$ & 4.2$\times10^{-3}$\\
S & 0.5 & 70.0 & 1.0 & Fail & Fail & Pass & Fail & Pass & Pass & Pass & Pass & Pass & Pass & Fail & Pass & Pass &1.9$\times10^{-8}$ & 4.1$\times10^{35}$ & 4.2$\times10^{-3}$\\
S & 0.5 & 70.0 & 10.0 & Fail & Fail & Pass & Fail & Pass & Pass & Pass & Fail & Pass & Fail & Fail & Fail & Fail &2.6$\times10^{-7}$ & 5.6$\times10^{36}$ & 4.2$\times10^{-3}$\\
S & 0.5 & 70.0 & 40.0 & Pass & Pass & Pass & Pass & Pass & Pass & Pass & Fail & Pass & Fail & Fail & Fail & Fail &7.7$\times10^{-7}$ & 1.7$\times10^{37}$ & 4.2$\times10^{-3}$\\
S & 0.5 & 70.0 & 160.0 & Pass & Fail & Pass & Fail & Pass & Pass & Pass & Fail & Pass & Fail & Fail & Fail & Fail &1.6$\times10^{-6}$ & 3.5$\times10^{37}$ & 4.2$\times10^{-3}$\\
S & 0.5 & 90.0 & 1.0 & Fail & Fail & Pass & Fail & Pass & Fail & Pass & Fail & Pass & Fail & Fail & Pass & Pass &1.9$\times10^{-8}$ & 4.0$\times10^{35}$ & 4.2$\times10^{-3}$\\
S & 0.5 & 90.0 & 10.0 & Fail & Fail & Pass & Fail & Pass & Fail & Pass & Fail & Pass & Fail & Fail & Fail & Fail &2.6$\times10^{-7}$ & 5.6$\times10^{36}$ & 4.2$\times10^{-3}$\\
S & 0.5 & 90.0 & 40.0 & Fail & Pass & Pass & Fail & Pass & Pass & Pass & Fail & Pass & Fail & Fail & Fail & Fail &7.9$\times10^{-7}$ & 1.7$\times10^{37}$ & 4.2$\times10^{-3}$\\
S & 0.5 & 90.0 & 160.0 & Pass & Fail & Pass & Fail & Pass & Fail & Pass & Fail & Pass & Fail & Fail & Fail & Fail &1.7$\times10^{-6}$ & 3.6$\times10^{37}$ & 4.2$\times10^{-3}$\\
S & 0.94 & 10.0 & 1.0 & Fail & Fail & Pass & Fail & Pass & Pass & Fail & Fail & Pass & Fail & Fail & Pass & Pass &7.2$\times10^{-9}$ & 4.4$\times10^{35}$ & 1.2$\times10^{-2}$\\
S & 0.94 & 10.0 & 10.0 & Pass & Pass & Pass & Pass & Pass & Pass & Fail & Pass & Pass & Fail & Fail & Fail & Pass &1.1$\times10^{-7}$ & 6.6$\times10^{36}$ & 1.2$\times10^{-2}$\\
S & 0.94 & 10.0 & 40.0 & Pass & Pass & Pass & Pass & Pass & Pass & Fail & Fail & Pass & Fail & Fail & Fail & Pass &5.5$\times10^{-7}$ & 3.3$\times10^{37}$ & 1.2$\times10^{-2}$\\
S & 0.94 & 10.0 & 160.0 & Pass & Pass & Pass & Pass & Pass & Pass & Pass & Fail & Pass & Fail & Fail & Fail & Pass &9.5$\times10^{-7}$ & 5.7$\times10^{37}$ & 1.2$\times10^{-2}$\\
S & 0.94 & 30.0 & 1.0 & Fail & Fail & Pass & Fail & Pass & Pass & Pass & Fail & Pass & Fail & Fail & Pass & Pass &7.1$\times10^{-9}$ & 4.2$\times10^{35}$ & 1.2$\times10^{-2}$\\
S & 0.94 & 30.0 & 10.0 & Fail & Pass & Pass & Fail & Pass & Pass & Pass & Pass & Pass & Pass & Fail & Fail & Pass &1.1$\times10^{-7}$ & 6.5$\times10^{36}$ & 1.2$\times10^{-2}$\\
S & 0.94 & 30.0 & 40.0 & Pass & Fail & Fail & Fail & Pass & Pass & Pass & Fail & Pass & Fail & Fail & Fail & Fail &6.4$\times10^{-7}$ & 3.9$\times10^{37}$ & 1.2$\times10^{-2}$\\
S & 0.94 & 30.0 & 160.0 & Pass & Fail & Pass & Fail & Pass & Pass & Pass & Fail & Pass & Fail & Fail & Fail & Fail &1.1$\times10^{-6}$ & 6.4$\times10^{37}$ & 1.2$\times10^{-2}$\\
S & 0.94 & 50.0 & 1.0 & Fail & Fail & Pass & Fail & Pass & Pass & Pass & Pass & Pass & Pass & Fail & Pass & Pass &6.8$\times10^{-9}$ & 4.1$\times10^{35}$ & 1.2$\times10^{-2}$\\
S & 0.94 & 50.0 & 10.0 & Fail & Pass & Fail & Fail & Pass & Fail & Pass & Pass & Fail & Fail & Fail & Pass & Pass &1.1$\times10^{-7}$ & 6.4$\times10^{36}$ & 1.2$\times10^{-2}$\\
S & 0.94 & 50.0 & 40.0 & Pass & Fail & Fail & Fail & Pass & Pass & Pass & Fail & Pass & Fail & Fail & Fail & Fail &7.2$\times10^{-7}$ & 4.3$\times10^{37}$ & 1.2$\times10^{-2}$\\
S & 0.94 & 50.0 & 160.0 & Pass & Fail & Fail & Fail & Pass & Pass & Pass & Fail & Pass & Fail & Fail & Fail & Fail &1.2$\times10^{-6}$ & 7.3$\times10^{37}$ & 1.2$\times10^{-2}$\\
S & 0.94 & 70.0 & 1.0 & Fail & Fail & Pass & Fail & Pass & Fail & Pass & Pass & Pass & Fail & Fail & Pass & Pass &6.7$\times10^{-9}$ & 4.0$\times10^{35}$ & 1.2$\times10^{-2}$\\
S & 0.94 & 70.0 & 10.0 & Fail & Pass & Fail & Fail & Pass & Fail & Pass & Fail & Pass & Fail & Fail & Pass & Fail &1.1$\times10^{-7}$ & 6.9$\times10^{36}$ & 1.2$\times10^{-2}$\\
S & 0.94 & 70.0 & 40.0 & Pass & Fail & Fail & Fail & Pass & Pass & Pass & Fail & Pass & Fail & Fail & Fail & Fail &7.3$\times10^{-7}$ & 4.4$\times10^{37}$ & 1.2$\times10^{-2}$\\
S & 0.94 & 70.0 & 160.0 & Pass & Fail & Fail & Fail & Pass & Fail & Pass & Pass & Pass & Fail & Fail & Fail & Fail &1.3$\times10^{-6}$ & 8.1$\times10^{37}$ & 1.2$\times10^{-2}$\\
S & 0.94 & 90.0 & 1.0 & Fail & Fail & Fail & Fail & Pass & Fail & Pass & Pass & Pass & Fail & Fail & Pass & Pass &6.7$\times10^{-9}$ & 4.0$\times10^{35}$ & 1.2$\times10^{-2}$\\
S & 0.94 & 90.0 & 10.0 & Fail & Pass & Fail & Fail & Pass & Fail & Pass & Pass & Pass & Fail & Fail & Pass & Fail &1.2$\times10^{-7}$ & 7.2$\times10^{36}$ & 1.2$\times10^{-2}$\\
S & 0.94 & 90.0 & 40.0 & Pass & Fail & Fail & Fail & Pass & Fail & Pass & Fail & Pass & Fail & Fail & Fail & Fail &7.1$\times10^{-7}$ & 4.3$\times10^{37}$ & 1.2$\times10^{-2}$\\
S & 0.94 & 90.0 & 160.0 & Pass & Fail & Fail & Fail & Pass & Fail & Pass & Pass & Pass & Fail & Fail & Fail & Fail &1.4$\times10^{-6}$ & 8.2$\times10^{37}$ & 1.2$\times10^{-2}$\\
M & -0.94 & 10.0 & 1.0 & Fail & Pass & Fail & Fail & Pass & Pass & Fail & Fail & Pass & Fail & Fail & Fail & Pass &4.3$\times10^{-8}$ & 4.1$\times10^{37}$ & 1.8$\times10^{-1}$\\
M & -0.94 & 10.0 & 10.0 & Pass & Pass & Fail & Fail & Pass & Pass & Pass & Fail & Pass & Fail & Fail & Fail & Pass &1.1$\times10^{-7}$ & 1.1$\times10^{38}$ & 1.8$\times10^{-1}$\\
M & -0.94 & 10.0 & 40.0 & Pass & Pass & Fail & Fail & Pass & Pass & Pass & Fail & Pass & Fail & Fail & Fail & Fail &1.9$\times10^{-7}$ & 1.8$\times10^{38}$ & 1.8$\times10^{-1}$\\
M & -0.94 & 10.0 & 160.0 & Pass & Pass & Pass & Pass & Pass & Pass & Pass & Fail & Pass & Fail & Fail & Fail & Fail &3.5$\times10^{-7}$ & 3.4$\times10^{38}$ & 1.8$\times10^{-1}$\\
M & -0.94 & 30.0 & 1.0 & Fail & Pass & Fail & Fail & Pass & Pass & Fail & Fail & Pass & Fail & Fail & Fail & Pass &4.2$\times10^{-8}$ & 4.0$\times10^{37}$ & 1.8$\times10^{-1}$\\
M & -0.94 & 30.0 & 10.0 & Pass & Pass & Fail & Fail & Pass & Pass & Pass & Fail & Pass & Fail & Fail & Fail & Fail &1.1$\times10^{-7}$ & 1.1$\times10^{38}$ & 1.8$\times10^{-1}$\\
M & -0.94 & 30.0 & 40.0 & Pass & Pass & Fail & Fail & Pass & Pass & Pass & Fail & Pass & Fail & Fail & Fail & Fail &1.8$\times10^{-7}$ & 1.7$\times10^{38}$ & 1.8$\times10^{-1}$\\
M & -0.94 & 30.0 & 160.0 & Pass & Pass & Pass & Pass & Pass & Fail & Pass & Fail & Pass & Fail & Fail & Fail & Fail &3.4$\times10^{-7}$ & 3.2$\times10^{38}$ & 1.8$\times10^{-1}$\\
M & -0.94 & 50.0 & 1.0 & Fail & Pass & Fail & Fail & Pass & Pass & Fail & Fail & Pass & Fail & Fail & Fail & Pass &4.0$\times10^{-8}$ & 3.8$\times10^{37}$ & 1.8$\times10^{-1}$\\
M & -0.94 & 50.0 & 10.0 & Pass & Pass & Fail & Fail & Pass & Pass & Pass & Fail & Pass & Fail & Fail & Fail & Fail &1.0$\times10^{-7}$ & 9.7$\times10^{37}$ & 1.8$\times10^{-1}$\\
M & -0.94 & 50.0 & 40.0 & Pass & Pass & Fail & Fail & Pass & Pass & Pass & Fail & Pass & Fail & Fail & Fail & Fail &1.7$\times10^{-7}$ & 1.6$\times10^{38}$ & 1.8$\times10^{-1}$\\
M & -0.94 & 50.0 & 160.0 & Pass & Pass & Pass & Pass & Pass & Pass & Pass & Fail & Pass & Fail & Fail & Fail & Fail &3.1$\times10^{-7}$ & 3.0$\times10^{38}$ & 1.8$\times10^{-1}$\\
M & -0.94 & 70.0 & 1.0 & Fail & Pass & Fail & Fail & Pass & Pass & Fail & Fail & Pass & Fail & Fail & Fail & Pass &3.7$\times10^{-8}$ & 3.5$\times10^{37}$ & 1.8$\times10^{-1}$\\
M & -0.94 & 70.0 & 10.0 & Pass & Pass & Fail & Fail & Pass & Pass & Pass & Fail & Pass & Fail & Fail & Fail & Fail &9.2$\times10^{-8}$ & 8.7$\times10^{37}$ & 1.8$\times10^{-1}$\\
M & -0.94 & 70.0 & 40.0 & Pass & Pass & Fail & Fail & Pass & Pass & Pass & Fail & Pass & Fail & Fail & Fail & Fail &1.5$\times10^{-7}$ & 1.4$\times10^{38}$ & 1.8$\times10^{-1}$\\
M & -0.94 & 70.0 & 160.0 & Pass & Pass & Pass & Pass & Pass & Pass & Pass & Fail & Pass & Fail & Fail & Fail & Fail &2.8$\times10^{-7}$ & 2.7$\times10^{38}$ & 1.8$\times10^{-1}$\\
M & -0.94 & 90.0 & 1.0 & Fail & Pass & Fail & Fail & Pass & Pass & Fail & Fail & Pass & Fail & Fail & Fail & Pass &3.5$\times10^{-8}$ & 3.3$\times10^{37}$ & 1.8$\times10^{-1}$\\
M & -0.94 & 90.0 & 10.0 & Pass & Pass & Fail & Fail & Pass & Fail & Pass & Fail & Pass & Fail & Fail & Fail & Fail &8.7$\times10^{-8}$ & 8.3$\times10^{37}$ & 1.8$\times10^{-1}$\\
M & -0.94 & 90.0 & 40.0 & Pass & Pass & Fail & Fail & Pass & Pass & Pass & Fail & Pass & Fail & Fail & Fail & Fail &1.4$\times10^{-7}$ & 1.4$\times10^{38}$ & 1.8$\times10^{-1}$\\
M & -0.94 & 90.0 & 160.0 & Pass & Pass & Pass & Pass & Pass & Pass & Pass & Fail & Pass & Fail & Fail & Fail & Fail &2.7$\times10^{-7}$ & 2.6$\times10^{38}$ & 1.8$\times10^{-1}$\\
M & -0.5 & 10.0 & 1.0 & Fail & Pass & Fail & Fail & Pass & Pass & Fail & Fail & Pass & Fail & Fail & Pass & Pass &3.8$\times10^{-8}$ & 9.1$\times10^{36}$ & 4.6$\times10^{-2}$\\
M & -0.5 & 10.0 & 10.0 & Fail & Pass & Fail & Fail & Pass & Pass & Fail & Fail & Pass & Fail & Fail & Fail & Pass &1.3$\times10^{-7}$ & 3.0$\times10^{37}$ & 4.6$\times10^{-2}$\\
M & -0.5 & 10.0 & 40.0 & Pass & Pass & Fail & Fail & Pass & Pass & Fail & Fail & Pass & Fail & Fail & Fail & Pass &2.1$\times10^{-7}$ & 5.0$\times10^{37}$ & 4.6$\times10^{-2}$\\
M & -0.5 & 10.0 & 160.0 & Pass & Fail & Pass & Fail & Pass & Pass & Fail & Fail & Pass & Fail & Fail & Fail & Fail &3.8$\times10^{-7}$ & 8.9$\times10^{37}$ & 4.6$\times10^{-2}$\\
M & -0.5 & 30.0 & 1.0 & Fail & Pass & Fail & Fail & Pass & Pass & Fail & Fail & Pass & Fail & Fail & Pass & Pass &3.7$\times10^{-8}$ & 8.8$\times10^{36}$ & 4.6$\times10^{-2}$\\
M & -0.5 & 30.0 & 10.0 & Fail & Pass & Fail & Fail & Pass & Pass & Fail & Fail & Pass & Fail & Fail & Fail & Pass &1.2$\times10^{-7}$ & 2.9$\times10^{37}$ & 4.6$\times10^{-2}$\\
M & -0.5 & 30.0 & 40.0 & Pass & Pass & Fail & Fail & Pass & Pass & Fail & Fail & Pass & Fail & Fail & Fail & Pass &2.0$\times10^{-7}$ & 4.8$\times10^{37}$ & 4.6$\times10^{-2}$\\
M & -0.5 & 30.0 & 160.0 & Pass & Pass & Pass & Pass & Pass & Pass & Fail & Fail & Pass & Fail & Fail & Fail & Fail &3.6$\times10^{-7}$ & 8.6$\times10^{37}$ & 4.6$\times10^{-2}$\\
M & -0.5 & 50.0 & 1.0 & Fail & Pass & Fail & Fail & Pass & Pass & Fail & Fail & Pass & Fail & Fail & Pass & Pass &3.6$\times10^{-8}$ & 8.4$\times10^{36}$ & 4.6$\times10^{-2}$\\
M & -0.5 & 50.0 & 10.0 & Pass & Pass & Fail & Fail & Pass & Pass & Pass & Fail & Pass & Fail & Fail & Fail & Pass &1.1$\times10^{-7}$ & 2.7$\times10^{37}$ & 4.6$\times10^{-2}$\\
M & -0.5 & 50.0 & 40.0 & Pass & Pass & Fail & Fail & Pass & Pass & Pass & Fail & Pass & Fail & Fail & Fail & Pass &1.9$\times10^{-7}$ & 4.4$\times10^{37}$ & 4.6$\times10^{-2}$\\
M & -0.5 & 50.0 & 160.0 & Pass & Pass & Pass & Pass & Pass & Pass & Fail & Fail & Pass & Fail & Fail & Fail & Fail &3.4$\times10^{-7}$ & 8.0$\times10^{37}$ & 4.6$\times10^{-2}$\\
M & -0.5 & 70.0 & 1.0 & Fail & Fail & Fail & Fail & Pass & Pass & Fail & Fail & Pass & Fail & Fail & Fail & Pass &3.4$\times10^{-8}$ & 8.0$\times10^{36}$ & 4.6$\times10^{-2}$\\
M & -0.5 & 70.0 & 10.0 & Pass & Pass & Fail & Fail & Pass & Pass & Fail & Fail & Pass & Fail & Fail & Fail & Pass &1.0$\times10^{-7}$ & 2.4$\times10^{37}$ & 4.6$\times10^{-2}$\\
M & -0.5 & 70.0 & 40.0 & Pass & Pass & Fail & Fail & Pass & Pass & Fail & Fail & Pass & Fail & Fail & Fail & Pass &1.7$\times10^{-7}$ & 3.9$\times10^{37}$ & 4.6$\times10^{-2}$\\
M & -0.5 & 70.0 & 160.0 & Pass & Pass & Pass & Pass & Pass & Pass & Fail & Fail & Pass & Fail & Fail & Fail & Fail &3.0$\times10^{-7}$ & 7.2$\times10^{37}$ & 4.6$\times10^{-2}$\\
M & -0.5 & 90.0 & 1.0 & Fail & Fail & Fail & Fail & Pass & Fail & Fail & Fail & Pass & Fail & Fail & Fail & Pass &3.3$\times10^{-8}$ & 7.8$\times10^{36}$ & 4.6$\times10^{-2}$\\
M & -0.5 & 90.0 & 10.0 & Pass & Pass & Fail & Fail & Pass & Pass & Fail & Fail & Pass & Fail & Fail & Fail & Pass &10.0$\times10^{-8}$ & 2.4$\times10^{37}$ & 4.6$\times10^{-2}$\\
M & -0.5 & 90.0 & 40.0 & Pass & Pass & Fail & Fail & Pass & Pass & Fail & Fail & Pass & Fail & Fail & Fail & Pass &1.7$\times10^{-7}$ & 3.9$\times10^{37}$ & 4.6$\times10^{-2}$\\
M & -0.5 & 90.0 & 160.0 & Pass & Pass & Pass & Pass & Pass & Pass & Fail & Fail & Pass & Fail & Fail & Fail & Fail &3.0$\times10^{-7}$ & 7.2$\times10^{37}$ & 4.6$\times10^{-2}$\\
M & 0.0 & 10.0 & 1.0 & Fail & Pass & Pass & Fail & Pass & Pass & Fail & Fail & Pass & Fail & Fail & Pass & Pass &2.4$\times10^{-8}$ & 1.7$\times10^{36}$ & 1.4$\times10^{-2}$\\
M & 0.0 & 10.0 & 10.0 & Pass & Fail & Fail & Fail & Pass & Pass & Fail & Fail & Pass & Fail & Fail & Fail & Pass &7.6$\times10^{-8}$ & 5.4$\times10^{36}$ & 1.4$\times10^{-2}$\\
M & 0.0 & 10.0 & 40.0 & Pass & Fail & Pass & Fail & Pass & Pass & Fail & Fail & Pass & Fail & Fail & Fail & Pass &1.2$\times10^{-7}$ & 8.5$\times10^{36}$ & 1.4$\times10^{-2}$\\
M & 0.0 & 10.0 & 160.0 & Pass & Fail & Pass & Fail & Pass & Pass & Fail & Fail & Pass & Fail & Fail & Pass & Pass &2.0$\times10^{-7}$ & 1.4$\times10^{37}$ & 1.4$\times10^{-2}$\\
M & 0.0 & 30.0 & 1.0 & Fail & Pass & Pass & Fail & Pass & Pass & Fail & Fail & Pass & Fail & Fail & Pass & Pass &2.4$\times10^{-8}$ & 1.7$\times10^{36}$ & 1.4$\times10^{-2}$\\
M & 0.0 & 30.0 & 10.0 & Pass & Pass & Fail & Fail & Pass & Pass & Fail & Fail & Pass & Fail & Fail & Fail & Pass &7.3$\times10^{-8}$ & 5.2$\times10^{36}$ & 1.4$\times10^{-2}$\\
M & 0.0 & 30.0 & 40.0 & Pass & Pass & Pass & Pass & Pass & Pass & Fail & Fail & Pass & Fail & Fail & Fail & Pass &1.2$\times10^{-7}$ & 8.2$\times10^{36}$ & 1.4$\times10^{-2}$\\
M & 0.0 & 30.0 & 160.0 & Pass & Pass & Pass & Pass & Pass & Pass & Fail & Fail & Pass & Fail & Fail & Fail & Pass &1.9$\times10^{-7}$ & 1.4$\times10^{37}$ & 1.4$\times10^{-2}$\\
M & 0.0 & 50.0 & 1.0 & Fail & Fail & Fail & Fail & Pass & Pass & Fail & Fail & Pass & Fail & Fail & Pass & Pass &2.3$\times10^{-8}$ & 1.6$\times10^{36}$ & 1.4$\times10^{-2}$\\
M & 0.0 & 50.0 & 10.0 & Pass & Fail & Fail & Fail & Pass & Pass & Pass & Fail & Pass & Fail & Fail & Fail & Pass &6.8$\times10^{-8}$ & 4.8$\times10^{36}$ & 1.4$\times10^{-2}$\\
M & 0.0 & 50.0 & 40.0 & Pass & Fail & Pass & Fail & Pass & Pass & Pass & Fail & Pass & Fail & Fail & Fail & Pass &1.1$\times10^{-7}$ & 7.6$\times10^{36}$ & 1.4$\times10^{-2}$\\
M & 0.0 & 50.0 & 160.0 & Pass & Pass & Pass & Pass & Pass & Pass & Fail & Fail & Pass & Fail & Fail & Fail & Pass &1.8$\times10^{-7}$ & 1.3$\times10^{37}$ & 1.4$\times10^{-2}$\\
M & 0.0 & 70.0 & 1.0 & Fail & Fail & Fail & Fail & Pass & Pass & Pass & Fail & Pass & Fail & Fail & Pass & Pass &2.2$\times10^{-8}$ & 1.6$\times10^{36}$ & 1.4$\times10^{-2}$\\
M & 0.0 & 70.0 & 10.0 & Pass & Fail & Fail & Fail & Pass & Pass & Fail & Fail & Pass & Fail & Fail & Fail & Pass &6.4$\times10^{-8}$ & 4.5$\times10^{36}$ & 1.4$\times10^{-2}$\\
M & 0.0 & 70.0 & 40.0 & Pass & Fail & Fail & Fail & Pass & Pass & Fail & Fail & Pass & Fail & Fail & Fail & Pass &9.9$\times10^{-8}$ & 7.1$\times10^{36}$ & 1.4$\times10^{-2}$\\
M & 0.0 & 70.0 & 160.0 & Pass & Fail & Pass & Fail & Pass & Pass & Fail & Fail & Pass & Fail & Fail & Fail & Pass &1.6$\times10^{-7}$ & 1.2$\times10^{37}$ & 1.4$\times10^{-2}$\\
M & 0.0 & 90.0 & 1.0 & Pass & Fail & Fail & Fail & Pass & Fail & Pass & Fail & Pass & Fail & Fail & Fail & Pass &2.1$\times10^{-8}$ & 1.5$\times10^{36}$ & 1.4$\times10^{-2}$\\
M & 0.0 & 90.0 & 10.0 & Pass & Fail & Fail & Fail & Pass & Fail & Pass & Fail & Pass & Fail & Fail & Fail & Pass &6.1$\times10^{-8}$ & 4.4$\times10^{36}$ & 1.4$\times10^{-2}$\\
M & 0.0 & 90.0 & 40.0 & Pass & Fail & Fail & Fail & Pass & Pass & Pass & Fail & Pass & Fail & Fail & Fail & Pass &9.7$\times10^{-8}$ & 6.9$\times10^{36}$ & 1.4$\times10^{-2}$\\
M & 0.0 & 90.0 & 160.0 & Pass & Fail & Pass & Fail & Pass & Pass & Fail & Fail & Pass & Fail & Fail & Fail & Pass &1.6$\times10^{-7}$ & 1.2$\times10^{37}$ & 1.4$\times10^{-2}$\\
M & 0.5 & 10.0 & 1.0 & Fail & Pass & Fail & Fail & Pass & Pass & Fail & Fail & Pass & Fail & Fail & Pass & Pass &1.5$\times10^{-8}$ & 1.3$\times10^{37}$ & 1.6$\times10^{-1}$\\
M & 0.5 & 10.0 & 10.0 & Fail & Pass & Fail & Fail & Pass & Pass & Pass & Fail & Pass & Fail & Fail & Fail & Pass &4.7$\times10^{-8}$ & 3.9$\times10^{37}$ & 1.6$\times10^{-1}$\\
M & 0.5 & 10.0 & 40.0 & Fail & Pass & Pass & Fail & Pass & Pass & Pass & Pass & Pass & Pass & Fail & Fail & Pass &8.0$\times10^{-8}$ & 6.6$\times10^{37}$ & 1.6$\times10^{-1}$\\
M & 0.5 & 10.0 & 160.0 & Fail & Pass & Pass & Fail & Pass & Pass & Pass & Fail & Pass & Fail & Fail & Fail & Pass &1.4$\times10^{-7}$ & 1.1$\times10^{38}$ & 1.6$\times10^{-1}$\\
M & 0.5 & 30.0 & 1.0 & Fail & Fail & Fail & Fail & Pass & Pass & Fail & Fail & Pass & Fail & Fail & Fail & Pass &1.5$\times10^{-8}$ & 1.2$\times10^{37}$ & 1.6$\times10^{-1}$\\
M & 0.5 & 30.0 & 10.0 & Pass & Pass & Fail & Fail & Pass & Pass & Pass & Fail & Pass & Fail & Fail & Fail & Pass &4.5$\times10^{-8}$ & 3.7$\times10^{37}$ & 1.6$\times10^{-1}$\\
M & 0.5 & 30.0 & 40.0 & Pass & Pass & Fail & Fail & Pass & Pass & Pass & Pass & Pass & Pass & Fail & Fail & Pass &7.7$\times10^{-8}$ & 6.3$\times10^{37}$ & 1.6$\times10^{-1}$\\
M & 0.5 & 30.0 & 160.0 & Pass & Pass & Pass & Pass & Pass & Pass & Pass & Pass & Pass & Pass & Pass & Fail & Pass &1.3$\times10^{-7}$ & 1.1$\times10^{38}$ & 1.6$\times10^{-1}$\\
M & 0.5 & 50.0 & 1.0 & Fail & Fail & Fail & Fail & Pass & Pass & Pass & Pass & Pass & Pass & Fail & Fail & Pass &1.5$\times10^{-8}$ & 1.2$\times10^{37}$ & 1.6$\times10^{-1}$\\
M & 0.5 & 50.0 & 10.0 & Pass & Pass & Fail & Fail & Pass & Pass & Pass & Pass & Pass & Pass & Fail & Fail & Pass &4.3$\times10^{-8}$ & 3.5$\times10^{37}$ & 1.6$\times10^{-1}$\\
M & 0.5 & 50.0 & 40.0 & Pass & Pass & Fail & Fail & Pass & Pass & Pass & Pass & Pass & Pass & Fail & Fail & Pass &7.4$\times10^{-8}$ & 6.1$\times10^{37}$ & 1.6$\times10^{-1}$\\
M & 0.5 & 50.0 & 160.0 & Pass & Pass & Pass & Pass & Pass & Pass & Pass & Pass & Pass & Pass & Pass & Fail & Pass &1.3$\times10^{-7}$ & 1.1$\times10^{38}$ & 1.6$\times10^{-1}$\\
M & 0.5 & 70.0 & 1.0 & Fail & Fail & Fail & Fail & Pass & Pass & Pass & Pass & Pass & Pass & Fail & Fail & Pass &1.4$\times10^{-8}$ & 1.2$\times10^{37}$ & 1.6$\times10^{-1}$\\
M & 0.5 & 70.0 & 10.0 & Pass & Pass & Fail & Fail & Pass & Fail & Pass & Fail & Fail & Fail & Fail & Fail & Pass &4.1$\times10^{-8}$ & 3.3$\times10^{37}$ & 1.6$\times10^{-1}$\\
M & 0.5 & 70.0 & 40.0 & Pass & Pass & Fail & Fail & Pass & Fail & Pass & Fail & Fail & Fail & Fail & Fail & Pass &7.1$\times10^{-8}$ & 5.8$\times10^{37}$ & 1.6$\times10^{-1}$\\
M & 0.5 & 70.0 & 160.0 & Pass & Pass & Fail & Fail & Pass & Pass & Pass & Fail & Pass & Fail & Fail & Fail & Pass &1.3$\times10^{-7}$ & 1.0$\times10^{38}$ & 1.6$\times10^{-1}$\\
M & 0.5 & 90.0 & 1.0 & Fail & Fail & Fail & Fail & Pass & Fail & Pass & Fail & Pass & Fail & Fail & Fail & Pass &1.3$\times10^{-8}$ & 1.1$\times10^{37}$ & 1.6$\times10^{-1}$\\
M & 0.5 & 90.0 & 10.0 & Pass & Pass & Fail & Fail & Pass & Fail & Pass & Fail & Fail & Fail & Fail & Fail & Pass &3.9$\times10^{-8}$ & 3.2$\times10^{37}$ & 1.6$\times10^{-1}$\\
M & 0.5 & 90.0 & 40.0 & Pass & Pass & Fail & Fail & Pass & Fail & Pass & Fail & Pass & Fail & Fail & Fail & Fail &6.9$\times10^{-8}$ & 5.7$\times10^{37}$ & 1.6$\times10^{-1}$\\
M & 0.5 & 90.0 & 160.0 & Pass & Pass & Fail & Fail & Pass & Pass & Pass & Fail & Pass & Fail & Fail & Fail & Fail &1.3$\times10^{-7}$ & 1.0$\times10^{38}$ & 1.6$\times10^{-1}$\\
M & 0.94 & 10.0 & 1.0 & Pass & Fail & Fail & Fail & Pass & Pass & Fail & Fail & Pass & Fail & Fail & Fail & Pass &8.5$\times10^{-9}$ & 4.1$\times10^{37}$ & 9.3$\times10^{-1}$\\
M & 0.94 & 10.0 & 10.0 & Pass & Pass & Fail & Fail & Pass & Pass & Pass & Pass & Pass & Pass & Fail & Fail & Pass &2.3$\times10^{-8}$ & 1.1$\times10^{38}$ & 9.3$\times10^{-1}$\\
M & 0.94 & 10.0 & 40.0 & Pass & Pass & Fail & Fail & Pass & Pass & Pass & Pass & Pass & Pass & Fail & Fail & Pass &3.8$\times10^{-8}$ & 1.8$\times10^{38}$ & 9.3$\times10^{-1}$\\
M & 0.94 & 10.0 & 160.0 & Pass & Pass & Pass & Pass & Pass & Pass & Pass & Pass & Pass & Pass & Pass & Fail & Pass &6.6$\times10^{-8}$ & 3.2$\times10^{38}$ & 9.3$\times10^{-1}$\\
M & 0.94 & 30.0 & 1.0 & Pass & Fail & Fail & Fail & Pass & Pass & Pass & Fail & Pass & Fail & Fail & Fail & Pass &8.4$\times10^{-9}$ & 4.0$\times10^{37}$ & 9.3$\times10^{-1}$\\
M & 0.94 & 30.0 & 10.0 & Pass & Pass & Fail & Fail & Pass & Pass & Pass & Pass & Pass & Pass & Fail & Fail & Pass &2.2$\times10^{-8}$ & 1.1$\times10^{38}$ & 9.3$\times10^{-1}$\\
M & 0.94 & 30.0 & 40.0 & Pass & Pass & Fail & Fail & Pass & Pass & Pass & Pass & Pass & Pass & Fail & Fail & Pass &3.8$\times10^{-8}$ & 1.8$\times10^{38}$ & 9.3$\times10^{-1}$\\
M & 0.94 & 30.0 & 160.0 & Pass & Pass & Pass & Pass & Pass & Pass & Pass & Pass & Pass & Pass & Pass & Fail & Pass &6.6$\times10^{-8}$ & 3.1$\times10^{38}$ & 9.3$\times10^{-1}$\\
M & 0.94 & 50.0 & 1.0 & Pass & Fail & Fail & Fail & Pass & Pass & Pass & Pass & Pass & Pass & Fail & Fail & Pass &8.2$\times10^{-9}$ & 3.9$\times10^{37}$ & 9.3$\times10^{-1}$\\
M & 0.94 & 50.0 & 10.0 & Pass & Pass & Fail & Fail & Pass & Pass & Pass & Pass & Fail & Fail & Fail & Fail & Pass &2.1$\times10^{-8}$ & 1.0$\times10^{38}$ & 9.3$\times10^{-1}$\\
M & 0.94 & 50.0 & 40.0 & Pass & Pass & Fail & Fail & Pass & Pass & Pass & Pass & Fail & Fail & Fail & Fail & Pass &3.7$\times10^{-8}$ & 1.8$\times10^{38}$ & 9.3$\times10^{-1}$\\
M & 0.94 & 50.0 & 160.0 & Pass & Pass & Fail & Fail & Pass & Pass & Pass & Pass & Pass & Pass & Fail & Fail & Pass &6.5$\times10^{-8}$ & 3.1$\times10^{38}$ & 9.3$\times10^{-1}$\\
M & 0.94 & 70.0 & 1.0 & Pass & Fail & Fail & Fail & Pass & Pass & Pass & Pass & Pass & Pass & Fail & Fail & Pass &7.9$\times10^{-9}$ & 3.8$\times10^{37}$ & 9.3$\times10^{-1}$\\
M & 0.94 & 70.0 & 10.0 & Pass & Pass & Fail & Fail & Pass & Fail & Pass & Fail & Fail & Fail & Fail & Fail & Pass &2.0$\times10^{-8}$ & 9.7$\times10^{37}$ & 9.3$\times10^{-1}$\\
M & 0.94 & 70.0 & 40.0 & Pass & Pass & Fail & Fail & Pass & Fail & Pass & Fail & Fail & Fail & Fail & Fail & Pass &3.6$\times10^{-8}$ & 1.7$\times10^{38}$ & 9.3$\times10^{-1}$\\
M & 0.94 & 70.0 & 160.0 & Pass & Pass & Fail & Fail & Pass & Fail & Pass & Fail & Pass & Fail & Fail & Fail & Pass &6.6$\times10^{-8}$ & 3.1$\times10^{38}$ & 9.3$\times10^{-1}$\\
M & 0.94 & 90.0 & 1.0 & Pass & Fail & Fail & Fail & Pass & Fail & Pass & Fail & Pass & Fail & Fail & Fail & Pass &7.7$\times10^{-9}$ & 3.7$\times10^{37}$ & 9.3$\times10^{-1}$\\
M & 0.94 & 90.0 & 10.0 & Pass & Pass & Fail & Fail & Pass & Fail & Pass & Fail & Fail & Fail & Fail & Fail & Pass &2.0$\times10^{-8}$ & 9.7$\times10^{37}$ & 9.3$\times10^{-1}$\\
M & 0.94 & 90.0 & 40.0 & Pass & Pass & Fail & Fail & Pass & Fail & Pass & Fail & Pass & Fail & Fail & Fail & Pass &3.7$\times10^{-8}$ & 1.8$\times10^{38}$ & 9.3$\times10^{-1}$\\
M & 0.94 & 90.0 & 160.0 & Pass & Pass & Fail & Fail & Pass & Fail & Pass & Fail & Pass & Fail & Fail & Fail & Pass &6.8$\times10^{-8}$ & 3.2$\times10^{38}$ & 9.3$\times10^{-1}$\\
\enddata
\end{deluxetable*}
\end{longrotatetable}
\input{Tables/Frankfurt_k5_Table}
\begin{longrotatetable}
\startlongtable
\begin{deluxetable*}{ccccc|ccc|c|ccccc|c|c|cc|ccc}
\tabletypesize{\scriptsize}
\tablecaption{Pass/Fail Table, Frankfurt Fixed Kappa, Variable Efficiency Models}
\label{tab:frankfurtfkPF}
\tablehead{ \colhead{$\epsilon$} & %
\colhead{M/S}  &  %
\colhead{Spin}  &  %
\colhead{$i$}  &  %
\colhead{$\Rh$}  &  %
\colhead{$F_{86}$}  &  %
\colhead{$\lambda_{maj,86}$}  &  %
\colhead{$F_{2\mu{\rm m}}$}  &  %
\colhead{non-EHT}  &  %
\colhead{$\lambda_{230}$}  &  %
\colhead{Nulls}  &  %
\colhead{Ring D}  &  %
\colhead{Ring W}  &  %
\colhead{Ring A}  &  %
\colhead{EHT}  &  %
\colhead{All}  &  %
\colhead{M$_3$} & %
\colhead{4G$\lambda$} & %
\colhead{$\dot{M}/\dot{M}_{Edd}$}  &  %
\colhead{$P_{out}$(cgs)}  &  %
\colhead{$P_{out}/(\dot{M} c^2)$}}
\startdata
0.05 & S & -0.94 & 10.0 & 1.0 & Fail & Fail & Pass & Fail & Fail & Pass & Pass & Fail & Pass & Fail & Fail & Pass & Pass &1.0$\times10^{-8}$ & 2.2$\times10^{35}$ & 4.2$\times10^{-3}$\\
0.05 & S & -0.94 & 10.0 & 10.0 & Pass & Pass & Fail & Fail & Pass & Pass & Fail & Fail & Pass & Fail & Fail & Fail & Pass &2.9$\times10^{-7}$ & 6.3$\times10^{36}$ & 4.2$\times10^{-3}$\\
0.05 & S & -0.94 & 10.0 & 40.0 & Pass & Pass & Fail & Fail & Pass & Pass & Fail & Fail & Pass & Fail & Fail & Fail & Pass &9.7$\times10^{-7}$ & 2.1$\times10^{37}$ & 4.2$\times10^{-3}$\\
0.05 & S & -0.94 & 10.0 & 160.0 & Pass & Pass & Fail & Fail & Pass & Pass & Pass & Fail & Pass & Fail & Fail & Fail & Pass &1.7$\times10^{-6}$ & 3.7$\times10^{37}$ & 4.2$\times10^{-3}$\\
0.05 & S & -0.94 & 30.0 & 1.0 & Fail & Fail & Pass & Fail & Pass & Pass & Pass & Fail & Pass & Fail & Fail & Pass & Pass &9.9$\times10^{-9}$ & 2.1$\times10^{35}$ & 4.2$\times10^{-3}$\\
0.05 & S & -0.94 & 30.0 & 10.0 & Pass & Pass & Fail & Fail & Pass & Pass & Fail & Fail & Pass & Fail & Fail & Fail & Fail &2.8$\times10^{-7}$ & 6.2$\times10^{36}$ & 4.2$\times10^{-3}$\\
0.05 & S & -0.94 & 30.0 & 40.0 & Pass & Pass & Fail & Fail & Pass & Pass & Fail & Fail & Pass & Fail & Fail & Fail & Pass &9.3$\times10^{-7}$ & 2.0$\times10^{37}$ & 4.2$\times10^{-3}$\\
0.05 & S & -0.94 & 30.0 & 160.0 & Pass & Pass & Fail & Fail & Pass & Pass & Pass & Fail & Pass & Fail & Fail & Fail & Fail &1.6$\times10^{-6}$ & 3.5$\times10^{37}$ & 4.2$\times10^{-3}$\\
0.05 & S & -0.94 & 50.0 & 1.0 & Fail & Pass & Pass & Fail & Pass & Pass & Pass & Fail & Pass & Fail & Fail & Pass & Pass &9.7$\times10^{-9}$ & 2.1$\times10^{35}$ & 4.2$\times10^{-3}$\\
0.05 & S & -0.94 & 50.0 & 10.0 & Pass & Pass & Fail & Fail & Pass & Pass & Fail & Fail & Pass & Fail & Fail & Fail & Fail &2.9$\times10^{-7}$ & 6.3$\times10^{36}$ & 4.2$\times10^{-3}$\\
0.05 & S & -0.94 & 50.0 & 40.0 & Pass & Pass & Fail & Fail & Pass & Pass & Fail & Fail & Pass & Fail & Fail & Fail & Pass &9.4$\times10^{-7}$ & 2.0$\times10^{37}$ & 4.2$\times10^{-3}$\\
0.05 & S & -0.94 & 50.0 & 160.0 & Pass & Pass & Fail & Fail & Pass & Pass & Pass & Fail & Pass & Fail & Fail & Fail & Fail &1.6$\times10^{-6}$ & 3.5$\times10^{37}$ & 4.2$\times10^{-3}$\\
0.05 & S & -0.94 & 70.0 & 1.0 & Fail & Pass & Pass & Fail & Pass & Pass & Fail & Fail & Pass & Fail & Fail & Pass & Pass &9.8$\times10^{-9}$ & 2.1$\times10^{35}$ & 4.2$\times10^{-3}$\\
0.05 & S & -0.94 & 70.0 & 10.0 & Pass & Pass & Fail & Fail & Pass & Pass & Pass & Fail & Pass & Fail & Fail & Fail & Fail &3.1$\times10^{-7}$ & 6.8$\times10^{36}$ & 4.2$\times10^{-3}$\\
0.05 & S & -0.94 & 70.0 & 40.0 & Pass & Pass & Fail & Fail & Pass & Pass & Pass & Fail & Pass & Fail & Fail & Fail & Fail &9.9$\times10^{-7}$ & 2.2$\times10^{37}$ & 4.2$\times10^{-3}$\\
0.05 & S & -0.94 & 70.0 & 160.0 & Pass & Pass & Fail & Fail & Pass & Pass & Pass & Fail & Pass & Fail & Fail & Fail & Fail &1.7$\times10^{-6}$ & 3.7$\times10^{37}$ & 4.2$\times10^{-3}$\\
0.05 & S & -0.94 & 90.0 & 1.0 & Fail & Pass & Pass & Fail & Pass & Pass & Pass & Fail & Pass & Fail & Fail & Pass & Pass &9.6$\times10^{-9}$ & 2.1$\times10^{35}$ & 4.2$\times10^{-3}$\\
0.05 & S & -0.94 & 90.0 & 10.0 & Pass & Pass & Fail & Fail & Pass & Pass & Pass & Fail & Pass & Fail & Fail & Fail & Fail &3.2$\times10^{-7}$ & 7.0$\times10^{36}$ & 4.2$\times10^{-3}$\\
0.05 & S & -0.94 & 90.0 & 40.0 & Pass & Fail & Fail & Fail & Pass & Pass & Pass & Fail & Pass & Fail & Fail & Fail & Fail &1.0$\times10^{-6}$ & 2.2$\times10^{37}$ & 4.2$\times10^{-3}$\\
0.05 & S & -0.94 & 90.0 & 160.0 & Pass & Pass & Fail & Fail & Pass & Fail & Pass & Fail & Pass & Fail & Fail & Fail & Fail &1.8$\times10^{-6}$ & 3.8$\times10^{37}$ & 4.2$\times10^{-3}$\\
0.05 & S & -0.5 & 10.0 & 1.0 & Fail & Fail & Pass & Fail & Fail & Pass & Pass & Fail & Pass & Fail & Fail & Pass & Pass &5.7$\times10^{-8}$ & 3.5$\times10^{35}$ & 1.2$\times10^{-3}$\\
0.05 & S & -0.5 & 10.0 & 10.0 & Pass & Pass & Fail & Fail & Pass & Pass & Fail & Fail & Pass & Fail & Fail & Fail & Fail &1.7$\times10^{-6}$ & 1.1$\times10^{37}$ & 1.2$\times10^{-3}$\\
0.05 & S & -0.5 & 10.0 & 40.0 & Pass & Pass & Fail & Fail & Pass & Pass & Fail & Fail & Pass & Fail & Fail & Pass & Pass &3.7$\times10^{-6}$ & 2.3$\times10^{37}$ & 1.2$\times10^{-3}$\\
0.05 & S & -0.5 & 10.0 & 160.0 & Pass & Pass & Fail & Fail & Pass & Pass & Fail & Fail & Pass & Fail & Fail & Pass & Pass &5.7$\times10^{-6}$ & 3.5$\times10^{37}$ & 1.2$\times10^{-3}$\\
0.05 & S & -0.5 & 30.0 & 1.0 & Fail & Pass & Pass & Fail & Fail & Pass & Pass & Fail & Pass & Fail & Fail & Pass & Pass &5.6$\times10^{-8}$ & 3.4$\times10^{35}$ & 1.2$\times10^{-3}$\\
0.05 & S & -0.5 & 30.0 & 10.0 & Pass & Pass & Fail & Fail & Pass & Pass & Fail & Fail & Pass & Fail & Fail & Fail & Pass &1.7$\times10^{-6}$ & 1.0$\times10^{37}$ & 1.2$\times10^{-3}$\\
0.05 & S & -0.5 & 30.0 & 40.0 & Pass & Fail & Fail & Fail & Pass & Pass & Fail & Pass & Pass & Fail & Fail & Pass & Pass &3.8$\times10^{-6}$ & 2.3$\times10^{37}$ & 1.2$\times10^{-3}$\\
0.05 & S & -0.5 & 30.0 & 160.0 & Pass & Pass & Fail & Fail & Pass & Pass & Fail & Fail & Pass & Fail & Fail & Pass & Pass &6.0$\times10^{-6}$ & 3.7$\times10^{37}$ & 1.2$\times10^{-3}$\\
0.05 & S & -0.5 & 50.0 & 1.0 & Fail & Pass & Pass & Fail & Pass & Pass & Pass & Fail & Pass & Fail & Fail & Pass & Pass &5.5$\times10^{-8}$ & 3.4$\times10^{35}$ & 1.2$\times10^{-3}$\\
0.05 & S & -0.5 & 50.0 & 10.0 & Pass & Pass & Fail & Fail & Pass & Pass & Fail & Fail & Pass & Fail & Fail & Fail & Pass &1.6$\times10^{-6}$ & 1.0$\times10^{37}$ & 1.2$\times10^{-3}$\\
0.05 & S & -0.5 & 50.0 & 40.0 & Pass & Fail & Fail & Fail & Pass & Pass & Fail & Fail & Pass & Fail & Fail & Pass & Pass &3.6$\times10^{-6}$ & 2.2$\times10^{37}$ & 1.2$\times10^{-3}$\\
0.05 & S & -0.5 & 50.0 & 160.0 & Pass & Fail & Fail & Fail & Pass & Fail & Fail & Fail & Pass & Fail & Fail & Pass & Pass &5.7$\times10^{-6}$ & 3.5$\times10^{37}$ & 1.2$\times10^{-3}$\\
0.05 & S & -0.5 & 70.0 & 1.0 & Fail & Fail & Pass & Fail & Pass & Pass & Pass & Fail & Pass & Fail & Fail & Pass & Pass &5.5$\times10^{-8}$ & 3.4$\times10^{35}$ & 1.2$\times10^{-3}$\\
0.05 & S & -0.5 & 70.0 & 10.0 & Pass & Fail & Fail & Fail & Pass & Pass & Fail & Fail & Pass & Fail & Fail & Fail & Pass &1.6$\times10^{-6}$ & 1.0$\times10^{37}$ & 1.2$\times10^{-3}$\\
0.05 & S & -0.5 & 70.0 & 40.0 & Pass & Fail & Fail & Fail & Pass & Fail & Pass & Pass & Pass & Fail & Fail & Pass & Pass &3.6$\times10^{-6}$ & 2.2$\times10^{37}$ & 1.2$\times10^{-3}$\\
0.05 & S & -0.5 & 70.0 & 160.0 & Pass & Fail & Fail & Fail & Pass & Fail & Pass & Pass & Pass & Fail & Fail & Pass & Pass &5.9$\times10^{-6}$ & 3.6$\times10^{37}$ & 1.2$\times10^{-3}$\\
0.05 & S & -0.5 & 90.0 & 1.0 & Fail & Fail & Pass & Fail & Pass & Pass & Pass & Fail & Pass & Fail & Fail & Pass & Pass &5.4$\times10^{-8}$ & 3.4$\times10^{35}$ & 1.2$\times10^{-3}$\\
0.05 & S & -0.5 & 90.0 & 10.0 & Pass & Fail & Fail & Fail & Pass & Pass & Fail & Fail & Pass & Fail & Fail & Fail & Pass &1.7$\times10^{-6}$ & 1.0$\times10^{37}$ & 1.2$\times10^{-3}$\\
0.05 & S & -0.5 & 90.0 & 40.0 & Pass & Fail & Fail & Fail & Pass & Fail & Pass & Fail & Pass & Fail & Fail & Pass & Pass &3.7$\times10^{-6}$ & 2.3$\times10^{37}$ & 1.2$\times10^{-3}$\\
0.05 & S & -0.5 & 90.0 & 160.0 & Pass & Fail & Fail & Fail & Pass & Fail & Pass & Fail & Pass & Fail & Fail & Pass & Pass &6.1$\times10^{-6}$ & 3.8$\times10^{37}$ & 1.2$\times10^{-3}$\\
0.05 & S & 0.0 & 10.0 & 1.0 & Fail & Pass & Pass & Fail & Fail & Pass & Pass & Fail & Pass & Fail & Fail & Pass & Pass &3.8$\times10^{-8}$ & 9.0$\times10^{34}$ & 4.5$\times10^{-4}$\\
0.05 & S & 0.0 & 10.0 & 10.0 & Pass & Fail & Fail & Fail & Pass & Pass & Fail & Fail & Pass & Fail & Fail & Fail & Fail &5.3$\times10^{-7}$ & 1.2$\times10^{36}$ & 4.5$\times10^{-4}$\\
0.05 & S & 0.0 & 10.0 & 40.0 & Pass & Fail & Fail & Fail & Pass & Pass & Fail & Fail & Pass & Fail & Fail & Pass & Pass &1.4$\times10^{-6}$ & 3.4$\times10^{36}$ & 4.5$\times10^{-4}$\\
0.05 & S & 0.0 & 10.0 & 160.0 & Pass & Fail & Fail & Fail & Pass & Pass & Fail & Fail & Pass & Fail & Fail & Pass & Pass &2.9$\times10^{-6}$ & 6.8$\times10^{36}$ & 4.5$\times10^{-4}$\\
0.05 & S & 0.0 & 30.0 & 1.0 & Fail & Pass & Pass & Fail & Pass & Pass & Pass & Fail & Pass & Fail & Fail & Pass & Pass &3.7$\times10^{-8}$ & 8.7$\times10^{34}$ & 4.5$\times10^{-4}$\\
0.05 & S & 0.0 & 30.0 & 10.0 & Pass & Fail & Fail & Fail & Pass & Pass & Fail & Fail & Pass & Fail & Fail & Fail & Fail &5.1$\times10^{-7}$ & 1.2$\times10^{36}$ & 4.5$\times10^{-4}$\\
0.05 & S & 0.0 & 30.0 & 40.0 & Pass & Pass & Fail & Fail & Pass & Pass & Fail & Pass & Pass & Fail & Fail & Pass & Pass &1.4$\times10^{-6}$ & 3.3$\times10^{36}$ & 4.5$\times10^{-4}$\\
0.05 & S & 0.0 & 30.0 & 160.0 & Pass & Pass & Fail & Fail & Pass & Pass & Fail & Fail & Pass & Fail & Fail & Pass & Pass &2.9$\times10^{-6}$ & 6.7$\times10^{36}$ & 4.5$\times10^{-4}$\\
0.05 & S & 0.0 & 50.0 & 1.0 & Fail & Fail & Pass & Fail & Pass & Pass & Pass & Fail & Pass & Fail & Fail & Pass & Pass &3.6$\times10^{-8}$ & 8.5$\times10^{34}$ & 4.5$\times10^{-4}$\\
0.05 & S & 0.0 & 50.0 & 10.0 & Fail & Pass & Fail & Fail & Pass & Pass & Pass & Pass & Pass & Pass & Fail & Fail & Fail &4.8$\times10^{-7}$ & 1.1$\times10^{36}$ & 4.5$\times10^{-4}$\\
0.05 & S & 0.0 & 50.0 & 40.0 & Pass & Fail & Fail & Fail & Pass & Pass & Fail & Fail & Pass & Fail & Fail & Pass & Pass &1.4$\times10^{-6}$ & 3.2$\times10^{36}$ & 4.5$\times10^{-4}$\\
0.05 & S & 0.0 & 50.0 & 160.0 & Pass & Fail & Fail & Fail & Pass & Pass & Pass & Fail & Pass & Fail & Fail & Pass & Pass &2.9$\times10^{-6}$ & 6.9$\times10^{36}$ & 4.5$\times10^{-4}$\\
0.05 & S & 0.0 & 70.0 & 1.0 & Fail & Fail & Pass & Fail & Pass & Pass & Pass & Pass & Pass & Pass & Fail & Pass & Pass &3.6$\times10^{-8}$ & 8.4$\times10^{34}$ & 4.5$\times10^{-4}$\\
0.05 & S & 0.0 & 70.0 & 10.0 & Fail & Fail & Fail & Fail & Pass & Pass & Fail & Fail & Pass & Fail & Fail & Fail & Fail &4.8$\times10^{-7}$ & 1.1$\times10^{36}$ & 4.5$\times10^{-4}$\\
0.05 & S & 0.0 & 70.0 & 40.0 & Pass & Fail & Fail & Fail & Pass & Pass & Pass & Fail & Pass & Fail & Fail & Pass & Pass &1.4$\times10^{-6}$ & 3.3$\times10^{36}$ & 4.5$\times10^{-4}$\\
0.05 & S & 0.0 & 70.0 & 160.0 & Fail & Fail & Fail & Fail & Pass & Pass & Pass & Fail & Pass & Fail & Fail & Pass & Pass &3.1$\times10^{-6}$ & 7.4$\times10^{36}$ & 4.5$\times10^{-4}$\\
0.05 & S & 0.0 & 90.0 & 1.0 & Fail & Fail & Pass & Fail & Pass & Pass & Pass & Fail & Pass & Fail & Fail & Pass & Pass &3.5$\times10^{-8}$ & 8.2$\times10^{34}$ & 4.5$\times10^{-4}$\\
0.05 & S & 0.0 & 90.0 & 10.0 & Fail & Fail & Fail & Fail & Pass & Pass & Pass & Fail & Pass & Fail & Fail & Fail & Fail &4.8$\times10^{-7}$ & 1.1$\times10^{36}$ & 4.5$\times10^{-4}$\\
0.05 & S & 0.0 & 90.0 & 40.0 & Fail & Fail & Fail & Fail & Pass & Fail & Pass & Fail & Pass & Fail & Fail & Pass & Pass &1.5$\times10^{-6}$ & 3.4$\times10^{36}$ & 4.5$\times10^{-4}$\\
0.05 & S & 0.0 & 90.0 & 160.0 & Fail & Fail & Fail & Fail & Pass & Pass & Pass & Fail & Pass & Fail & Fail & Pass & Pass &3.2$\times10^{-6}$ & 7.5$\times10^{36}$ & 4.5$\times10^{-4}$\\
0.05 & S & 0.5 & 10.0 & 1.0 & Fail & Pass & Pass & Fail & Pass & Pass & Pass & Fail & Pass & Fail & Fail & Pass & Pass &1.7$\times10^{-7}$ & 3.6$\times10^{36}$ & 4.2$\times10^{-3}$\\
0.05 & S & 0.5 & 10.0 & 10.0 & Pass & Fail & Fail & Fail & Pass & Pass & Pass & Pass & Pass & Pass & Fail & Fail & Pass &2.3$\times10^{-6}$ & 4.9$\times10^{37}$ & 4.2$\times10^{-3}$\\
0.05 & S & 0.5 & 10.0 & 40.0 & Pass & Pass & Fail & Fail & Pass & Pass & Pass & Pass & Pass & Pass & Fail & Fail & Pass &4.1$\times10^{-6}$ & 8.8$\times10^{37}$ & 4.2$\times10^{-3}$\\
0.05 & S & 0.5 & 10.0 & 160.0 & Pass & Pass & Fail & Fail & Pass & Pass & Pass & Pass & Pass & Pass & Fail & Fail & Pass &6.1$\times10^{-6}$ & 1.3$\times10^{38}$ & 4.2$\times10^{-3}$\\
0.05 & S & 0.5 & 30.0 & 1.0 & Fail & Fail & Pass & Fail & Pass & Pass & Pass & Fail & Pass & Fail & Fail & Pass & Pass &1.6$\times10^{-7}$ & 3.5$\times10^{36}$ & 4.2$\times10^{-3}$\\
0.05 & S & 0.5 & 30.0 & 10.0 & Pass & Fail & Fail & Fail & Pass & Pass & Pass & Pass & Pass & Pass & Fail & Fail & Pass &2.2$\times10^{-6}$ & 4.8$\times10^{37}$ & 4.2$\times10^{-3}$\\
0.05 & S & 0.5 & 30.0 & 40.0 & Pass & Pass & Fail & Fail & Pass & Pass & Pass & Pass & Pass & Pass & Fail & Fail & Pass &4.2$\times10^{-6}$ & 9.0$\times10^{37}$ & 4.2$\times10^{-3}$\\
0.05 & S & 0.5 & 30.0 & 160.0 & Pass & Pass & Fail & Fail & Pass & Pass & Pass & Pass & Pass & Pass & Fail & Fail & Pass &6.5$\times10^{-6}$ & 1.4$\times10^{38}$ & 4.2$\times10^{-3}$\\
0.05 & S & 0.5 & 50.0 & 1.0 & Fail & Fail & Pass & Fail & Pass & Pass & Pass & Fail & Pass & Fail & Fail & Pass & Pass &1.6$\times10^{-7}$ & 3.4$\times10^{36}$ & 4.2$\times10^{-3}$\\
0.05 & S & 0.5 & 50.0 & 10.0 & Fail & Fail & Fail & Fail & Pass & Pass & Pass & Fail & Pass & Fail & Fail & Fail & Pass &2.1$\times10^{-6}$ & 4.4$\times10^{37}$ & 4.2$\times10^{-3}$\\
0.05 & S & 0.5 & 50.0 & 40.0 & Pass & Pass & Fail & Fail & Pass & Pass & Pass & Fail & Pass & Fail & Fail & Fail & Pass &3.9$\times10^{-6}$ & 8.3$\times10^{37}$ & 4.2$\times10^{-3}$\\
0.05 & S & 0.5 & 50.0 & 160.0 & Pass & Fail & Fail & Fail & Pass & Pass & Pass & Fail & Pass & Fail & Fail & Fail & Pass &6.0$\times10^{-6}$ & 1.3$\times10^{38}$ & 4.2$\times10^{-3}$\\
0.05 & S & 0.5 & 70.0 & 1.0 & Fail & Fail & Pass & Fail & Pass & Pass & Pass & Pass & Pass & Pass & Fail & Pass & Pass &1.6$\times10^{-7}$ & 3.4$\times10^{36}$ & 4.2$\times10^{-3}$\\
0.05 & S & 0.5 & 70.0 & 10.0 & Fail & Fail & Fail & Fail & Pass & Pass & Pass & Fail & Pass & Fail & Fail & Fail & Fail &2.0$\times10^{-6}$ & 4.2$\times10^{37}$ & 4.2$\times10^{-3}$\\
0.05 & S & 0.5 & 70.0 & 40.0 & Fail & Pass & Fail & Fail & Pass & Fail & Pass & Fail & Pass & Fail & Fail & Fail & Fail &3.8$\times10^{-6}$ & 8.1$\times10^{37}$ & 4.2$\times10^{-3}$\\
0.05 & S & 0.5 & 70.0 & 160.0 & Pass & Fail & Fail & Fail & Pass & Fail & Pass & Fail & Pass & Fail & Fail & Fail & Pass &6.1$\times10^{-6}$ & 1.3$\times10^{38}$ & 4.2$\times10^{-3}$\\
0.05 & S & 0.5 & 90.0 & 1.0 & Fail & Fail & Pass & Fail & Pass & Fail & Pass & Fail & Pass & Fail & Fail & Pass & Pass &1.5$\times10^{-7}$ & 3.3$\times10^{36}$ & 4.2$\times10^{-3}$\\
0.05 & S & 0.5 & 90.0 & 10.0 & Fail & Fail & Fail & Fail & Pass & Pass & Pass & Fail & Pass & Fail & Fail & Fail & Fail &1.9$\times10^{-6}$ & 4.2$\times10^{37}$ & 4.2$\times10^{-3}$\\
0.05 & S & 0.5 & 90.0 & 40.0 & Fail & Pass & Fail & Fail & Pass & Fail & Pass & Fail & Pass & Fail & Fail & Fail & Fail &3.9$\times10^{-6}$ & 8.3$\times10^{37}$ & 4.2$\times10^{-3}$\\
0.05 & S & 0.5 & 90.0 & 160.0 & Pass & Fail & Fail & Fail & Pass & Fail & Pass & Pass & Pass & Fail & Fail & Fail & Fail &6.3$\times10^{-6}$ & 1.4$\times10^{38}$ & 4.2$\times10^{-3}$\\
0.05 & S & 0.94 & 10.0 & 1.0 & Fail & Fail & Pass & Fail & Pass & Pass & Fail & Fail & Pass & Fail & Fail & Pass & Pass &7.3$\times10^{-9}$ & 4.4$\times10^{35}$ & 1.2$\times10^{-2}$\\
0.05 & S & 0.94 & 10.0 & 10.0 & Pass & Pass & Pass & Pass & Pass & Pass & Pass & Pass & Pass & Pass & Pass & Pass & Pass &1.1$\times10^{-7}$ & 6.6$\times10^{36}$ & 1.2$\times10^{-2}$\\
0.05 & S & 0.94 & 10.0 & 40.0 & Pass & Fail & Fail & Fail & Pass & Pass & Fail & Pass & Pass & Fail & Fail & Fail & Pass &5.5$\times10^{-7}$ & 3.3$\times10^{37}$ & 1.2$\times10^{-2}$\\
0.05 & S & 0.94 & 10.0 & 160.0 & Fail & Fail & Fail & Fail & Pass & Pass & Pass & Pass & Pass & Pass & Fail & Fail & Pass &9.7$\times10^{-7}$ & 5.9$\times10^{37}$ & 1.2$\times10^{-2}$\\
0.05 & S & 0.94 & 30.0 & 1.0 & Fail & Fail & Pass & Fail & Pass & Pass & Pass & Fail & Pass & Fail & Fail & Pass & Pass &7.1$\times10^{-9}$ & 4.2$\times10^{35}$ & 1.2$\times10^{-2}$\\
0.05 & S & 0.94 & 30.0 & 10.0 & Fail & Pass & Pass & Fail & Pass & Pass & Pass & Pass & Pass & Pass & Fail & Pass & Pass &1.1$\times10^{-7}$ & 6.5$\times10^{36}$ & 1.2$\times10^{-2}$\\
0.05 & S & 0.94 & 30.0 & 40.0 & Pass & Fail & Fail & Fail & Pass & Pass & Pass & Fail & Pass & Fail & Fail & Pass & Pass &6.4$\times10^{-7}$ & 3.9$\times10^{37}$ & 1.2$\times10^{-2}$\\
0.05 & S & 0.94 & 30.0 & 160.0 & Fail & Fail & Fail & Fail & Pass & Pass & Pass & Pass & Pass & Pass & Fail & Pass & Pass &1.1$\times10^{-6}$ & 6.6$\times10^{37}$ & 1.2$\times10^{-2}$\\
0.05 & S & 0.94 & 50.0 & 1.0 & Fail & Fail & Pass & Fail & Pass & Pass & Pass & Pass & Pass & Pass & Fail & Pass & Pass &6.8$\times10^{-9}$ & 4.1$\times10^{35}$ & 1.2$\times10^{-2}$\\
0.05 & S & 0.94 & 50.0 & 10.0 & Fail & Pass & Fail & Fail & Pass & Pass & Pass & Pass & Fail & Fail & Fail & Pass & Fail &1.1$\times10^{-7}$ & 6.4$\times10^{36}$ & 1.2$\times10^{-2}$\\
0.05 & S & 0.94 & 50.0 & 40.0 & Pass & Fail & Fail & Fail & Pass & Pass & Pass & Fail & Pass & Fail & Fail & Fail & Pass &7.2$\times10^{-7}$ & 4.3$\times10^{37}$ & 1.2$\times10^{-2}$\\
0.05 & S & 0.94 & 50.0 & 160.0 & Fail & Fail & Fail & Fail & Pass & Fail & Pass & Pass & Pass & Fail & Fail & Pass & Pass &1.3$\times10^{-6}$ & 7.7$\times10^{37}$ & 1.2$\times10^{-2}$\\
0.05 & S & 0.94 & 70.0 & 1.0 & Fail & Fail & Pass & Fail & Pass & Fail & Pass & Pass & Pass & Fail & Fail & Pass & Pass &6.7$\times10^{-9}$ & 4.0$\times10^{35}$ & 1.2$\times10^{-2}$\\
0.05 & S & 0.94 & 70.0 & 10.0 & Fail & Pass & Fail & Fail & Pass & Pass & Pass & Pass & Fail & Fail & Fail & Pass & Fail &1.1$\times10^{-7}$ & 6.8$\times10^{36}$ & 1.2$\times10^{-2}$\\
0.05 & S & 0.94 & 70.0 & 40.0 & Pass & Fail & Fail & Fail & Pass & Fail & Pass & Fail & Pass & Fail & Fail & Pass & Fail &7.2$\times10^{-7}$ & 4.4$\times10^{37}$ & 1.2$\times10^{-2}$\\
0.05 & S & 0.94 & 70.0 & 160.0 & Pass & Fail & Fail & Fail & Pass & Fail & Pass & Fail & Pass & Fail & Fail & Pass & Pass &1.4$\times10^{-6}$ & 8.3$\times10^{37}$ & 1.2$\times10^{-2}$\\
0.05 & S & 0.94 & 90.0 & 1.0 & Fail & Fail & Pass & Fail & Pass & Fail & Pass & Pass & Pass & Fail & Fail & Pass & Pass &6.7$\times10^{-9}$ & 4.0$\times10^{35}$ & 1.2$\times10^{-2}$\\
0.05 & S & 0.94 & 90.0 & 10.0 & Fail & Pass & Fail & Fail & Pass & Fail & Pass & Pass & Pass & Fail & Fail & Pass & Fail &1.2$\times10^{-7}$ & 7.1$\times10^{36}$ & 1.2$\times10^{-2}$\\
0.05 & S & 0.94 & 90.0 & 40.0 & Pass & Fail & Fail & Fail & Pass & Fail & Pass & Fail & Pass & Fail & Fail & Pass & Fail &7.1$\times10^{-7}$ & 4.3$\times10^{37}$ & 1.2$\times10^{-2}$\\
0.05 & S & 0.94 & 90.0 & 160.0 & Pass & Fail & Fail & Fail & Pass & Fail & Pass & Fail & Pass & Fail & Fail & Pass & Pass &1.4$\times10^{-6}$ & 8.4$\times10^{37}$ & 1.2$\times10^{-2}$\\
0.05 & M & -0.94 & 10.0 & 1.0 & Fail & Fail & Fail & Fail & Pass & Pass & Fail & Fail & Pass & Fail & Fail & Fail & Pass &1.8$\times10^{-8}$ & 1.7$\times10^{37}$ & 1.8$\times10^{-1}$\\
0.05 & M & -0.94 & 10.0 & 10.0 & Pass & Pass & Fail & Fail & Pass & Pass & Pass & Fail & Pass & Fail & Fail & Fail & Fail &5.6$\times10^{-8}$ & 5.3$\times10^{37}$ & 1.8$\times10^{-1}$\\
0.05 & M & -0.94 & 10.0 & 40.0 & Pass & Pass & Fail & Fail & Pass & Pass & Pass & Fail & Pass & Fail & Fail & Fail & Fail &8.8$\times10^{-8}$ & 8.3$\times10^{37}$ & 1.8$\times10^{-1}$\\
0.05 & M & -0.94 & 10.0 & 160.0 & Pass & Pass & Fail & Fail & Pass & Pass & Pass & Fail & Pass & Fail & Fail & Fail & Fail &1.5$\times10^{-7}$ & 1.4$\times10^{38}$ & 1.8$\times10^{-1}$\\
0.05 & M & -0.94 & 30.0 & 1.0 & Fail & Pass & Fail & Fail & Pass & Pass & Fail & Fail & Pass & Fail & Fail & Fail & Pass &1.7$\times10^{-8}$ & 1.6$\times10^{37}$ & 1.8$\times10^{-1}$\\
0.05 & M & -0.94 & 30.0 & 10.0 & Pass & Pass & Fail & Fail & Pass & Pass & Pass & Fail & Pass & Fail & Fail & Fail & Fail &5.4$\times10^{-8}$ & 5.1$\times10^{37}$ & 1.8$\times10^{-1}$\\
0.05 & M & -0.94 & 30.0 & 40.0 & Pass & Pass & Fail & Fail & Pass & Pass & Pass & Fail & Pass & Fail & Fail & Fail & Fail &8.4$\times10^{-8}$ & 8.0$\times10^{37}$ & 1.8$\times10^{-1}$\\
0.05 & M & -0.94 & 30.0 & 160.0 & Pass & Pass & Fail & Fail & Pass & Pass & Pass & Fail & Pass & Fail & Fail & Fail & Fail &1.4$\times10^{-7}$ & 1.3$\times10^{38}$ & 1.8$\times10^{-1}$\\
0.05 & M & -0.94 & 50.0 & 1.0 & Fail & Pass & Fail & Fail & Pass & Pass & Fail & Fail & Pass & Fail & Fail & Fail & Pass &1.7$\times10^{-8}$ & 1.6$\times10^{37}$ & 1.8$\times10^{-1}$\\
0.05 & M & -0.94 & 50.0 & 10.0 & Pass & Pass & Fail & Fail & Pass & Pass & Pass & Fail & Pass & Fail & Fail & Fail & Fail &5.0$\times10^{-8}$ & 4.7$\times10^{37}$ & 1.8$\times10^{-1}$\\
0.05 & M & -0.94 & 50.0 & 40.0 & Pass & Pass & Fail & Fail & Pass & Pass & Pass & Fail & Pass & Fail & Fail & Fail & Fail &7.8$\times10^{-8}$ & 7.4$\times10^{37}$ & 1.8$\times10^{-1}$\\
0.05 & M & -0.94 & 50.0 & 160.0 & Pass & Pass & Fail & Fail & Pass & Pass & Pass & Fail & Pass & Fail & Fail & Fail & Fail &1.3$\times10^{-7}$ & 1.2$\times10^{38}$ & 1.8$\times10^{-1}$\\
0.05 & M & -0.94 & 70.0 & 1.0 & Fail & Pass & Fail & Fail & Pass & Pass & Fail & Fail & Pass & Fail & Fail & Fail & Pass &1.6$\times10^{-8}$ & 1.5$\times10^{37}$ & 1.8$\times10^{-1}$\\
0.05 & M & -0.94 & 70.0 & 10.0 & Pass & Pass & Fail & Fail & Pass & Pass & Pass & Fail & Pass & Fail & Fail & Fail & Fail &4.7$\times10^{-8}$ & 4.4$\times10^{37}$ & 1.8$\times10^{-1}$\\
0.05 & M & -0.94 & 70.0 & 40.0 & Pass & Pass & Fail & Fail & Pass & Pass & Pass & Fail & Pass & Fail & Fail & Fail & Fail &7.3$\times10^{-8}$ & 6.9$\times10^{37}$ & 1.8$\times10^{-1}$\\
0.05 & M & -0.94 & 70.0 & 160.0 & Pass & Pass & Fail & Fail & Pass & Pass & Pass & Fail & Pass & Fail & Fail & Fail & Fail &1.2$\times10^{-7}$ & 1.1$\times10^{38}$ & 1.8$\times10^{-1}$\\
0.05 & M & -0.94 & 90.0 & 1.0 & Fail & Pass & Fail & Fail & Pass & Fail & Fail & Fail & Pass & Fail & Fail & Fail & Pass &1.5$\times10^{-8}$ & 1.5$\times10^{37}$ & 1.8$\times10^{-1}$\\
0.05 & M & -0.94 & 90.0 & 10.0 & Pass & Pass & Fail & Fail & Pass & Pass & Pass & Fail & Pass & Fail & Fail & Fail & Fail &4.5$\times10^{-8}$ & 4.3$\times10^{37}$ & 1.8$\times10^{-1}$\\
0.05 & M & -0.94 & 90.0 & 40.0 & Pass & Pass & Fail & Fail & Pass & Pass & Pass & Fail & Pass & Fail & Fail & Fail & Fail &7.1$\times10^{-8}$ & 6.7$\times10^{37}$ & 1.8$\times10^{-1}$\\
0.05 & M & -0.94 & 90.0 & 160.0 & Pass & Pass & Fail & Fail & Pass & Pass & Pass & Fail & Pass & Fail & Fail & Fail & Fail &1.2$\times10^{-7}$ & 1.1$\times10^{38}$ & 1.8$\times10^{-1}$\\
0.05 & M & -0.5 & 10.0 & 1.0 & Fail & Pass & Fail & Fail & Pass & Pass & Fail & Fail & Pass & Fail & Fail & Pass & Pass &3.9$\times10^{-8}$ & 9.1$\times10^{36}$ & 4.6$\times10^{-2}$\\
0.05 & M & -0.5 & 10.0 & 10.0 & Fail & Pass & Fail & Fail & Pass & Pass & Fail & Fail & Pass & Fail & Fail & Pass & Pass &1.3$\times10^{-7}$ & 3.1$\times10^{37}$ & 4.6$\times10^{-2}$\\
0.05 & M & -0.5 & 10.0 & 40.0 & Pass & Pass & Fail & Fail & Pass & Pass & Fail & Fail & Pass & Fail & Fail & Pass & Pass &2.1$\times10^{-7}$ & 5.0$\times10^{37}$ & 4.6$\times10^{-2}$\\
0.05 & M & -0.5 & 10.0 & 160.0 & Pass & Fail & Fail & Fail & Pass & Pass & Fail & Fail & Pass & Fail & Fail & Pass & Fail &3.8$\times10^{-7}$ & 9.0$\times10^{37}$ & 4.6$\times10^{-2}$\\
0.05 & M & -0.5 & 30.0 & 1.0 & Fail & Pass & Fail & Fail & Pass & Pass & Fail & Fail & Pass & Fail & Fail & Pass & Pass &3.7$\times10^{-8}$ & 8.8$\times10^{36}$ & 4.6$\times10^{-2}$\\
0.05 & M & -0.5 & 30.0 & 10.0 & Pass & Pass & Fail & Fail & Pass & Pass & Fail & Fail & Pass & Fail & Fail & Pass & Pass &1.2$\times10^{-7}$ & 2.9$\times10^{37}$ & 4.6$\times10^{-2}$\\
0.05 & M & -0.5 & 30.0 & 40.0 & Pass & Pass & Fail & Fail & Pass & Pass & Fail & Fail & Pass & Fail & Fail & Pass & Pass &2.0$\times10^{-7}$ & 4.8$\times10^{37}$ & 4.6$\times10^{-2}$\\
0.05 & M & -0.5 & 30.0 & 160.0 & Pass & Pass & Fail & Fail & Pass & Pass & Fail & Fail & Pass & Fail & Fail & Fail & Fail &3.7$\times10^{-7}$ & 8.7$\times10^{37}$ & 4.6$\times10^{-2}$\\
0.05 & M & -0.5 & 50.0 & 1.0 & Fail & Fail & Fail & Fail & Pass & Pass & Fail & Pass & Pass & Fail & Fail & Pass & Pass &3.6$\times10^{-8}$ & 8.4$\times10^{36}$ & 4.6$\times10^{-2}$\\
0.05 & M & -0.5 & 50.0 & 10.0 & Pass & Pass & Fail & Fail & Pass & Pass & Pass & Fail & Pass & Fail & Fail & Fail & Pass &1.1$\times10^{-7}$ & 2.7$\times10^{37}$ & 4.6$\times10^{-2}$\\
0.05 & M & -0.5 & 50.0 & 40.0 & Pass & Pass & Fail & Fail & Pass & Pass & Fail & Fail & Pass & Fail & Fail & Fail & Pass &1.9$\times10^{-7}$ & 4.5$\times10^{37}$ & 4.6$\times10^{-2}$\\
0.05 & M & -0.5 & 50.0 & 160.0 & Pass & Pass & Fail & Fail & Pass & Pass & Fail & Fail & Pass & Fail & Fail & Fail & Fail &3.4$\times10^{-7}$ & 8.1$\times10^{37}$ & 4.6$\times10^{-2}$\\
0.05 & M & -0.5 & 70.0 & 1.0 & Fail & Fail & Fail & Fail & Pass & Pass & Fail & Fail & Pass & Fail & Fail & Pass & Pass &3.4$\times10^{-8}$ & 8.0$\times10^{36}$ & 4.6$\times10^{-2}$\\
0.05 & M & -0.5 & 70.0 & 10.0 & Pass & Pass & Fail & Fail & Pass & Fail & Fail & Fail & Pass & Fail & Fail & Fail & Pass &1.0$\times10^{-7}$ & 2.4$\times10^{37}$ & 4.6$\times10^{-2}$\\
0.05 & M & -0.5 & 70.0 & 40.0 & Pass & Pass & Fail & Fail & Pass & Fail & Fail & Fail & Pass & Fail & Fail & Fail & Pass &1.7$\times10^{-7}$ & 4.0$\times10^{37}$ & 4.6$\times10^{-2}$\\
0.05 & M & -0.5 & 70.0 & 160.0 & Pass & Pass & Fail & Fail & Pass & Pass & Fail & Fail & Pass & Fail & Fail & Fail & Fail &3.1$\times10^{-7}$ & 7.2$\times10^{37}$ & 4.6$\times10^{-2}$\\
0.05 & M & -0.5 & 90.0 & 1.0 & Fail & Fail & Fail & Fail & Pass & Fail & Fail & Fail & Pass & Fail & Fail & Pass & Pass &3.3$\times10^{-8}$ & 7.9$\times10^{36}$ & 4.6$\times10^{-2}$\\
0.05 & M & -0.5 & 90.0 & 10.0 & Pass & Pass & Fail & Fail & Pass & Fail & Fail & Fail & Pass & Fail & Fail & Fail & Pass &1.0$\times10^{-7}$ & 2.4$\times10^{37}$ & 4.6$\times10^{-2}$\\
0.05 & M & -0.5 & 90.0 & 40.0 & Pass & Pass & Fail & Fail & Pass & Fail & Fail & Fail & Pass & Fail & Fail & Fail & Pass &1.7$\times10^{-7}$ & 3.9$\times10^{37}$ & 4.6$\times10^{-2}$\\
0.05 & M & -0.5 & 90.0 & 160.0 & Pass & Pass & Fail & Fail & Pass & Pass & Fail & Fail & Pass & Fail & Fail & Fail & Fail &3.1$\times10^{-7}$ & 7.2$\times10^{37}$ & 4.6$\times10^{-2}$\\
0.05 & M & 0.0 & 10.0 & 1.0 & Fail & Pass & Fail & Fail & Pass & Pass & Fail & Fail & Pass & Fail & Fail & Fail & Pass &2.3$\times10^{-8}$ & 1.7$\times10^{36}$ & 1.4$\times10^{-2}$\\
0.05 & M & 0.0 & 10.0 & 10.0 & Pass & Fail & Fail & Fail & Pass & Pass & Fail & Fail & Pass & Fail & Fail & Pass & Pass &7.2$\times10^{-8}$ & 5.1$\times10^{36}$ & 1.4$\times10^{-2}$\\
0.05 & M & 0.0 & 10.0 & 40.0 & Pass & Fail & Fail & Fail & Pass & Pass & Fail & Fail & Pass & Fail & Fail & Pass & Pass &1.2$\times10^{-7}$ & 8.7$\times10^{36}$ & 1.4$\times10^{-2}$\\
0.05 & M & 0.0 & 10.0 & 160.0 & Pass & Fail & Fail & Fail & Pass & Pass & Fail & Fail & Pass & Fail & Fail & Pass & Fail &2.1$\times10^{-7}$ & 1.5$\times10^{37}$ & 1.4$\times10^{-2}$\\
0.05 & M & 0.0 & 30.0 & 1.0 & Fail & Fail & Fail & Fail & Pass & Pass & Fail & Fail & Pass & Fail & Fail & Pass & Pass &2.3$\times10^{-8}$ & 1.6$\times10^{36}$ & 1.4$\times10^{-2}$\\
0.05 & M & 0.0 & 30.0 & 10.0 & Pass & Pass & Fail & Fail & Pass & Pass & Fail & Fail & Pass & Fail & Fail & Fail & Pass &6.9$\times10^{-8}$ & 4.9$\times10^{36}$ & 1.4$\times10^{-2}$\\
0.05 & M & 0.0 & 30.0 & 40.0 & Pass & Pass & Fail & Fail & Pass & Pass & Fail & Fail & Pass & Fail & Fail & Fail & Pass &1.2$\times10^{-7}$ & 8.4$\times10^{36}$ & 1.4$\times10^{-2}$\\
0.05 & M & 0.0 & 30.0 & 160.0 & Pass & Pass & Fail & Fail & Pass & Pass & Fail & Fail & Pass & Fail & Fail & Pass & Fail &2.0$\times10^{-7}$ & 1.4$\times10^{37}$ & 1.4$\times10^{-2}$\\
0.05 & M & 0.0 & 50.0 & 1.0 & Fail & Fail & Fail & Fail & Pass & Pass & Fail & Fail & Pass & Fail & Fail & Pass & Pass &2.2$\times10^{-8}$ & 1.6$\times10^{36}$ & 1.4$\times10^{-2}$\\
0.05 & M & 0.0 & 50.0 & 10.0 & Pass & Fail & Fail & Fail & Pass & Pass & Pass & Pass & Pass & Pass & Fail & Fail & Pass &6.6$\times10^{-8}$ & 4.7$\times10^{36}$ & 1.4$\times10^{-2}$\\
0.05 & M & 0.0 & 50.0 & 40.0 & Pass & Fail & Fail & Fail & Pass & Pass & Pass & Fail & Pass & Fail & Fail & Fail & Pass &1.1$\times10^{-7}$ & 8.0$\times10^{36}$ & 1.4$\times10^{-2}$\\
0.05 & M & 0.0 & 50.0 & 160.0 & Pass & Pass & Fail & Fail & Pass & Pass & Pass & Fail & Pass & Fail & Fail & Fail & Fail &2.0$\times10^{-7}$ & 1.4$\times10^{37}$ & 1.4$\times10^{-2}$\\
0.05 & M & 0.0 & 70.0 & 1.0 & Fail & Fail & Fail & Fail & Pass & Pass & Pass & Fail & Pass & Fail & Fail & Pass & Pass &2.2$\times10^{-8}$ & 1.5$\times10^{36}$ & 1.4$\times10^{-2}$\\
0.05 & M & 0.0 & 70.0 & 10.0 & Pass & Fail & Fail & Fail & Pass & Pass & Fail & Fail & Pass & Fail & Fail & Fail & Pass &6.2$\times10^{-8}$ & 4.4$\times10^{36}$ & 1.4$\times10^{-2}$\\
0.05 & M & 0.0 & 70.0 & 40.0 & Pass & Fail & Fail & Fail & Pass & Pass & Fail & Fail & Pass & Fail & Fail & Fail & Pass &1.1$\times10^{-7}$ & 7.6$\times10^{36}$ & 1.4$\times10^{-2}$\\
0.05 & M & 0.0 & 70.0 & 160.0 & Pass & Fail & Fail & Fail & Pass & Pass & Fail & Fail & Pass & Fail & Fail & Fail & Fail &1.9$\times10^{-7}$ & 1.4$\times10^{37}$ & 1.4$\times10^{-2}$\\
0.05 & M & 0.0 & 90.0 & 1.0 & Pass & Fail & Fail & Fail & Pass & Fail & Pass & Fail & Pass & Fail & Fail & Pass & Pass &2.0$\times10^{-8}$ & 1.4$\times10^{36}$ & 1.4$\times10^{-2}$\\
0.05 & M & 0.0 & 90.0 & 10.0 & Pass & Fail & Fail & Fail & Pass & Fail & Pass & Fail & Pass & Fail & Fail & Fail & Pass &5.9$\times10^{-8}$ & 4.2$\times10^{36}$ & 1.4$\times10^{-2}$\\
0.05 & M & 0.0 & 90.0 & 40.0 & Pass & Fail & Fail & Fail & Pass & Pass & Pass & Fail & Pass & Fail & Fail & Fail & Pass &1.1$\times10^{-7}$ & 7.5$\times10^{36}$ & 1.4$\times10^{-2}$\\
0.05 & M & 0.0 & 90.0 & 160.0 & Pass & Fail & Fail & Fail & Pass & Pass & Fail & Fail & Pass & Fail & Fail & Pass & Fail &1.9$\times10^{-7}$ & 1.4$\times10^{37}$ & 1.4$\times10^{-2}$\\
0.05 & M & 0.5 & 10.0 & 1.0 & Fail & Fail & Fail & Fail & Pass & Pass & Fail & Fail & Pass & Fail & Fail & Pass & Pass &3.9$\times10^{-8}$ & 3.2$\times10^{37}$ & 1.6$\times10^{-1}$\\
0.05 & M & 0.5 & 10.0 & 10.0 & Fail & Pass & Fail & Fail & Pass & Pass & Pass & Fail & Pass & Fail & Fail & Fail & Pass &1.1$\times10^{-7}$ & 8.6$\times10^{37}$ & 1.6$\times10^{-1}$\\
0.05 & M & 0.5 & 10.0 & 40.0 & Fail & Pass & Fail & Fail & Pass & Pass & Pass & Pass & Pass & Pass & Fail & Pass & Pass &1.7$\times10^{-7}$ & 1.4$\times10^{38}$ & 1.6$\times10^{-1}$\\
0.05 & M & 0.5 & 10.0 & 160.0 & Fail & Pass & Fail & Fail & Pass & Pass & Pass & Pass & Pass & Pass & Fail & Pass & Pass &3.2$\times10^{-7}$ & 2.7$\times10^{38}$ & 1.6$\times10^{-1}$\\
0.05 & M & 0.5 & 30.0 & 1.0 & Fail & Fail & Fail & Fail & Pass & Pass & Fail & Fail & Pass & Fail & Fail & Pass & Pass &3.8$\times10^{-8}$ & 3.1$\times10^{37}$ & 1.6$\times10^{-1}$\\
0.05 & M & 0.5 & 30.0 & 10.0 & Fail & Pass & Fail & Fail & Pass & Pass & Pass & Pass & Pass & Pass & Fail & Fail & Pass &1.0$\times10^{-7}$ & 8.4$\times10^{37}$ & 1.6$\times10^{-1}$\\
0.05 & M & 0.5 & 30.0 & 40.0 & Pass & Pass & Fail & Fail & Pass & Pass & Pass & Pass & Pass & Pass & Fail & Fail & Pass &1.7$\times10^{-7}$ & 1.4$\times10^{38}$ & 1.6$\times10^{-1}$\\
0.05 & M & 0.5 & 30.0 & 160.0 & Pass & Pass & Fail & Fail & Pass & Pass & Pass & Pass & Pass & Pass & Fail & Pass & Pass &3.1$\times10^{-7}$ & 2.6$\times10^{38}$ & 1.6$\times10^{-1}$\\
0.05 & M & 0.5 & 50.0 & 1.0 & Fail & Fail & Fail & Fail & Pass & Pass & Pass & Pass & Pass & Pass & Fail & Pass & Pass &3.6$\times10^{-8}$ & 3.0$\times10^{37}$ & 1.6$\times10^{-1}$\\
0.05 & M & 0.5 & 50.0 & 10.0 & Pass & Pass & Fail & Fail & Pass & Pass & Pass & Pass & Pass & Pass & Fail & Fail & Pass &9.4$\times10^{-8}$ & 7.7$\times10^{37}$ & 1.6$\times10^{-1}$\\
0.05 & M & 0.5 & 50.0 & 40.0 & Pass & Pass & Fail & Fail & Pass & Pass & Pass & Pass & Pass & Pass & Fail & Fail & Pass &1.5$\times10^{-7}$ & 1.3$\times10^{38}$ & 1.6$\times10^{-1}$\\
0.05 & M & 0.5 & 50.0 & 160.0 & Pass & Pass & Fail & Fail & Pass & Pass & Pass & Pass & Pass & Pass & Fail & Pass & Pass &2.9$\times10^{-7}$ & 2.4$\times10^{38}$ & 1.6$\times10^{-1}$\\
0.05 & M & 0.5 & 70.0 & 1.0 & Fail & Fail & Fail & Fail & Pass & Pass & Pass & Pass & Pass & Pass & Fail & Fail & Pass &3.4$\times10^{-8}$ & 2.8$\times10^{37}$ & 1.6$\times10^{-1}$\\
0.05 & M & 0.5 & 70.0 & 10.0 & Pass & Pass & Fail & Fail & Pass & Fail & Pass & Fail & Fail & Fail & Fail & Fail & Pass &8.4$\times10^{-8}$ & 6.9$\times10^{37}$ & 1.6$\times10^{-1}$\\
0.05 & M & 0.5 & 70.0 & 40.0 & Pass & Pass & Fail & Fail & Pass & Fail & Pass & Fail & Fail & Fail & Fail & Fail & Pass &1.4$\times10^{-7}$ & 1.1$\times10^{38}$ & 1.6$\times10^{-1}$\\
0.05 & M & 0.5 & 70.0 & 160.0 & Pass & Pass & Fail & Fail & Pass & Pass & Pass & Fail & Pass & Fail & Fail & Fail & Pass &2.6$\times10^{-7}$ & 2.1$\times10^{38}$ & 1.6$\times10^{-1}$\\
0.05 & M & 0.5 & 90.0 & 1.0 & Fail & Fail & Fail & Fail & Pass & Fail & Pass & Fail & Pass & Fail & Fail & Pass & Pass &3.2$\times10^{-8}$ & 2.6$\times10^{37}$ & 1.6$\times10^{-1}$\\
0.05 & M & 0.5 & 90.0 & 10.0 & Pass & Pass & Fail & Fail & Pass & Fail & Pass & Fail & Fail & Fail & Fail & Fail & Pass &8.0$\times10^{-8}$ & 6.5$\times10^{37}$ & 1.6$\times10^{-1}$\\
0.05 & M & 0.5 & 90.0 & 40.0 & Pass & Pass & Fail & Fail & Pass & Fail & Pass & Fail & Pass & Fail & Fail & Fail & Pass &1.3$\times10^{-7}$ & 1.1$\times10^{38}$ & 1.6$\times10^{-1}$\\
0.05 & M & 0.5 & 90.0 & 160.0 & Pass & Pass & Fail & Fail & Pass & Pass & Pass & Fail & Pass & Fail & Fail & Fail & Pass &2.5$\times10^{-7}$ & 2.1$\times10^{38}$ & 1.6$\times10^{-1}$\\
0.05 & M & 0.94 & 10.0 & 1.0 & Pass & Fail & Fail & Fail & Pass & Pass & Fail & Fail & Pass & Fail & Fail & Pass & Pass &8.5$\times10^{-9}$ & 4.1$\times10^{37}$ & 9.3$\times10^{-1}$\\
0.05 & M & 0.94 & 10.0 & 10.0 & Pass & Pass & Fail & Fail & Pass & Pass & Pass & Pass & Pass & Pass & Fail & Fail & Pass &2.3$\times10^{-8}$ & 1.1$\times10^{38}$ & 9.3$\times10^{-1}$\\
0.05 & M & 0.94 & 10.0 & 40.0 & Pass & Pass & Fail & Fail & Pass & Pass & Pass & Pass & Pass & Pass & Fail & Pass & Pass &3.9$\times10^{-8}$ & 1.8$\times10^{38}$ & 9.3$\times10^{-1}$\\
0.05 & M & 0.94 & 10.0 & 160.0 & Pass & Pass & Fail & Fail & Pass & Pass & Pass & Pass & Pass & Pass & Fail & Fail & Pass &6.6$\times10^{-8}$ & 3.2$\times10^{38}$ & 9.3$\times10^{-1}$\\
0.05 & M & 0.94 & 30.0 & 1.0 & Pass & Fail & Fail & Fail & Pass & Pass & Pass & Fail & Pass & Fail & Fail & Pass & Pass &8.4$\times10^{-9}$ & 4.0$\times10^{37}$ & 9.3$\times10^{-1}$\\
0.05 & M & 0.94 & 30.0 & 10.0 & Pass & Pass & Fail & Fail & Pass & Pass & Pass & Pass & Pass & Pass & Fail & Fail & Pass &2.2$\times10^{-8}$ & 1.1$\times10^{38}$ & 9.3$\times10^{-1}$\\
0.05 & M & 0.94 & 30.0 & 40.0 & Pass & Pass & Fail & Fail & Pass & Pass & Pass & Pass & Pass & Pass & Fail & Fail & Pass &3.8$\times10^{-8}$ & 1.8$\times10^{38}$ & 9.3$\times10^{-1}$\\
0.05 & M & 0.94 & 30.0 & 160.0 & Pass & Pass & Fail & Fail & Pass & Pass & Pass & Pass & Pass & Pass & Fail & Fail & Pass &6.6$\times10^{-8}$ & 3.2$\times10^{38}$ & 9.3$\times10^{-1}$\\
0.05 & M & 0.94 & 50.0 & 1.0 & Pass & Fail & Fail & Fail & Pass & Pass & Pass & Pass & Pass & Pass & Fail & Pass & Pass &8.2$\times10^{-9}$ & 3.9$\times10^{37}$ & 9.3$\times10^{-1}$\\
0.05 & M & 0.94 & 50.0 & 10.0 & Pass & Pass & Fail & Fail & Pass & Pass & Pass & Pass & Fail & Fail & Fail & Fail & Pass &2.1$\times10^{-8}$ & 1.0$\times10^{38}$ & 9.3$\times10^{-1}$\\
0.05 & M & 0.94 & 50.0 & 40.0 & Pass & Pass & Fail & Fail & Pass & Pass & Pass & Pass & Pass & Pass & Fail & Fail & Pass &3.7$\times10^{-8}$ & 1.8$\times10^{38}$ & 9.3$\times10^{-1}$\\
0.05 & M & 0.94 & 50.0 & 160.0 & Pass & Pass & Fail & Fail & Pass & Pass & Pass & Pass & Pass & Pass & Fail & Fail & Pass &6.5$\times10^{-8}$ & 3.1$\times10^{38}$ & 9.3$\times10^{-1}$\\
0.05 & M & 0.94 & 70.0 & 1.0 & Pass & Fail & Fail & Fail & Pass & Fail & Pass & Pass & Pass & Fail & Fail & Pass & Pass &8.0$\times10^{-9}$ & 3.8$\times10^{37}$ & 9.3$\times10^{-1}$\\
0.05 & M & 0.94 & 70.0 & 10.0 & Pass & Pass & Fail & Fail & Pass & Fail & Pass & Fail & Fail & Fail & Fail & Fail & Pass &2.0$\times10^{-8}$ & 9.8$\times10^{37}$ & 9.3$\times10^{-1}$\\
0.05 & M & 0.94 & 70.0 & 40.0 & Pass & Pass & Fail & Fail & Pass & Fail & Pass & Fail & Fail & Fail & Fail & Pass & Pass &3.7$\times10^{-8}$ & 1.8$\times10^{38}$ & 9.3$\times10^{-1}$\\
0.05 & M & 0.94 & 70.0 & 160.0 & Pass & Pass & Fail & Fail & Pass & Pass & Pass & Fail & Pass & Fail & Fail & Pass & Pass &6.6$\times10^{-8}$ & 3.2$\times10^{38}$ & 9.3$\times10^{-1}$\\
0.05 & M & 0.94 & 90.0 & 1.0 & Pass & Fail & Fail & Fail & Pass & Fail & Pass & Fail & Pass & Fail & Fail & Fail & Pass &7.7$\times10^{-9}$ & 3.7$\times10^{37}$ & 9.3$\times10^{-1}$\\
0.05 & M & 0.94 & 90.0 & 10.0 & Pass & Pass & Fail & Fail & Pass & Fail & Pass & Fail & Fail & Fail & Fail & Fail & Pass &2.0$\times10^{-8}$ & 9.7$\times10^{37}$ & 9.3$\times10^{-1}$\\
0.05 & M & 0.94 & 90.0 & 40.0 & Pass & Pass & Fail & Fail & Pass & Fail & Pass & Fail & Fail & Fail & Fail & Pass & Pass &3.7$\times10^{-8}$ & 1.8$\times10^{38}$ & 9.3$\times10^{-1}$\\
0.05 & M & 0.94 & 90.0 & 160.0 & Pass & Pass & Fail & Fail & Pass & Pass & Pass & Fail & Pass & Fail & Fail & Pass & Fail &6.8$\times10^{-8}$ & 3.3$\times10^{38}$ & 9.3$\times10^{-1}$\\
0.10 & S & -0.94 & 10.0 & 1.0 & Fail & Fail & Pass & Fail & Fail & Pass & Pass & Fail & Pass & Fail & Fail & Pass & Pass &1.0$\times10^{-8}$ & 2.2$\times10^{35}$ & 4.2$\times10^{-3}$\\
0.10 & S & -0.94 & 10.0 & 10.0 & Pass & Pass & Fail & Fail & Pass & Pass & Fail & Fail & Pass & Fail & Fail & Fail & Pass &2.9$\times10^{-7}$ & 6.2$\times10^{36}$ & 4.2$\times10^{-3}$\\
0.10 & S & -0.94 & 10.0 & 40.0 & Pass & Pass & Fail & Fail & Pass & Pass & Fail & Fail & Pass & Fail & Fail & Pass & Pass &9.8$\times10^{-7}$ & 2.1$\times10^{37}$ & 4.2$\times10^{-3}$\\
0.10 & S & -0.94 & 10.0 & 160.0 & Pass & Pass & Fail & Fail & Pass & Pass & Pass & Fail & Pass & Fail & Fail & Fail & Pass &1.8$\times10^{-6}$ & 3.9$\times10^{37}$ & 4.2$\times10^{-3}$\\
0.10 & S & -0.94 & 30.0 & 1.0 & Fail & Fail & Pass & Fail & Pass & Pass & Pass & Fail & Pass & Fail & Fail & Pass & Pass &9.9$\times10^{-9}$ & 2.2$\times10^{35}$ & 4.2$\times10^{-3}$\\
0.10 & S & -0.94 & 30.0 & 10.0 & Pass & Pass & Fail & Fail & Pass & Pass & Fail & Fail & Pass & Fail & Fail & Fail & Fail &2.8$\times10^{-7}$ & 6.1$\times10^{36}$ & 4.2$\times10^{-3}$\\
0.10 & S & -0.94 & 30.0 & 40.0 & Pass & Pass & Fail & Fail & Pass & Pass & Pass & Fail & Pass & Fail & Fail & Fail & Pass &9.3$\times10^{-7}$ & 2.0$\times10^{37}$ & 4.2$\times10^{-3}$\\
0.10 & S & -0.94 & 30.0 & 160.0 & Pass & Pass & Fail & Fail & Pass & Pass & Pass & Fail & Pass & Fail & Fail & Fail & Fail &1.7$\times10^{-6}$ & 3.7$\times10^{37}$ & 4.2$\times10^{-3}$\\
0.10 & S & -0.94 & 50.0 & 1.0 & Fail & Pass & Pass & Fail & Pass & Pass & Pass & Fail & Pass & Fail & Fail & Pass & Pass &9.7$\times10^{-9}$ & 2.1$\times10^{35}$ & 4.2$\times10^{-3}$\\
0.10 & S & -0.94 & 50.0 & 10.0 & Pass & Pass & Fail & Fail & Pass & Pass & Fail & Fail & Pass & Fail & Fail & Fail & Fail &2.8$\times10^{-7}$ & 6.2$\times10^{36}$ & 4.2$\times10^{-3}$\\
0.10 & S & -0.94 & 50.0 & 40.0 & Pass & Pass & Fail & Fail & Pass & Pass & Fail & Fail & Pass & Fail & Fail & Fail & Pass &9.5$\times10^{-7}$ & 2.1$\times10^{37}$ & 4.2$\times10^{-3}$\\
0.10 & S & -0.94 & 50.0 & 160.0 & Pass & Pass & Fail & Fail & Pass & Pass & Pass & Fail & Pass & Fail & Fail & Fail & Fail &1.7$\times10^{-6}$ & 3.7$\times10^{37}$ & 4.2$\times10^{-3}$\\
0.10 & S & -0.94 & 70.0 & 1.0 & Fail & Pass & Pass & Fail & Pass & Pass & Fail & Fail & Pass & Fail & Fail & Pass & Pass &9.8$\times10^{-9}$ & 2.1$\times10^{35}$ & 4.2$\times10^{-3}$\\
0.10 & S & -0.94 & 70.0 & 10.0 & Pass & Pass & Fail & Fail & Pass & Pass & Pass & Fail & Pass & Fail & Fail & Fail & Fail &3.0$\times10^{-7}$ & 6.6$\times10^{36}$ & 4.2$\times10^{-3}$\\
0.10 & S & -0.94 & 70.0 & 40.0 & Pass & Pass & Fail & Fail & Pass & Pass & Pass & Fail & Pass & Fail & Fail & Fail & Fail &9.9$\times10^{-7}$ & 2.2$\times10^{37}$ & 4.2$\times10^{-3}$\\
0.10 & S & -0.94 & 70.0 & 160.0 & Pass & Pass & Fail & Fail & Pass & Pass & Pass & Fail & Pass & Fail & Fail & Fail & Fail &1.8$\times10^{-6}$ & 3.9$\times10^{37}$ & 4.2$\times10^{-3}$\\
0.10 & S & -0.94 & 90.0 & 1.0 & Fail & Pass & Pass & Fail & Pass & Pass & Pass & Fail & Pass & Fail & Fail & Pass & Pass &9.6$\times10^{-9}$ & 2.1$\times10^{35}$ & 4.2$\times10^{-3}$\\
0.10 & S & -0.94 & 90.0 & 10.0 & Pass & Pass & Fail & Fail & Pass & Pass & Pass & Fail & Pass & Fail & Fail & Fail & Fail &3.1$\times10^{-7}$ & 6.8$\times10^{36}$ & 4.2$\times10^{-3}$\\
0.10 & S & -0.94 & 90.0 & 40.0 & Pass & Fail & Fail & Fail & Pass & Pass & Pass & Fail & Pass & Fail & Fail & Fail & Fail &1.0$\times10^{-6}$ & 2.2$\times10^{37}$ & 4.2$\times10^{-3}$\\
0.10 & S & -0.94 & 90.0 & 160.0 & Pass & Pass & Fail & Fail & Pass & Fail & Pass & Fail & Pass & Fail & Fail & Fail & Fail &1.8$\times10^{-6}$ & 3.9$\times10^{37}$ & 4.2$\times10^{-3}$\\
0.10 & S & -0.5 & 10.0 & 1.0 & Fail & Fail & Pass & Fail & Fail & Pass & Pass & Fail & Pass & Fail & Fail & Pass & Pass &5.7$\times10^{-8}$ & 3.5$\times10^{35}$ & 1.2$\times10^{-3}$\\
0.10 & S & -0.5 & 10.0 & 10.0 & Pass & Pass & Fail & Fail & Pass & Pass & Fail & Fail & Pass & Fail & Fail & Fail & Fail &1.7$\times10^{-6}$ & 1.0$\times10^{37}$ & 1.2$\times10^{-3}$\\
0.10 & S & -0.5 & 10.0 & 40.0 & Pass & Pass & Fail & Fail & Pass & Pass & Fail & Fail & Pass & Fail & Fail & Pass & Pass &3.7$\times10^{-6}$ & 2.3$\times10^{37}$ & 1.2$\times10^{-3}$\\
0.10 & S & -0.5 & 10.0 & 160.0 & Pass & Pass & Fail & Fail & Pass & Pass & Fail & Fail & Pass & Fail & Fail & Pass & Pass &5.9$\times10^{-6}$ & 3.6$\times10^{37}$ & 1.2$\times10^{-3}$\\
0.10 & S & -0.5 & 30.0 & 1.0 & Fail & Pass & Pass & Fail & Fail & Pass & Pass & Fail & Pass & Fail & Fail & Pass & Pass &5.6$\times10^{-8}$ & 3.4$\times10^{35}$ & 1.2$\times10^{-3}$\\
0.10 & S & -0.5 & 30.0 & 10.0 & Pass & Pass & Fail & Fail & Pass & Pass & Fail & Fail & Pass & Fail & Fail & Fail & Pass &1.7$\times10^{-6}$ & 1.0$\times10^{37}$ & 1.2$\times10^{-3}$\\
0.10 & S & -0.5 & 30.0 & 40.0 & Pass & Fail & Fail & Fail & Pass & Pass & Fail & Pass & Pass & Fail & Fail & Pass & Pass &3.8$\times10^{-6}$ & 2.3$\times10^{37}$ & 1.2$\times10^{-3}$\\
0.10 & S & -0.5 & 30.0 & 160.0 & Pass & Fail & Fail & Fail & Pass & Pass & Fail & Fail & Pass & Fail & Fail & Pass & Pass &6.2$\times10^{-6}$ & 3.8$\times10^{37}$ & 1.2$\times10^{-3}$\\
0.10 & S & -0.5 & 50.0 & 1.0 & Fail & Pass & Pass & Fail & Pass & Pass & Pass & Fail & Pass & Fail & Fail & Pass & Pass &5.5$\times10^{-8}$ & 3.4$\times10^{35}$ & 1.2$\times10^{-3}$\\
0.10 & S & -0.5 & 50.0 & 10.0 & Pass & Pass & Fail & Fail & Pass & Pass & Fail & Fail & Pass & Fail & Fail & Fail & Pass &1.6$\times10^{-6}$ & 9.9$\times10^{36}$ & 1.2$\times10^{-3}$\\
0.10 & S & -0.5 & 50.0 & 40.0 & Pass & Fail & Fail & Fail & Pass & Pass & Fail & Fail & Pass & Fail & Fail & Pass & Pass &3.6$\times10^{-6}$ & 2.2$\times10^{37}$ & 1.2$\times10^{-3}$\\
0.10 & S & -0.5 & 50.0 & 160.0 & Pass & Fail & Fail & Fail & Pass & Fail & Fail & Fail & Pass & Fail & Fail & Pass & Pass &5.9$\times10^{-6}$ & 3.7$\times10^{37}$ & 1.2$\times10^{-3}$\\
0.10 & S & -0.5 & 70.0 & 1.0 & Fail & Fail & Pass & Fail & Pass & Pass & Pass & Fail & Pass & Fail & Fail & Pass & Pass &5.5$\times10^{-8}$ & 3.4$\times10^{35}$ & 1.2$\times10^{-3}$\\
0.10 & S & -0.5 & 70.0 & 10.0 & Pass & Fail & Fail & Fail & Pass & Pass & Fail & Fail & Pass & Fail & Fail & Fail & Pass &1.6$\times10^{-6}$ & 1.0$\times10^{37}$ & 1.2$\times10^{-3}$\\
0.10 & S & -0.5 & 70.0 & 40.0 & Pass & Fail & Fail & Fail & Pass & Pass & Pass & Pass & Pass & Pass & Fail & Pass & Pass &3.7$\times10^{-6}$ & 2.3$\times10^{37}$ & 1.2$\times10^{-3}$\\
0.10 & S & -0.5 & 70.0 & 160.0 & Pass & Fail & Fail & Fail & Pass & Fail & Pass & Pass & Pass & Fail & Fail & Pass & Pass &6.1$\times10^{-6}$ & 3.7$\times10^{37}$ & 1.2$\times10^{-3}$\\
0.10 & S & -0.5 & 90.0 & 1.0 & Fail & Fail & Pass & Fail & Pass & Pass & Pass & Fail & Pass & Fail & Fail & Pass & Pass &5.4$\times10^{-8}$ & 3.4$\times10^{35}$ & 1.2$\times10^{-3}$\\
0.10 & S & -0.5 & 90.0 & 10.0 & Pass & Fail & Fail & Fail & Pass & Pass & Fail & Fail & Pass & Fail & Fail & Fail & Pass &1.6$\times10^{-6}$ & 1.0$\times10^{37}$ & 1.2$\times10^{-3}$\\
0.10 & S & -0.5 & 90.0 & 40.0 & Pass & Fail & Fail & Fail & Pass & Fail & Pass & Pass & Pass & Fail & Fail & Pass & Pass &3.7$\times10^{-6}$ & 2.3$\times10^{37}$ & 1.2$\times10^{-3}$\\
0.10 & S & -0.5 & 90.0 & 160.0 & Pass & Fail & Fail & Fail & Pass & Fail & Pass & Fail & Pass & Fail & Fail & Pass & Pass &6.2$\times10^{-6}$ & 3.9$\times10^{37}$ & 1.2$\times10^{-3}$\\
0.10 & S & 0.0 & 10.0 & 1.0 & Fail & Pass & Pass & Fail & Fail & Pass & Pass & Fail & Pass & Fail & Fail & Pass & Pass &3.8$\times10^{-8}$ & 9.0$\times10^{34}$ & 4.5$\times10^{-4}$\\
0.10 & S & 0.0 & 10.0 & 10.0 & Pass & Fail & Fail & Fail & Pass & Pass & Fail & Fail & Pass & Fail & Fail & Fail & Fail &5.3$\times10^{-7}$ & 1.2$\times10^{36}$ & 4.5$\times10^{-4}$\\
0.10 & S & 0.0 & 10.0 & 40.0 & Fail & Fail & Fail & Fail & Pass & Pass & Fail & Fail & Pass & Fail & Fail & Pass & Pass &1.4$\times10^{-6}$ & 3.4$\times10^{36}$ & 4.5$\times10^{-4}$\\
0.10 & S & 0.0 & 10.0 & 160.0 & Pass & Fail & Fail & Fail & Pass & Pass & Fail & Fail & Pass & Fail & Fail & Pass & Pass &3.0$\times10^{-6}$ & 6.9$\times10^{36}$ & 4.5$\times10^{-4}$\\
0.10 & S & 0.0 & 30.0 & 1.0 & Fail & Pass & Pass & Fail & Pass & Pass & Pass & Fail & Pass & Fail & Fail & Pass & Pass &3.7$\times10^{-8}$ & 8.7$\times10^{34}$ & 4.5$\times10^{-4}$\\
0.10 & S & 0.0 & 30.0 & 10.0 & Pass & Fail & Fail & Fail & Pass & Pass & Fail & Fail & Pass & Fail & Fail & Fail & Fail &5.0$\times10^{-7}$ & 1.2$\times10^{36}$ & 4.5$\times10^{-4}$\\
0.10 & S & 0.0 & 30.0 & 40.0 & Pass & Pass & Fail & Fail & Pass & Pass & Fail & Pass & Pass & Fail & Fail & Pass & Pass &1.4$\times10^{-6}$ & 3.3$\times10^{36}$ & 4.5$\times10^{-4}$\\
0.10 & S & 0.0 & 30.0 & 160.0 & Pass & Pass & Fail & Fail & Pass & Pass & Fail & Pass & Pass & Fail & Fail & Pass & Pass &2.9$\times10^{-6}$ & 6.9$\times10^{36}$ & 4.5$\times10^{-4}$\\
0.10 & S & 0.0 & 50.0 & 1.0 & Fail & Fail & Pass & Fail & Pass & Pass & Pass & Fail & Pass & Fail & Fail & Pass & Pass &3.6$\times10^{-8}$ & 8.5$\times10^{34}$ & 4.5$\times10^{-4}$\\
0.10 & S & 0.0 & 50.0 & 10.0 & Fail & Pass & Fail & Fail & Pass & Pass & Pass & Pass & Pass & Pass & Fail & Fail & Fail &4.8$\times10^{-7}$ & 1.1$\times10^{36}$ & 4.5$\times10^{-4}$\\
0.10 & S & 0.0 & 50.0 & 40.0 & Pass & Fail & Fail & Fail & Pass & Pass & Pass & Fail & Pass & Fail & Fail & Pass & Pass &1.4$\times10^{-6}$ & 3.2$\times10^{36}$ & 4.5$\times10^{-4}$\\
0.10 & S & 0.0 & 50.0 & 160.0 & Pass & Fail & Fail & Fail & Pass & Pass & Pass & Fail & Pass & Fail & Fail & Pass & Pass &3.0$\times10^{-6}$ & 7.1$\times10^{36}$ & 4.5$\times10^{-4}$\\
0.10 & S & 0.0 & 70.0 & 1.0 & Fail & Fail & Pass & Fail & Pass & Pass & Pass & Pass & Pass & Pass & Fail & Pass & Pass &3.6$\times10^{-8}$ & 8.4$\times10^{34}$ & 4.5$\times10^{-4}$\\
0.10 & S & 0.0 & 70.0 & 10.0 & Fail & Fail & Fail & Fail & Pass & Pass & Pass & Fail & Pass & Fail & Fail & Fail & Fail &4.7$\times10^{-7}$ & 1.1$\times10^{36}$ & 4.5$\times10^{-4}$\\
0.10 & S & 0.0 & 70.0 & 40.0 & Pass & Fail & Fail & Fail & Pass & Pass & Pass & Fail & Pass & Fail & Fail & Pass & Pass &1.4$\times10^{-6}$ & 3.3$\times10^{36}$ & 4.5$\times10^{-4}$\\
0.10 & S & 0.0 & 70.0 & 160.0 & Fail & Fail & Fail & Fail & Pass & Pass & Pass & Fail & Pass & Fail & Fail & Pass & Pass &3.2$\times10^{-6}$ & 7.5$\times10^{36}$ & 4.5$\times10^{-4}$\\
0.10 & S & 0.0 & 90.0 & 1.0 & Fail & Fail & Pass & Fail & Pass & Pass & Pass & Fail & Pass & Fail & Fail & Pass & Pass &3.5$\times10^{-8}$ & 8.2$\times10^{34}$ & 4.5$\times10^{-4}$\\
0.10 & S & 0.0 & 90.0 & 10.0 & Fail & Fail & Fail & Fail & Pass & Pass & Pass & Fail & Pass & Fail & Fail & Fail & Fail &4.7$\times10^{-7}$ & 1.1$\times10^{36}$ & 4.5$\times10^{-4}$\\
0.10 & S & 0.0 & 90.0 & 40.0 & Pass & Fail & Fail & Fail & Pass & Pass & Pass & Fail & Pass & Fail & Fail & Pass & Pass &1.4$\times10^{-6}$ & 3.4$\times10^{36}$ & 4.5$\times10^{-4}$\\
0.10 & S & 0.0 & 90.0 & 160.0 & Fail & Fail & Fail & Fail & Pass & Pass & Pass & Fail & Pass & Fail & Fail & Pass & Pass &3.3$\times10^{-6}$ & 7.7$\times10^{36}$ & 4.5$\times10^{-4}$\\
0.10 & S & 0.5 & 10.0 & 1.0 & Fail & Pass & Pass & Fail & Pass & Pass & Pass & Fail & Pass & Fail & Fail & Pass & Pass &1.7$\times10^{-7}$ & 3.6$\times10^{36}$ & 4.2$\times10^{-3}$\\
0.10 & S & 0.5 & 10.0 & 10.0 & Pass & Fail & Fail & Fail & Pass & Pass & Pass & Pass & Pass & Pass & Fail & Fail & Pass &2.3$\times10^{-6}$ & 4.9$\times10^{37}$ & 4.2$\times10^{-3}$\\
0.10 & S & 0.5 & 10.0 & 40.0 & Pass & Pass & Fail & Fail & Pass & Pass & Pass & Pass & Pass & Pass & Fail & Fail & Pass &4.1$\times10^{-6}$ & 8.8$\times10^{37}$ & 4.2$\times10^{-3}$\\
0.10 & S & 0.5 & 10.0 & 160.0 & Pass & Pass & Fail & Fail & Pass & Pass & Pass & Pass & Pass & Pass & Fail & Fail & Pass &6.2$\times10^{-6}$ & 1.3$\times10^{38}$ & 4.2$\times10^{-3}$\\
0.10 & S & 0.5 & 30.0 & 1.0 & Fail & Fail & Pass & Fail & Pass & Pass & Pass & Fail & Pass & Fail & Fail & Pass & Pass &1.6$\times10^{-7}$ & 3.5$\times10^{36}$ & 4.2$\times10^{-3}$\\
0.10 & S & 0.5 & 30.0 & 10.0 & Pass & Fail & Fail & Fail & Pass & Pass & Pass & Pass & Pass & Pass & Fail & Fail & Pass &2.2$\times10^{-6}$ & 4.7$\times10^{37}$ & 4.2$\times10^{-3}$\\
0.10 & S & 0.5 & 30.0 & 40.0 & Pass & Pass & Fail & Fail & Pass & Pass & Pass & Pass & Pass & Pass & Fail & Fail & Pass &4.2$\times10^{-6}$ & 9.0$\times10^{37}$ & 4.2$\times10^{-3}$\\
0.10 & S & 0.5 & 30.0 & 160.0 & Pass & Pass & Fail & Fail & Pass & Pass & Pass & Pass & Pass & Pass & Fail & Fail & Pass &6.6$\times10^{-6}$ & 1.4$\times10^{38}$ & 4.2$\times10^{-3}$\\
0.10 & S & 0.5 & 50.0 & 1.0 & Fail & Fail & Pass & Fail & Pass & Pass & Pass & Fail & Pass & Fail & Fail & Pass & Pass &1.6$\times10^{-7}$ & 3.4$\times10^{36}$ & 4.2$\times10^{-3}$\\
0.10 & S & 0.5 & 50.0 & 10.0 & Pass & Fail & Fail & Fail & Pass & Pass & Pass & Fail & Pass & Fail & Fail & Fail & Pass &2.0$\times10^{-6}$ & 4.4$\times10^{37}$ & 4.2$\times10^{-3}$\\
0.10 & S & 0.5 & 50.0 & 40.0 & Pass & Pass & Fail & Fail & Pass & Pass & Pass & Fail & Pass & Fail & Fail & Fail & Pass &3.9$\times10^{-6}$ & 8.3$\times10^{37}$ & 4.2$\times10^{-3}$\\
0.10 & S & 0.5 & 50.0 & 160.0 & Pass & Fail & Fail & Fail & Pass & Pass & Pass & Fail & Pass & Fail & Fail & Fail & Pass &6.1$\times10^{-6}$ & 1.3$\times10^{38}$ & 4.2$\times10^{-3}$\\
0.10 & S & 0.5 & 70.0 & 1.0 & Fail & Fail & Pass & Fail & Pass & Pass & Pass & Pass & Pass & Pass & Fail & Pass & Pass &1.6$\times10^{-7}$ & 3.4$\times10^{36}$ & 4.2$\times10^{-3}$\\
0.10 & S & 0.5 & 70.0 & 10.0 & Fail & Fail & Fail & Fail & Pass & Pass & Pass & Fail & Pass & Fail & Fail & Fail & Fail &1.9$\times10^{-6}$ & 4.1$\times10^{37}$ & 4.2$\times10^{-3}$\\
0.10 & S & 0.5 & 70.0 & 40.0 & Fail & Pass & Fail & Fail & Pass & Fail & Pass & Fail & Pass & Fail & Fail & Fail & Fail &3.8$\times10^{-6}$ & 8.1$\times10^{37}$ & 4.2$\times10^{-3}$\\
0.10 & S & 0.5 & 70.0 & 160.0 & Pass & Fail & Fail & Fail & Pass & Fail & Pass & Fail & Pass & Fail & Fail & Fail & Pass &6.1$\times10^{-6}$ & 1.3$\times10^{38}$ & 4.2$\times10^{-3}$\\
0.10 & S & 0.5 & 90.0 & 1.0 & Fail & Fail & Pass & Fail & Pass & Fail & Pass & Fail & Pass & Fail & Fail & Pass & Pass &1.5$\times10^{-7}$ & 3.3$\times10^{36}$ & 4.2$\times10^{-3}$\\
0.10 & S & 0.5 & 90.0 & 10.0 & Fail & Fail & Fail & Fail & Pass & Fail & Pass & Fail & Pass & Fail & Fail & Fail & Fail &1.9$\times10^{-6}$ & 4.1$\times10^{37}$ & 4.2$\times10^{-3}$\\
0.10 & S & 0.5 & 90.0 & 40.0 & Fail & Pass & Fail & Fail & Pass & Fail & Pass & Fail & Pass & Fail & Fail & Fail & Fail &3.8$\times10^{-6}$ & 8.3$\times10^{37}$ & 4.2$\times10^{-3}$\\
0.10 & S & 0.5 & 90.0 & 160.0 & Pass & Fail & Fail & Fail & Pass & Fail & Pass & Fail & Pass & Fail & Fail & Fail & Fail &6.4$\times10^{-6}$ & 1.4$\times10^{38}$ & 4.2$\times10^{-3}$\\
0.10 & S & 0.94 & 10.0 & 1.0 & Fail & Fail & Pass & Fail & Pass & Pass & Fail & Fail & Pass & Fail & Fail & Pass & Pass &7.3$\times10^{-9}$ & 4.4$\times10^{35}$ & 1.2$\times10^{-2}$\\
0.10 & S & 0.94 & 10.0 & 10.0 & Pass & Pass & Fail & Fail & Pass & Pass & Pass & Pass & Pass & Pass & Fail & Pass & Pass &1.1$\times10^{-7}$ & 6.6$\times10^{36}$ & 1.2$\times10^{-2}$\\
0.10 & S & 0.94 & 10.0 & 40.0 & Pass & Fail & Fail & Fail & Pass & Pass & Fail & Pass & Pass & Fail & Fail & Fail & Pass &5.5$\times10^{-7}$ & 3.3$\times10^{37}$ & 1.2$\times10^{-2}$\\
0.10 & S & 0.94 & 10.0 & 160.0 & Fail & Fail & Fail & Fail & Pass & Pass & Pass & Pass & Pass & Pass & Fail & Fail & Pass &9.9$\times10^{-7}$ & 6.0$\times10^{37}$ & 1.2$\times10^{-2}$\\
0.10 & S & 0.94 & 30.0 & 1.0 & Fail & Fail & Pass & Fail & Pass & Pass & Pass & Fail & Pass & Fail & Fail & Pass & Pass &7.1$\times10^{-9}$ & 4.3$\times10^{35}$ & 1.2$\times10^{-2}$\\
0.10 & S & 0.94 & 30.0 & 10.0 & Fail & Pass & Fail & Fail & Pass & Pass & Pass & Pass & Pass & Pass & Fail & Pass & Pass &1.1$\times10^{-7}$ & 6.5$\times10^{36}$ & 1.2$\times10^{-2}$\\
0.10 & S & 0.94 & 30.0 & 40.0 & Pass & Fail & Fail & Fail & Pass & Pass & Pass & Fail & Pass & Fail & Fail & Pass & Pass &6.4$\times10^{-7}$ & 3.9$\times10^{37}$ & 1.2$\times10^{-2}$\\
0.10 & S & 0.94 & 30.0 & 160.0 & Fail & Fail & Fail & Fail & Pass & Pass & Pass & Fail & Pass & Fail & Fail & Pass & Pass &1.1$\times10^{-6}$ & 6.9$\times10^{37}$ & 1.2$\times10^{-2}$\\
0.10 & S & 0.94 & 50.0 & 1.0 & Fail & Fail & Pass & Fail & Pass & Pass & Pass & Pass & Pass & Pass & Fail & Pass & Pass &6.8$\times10^{-9}$ & 4.1$\times10^{35}$ & 1.2$\times10^{-2}$\\
0.10 & S & 0.94 & 50.0 & 10.0 & Fail & Pass & Fail & Fail & Pass & Pass & Pass & Pass & Fail & Fail & Fail & Pass & Fail &1.1$\times10^{-7}$ & 6.4$\times10^{36}$ & 1.2$\times10^{-2}$\\
0.10 & S & 0.94 & 50.0 & 40.0 & Pass & Fail & Fail & Fail & Pass & Pass & Pass & Fail & Pass & Fail & Fail & Pass & Pass &7.2$\times10^{-7}$ & 4.3$\times10^{37}$ & 1.2$\times10^{-2}$\\
0.10 & S & 0.94 & 50.0 & 160.0 & Fail & Fail & Fail & Fail & Pass & Fail & Pass & Fail & Pass & Fail & Fail & Pass & Fail &1.3$\times10^{-6}$ & 8.0$\times10^{37}$ & 1.2$\times10^{-2}$\\
0.10 & S & 0.94 & 70.0 & 1.0 & Fail & Fail & Pass & Fail & Pass & Fail & Pass & Pass & Pass & Fail & Fail & Pass & Pass &6.7$\times10^{-9}$ & 4.0$\times10^{35}$ & 1.2$\times10^{-2}$\\
0.10 & S & 0.94 & 70.0 & 10.0 & Fail & Pass & Fail & Fail & Pass & Pass & Pass & Fail & Pass & Fail & Fail & Pass & Fail &1.1$\times10^{-7}$ & 6.8$\times10^{36}$ & 1.2$\times10^{-2}$\\
0.10 & S & 0.94 & 70.0 & 40.0 & Pass & Fail & Fail & Fail & Pass & Fail & Pass & Fail & Pass & Fail & Fail & Pass & Fail &7.2$\times10^{-7}$ & 4.3$\times10^{37}$ & 1.2$\times10^{-2}$\\
0.10 & S & 0.94 & 70.0 & 160.0 & Pass & Fail & Fail & Fail & Pass & Fail & Pass & Fail & Pass & Fail & Fail & Pass & Pass &1.4$\times10^{-6}$ & 8.5$\times10^{37}$ & 1.2$\times10^{-2}$\\
0.10 & S & 0.94 & 90.0 & 1.0 & Fail & Fail & Fail & Fail & Pass & Fail & Pass & Pass & Pass & Fail & Fail & Pass & Pass &6.7$\times10^{-9}$ & 4.0$\times10^{35}$ & 1.2$\times10^{-2}$\\
0.10 & S & 0.94 & 90.0 & 10.0 & Fail & Pass & Fail & Fail & Pass & Fail & Pass & Pass & Pass & Fail & Fail & Pass & Fail &1.2$\times10^{-7}$ & 7.1$\times10^{36}$ & 1.2$\times10^{-2}$\\
0.10 & S & 0.94 & 90.0 & 40.0 & Pass & Fail & Fail & Fail & Pass & Fail & Pass & Fail & Pass & Fail & Fail & Pass & Fail &7.1$\times10^{-7}$ & 4.3$\times10^{37}$ & 1.2$\times10^{-2}$\\
0.10 & S & 0.94 & 90.0 & 160.0 & Pass & Fail & Fail & Fail & Pass & Fail & Pass & Fail & Pass & Fail & Fail & Pass & Pass &1.4$\times10^{-6}$ & 8.5$\times10^{37}$ & 1.2$\times10^{-2}$\\
0.10 & M & -0.94 & 10.0 & 1.0 & Fail & Fail & Fail & Fail & Pass & Pass & Fail & Fail & Pass & Fail & Fail & Fail & Pass &1.8$\times10^{-8}$ & 1.7$\times10^{37}$ & 1.8$\times10^{-1}$\\
0.10 & M & -0.94 & 10.0 & 10.0 & Pass & Pass & Fail & Fail & Pass & Pass & Pass & Fail & Pass & Fail & Fail & Fail & Fail &5.6$\times10^{-8}$ & 5.3$\times10^{37}$ & 1.8$\times10^{-1}$\\
0.10 & M & -0.94 & 10.0 & 40.0 & Pass & Pass & Fail & Fail & Pass & Pass & Pass & Fail & Pass & Fail & Fail & Fail & Fail &8.8$\times10^{-8}$ & 8.4$\times10^{37}$ & 1.8$\times10^{-1}$\\
0.10 & M & -0.94 & 10.0 & 160.0 & Pass & Pass & Fail & Fail & Pass & Pass & Pass & Fail & Pass & Fail & Fail & Fail & Fail &1.5$\times10^{-7}$ & 1.4$\times10^{38}$ & 1.8$\times10^{-1}$\\
0.10 & M & -0.94 & 30.0 & 1.0 & Fail & Pass & Fail & Fail & Pass & Pass & Fail & Fail & Pass & Fail & Fail & Fail & Pass &1.7$\times10^{-8}$ & 1.7$\times10^{37}$ & 1.8$\times10^{-1}$\\
0.10 & M & -0.94 & 30.0 & 10.0 & Pass & Pass & Fail & Fail & Pass & Pass & Pass & Fail & Pass & Fail & Fail & Fail & Fail &5.4$\times10^{-8}$ & 5.1$\times10^{37}$ & 1.8$\times10^{-1}$\\
0.10 & M & -0.94 & 30.0 & 40.0 & Pass & Pass & Fail & Fail & Pass & Pass & Pass & Fail & Pass & Fail & Fail & Fail & Fail &8.5$\times10^{-8}$ & 8.1$\times10^{37}$ & 1.8$\times10^{-1}$\\
0.10 & M & -0.94 & 30.0 & 160.0 & Pass & Pass & Fail & Fail & Pass & Pass & Pass & Fail & Pass & Fail & Fail & Fail & Fail &1.4$\times10^{-7}$ & 1.3$\times10^{38}$ & 1.8$\times10^{-1}$\\
0.10 & M & -0.94 & 50.0 & 1.0 & Fail & Pass & Fail & Fail & Pass & Pass & Fail & Fail & Pass & Fail & Fail & Fail & Pass &1.7$\times10^{-8}$ & 1.6$\times10^{37}$ & 1.8$\times10^{-1}$\\
0.10 & M & -0.94 & 50.0 & 10.0 & Fail & Pass & Fail & Fail & Pass & Pass & Pass & Fail & Pass & Fail & Fail & Fail & Fail &5.0$\times10^{-8}$ & 4.8$\times10^{37}$ & 1.8$\times10^{-1}$\\
0.10 & M & -0.94 & 50.0 & 40.0 & Pass & Pass & Fail & Fail & Pass & Pass & Pass & Fail & Pass & Fail & Fail & Fail & Fail &7.9$\times10^{-8}$ & 7.5$\times10^{37}$ & 1.8$\times10^{-1}$\\
0.10 & M & -0.94 & 50.0 & 160.0 & Pass & Pass & Fail & Fail & Pass & Pass & Pass & Fail & Pass & Fail & Fail & Fail & Fail &1.3$\times10^{-7}$ & 1.2$\times10^{38}$ & 1.8$\times10^{-1}$\\
0.10 & M & -0.94 & 70.0 & 1.0 & Fail & Pass & Fail & Fail & Pass & Pass & Fail & Fail & Pass & Fail & Fail & Fail & Pass &1.6$\times10^{-8}$ & 1.5$\times10^{37}$ & 1.8$\times10^{-1}$\\
0.10 & M & -0.94 & 70.0 & 10.0 & Pass & Pass & Fail & Fail & Pass & Pass & Pass & Fail & Pass & Fail & Fail & Fail & Fail &4.7$\times10^{-8}$ & 4.5$\times10^{37}$ & 1.8$\times10^{-1}$\\
0.10 & M & -0.94 & 70.0 & 40.0 & Pass & Pass & Fail & Fail & Pass & Pass & Pass & Fail & Pass & Fail & Fail & Fail & Fail &7.3$\times10^{-8}$ & 7.0$\times10^{37}$ & 1.8$\times10^{-1}$\\
0.10 & M & -0.94 & 70.0 & 160.0 & Pass & Pass & Fail & Fail & Pass & Pass & Pass & Fail & Pass & Fail & Fail & Fail & Fail &1.2$\times10^{-7}$ & 1.2$\times10^{38}$ & 1.8$\times10^{-1}$\\
0.10 & M & -0.94 & 90.0 & 1.0 & Fail & Pass & Fail & Fail & Pass & Fail & Fail & Fail & Pass & Fail & Fail & Fail & Pass &1.5$\times10^{-8}$ & 1.5$\times10^{37}$ & 1.8$\times10^{-1}$\\
0.10 & M & -0.94 & 90.0 & 10.0 & Pass & Pass & Fail & Fail & Pass & Pass & Pass & Fail & Pass & Fail & Fail & Fail & Fail &4.5$\times10^{-8}$ & 4.3$\times10^{37}$ & 1.8$\times10^{-1}$\\
0.10 & M & -0.94 & 90.0 & 40.0 & Pass & Pass & Fail & Fail & Pass & Pass & Pass & Fail & Pass & Fail & Fail & Fail & Fail &7.1$\times10^{-8}$ & 6.8$\times10^{37}$ & 1.8$\times10^{-1}$\\
0.10 & M & -0.94 & 90.0 & 160.0 & Pass & Pass & Fail & Fail & Pass & Pass & Pass & Fail & Pass & Fail & Fail & Fail & Fail &1.2$\times10^{-7}$ & 1.1$\times10^{38}$ & 1.8$\times10^{-1}$\\
0.10 & M & -0.5 & 10.0 & 1.0 & Fail & Pass & Fail & Fail & Pass & Pass & Fail & Fail & Pass & Fail & Fail & Pass & Pass &3.9$\times10^{-8}$ & 9.2$\times10^{36}$ & 4.6$\times10^{-2}$\\
0.10 & M & -0.5 & 10.0 & 10.0 & Fail & Pass & Fail & Fail & Pass & Pass & Fail & Fail & Pass & Fail & Fail & Pass & Pass &1.3$\times10^{-7}$ & 3.1$\times10^{37}$ & 4.6$\times10^{-2}$\\
0.10 & M & -0.5 & 10.0 & 40.0 & Pass & Pass & Fail & Fail & Pass & Pass & Fail & Fail & Pass & Fail & Fail & Pass & Pass &2.1$\times10^{-7}$ & 5.0$\times10^{37}$ & 4.6$\times10^{-2}$\\
0.10 & M & -0.5 & 10.0 & 160.0 & Pass & Fail & Fail & Fail & Pass & Pass & Fail & Fail & Pass & Fail & Fail & Pass & Fail &3.8$\times10^{-7}$ & 9.0$\times10^{37}$ & 4.6$\times10^{-2}$\\
0.10 & M & -0.5 & 30.0 & 1.0 & Fail & Pass & Fail & Fail & Pass & Pass & Fail & Fail & Pass & Fail & Fail & Pass & Pass &3.8$\times10^{-8}$ & 8.9$\times10^{36}$ & 4.6$\times10^{-2}$\\
0.10 & M & -0.5 & 30.0 & 10.0 & Fail & Pass & Fail & Fail & Pass & Pass & Fail & Fail & Pass & Fail & Fail & Pass & Pass &1.2$\times10^{-7}$ & 2.9$\times10^{37}$ & 4.6$\times10^{-2}$\\
0.10 & M & -0.5 & 30.0 & 40.0 & Pass & Pass & Fail & Fail & Pass & Pass & Fail & Fail & Pass & Fail & Fail & Pass & Pass &2.0$\times10^{-7}$ & 4.8$\times10^{37}$ & 4.6$\times10^{-2}$\\
0.10 & M & -0.5 & 30.0 & 160.0 & Pass & Pass & Fail & Fail & Pass & Pass & Fail & Fail & Pass & Fail & Fail & Fail & Fail &3.7$\times10^{-7}$ & 8.7$\times10^{37}$ & 4.6$\times10^{-2}$\\
0.10 & M & -0.5 & 50.0 & 1.0 & Fail & Fail & Fail & Fail & Pass & Pass & Fail & Pass & Pass & Fail & Fail & Pass & Pass &3.6$\times10^{-8}$ & 8.5$\times10^{36}$ & 4.6$\times10^{-2}$\\
0.10 & M & -0.5 & 50.0 & 10.0 & Pass & Pass & Fail & Fail & Pass & Pass & Pass & Fail & Pass & Fail & Fail & Fail & Pass &1.1$\times10^{-7}$ & 2.7$\times10^{37}$ & 4.6$\times10^{-2}$\\
0.10 & M & -0.5 & 50.0 & 40.0 & Pass & Pass & Fail & Fail & Pass & Pass & Fail & Fail & Pass & Fail & Fail & Fail & Pass &1.9$\times10^{-7}$ & 4.5$\times10^{37}$ & 4.6$\times10^{-2}$\\
0.10 & M & -0.5 & 50.0 & 160.0 & Pass & Pass & Fail & Fail & Pass & Pass & Fail & Fail & Pass & Fail & Fail & Fail & Fail &3.4$\times10^{-7}$ & 8.1$\times10^{37}$ & 4.6$\times10^{-2}$\\
0.10 & M & -0.5 & 70.0 & 1.0 & Fail & Fail & Fail & Fail & Pass & Pass & Fail & Fail & Pass & Fail & Fail & Pass & Pass &3.4$\times10^{-8}$ & 8.0$\times10^{36}$ & 4.6$\times10^{-2}$\\
0.10 & M & -0.5 & 70.0 & 10.0 & Pass & Pass & Fail & Fail & Pass & Fail & Fail & Fail & Pass & Fail & Fail & Fail & Pass &1.0$\times10^{-7}$ & 2.4$\times10^{37}$ & 4.6$\times10^{-2}$\\
0.10 & M & -0.5 & 70.0 & 40.0 & Pass & Pass & Fail & Fail & Pass & Fail & Fail & Fail & Pass & Fail & Fail & Fail & Pass &1.7$\times10^{-7}$ & 4.0$\times10^{37}$ & 4.6$\times10^{-2}$\\
0.10 & M & -0.5 & 70.0 & 160.0 & Pass & Pass & Fail & Fail & Pass & Pass & Fail & Fail & Pass & Fail & Fail & Fail & Fail &3.1$\times10^{-7}$ & 7.3$\times10^{37}$ & 4.6$\times10^{-2}$\\
0.10 & M & -0.5 & 90.0 & 1.0 & Fail & Fail & Fail & Fail & Pass & Fail & Fail & Fail & Pass & Fail & Fail & Pass & Pass &3.3$\times10^{-8}$ & 7.9$\times10^{36}$ & 4.6$\times10^{-2}$\\
0.10 & M & -0.5 & 90.0 & 10.0 & Pass & Pass & Fail & Fail & Pass & Fail & Fail & Fail & Pass & Fail & Fail & Fail & Pass &1.0$\times10^{-7}$ & 2.4$\times10^{37}$ & 4.6$\times10^{-2}$\\
0.10 & M & -0.5 & 90.0 & 40.0 & Pass & Pass & Fail & Fail & Pass & Fail & Fail & Fail & Pass & Fail & Fail & Fail & Pass &1.7$\times10^{-7}$ & 4.0$\times10^{37}$ & 4.6$\times10^{-2}$\\
0.10 & M & -0.5 & 90.0 & 160.0 & Pass & Pass & Fail & Fail & Pass & Pass & Fail & Fail & Pass & Fail & Fail & Fail & Fail &3.1$\times10^{-7}$ & 7.3$\times10^{37}$ & 4.6$\times10^{-2}$\\
0.10 & M & 0.0 & 10.0 & 1.0 & Fail & Pass & Fail & Fail & Pass & Pass & Fail & Fail & Pass & Fail & Fail & Fail & Pass &2.4$\times10^{-8}$ & 1.7$\times10^{36}$ & 1.4$\times10^{-2}$\\
0.10 & M & 0.0 & 10.0 & 10.0 & Pass & Fail & Fail & Fail & Pass & Pass & Fail & Fail & Pass & Fail & Fail & Pass & Pass &7.3$\times10^{-8}$ & 5.2$\times10^{36}$ & 1.4$\times10^{-2}$\\
0.10 & M & 0.0 & 10.0 & 40.0 & Pass & Fail & Fail & Fail & Pass & Pass & Fail & Fail & Pass & Fail & Fail & Pass & Pass &1.2$\times10^{-7}$ & 8.8$\times10^{36}$ & 1.4$\times10^{-2}$\\
0.10 & M & 0.0 & 10.0 & 160.0 & Pass & Fail & Fail & Fail & Pass & Pass & Fail & Fail & Pass & Fail & Fail & Pass & Fail &2.1$\times10^{-7}$ & 1.5$\times10^{37}$ & 1.4$\times10^{-2}$\\
0.10 & M & 0.0 & 30.0 & 1.0 & Fail & Fail & Fail & Fail & Pass & Pass & Fail & Fail & Pass & Fail & Fail & Pass & Pass &2.3$\times10^{-8}$ & 1.6$\times10^{36}$ & 1.4$\times10^{-2}$\\
0.10 & M & 0.0 & 30.0 & 10.0 & Pass & Pass & Fail & Fail & Pass & Pass & Fail & Fail & Pass & Fail & Fail & Fail & Pass &7.0$\times10^{-8}$ & 4.9$\times10^{36}$ & 1.4$\times10^{-2}$\\
0.10 & M & 0.0 & 30.0 & 40.0 & Pass & Pass & Fail & Fail & Pass & Pass & Fail & Fail & Pass & Fail & Fail & Fail & Pass &1.2$\times10^{-7}$ & 8.4$\times10^{36}$ & 1.4$\times10^{-2}$\\
0.10 & M & 0.0 & 30.0 & 160.0 & Pass & Pass & Fail & Fail & Pass & Pass & Fail & Fail & Pass & Fail & Fail & Pass & Fail &2.1$\times10^{-7}$ & 1.5$\times10^{37}$ & 1.4$\times10^{-2}$\\
0.10 & M & 0.0 & 50.0 & 1.0 & Fail & Fail & Fail & Fail & Pass & Pass & Fail & Pass & Pass & Fail & Fail & Pass & Pass &2.3$\times10^{-8}$ & 1.6$\times10^{36}$ & 1.4$\times10^{-2}$\\
0.10 & M & 0.0 & 50.0 & 10.0 & Pass & Fail & Fail & Fail & Pass & Pass & Pass & Fail & Pass & Fail & Fail & Fail & Pass &6.6$\times10^{-8}$ & 4.7$\times10^{36}$ & 1.4$\times10^{-2}$\\
0.10 & M & 0.0 & 50.0 & 40.0 & Pass & Fail & Fail & Fail & Pass & Pass & Pass & Fail & Pass & Fail & Fail & Fail & Pass &1.1$\times10^{-7}$ & 8.0$\times10^{36}$ & 1.4$\times10^{-2}$\\
0.10 & M & 0.0 & 50.0 & 160.0 & Pass & Pass & Fail & Fail & Pass & Pass & Fail & Fail & Pass & Fail & Fail & Fail & Fail &2.0$\times10^{-7}$ & 1.4$\times10^{37}$ & 1.4$\times10^{-2}$\\
0.10 & M & 0.0 & 70.0 & 1.0 & Fail & Fail & Fail & Fail & Pass & Pass & Fail & Pass & Pass & Fail & Fail & Pass & Pass &2.2$\times10^{-8}$ & 1.6$\times10^{36}$ & 1.4$\times10^{-2}$\\
0.10 & M & 0.0 & 70.0 & 10.0 & Pass & Fail & Fail & Fail & Pass & Pass & Fail & Fail & Pass & Fail & Fail & Fail & Pass &6.2$\times10^{-8}$ & 4.4$\times10^{36}$ & 1.4$\times10^{-2}$\\
0.10 & M & 0.0 & 70.0 & 40.0 & Pass & Fail & Fail & Fail & Pass & Pass & Fail & Fail & Pass & Fail & Fail & Fail & Pass &1.1$\times10^{-7}$ & 7.7$\times10^{36}$ & 1.4$\times10^{-2}$\\
0.10 & M & 0.0 & 70.0 & 160.0 & Pass & Fail & Fail & Fail & Pass & Pass & Fail & Fail & Pass & Fail & Fail & Fail & Fail &1.9$\times10^{-7}$ & 1.4$\times10^{37}$ & 1.4$\times10^{-2}$\\
0.10 & M & 0.0 & 90.0 & 1.0 & Pass & Fail & Fail & Fail & Pass & Fail & Fail & Fail & Pass & Fail & Fail & Pass & Pass &2.0$\times10^{-8}$ & 1.4$\times10^{36}$ & 1.4$\times10^{-2}$\\
0.10 & M & 0.0 & 90.0 & 10.0 & Pass & Fail & Fail & Fail & Pass & Fail & Pass & Fail & Pass & Fail & Fail & Fail & Pass &5.9$\times10^{-8}$ & 4.2$\times10^{36}$ & 1.4$\times10^{-2}$\\
0.10 & M & 0.0 & 90.0 & 40.0 & Pass & Fail & Fail & Fail & Pass & Pass & Pass & Fail & Pass & Fail & Fail & Fail & Pass &1.1$\times10^{-7}$ & 7.5$\times10^{36}$ & 1.4$\times10^{-2}$\\
0.10 & M & 0.0 & 90.0 & 160.0 & Pass & Fail & Fail & Fail & Pass & Pass & Fail & Fail & Pass & Fail & Fail & Pass & Fail &1.9$\times10^{-7}$ & 1.4$\times10^{37}$ & 1.4$\times10^{-2}$\\
0.10 & M & 0.5 & 10.0 & 1.0 & Fail & Fail & Fail & Fail & Pass & Pass & Fail & Fail & Pass & Fail & Fail & Pass & Pass &3.9$\times10^{-8}$ & 3.2$\times10^{37}$ & 1.6$\times10^{-1}$\\
0.10 & M & 0.5 & 10.0 & 10.0 & Fail & Pass & Fail & Fail & Pass & Pass & Pass & Fail & Pass & Fail & Fail & Fail & Pass &1.1$\times10^{-7}$ & 8.7$\times10^{37}$ & 1.6$\times10^{-1}$\\
0.10 & M & 0.5 & 10.0 & 40.0 & Fail & Pass & Fail & Fail & Pass & Pass & Pass & Pass & Pass & Pass & Fail & Pass & Pass &1.7$\times10^{-7}$ & 1.4$\times10^{38}$ & 1.6$\times10^{-1}$\\
0.10 & M & 0.5 & 10.0 & 160.0 & Fail & Pass & Fail & Fail & Pass & Pass & Pass & Fail & Pass & Fail & Fail & Pass & Pass &3.3$\times10^{-7}$ & 2.7$\times10^{38}$ & 1.6$\times10^{-1}$\\
0.10 & M & 0.5 & 30.0 & 1.0 & Fail & Fail & Fail & Fail & Pass & Pass & Fail & Fail & Pass & Fail & Fail & Pass & Pass &3.8$\times10^{-8}$ & 3.2$\times10^{37}$ & 1.6$\times10^{-1}$\\
0.10 & M & 0.5 & 30.0 & 10.0 & Fail & Pass & Fail & Fail & Pass & Pass & Pass & Pass & Pass & Pass & Fail & Fail & Pass &1.0$\times10^{-7}$ & 8.4$\times10^{37}$ & 1.6$\times10^{-1}$\\
0.10 & M & 0.5 & 30.0 & 40.0 & Pass & Pass & Fail & Fail & Pass & Pass & Pass & Pass & Pass & Pass & Fail & Fail & Pass &1.7$\times10^{-7}$ & 1.4$\times10^{38}$ & 1.6$\times10^{-1}$\\
0.10 & M & 0.5 & 30.0 & 160.0 & Pass & Pass & Fail & Fail & Pass & Pass & Pass & Pass & Pass & Pass & Fail & Pass & Pass &3.1$\times10^{-7}$ & 2.6$\times10^{38}$ & 1.6$\times10^{-1}$\\
0.10 & M & 0.5 & 50.0 & 1.0 & Fail & Fail & Fail & Fail & Pass & Pass & Pass & Pass & Pass & Pass & Fail & Pass & Pass &3.6$\times10^{-8}$ & 3.0$\times10^{37}$ & 1.6$\times10^{-1}$\\
0.10 & M & 0.5 & 50.0 & 10.0 & Pass & Pass & Fail & Fail & Pass & Pass & Pass & Pass & Pass & Pass & Fail & Fail & Pass &9.4$\times10^{-8}$ & 7.7$\times10^{37}$ & 1.6$\times10^{-1}$\\
0.10 & M & 0.5 & 50.0 & 40.0 & Pass & Pass & Fail & Fail & Pass & Pass & Pass & Pass & Pass & Pass & Fail & Fail & Pass &1.5$\times10^{-7}$ & 1.3$\times10^{38}$ & 1.6$\times10^{-1}$\\
0.10 & M & 0.5 & 50.0 & 160.0 & Pass & Pass & Fail & Fail & Pass & Pass & Pass & Pass & Pass & Pass & Fail & Pass & Pass &2.9$\times10^{-7}$ & 2.4$\times10^{38}$ & 1.6$\times10^{-1}$\\
0.10 & M & 0.5 & 70.0 & 1.0 & Fail & Fail & Fail & Fail & Pass & Pass & Pass & Pass & Pass & Pass & Fail & Fail & Pass &3.4$\times10^{-8}$ & 2.8$\times10^{37}$ & 1.6$\times10^{-1}$\\
0.10 & M & 0.5 & 70.0 & 10.0 & Pass & Pass & Fail & Fail & Pass & Fail & Pass & Fail & Fail & Fail & Fail & Fail & Pass &8.5$\times10^{-8}$ & 7.0$\times10^{37}$ & 1.6$\times10^{-1}$\\
0.10 & M & 0.5 & 70.0 & 40.0 & Pass & Pass & Fail & Fail & Pass & Fail & Pass & Fail & Fail & Fail & Fail & Fail & Pass &1.4$\times10^{-7}$ & 1.1$\times10^{38}$ & 1.6$\times10^{-1}$\\
0.10 & M & 0.5 & 70.0 & 160.0 & Pass & Pass & Fail & Fail & Pass & Pass & Pass & Fail & Pass & Fail & Fail & Fail & Pass &2.6$\times10^{-7}$ & 2.1$\times10^{38}$ & 1.6$\times10^{-1}$\\
0.10 & M & 0.5 & 90.0 & 1.0 & Fail & Fail & Fail & Fail & Pass & Fail & Pass & Fail & Pass & Fail & Fail & Pass & Pass &3.2$\times10^{-8}$ & 2.6$\times10^{37}$ & 1.6$\times10^{-1}$\\
0.10 & M & 0.5 & 90.0 & 10.0 & Pass & Pass & Fail & Fail & Pass & Fail & Pass & Fail & Fail & Fail & Fail & Fail & Pass &8.0$\times10^{-8}$ & 6.6$\times10^{37}$ & 1.6$\times10^{-1}$\\
0.10 & M & 0.5 & 90.0 & 40.0 & Pass & Pass & Fail & Fail & Pass & Fail & Pass & Fail & Pass & Fail & Fail & Fail & Pass &1.3$\times10^{-7}$ & 1.1$\times10^{38}$ & 1.6$\times10^{-1}$\\
0.10 & M & 0.5 & 90.0 & 160.0 & Pass & Pass & Fail & Fail & Pass & Pass & Pass & Fail & Pass & Fail & Fail & Fail & Pass &2.5$\times10^{-7}$ & 2.1$\times10^{38}$ & 1.6$\times10^{-1}$\\
0.10 & M & 0.94 & 10.0 & 1.0 & Pass & Pass & Fail & Fail & Pass & Pass & Fail & Fail & Pass & Fail & Fail & Pass & Pass &8.5$\times10^{-9}$ & 4.1$\times10^{37}$ & 9.3$\times10^{-1}$\\
0.10 & M & 0.94 & 10.0 & 10.0 & Pass & Pass & Fail & Fail & Pass & Pass & Pass & Pass & Pass & Pass & Fail & Fail & Pass &2.3$\times10^{-8}$ & 1.1$\times10^{38}$ & 9.3$\times10^{-1}$\\
0.10 & M & 0.94 & 10.0 & 40.0 & Pass & Pass & Fail & Fail & Pass & Pass & Pass & Pass & Pass & Pass & Fail & Pass & Pass &3.9$\times10^{-8}$ & 1.9$\times10^{38}$ & 9.3$\times10^{-1}$\\
0.10 & M & 0.94 & 10.0 & 160.0 & Pass & Pass & Fail & Fail & Pass & Pass & Pass & Pass & Pass & Pass & Fail & Fail & Pass &6.7$\times10^{-8}$ & 3.2$\times10^{38}$ & 9.3$\times10^{-1}$\\
0.10 & M & 0.94 & 30.0 & 1.0 & Pass & Fail & Fail & Fail & Pass & Pass & Fail & Fail & Pass & Fail & Fail & Pass & Pass &8.4$\times10^{-9}$ & 4.0$\times10^{37}$ & 9.3$\times10^{-1}$\\
0.10 & M & 0.94 & 30.0 & 10.0 & Pass & Pass & Fail & Fail & Pass & Pass & Pass & Pass & Pass & Pass & Fail & Fail & Pass &2.3$\times10^{-8}$ & 1.1$\times10^{38}$ & 9.3$\times10^{-1}$\\
0.10 & M & 0.94 & 30.0 & 40.0 & Pass & Pass & Fail & Fail & Pass & Pass & Pass & Pass & Pass & Pass & Fail & Fail & Pass &3.8$\times10^{-8}$ & 1.8$\times10^{38}$ & 9.3$\times10^{-1}$\\
0.10 & M & 0.94 & 30.0 & 160.0 & Pass & Pass & Fail & Fail & Pass & Pass & Pass & Pass & Pass & Pass & Fail & Fail & Pass &6.7$\times10^{-8}$ & 3.2$\times10^{38}$ & 9.3$\times10^{-1}$\\
0.10 & M & 0.94 & 50.0 & 1.0 & Pass & Fail & Fail & Fail & Pass & Pass & Pass & Pass & Pass & Pass & Fail & Pass & Pass &8.2$\times10^{-9}$ & 3.9$\times10^{37}$ & 9.3$\times10^{-1}$\\
0.10 & M & 0.94 & 50.0 & 10.0 & Pass & Pass & Fail & Fail & Pass & Pass & Pass & Pass & Fail & Fail & Fail & Fail & Pass &2.2$\times10^{-8}$ & 1.0$\times10^{38}$ & 9.3$\times10^{-1}$\\
0.10 & M & 0.94 & 50.0 & 40.0 & Pass & Pass & Fail & Fail & Pass & Pass & Pass & Pass & Fail & Fail & Fail & Fail & Pass &3.7$\times10^{-8}$ & 1.8$\times10^{38}$ & 9.3$\times10^{-1}$\\
0.10 & M & 0.94 & 50.0 & 160.0 & Pass & Pass & Fail & Fail & Pass & Pass & Pass & Pass & Pass & Pass & Fail & Fail & Pass &6.6$\times10^{-8}$ & 3.2$\times10^{38}$ & 9.3$\times10^{-1}$\\
0.10 & M & 0.94 & 70.0 & 1.0 & Pass & Fail & Fail & Fail & Pass & Fail & Pass & Pass & Pass & Fail & Fail & Pass & Pass &8.0$\times10^{-9}$ & 3.8$\times10^{37}$ & 9.3$\times10^{-1}$\\
0.10 & M & 0.94 & 70.0 & 10.0 & Pass & Pass & Fail & Fail & Pass & Fail & Pass & Fail & Fail & Fail & Fail & Fail & Pass &2.1$\times10^{-8}$ & 9.8$\times10^{37}$ & 9.3$\times10^{-1}$\\
0.10 & M & 0.94 & 70.0 & 40.0 & Pass & Pass & Fail & Fail & Pass & Fail & Pass & Fail & Fail & Fail & Fail & Pass & Pass &3.7$\times10^{-8}$ & 1.8$\times10^{38}$ & 9.3$\times10^{-1}$\\
0.10 & M & 0.94 & 70.0 & 160.0 & Pass & Pass & Fail & Fail & Pass & Pass & Pass & Fail & Pass & Fail & Fail & Fail & Pass &6.6$\times10^{-8}$ & 3.2$\times10^{38}$ & 9.3$\times10^{-1}$\\
0.10 & M & 0.94 & 90.0 & 1.0 & Pass & Fail & Fail & Fail & Pass & Fail & Pass & Fail & Pass & Fail & Fail & Fail & Pass &7.7$\times10^{-9}$ & 3.7$\times10^{37}$ & 9.3$\times10^{-1}$\\
0.10 & M & 0.94 & 90.0 & 10.0 & Pass & Pass & Fail & Fail & Pass & Fail & Pass & Fail & Fail & Fail & Fail & Fail & Pass &2.0$\times10^{-8}$ & 9.8$\times10^{37}$ & 9.3$\times10^{-1}$\\
0.10 & M & 0.94 & 90.0 & 40.0 & Pass & Pass & Fail & Fail & Pass & Fail & Pass & Fail & Fail & Fail & Fail & Pass & Pass &3.7$\times10^{-8}$ & 1.8$\times10^{38}$ & 9.3$\times10^{-1}$\\
0.10 & M & 0.94 & 90.0 & 160.0 & Pass & Pass & Fail & Fail & Pass & Pass & Pass & Fail & Pass & Fail & Fail & Pass & Fail &6.8$\times10^{-8}$ & 3.3$\times10^{38}$ & 9.3$\times10^{-1}$\\
0.20 & S & -0.94 & 10.0 & 1.0 & Fail & Fail & Pass & Fail & Fail & Pass & Pass & Fail & Pass & Fail & Fail & Pass & Pass &1.0$\times10^{-8}$ & 2.2$\times10^{35}$ & 4.2$\times10^{-3}$\\
0.20 & S & -0.94 & 10.0 & 10.0 & Pass & Pass & Fail & Fail & Pass & Pass & Fail & Fail & Pass & Fail & Fail & Fail & Fail &2.8$\times10^{-7}$ & 6.1$\times10^{36}$ & 4.2$\times10^{-3}$\\
0.20 & S & -0.94 & 10.0 & 40.0 & Pass & Pass & Fail & Fail & Pass & Pass & Fail & Fail & Pass & Fail & Fail & Pass & Pass &9.8$\times10^{-7}$ & 2.1$\times10^{37}$ & 4.2$\times10^{-3}$\\
0.20 & S & -0.94 & 10.0 & 160.0 & Pass & Pass & Fail & Fail & Pass & Pass & Pass & Fail & Pass & Fail & Fail & Fail & Pass &2.0$\times10^{-6}$ & 4.4$\times10^{37}$ & 4.2$\times10^{-3}$\\
0.20 & S & -0.94 & 30.0 & 1.0 & Fail & Fail & Pass & Fail & Pass & Pass & Pass & Fail & Pass & Fail & Fail & Pass & Pass &9.9$\times10^{-9}$ & 2.2$\times10^{35}$ & 4.2$\times10^{-3}$\\
0.20 & S & -0.94 & 30.0 & 10.0 & Pass & Pass & Fail & Fail & Pass & Pass & Fail & Fail & Pass & Fail & Fail & Fail & Fail &2.7$\times10^{-7}$ & 5.9$\times10^{36}$ & 4.2$\times10^{-3}$\\
0.20 & S & -0.94 & 30.0 & 40.0 & Pass & Pass & Fail & Fail & Pass & Pass & Pass & Fail & Pass & Fail & Fail & Fail & Pass &9.4$\times10^{-7}$ & 2.0$\times10^{37}$ & 4.2$\times10^{-3}$\\
0.20 & S & -0.94 & 30.0 & 160.0 & Pass & Pass & Fail & Fail & Pass & Pass & Pass & Fail & Pass & Fail & Fail & Fail & Fail &1.8$\times10^{-6}$ & 4.0$\times10^{37}$ & 4.2$\times10^{-3}$\\
0.20 & S & -0.94 & 50.0 & 1.0 & Fail & Pass & Pass & Fail & Pass & Pass & Pass & Fail & Pass & Fail & Fail & Pass & Pass &9.7$\times10^{-9}$ & 2.1$\times10^{35}$ & 4.2$\times10^{-3}$\\
0.20 & S & -0.94 & 50.0 & 10.0 & Pass & Pass & Fail & Fail & Pass & Pass & Fail & Fail & Pass & Fail & Fail & Fail & Fail &2.7$\times10^{-7}$ & 6.0$\times10^{36}$ & 4.2$\times10^{-3}$\\
0.20 & S & -0.94 & 50.0 & 40.0 & Pass & Pass & Fail & Fail & Pass & Pass & Fail & Fail & Pass & Fail & Fail & Fail & Pass &9.6$\times10^{-7}$ & 2.1$\times10^{37}$ & 4.2$\times10^{-3}$\\
0.20 & S & -0.94 & 50.0 & 160.0 & Pass & Pass & Fail & Fail & Pass & Pass & Pass & Fail & Pass & Fail & Fail & Fail & Fail &1.8$\times10^{-6}$ & 4.0$\times10^{37}$ & 4.2$\times10^{-3}$\\
0.20 & S & -0.94 & 70.0 & 1.0 & Fail & Pass & Pass & Fail & Pass & Pass & Fail & Fail & Pass & Fail & Fail & Pass & Pass &9.8$\times10^{-9}$ & 2.1$\times10^{35}$ & 4.2$\times10^{-3}$\\
0.20 & S & -0.94 & 70.0 & 10.0 & Pass & Pass & Fail & Fail & Pass & Pass & Pass & Fail & Pass & Fail & Fail & Fail & Fail &2.9$\times10^{-7}$ & 6.4$\times10^{36}$ & 4.2$\times10^{-3}$\\
0.20 & S & -0.94 & 70.0 & 40.0 & Pass & Pass & Fail & Fail & Pass & Pass & Pass & Fail & Pass & Fail & Fail & Fail & Fail &10.0$\times10^{-7}$ & 2.2$\times10^{37}$ & 4.2$\times10^{-3}$\\
0.20 & S & -0.94 & 70.0 & 160.0 & Pass & Pass & Fail & Fail & Pass & Pass & Pass & Fail & Pass & Fail & Fail & Fail & Fail &1.9$\times10^{-6}$ & 4.0$\times10^{37}$ & 4.2$\times10^{-3}$\\
0.20 & S & -0.94 & 90.0 & 1.0 & Fail & Pass & Pass & Fail & Pass & Pass & Fail & Fail & Pass & Fail & Fail & Pass & Pass &9.6$\times10^{-9}$ & 2.1$\times10^{35}$ & 4.2$\times10^{-3}$\\
0.20 & S & -0.94 & 90.0 & 10.0 & Fail & Pass & Fail & Fail & Pass & Pass & Pass & Fail & Pass & Fail & Fail & Fail & Fail &3.0$\times10^{-7}$ & 6.6$\times10^{36}$ & 4.2$\times10^{-3}$\\
0.20 & S & -0.94 & 90.0 & 40.0 & Pass & Fail & Fail & Fail & Pass & Pass & Pass & Fail & Pass & Fail & Fail & Fail & Fail &1.0$\times10^{-6}$ & 2.2$\times10^{37}$ & 4.2$\times10^{-3}$\\
0.20 & S & -0.94 & 90.0 & 160.0 & Pass & Pass & Fail & Fail & Pass & Fail & Pass & Fail & Pass & Fail & Fail & Fail & Fail &1.9$\times10^{-6}$ & 4.1$\times10^{37}$ & 4.2$\times10^{-3}$\\
0.20 & S & -0.5 & 10.0 & 1.0 & Fail & Fail & Pass & Fail & Fail & Pass & Pass & Fail & Pass & Fail & Fail & Pass & Pass &5.7$\times10^{-8}$ & 3.5$\times10^{35}$ & 1.2$\times10^{-3}$\\
0.20 & S & -0.5 & 10.0 & 10.0 & Pass & Pass & Fail & Fail & Pass & Pass & Fail & Fail & Pass & Fail & Fail & Fail & Fail &1.6$\times10^{-6}$ & 1.0$\times10^{37}$ & 1.2$\times10^{-3}$\\
0.20 & S & -0.5 & 10.0 & 40.0 & Pass & Pass & Fail & Fail & Pass & Pass & Fail & Fail & Pass & Fail & Fail & Pass & Pass &3.7$\times10^{-6}$ & 2.3$\times10^{37}$ & 1.2$\times10^{-3}$\\
0.20 & S & -0.5 & 10.0 & 160.0 & Pass & Pass & Fail & Fail & Pass & Pass & Fail & Fail & Pass & Fail & Fail & Pass & Pass &6.3$\times10^{-6}$ & 3.9$\times10^{37}$ & 1.2$\times10^{-3}$\\
0.20 & S & -0.5 & 30.0 & 1.0 & Fail & Pass & Pass & Fail & Fail & Pass & Pass & Fail & Pass & Fail & Fail & Pass & Pass &5.6$\times10^{-8}$ & 3.4$\times10^{35}$ & 1.2$\times10^{-3}$\\
0.20 & S & -0.5 & 30.0 & 10.0 & Pass & Pass & Fail & Fail & Pass & Pass & Fail & Fail & Pass & Fail & Fail & Fail & Fail &1.6$\times10^{-6}$ & 9.9$\times10^{36}$ & 1.2$\times10^{-3}$\\
0.20 & S & -0.5 & 30.0 & 40.0 & Pass & Fail & Fail & Fail & Pass & Pass & Fail & Pass & Pass & Fail & Fail & Pass & Pass &3.9$\times10^{-6}$ & 2.4$\times10^{37}$ & 1.2$\times10^{-3}$\\
0.20 & S & -0.5 & 30.0 & 160.0 & Pass & Fail & Fail & Fail & Pass & Pass & Fail & Fail & Pass & Fail & Fail & Pass & Pass &6.6$\times10^{-6}$ & 4.1$\times10^{37}$ & 1.2$\times10^{-3}$\\
0.20 & S & -0.5 & 50.0 & 1.0 & Fail & Pass & Pass & Fail & Pass & Pass & Pass & Fail & Pass & Fail & Fail & Pass & Pass &5.5$\times10^{-8}$ & 3.4$\times10^{35}$ & 1.2$\times10^{-3}$\\
0.20 & S & -0.5 & 50.0 & 10.0 & Pass & Pass & Fail & Fail & Pass & Pass & Fail & Fail & Pass & Fail & Fail & Fail & Pass &1.5$\times10^{-6}$ & 9.6$\times10^{36}$ & 1.2$\times10^{-3}$\\
0.20 & S & -0.5 & 50.0 & 40.0 & Pass & Fail & Fail & Fail & Pass & Pass & Fail & Fail & Pass & Fail & Fail & Pass & Pass &3.7$\times10^{-6}$ & 2.3$\times10^{37}$ & 1.2$\times10^{-3}$\\
0.20 & S & -0.5 & 50.0 & 160.0 & Pass & Fail & Fail & Fail & Pass & Fail & Fail & Fail & Pass & Fail & Fail & Pass & Pass &6.3$\times10^{-6}$ & 3.9$\times10^{37}$ & 1.2$\times10^{-3}$\\
0.20 & S & -0.5 & 70.0 & 1.0 & Fail & Fail & Pass & Fail & Pass & Pass & Pass & Fail & Pass & Fail & Fail & Pass & Pass &5.5$\times10^{-8}$ & 3.4$\times10^{35}$ & 1.2$\times10^{-3}$\\
0.20 & S & -0.5 & 70.0 & 10.0 & Fail & Fail & Fail & Fail & Pass & Pass & Fail & Fail & Pass & Fail & Fail & Fail & Pass &1.6$\times10^{-6}$ & 9.7$\times10^{36}$ & 1.2$\times10^{-3}$\\
0.20 & S & -0.5 & 70.0 & 40.0 & Pass & Fail & Fail & Fail & Pass & Pass & Pass & Pass & Pass & Pass & Fail & Pass & Pass &3.7$\times10^{-6}$ & 2.3$\times10^{37}$ & 1.2$\times10^{-3}$\\
0.20 & S & -0.5 & 70.0 & 160.0 & Pass & Fail & Fail & Fail & Pass & Fail & Pass & Pass & Pass & Fail & Fail & Pass & Pass &6.3$\times10^{-6}$ & 3.9$\times10^{37}$ & 1.2$\times10^{-3}$\\
0.20 & S & -0.5 & 90.0 & 1.0 & Fail & Fail & Pass & Fail & Pass & Pass & Pass & Fail & Pass & Fail & Fail & Pass & Pass &5.4$\times10^{-8}$ & 3.4$\times10^{35}$ & 1.2$\times10^{-3}$\\
0.20 & S & -0.5 & 90.0 & 10.0 & Fail & Fail & Fail & Fail & Pass & Pass & Fail & Fail & Pass & Fail & Fail & Fail & Pass &1.6$\times10^{-6}$ & 9.7$\times10^{36}$ & 1.2$\times10^{-3}$\\
0.20 & S & -0.5 & 90.0 & 40.0 & Pass & Fail & Fail & Fail & Pass & Fail & Pass & Pass & Pass & Fail & Fail & Pass & Pass &3.7$\times10^{-6}$ & 2.3$\times10^{37}$ & 1.2$\times10^{-3}$\\
0.20 & S & -0.5 & 90.0 & 160.0 & Pass & Fail & Fail & Fail & Pass & Fail & Pass & Fail & Pass & Fail & Fail & Pass & Pass &6.4$\times10^{-6}$ & 4.0$\times10^{37}$ & 1.2$\times10^{-3}$\\
0.20 & S & 0.0 & 10.0 & 1.0 & Fail & Pass & Pass & Fail & Fail & Pass & Pass & Fail & Pass & Fail & Fail & Pass & Pass &3.8$\times10^{-8}$ & 9.0$\times10^{34}$ & 4.5$\times10^{-4}$\\
0.20 & S & 0.0 & 10.0 & 10.0 & Pass & Fail & Fail & Fail & Pass & Pass & Fail & Fail & Pass & Fail & Fail & Fail & Fail &5.2$\times10^{-7}$ & 1.2$\times10^{36}$ & 4.5$\times10^{-4}$\\
0.20 & S & 0.0 & 10.0 & 40.0 & Fail & Fail & Fail & Fail & Pass & Pass & Fail & Fail & Pass & Fail & Fail & Pass & Pass &1.4$\times10^{-6}$ & 3.3$\times10^{36}$ & 4.5$\times10^{-4}$\\
0.20 & S & 0.0 & 10.0 & 160.0 & Pass & Fail & Fail & Fail & Pass & Pass & Pass & Fail & Pass & Fail & Fail & Pass & Pass &3.1$\times10^{-6}$ & 7.2$\times10^{36}$ & 4.5$\times10^{-4}$\\
0.20 & S & 0.0 & 30.0 & 1.0 & Fail & Pass & Pass & Fail & Pass & Pass & Pass & Fail & Pass & Fail & Fail & Pass & Pass &3.7$\times10^{-8}$ & 8.8$\times10^{34}$ & 4.5$\times10^{-4}$\\
0.20 & S & 0.0 & 30.0 & 10.0 & Pass & Fail & Fail & Fail & Pass & Pass & Fail & Fail & Pass & Fail & Fail & Fail & Fail &5.0$\times10^{-7}$ & 1.2$\times10^{36}$ & 4.5$\times10^{-4}$\\
0.20 & S & 0.0 & 30.0 & 40.0 & Pass & Fail & Fail & Fail & Pass & Pass & Fail & Pass & Pass & Fail & Fail & Pass & Pass &1.4$\times10^{-6}$ & 3.2$\times10^{36}$ & 4.5$\times10^{-4}$\\
0.20 & S & 0.0 & 30.0 & 160.0 & Pass & Fail & Fail & Fail & Pass & Pass & Fail & Pass & Pass & Fail & Fail & Pass & Pass &3.1$\times10^{-6}$ & 7.2$\times10^{36}$ & 4.5$\times10^{-4}$\\
0.20 & S & 0.0 & 50.0 & 1.0 & Fail & Fail & Pass & Fail & Pass & Pass & Pass & Fail & Pass & Fail & Fail & Pass & Pass &3.6$\times10^{-8}$ & 8.6$\times10^{34}$ & 4.5$\times10^{-4}$\\
0.20 & S & 0.0 & 50.0 & 10.0 & Fail & Pass & Fail & Fail & Pass & Pass & Pass & Pass & Pass & Pass & Fail & Fail & Fail &4.7$\times10^{-7}$ & 1.1$\times10^{36}$ & 4.5$\times10^{-4}$\\
0.20 & S & 0.0 & 50.0 & 40.0 & Pass & Fail & Fail & Fail & Pass & Pass & Pass & Fail & Pass & Fail & Fail & Pass & Pass &1.3$\times10^{-6}$ & 3.2$\times10^{36}$ & 4.5$\times10^{-4}$\\
0.20 & S & 0.0 & 50.0 & 160.0 & Pass & Fail & Fail & Fail & Pass & Pass & Pass & Fail & Pass & Fail & Fail & Pass & Pass &3.2$\times10^{-6}$ & 7.4$\times10^{36}$ & 4.5$\times10^{-4}$\\
0.20 & S & 0.0 & 70.0 & 1.0 & Fail & Fail & Pass & Fail & Pass & Pass & Pass & Pass & Pass & Pass & Fail & Pass & Pass &3.6$\times10^{-8}$ & 8.5$\times10^{34}$ & 4.5$\times10^{-4}$\\
0.20 & S & 0.0 & 70.0 & 10.0 & Fail & Fail & Fail & Fail & Pass & Pass & Pass & Fail & Pass & Fail & Fail & Fail & Fail &4.6$\times10^{-7}$ & 1.1$\times10^{36}$ & 4.5$\times10^{-4}$\\
0.20 & S & 0.0 & 70.0 & 40.0 & Pass & Fail & Fail & Fail & Pass & Pass & Pass & Fail & Pass & Fail & Fail & Pass & Pass &1.4$\times10^{-6}$ & 3.2$\times10^{36}$ & 4.5$\times10^{-4}$\\
0.20 & S & 0.0 & 70.0 & 160.0 & Fail & Fail & Fail & Fail & Pass & Pass & Pass & Fail & Pass & Fail & Fail & Pass & Pass &3.3$\times10^{-6}$ & 7.7$\times10^{36}$ & 4.5$\times10^{-4}$\\
0.20 & S & 0.0 & 90.0 & 1.0 & Fail & Fail & Pass & Fail & Pass & Pass & Pass & Fail & Pass & Fail & Fail & Pass & Pass &3.5$\times10^{-8}$ & 8.3$\times10^{34}$ & 4.5$\times10^{-4}$\\
0.20 & S & 0.0 & 90.0 & 10.0 & Fail & Fail & Fail & Fail & Pass & Pass & Pass & Fail & Pass & Fail & Fail & Fail & Fail &4.6$\times10^{-7}$ & 1.1$\times10^{36}$ & 4.5$\times10^{-4}$\\
0.20 & S & 0.0 & 90.0 & 40.0 & Pass & Fail & Fail & Fail & Pass & Pass & Pass & Fail & Pass & Fail & Fail & Pass & Pass &1.4$\times10^{-6}$ & 3.3$\times10^{36}$ & 4.5$\times10^{-4}$\\
0.20 & S & 0.0 & 90.0 & 160.0 & Fail & Fail & Fail & Fail & Pass & Pass & Pass & Fail & Pass & Fail & Fail & Pass & Pass &3.3$\times10^{-6}$ & 7.9$\times10^{36}$ & 4.5$\times10^{-4}$\\
0.20 & S & 0.5 & 10.0 & 1.0 & Fail & Pass & Pass & Fail & Pass & Pass & Pass & Fail & Pass & Fail & Fail & Pass & Pass &1.7$\times10^{-7}$ & 3.6$\times10^{36}$ & 4.2$\times10^{-3}$\\
0.20 & S & 0.5 & 10.0 & 10.0 & Pass & Fail & Fail & Fail & Pass & Pass & Pass & Pass & Pass & Pass & Fail & Fail & Pass &2.2$\times10^{-6}$ & 4.8$\times10^{37}$ & 4.2$\times10^{-3}$\\
0.20 & S & 0.5 & 10.0 & 40.0 & Pass & Pass & Fail & Fail & Pass & Pass & Pass & Pass & Pass & Pass & Fail & Fail & Pass &4.1$\times10^{-6}$ & 8.8$\times10^{37}$ & 4.2$\times10^{-3}$\\
0.20 & S & 0.5 & 10.0 & 160.0 & Pass & Pass & Fail & Fail & Pass & Pass & Pass & Pass & Pass & Pass & Fail & Fail & Pass &6.4$\times10^{-6}$ & 1.4$\times10^{38}$ & 4.2$\times10^{-3}$\\
0.20 & S & 0.5 & 30.0 & 1.0 & Fail & Fail & Pass & Fail & Pass & Pass & Pass & Fail & Pass & Fail & Fail & Pass & Pass &1.6$\times10^{-7}$ & 3.5$\times10^{36}$ & 4.2$\times10^{-3}$\\
0.20 & S & 0.5 & 30.0 & 10.0 & Pass & Fail & Fail & Fail & Pass & Pass & Pass & Pass & Pass & Pass & Fail & Fail & Pass &2.2$\times10^{-6}$ & 4.6$\times10^{37}$ & 4.2$\times10^{-3}$\\
0.20 & S & 0.5 & 30.0 & 40.0 & Pass & Pass & Fail & Fail & Pass & Pass & Pass & Pass & Fail & Fail & Fail & Fail & Pass &4.2$\times10^{-6}$ & 9.0$\times10^{37}$ & 4.2$\times10^{-3}$\\
0.20 & S & 0.5 & 30.0 & 160.0 & Pass & Fail & Fail & Fail & Pass & Pass & Pass & Pass & Pass & Pass & Fail & Fail & Pass &6.7$\times10^{-6}$ & 1.4$\times10^{38}$ & 4.2$\times10^{-3}$\\
0.20 & S & 0.5 & 50.0 & 1.0 & Fail & Fail & Pass & Fail & Pass & Pass & Pass & Fail & Pass & Fail & Fail & Pass & Pass &1.6$\times10^{-7}$ & 3.4$\times10^{36}$ & 4.2$\times10^{-3}$\\
0.20 & S & 0.5 & 50.0 & 10.0 & Pass & Fail & Fail & Fail & Pass & Pass & Pass & Fail & Pass & Fail & Fail & Fail & Fail &2.0$\times10^{-6}$ & 4.3$\times10^{37}$ & 4.2$\times10^{-3}$\\
0.20 & S & 0.5 & 50.0 & 40.0 & Pass & Pass & Fail & Fail & Pass & Pass & Pass & Fail & Pass & Fail & Fail & Fail & Pass &3.9$\times10^{-6}$ & 8.3$\times10^{37}$ & 4.2$\times10^{-3}$\\
0.20 & S & 0.5 & 50.0 & 160.0 & Pass & Fail & Fail & Fail & Pass & Pass & Pass & Fail & Pass & Fail & Fail & Fail & Fail &6.2$\times10^{-6}$ & 1.3$\times10^{38}$ & 4.2$\times10^{-3}$\\
0.20 & S & 0.5 & 70.0 & 1.0 & Fail & Fail & Pass & Fail & Pass & Pass & Pass & Pass & Pass & Pass & Fail & Pass & Pass &1.6$\times10^{-7}$ & 3.4$\times10^{36}$ & 4.2$\times10^{-3}$\\
0.20 & S & 0.5 & 70.0 & 10.0 & Fail & Fail & Fail & Fail & Pass & Fail & Pass & Fail & Pass & Fail & Fail & Fail & Fail &1.9$\times10^{-6}$ & 4.0$\times10^{37}$ & 4.2$\times10^{-3}$\\
0.20 & S & 0.5 & 70.0 & 40.0 & Fail & Pass & Fail & Fail & Pass & Fail & Pass & Fail & Pass & Fail & Fail & Fail & Fail &3.7$\times10^{-6}$ & 8.0$\times10^{37}$ & 4.2$\times10^{-3}$\\
0.20 & S & 0.5 & 70.0 & 160.0 & Fail & Fail & Fail & Fail & Pass & Fail & Pass & Fail & Pass & Fail & Fail & Fail & Fail &6.2$\times10^{-6}$ & 1.3$\times10^{38}$ & 4.2$\times10^{-3}$\\
0.20 & S & 0.5 & 90.0 & 1.0 & Fail & Fail & Pass & Fail & Pass & Fail & Pass & Fail & Pass & Fail & Fail & Pass & Pass &1.5$\times10^{-7}$ & 3.3$\times10^{36}$ & 4.2$\times10^{-3}$\\
0.20 & S & 0.5 & 90.0 & 10.0 & Fail & Fail & Fail & Fail & Pass & Fail & Pass & Fail & Pass & Fail & Fail & Fail & Fail &1.8$\times10^{-6}$ & 4.0$\times10^{37}$ & 4.2$\times10^{-3}$\\
0.20 & S & 0.5 & 90.0 & 40.0 & Fail & Pass & Fail & Fail & Pass & Fail & Pass & Fail & Pass & Fail & Fail & Fail & Fail &3.8$\times10^{-6}$ & 8.2$\times10^{37}$ & 4.2$\times10^{-3}$\\
0.20 & S & 0.5 & 90.0 & 160.0 & Fail & Fail & Fail & Fail & Pass & Fail & Pass & Fail & Pass & Fail & Fail & Fail & Fail &6.4$\times10^{-6}$ & 1.4$\times10^{38}$ & 4.2$\times10^{-3}$\\
0.20 & S & 0.94 & 10.0 & 1.0 & Fail & Fail & Pass & Fail & Pass & Pass & Fail & Fail & Pass & Fail & Fail & Pass & Pass &7.3$\times10^{-9}$ & 4.4$\times10^{35}$ & 1.2$\times10^{-2}$\\
0.20 & S & 0.94 & 10.0 & 10.0 & Pass & Pass & Fail & Fail & Pass & Pass & Pass & Pass & Pass & Pass & Fail & Pass & Pass &1.1$\times10^{-7}$ & 6.6$\times10^{36}$ & 1.2$\times10^{-2}$\\
0.20 & S & 0.94 & 10.0 & 40.0 & Pass & Fail & Fail & Fail & Pass & Pass & Fail & Pass & Pass & Fail & Fail & Fail & Pass &5.5$\times10^{-7}$ & 3.3$\times10^{37}$ & 1.2$\times10^{-2}$\\
0.20 & S & 0.94 & 10.0 & 160.0 & Fail & Fail & Fail & Fail & Pass & Pass & Pass & Pass & Pass & Pass & Fail & Pass & Pass &1.0$\times10^{-6}$ & 6.2$\times10^{37}$ & 1.2$\times10^{-2}$\\
0.20 & S & 0.94 & 30.0 & 1.0 & Fail & Fail & Pass & Fail & Pass & Pass & Pass & Fail & Pass & Fail & Fail & Pass & Pass &7.1$\times10^{-9}$ & 4.3$\times10^{35}$ & 1.2$\times10^{-2}$\\
0.20 & S & 0.94 & 30.0 & 10.0 & Fail & Pass & Fail & Fail & Pass & Pass & Pass & Pass & Pass & Pass & Fail & Pass & Pass &1.1$\times10^{-7}$ & 6.5$\times10^{36}$ & 1.2$\times10^{-2}$\\
0.20 & S & 0.94 & 30.0 & 40.0 & Pass & Fail & Fail & Fail & Pass & Pass & Pass & Fail & Pass & Fail & Fail & Pass & Pass &6.5$\times10^{-7}$ & 3.9$\times10^{37}$ & 1.2$\times10^{-2}$\\
0.20 & S & 0.94 & 30.0 & 160.0 & Fail & Fail & Fail & Fail & Pass & Pass & Pass & Fail & Pass & Fail & Fail & Pass & Pass &1.2$\times10^{-6}$ & 7.3$\times10^{37}$ & 1.2$\times10^{-2}$\\
0.20 & S & 0.94 & 50.0 & 1.0 & Fail & Fail & Pass & Fail & Pass & Pass & Pass & Pass & Pass & Pass & Fail & Pass & Pass &6.8$\times10^{-9}$ & 4.1$\times10^{35}$ & 1.2$\times10^{-2}$\\
0.20 & S & 0.94 & 50.0 & 10.0 & Fail & Pass & Fail & Fail & Pass & Pass & Pass & Pass & Fail & Fail & Fail & Pass & Fail &1.1$\times10^{-7}$ & 6.4$\times10^{36}$ & 1.2$\times10^{-2}$\\
0.20 & S & 0.94 & 50.0 & 40.0 & Pass & Fail & Fail & Fail & Pass & Pass & Pass & Fail & Pass & Fail & Fail & Pass & Fail &7.2$\times10^{-7}$ & 4.3$\times10^{37}$ & 1.2$\times10^{-2}$\\
0.20 & S & 0.94 & 50.0 & 160.0 & Fail & Fail & Fail & Fail & Pass & Pass & Pass & Fail & Pass & Fail & Fail & Pass & Fail &1.4$\times10^{-6}$ & 8.4$\times10^{37}$ & 1.2$\times10^{-2}$\\
0.20 & S & 0.94 & 70.0 & 1.0 & Fail & Fail & Pass & Fail & Pass & Fail & Pass & Pass & Pass & Fail & Fail & Pass & Pass &6.7$\times10^{-9}$ & 4.0$\times10^{35}$ & 1.2$\times10^{-2}$\\
0.20 & S & 0.94 & 70.0 & 10.0 & Fail & Pass & Fail & Fail & Pass & Pass & Pass & Fail & Pass & Fail & Fail & Pass & Fail &1.1$\times10^{-7}$ & 6.8$\times10^{36}$ & 1.2$\times10^{-2}$\\
0.20 & S & 0.94 & 70.0 & 40.0 & Pass & Fail & Fail & Fail & Pass & Fail & Pass & Fail & Pass & Fail & Fail & Fail & Fail &7.2$\times10^{-7}$ & 4.3$\times10^{37}$ & 1.2$\times10^{-2}$\\
0.20 & S & 0.94 & 70.0 & 160.0 & Pass & Fail & Fail & Fail & Pass & Fail & Pass & Fail & Pass & Fail & Fail & Pass & Pass &1.5$\times10^{-6}$ & 8.8$\times10^{37}$ & 1.2$\times10^{-2}$\\
0.20 & S & 0.94 & 90.0 & 1.0 & Fail & Fail & Fail & Fail & Pass & Fail & Pass & Pass & Pass & Fail & Fail & Pass & Pass &6.7$\times10^{-9}$ & 4.0$\times10^{35}$ & 1.2$\times10^{-2}$\\
0.20 & S & 0.94 & 90.0 & 10.0 & Fail & Pass & Fail & Fail & Pass & Fail & Pass & Pass & Pass & Fail & Fail & Pass & Fail &1.2$\times10^{-7}$ & 7.1$\times10^{36}$ & 1.2$\times10^{-2}$\\
0.20 & S & 0.94 & 90.0 & 40.0 & Pass & Fail & Fail & Fail & Pass & Fail & Pass & Fail & Pass & Fail & Fail & Fail & Fail &7.0$\times10^{-7}$ & 4.2$\times10^{37}$ & 1.2$\times10^{-2}$\\
0.20 & S & 0.94 & 90.0 & 160.0 & Pass & Fail & Fail & Fail & Pass & Fail & Pass & Fail & Pass & Fail & Fail & Pass & Pass &1.5$\times10^{-6}$ & 8.8$\times10^{37}$ & 1.2$\times10^{-2}$\\
0.20 & M & -0.94 & 10.0 & 1.0 & Fail & Fail & Fail & Fail & Pass & Pass & Fail & Fail & Pass & Fail & Fail & Fail & Pass &1.8$\times10^{-8}$ & 1.7$\times10^{37}$ & 1.8$\times10^{-1}$\\
0.20 & M & -0.94 & 10.0 & 10.0 & Pass & Pass & Fail & Fail & Pass & Pass & Pass & Fail & Pass & Fail & Fail & Fail & Pass &5.7$\times10^{-8}$ & 5.4$\times10^{37}$ & 1.8$\times10^{-1}$\\
0.20 & M & -0.94 & 10.0 & 40.0 & Pass & Pass & Fail & Fail & Pass & Pass & Pass & Fail & Pass & Fail & Fail & Fail & Fail &9.0$\times10^{-8}$ & 8.6$\times10^{37}$ & 1.8$\times10^{-1}$\\
0.20 & M & -0.94 & 10.0 & 160.0 & Pass & Pass & Fail & Fail & Pass & Pass & Pass & Fail & Pass & Fail & Fail & Fail & Fail &1.5$\times10^{-7}$ & 1.4$\times10^{38}$ & 1.8$\times10^{-1}$\\
0.20 & M & -0.94 & 30.0 & 1.0 & Fail & Pass & Fail & Fail & Pass & Pass & Fail & Fail & Pass & Fail & Fail & Fail & Pass &1.8$\times10^{-8}$ & 1.7$\times10^{37}$ & 1.8$\times10^{-1}$\\
0.20 & M & -0.94 & 30.0 & 10.0 & Pass & Pass & Fail & Fail & Pass & Pass & Pass & Fail & Pass & Fail & Fail & Fail & Fail &5.5$\times10^{-8}$ & 5.2$\times10^{37}$ & 1.8$\times10^{-1}$\\
0.20 & M & -0.94 & 30.0 & 40.0 & Pass & Pass & Fail & Fail & Pass & Pass & Pass & Fail & Pass & Fail & Fail & Fail & Fail &8.7$\times10^{-8}$ & 8.2$\times10^{37}$ & 1.8$\times10^{-1}$\\
0.20 & M & -0.94 & 30.0 & 160.0 & Pass & Pass & Fail & Fail & Pass & Pass & Pass & Fail & Pass & Fail & Fail & Fail & Fail &1.4$\times10^{-7}$ & 1.4$\times10^{38}$ & 1.8$\times10^{-1}$\\
0.20 & M & -0.94 & 50.0 & 1.0 & Fail & Pass & Fail & Fail & Pass & Pass & Fail & Fail & Pass & Fail & Fail & Fail & Pass &1.7$\times10^{-8}$ & 1.6$\times10^{37}$ & 1.8$\times10^{-1}$\\
0.20 & M & -0.94 & 50.0 & 10.0 & Fail & Pass & Fail & Fail & Pass & Pass & Pass & Fail & Pass & Fail & Fail & Fail & Fail &5.1$\times10^{-8}$ & 4.9$\times10^{37}$ & 1.8$\times10^{-1}$\\
0.20 & M & -0.94 & 50.0 & 40.0 & Pass & Pass & Fail & Fail & Pass & Pass & Pass & Fail & Pass & Fail & Fail & Fail & Fail &8.0$\times10^{-8}$ & 7.6$\times10^{37}$ & 1.8$\times10^{-1}$\\
0.20 & M & -0.94 & 50.0 & 160.0 & Pass & Pass & Fail & Fail & Pass & Pass & Pass & Fail & Pass & Fail & Fail & Fail & Fail &1.3$\times10^{-7}$ & 1.3$\times10^{38}$ & 1.8$\times10^{-1}$\\
0.20 & M & -0.94 & 70.0 & 1.0 & Fail & Pass & Fail & Fail & Pass & Pass & Fail & Fail & Pass & Fail & Fail & Fail & Pass &1.6$\times10^{-8}$ & 1.5$\times10^{37}$ & 1.8$\times10^{-1}$\\
0.20 & M & -0.94 & 70.0 & 10.0 & Pass & Pass & Fail & Fail & Pass & Pass & Pass & Fail & Pass & Fail & Fail & Fail & Fail &4.8$\times10^{-8}$ & 4.5$\times10^{37}$ & 1.8$\times10^{-1}$\\
0.20 & M & -0.94 & 70.0 & 40.0 & Pass & Pass & Fail & Fail & Pass & Pass & Pass & Fail & Pass & Fail & Fail & Fail & Fail &7.5$\times10^{-8}$ & 7.1$\times10^{37}$ & 1.8$\times10^{-1}$\\
0.20 & M & -0.94 & 70.0 & 160.0 & Pass & Pass & Fail & Fail & Pass & Pass & Pass & Fail & Pass & Fail & Fail & Fail & Fail &1.2$\times10^{-7}$ & 1.2$\times10^{38}$ & 1.8$\times10^{-1}$\\
0.20 & M & -0.94 & 90.0 & 1.0 & Fail & Pass & Fail & Fail & Pass & Fail & Fail & Fail & Pass & Fail & Fail & Fail & Pass &1.6$\times10^{-8}$ & 1.5$\times10^{37}$ & 1.8$\times10^{-1}$\\
0.20 & M & -0.94 & 90.0 & 10.0 & Pass & Pass & Fail & Fail & Pass & Pass & Pass & Fail & Pass & Fail & Fail & Fail & Fail &4.6$\times10^{-8}$ & 4.3$\times10^{37}$ & 1.8$\times10^{-1}$\\
0.20 & M & -0.94 & 90.0 & 40.0 & Pass & Pass & Fail & Fail & Pass & Pass & Pass & Fail & Pass & Fail & Fail & Fail & Fail &7.2$\times10^{-8}$ & 6.9$\times10^{37}$ & 1.8$\times10^{-1}$\\
0.20 & M & -0.94 & 90.0 & 160.0 & Pass & Pass & Fail & Fail & Pass & Pass & Pass & Fail & Pass & Fail & Fail & Fail & Fail &1.2$\times10^{-7}$ & 1.2$\times10^{38}$ & 1.8$\times10^{-1}$\\
0.20 & M & -0.5 & 10.0 & 1.0 & Fail & Pass & Fail & Fail & Pass & Pass & Fail & Fail & Pass & Fail & Fail & Pass & Pass &3.9$\times10^{-8}$ & 9.2$\times10^{36}$ & 4.6$\times10^{-2}$\\
0.20 & M & -0.5 & 10.0 & 10.0 & Fail & Pass & Fail & Fail & Pass & Pass & Fail & Fail & Pass & Fail & Fail & Pass & Pass &1.3$\times10^{-7}$ & 3.1$\times10^{37}$ & 4.6$\times10^{-2}$\\
0.20 & M & -0.5 & 10.0 & 40.0 & Pass & Pass & Fail & Fail & Pass & Pass & Fail & Fail & Pass & Fail & Fail & Pass & Pass &2.2$\times10^{-7}$ & 5.1$\times10^{37}$ & 4.6$\times10^{-2}$\\
0.20 & M & -0.5 & 10.0 & 160.0 & Pass & Fail & Fail & Fail & Pass & Pass & Fail & Fail & Pass & Fail & Fail & Pass & Pass &3.9$\times10^{-7}$ & 9.2$\times10^{37}$ & 4.6$\times10^{-2}$\\
0.20 & M & -0.5 & 30.0 & 1.0 & Fail & Pass & Fail & Fail & Pass & Pass & Fail & Fail & Pass & Fail & Fail & Pass & Pass &3.8$\times10^{-8}$ & 9.0$\times10^{36}$ & 4.6$\times10^{-2}$\\
0.20 & M & -0.5 & 30.0 & 10.0 & Fail & Pass & Fail & Fail & Pass & Pass & Fail & Fail & Pass & Fail & Fail & Pass & Pass &1.3$\times10^{-7}$ & 3.0$\times10^{37}$ & 4.6$\times10^{-2}$\\
0.20 & M & -0.5 & 30.0 & 40.0 & Pass & Pass & Fail & Fail & Pass & Pass & Fail & Fail & Pass & Fail & Fail & Pass & Pass &2.1$\times10^{-7}$ & 4.9$\times10^{37}$ & 4.6$\times10^{-2}$\\
0.20 & M & -0.5 & 30.0 & 160.0 & Pass & Pass & Fail & Fail & Pass & Pass & Fail & Fail & Pass & Fail & Fail & Fail & Fail &3.7$\times10^{-7}$ & 8.9$\times10^{37}$ & 4.6$\times10^{-2}$\\
0.20 & M & -0.5 & 50.0 & 1.0 & Fail & Fail & Fail & Fail & Pass & Pass & Fail & Pass & Pass & Fail & Fail & Pass & Pass &3.6$\times10^{-8}$ & 8.6$\times10^{36}$ & 4.6$\times10^{-2}$\\
0.20 & M & -0.5 & 50.0 & 10.0 & Pass & Pass & Fail & Fail & Pass & Pass & Pass & Fail & Pass & Fail & Fail & Fail & Pass &1.2$\times10^{-7}$ & 2.7$\times10^{37}$ & 4.6$\times10^{-2}$\\
0.20 & M & -0.5 & 50.0 & 40.0 & Pass & Pass & Fail & Fail & Pass & Pass & Fail & Fail & Pass & Fail & Fail & Fail & Pass &1.9$\times10^{-7}$ & 4.5$\times10^{37}$ & 4.6$\times10^{-2}$\\
0.20 & M & -0.5 & 50.0 & 160.0 & Pass & Pass & Fail & Fail & Pass & Pass & Fail & Fail & Pass & Fail & Fail & Fail & Fail &3.5$\times10^{-7}$ & 8.3$\times10^{37}$ & 4.6$\times10^{-2}$\\
0.20 & M & -0.5 & 70.0 & 1.0 & Fail & Fail & Fail & Fail & Pass & Fail & Fail & Fail & Pass & Fail & Fail & Pass & Pass &3.4$\times10^{-8}$ & 8.1$\times10^{36}$ & 4.6$\times10^{-2}$\\
0.20 & M & -0.5 & 70.0 & 10.0 & Pass & Pass & Fail & Fail & Pass & Fail & Fail & Fail & Pass & Fail & Fail & Fail & Pass &1.0$\times10^{-7}$ & 2.5$\times10^{37}$ & 4.6$\times10^{-2}$\\
0.20 & M & -0.5 & 70.0 & 40.0 & Pass & Fail & Fail & Fail & Pass & Fail & Fail & Fail & Pass & Fail & Fail & Fail & Pass &1.7$\times10^{-7}$ & 4.0$\times10^{37}$ & 4.6$\times10^{-2}$\\
0.20 & M & -0.5 & 70.0 & 160.0 & Pass & Pass & Fail & Fail & Pass & Fail & Fail & Fail & Pass & Fail & Fail & Fail & Fail &3.1$\times10^{-7}$ & 7.4$\times10^{37}$ & 4.6$\times10^{-2}$\\
0.20 & M & -0.5 & 90.0 & 1.0 & Fail & Fail & Fail & Fail & Pass & Fail & Fail & Fail & Pass & Fail & Fail & Pass & Pass &3.4$\times10^{-8}$ & 7.9$\times10^{36}$ & 4.6$\times10^{-2}$\\
0.20 & M & -0.5 & 90.0 & 10.0 & Pass & Pass & Fail & Fail & Pass & Fail & Fail & Fail & Pass & Fail & Fail & Fail & Pass &1.0$\times10^{-7}$ & 2.4$\times10^{37}$ & 4.6$\times10^{-2}$\\
0.20 & M & -0.5 & 90.0 & 40.0 & Pass & Pass & Fail & Fail & Pass & Fail & Fail & Fail & Pass & Fail & Fail & Fail & Pass &1.7$\times10^{-7}$ & 4.0$\times10^{37}$ & 4.6$\times10^{-2}$\\
0.20 & M & -0.5 & 90.0 & 160.0 & Pass & Pass & Fail & Fail & Pass & Pass & Fail & Fail & Pass & Fail & Fail & Fail & Fail &3.1$\times10^{-7}$ & 7.3$\times10^{37}$ & 4.6$\times10^{-2}$\\
0.20 & M & 0.0 & 10.0 & 1.0 & Fail & Pass & Fail & Fail & Pass & Pass & Fail & Fail & Pass & Fail & Fail & Fail & Pass &2.4$\times10^{-8}$ & 1.7$\times10^{36}$ & 1.4$\times10^{-2}$\\
0.20 & M & 0.0 & 10.0 & 10.0 & Pass & Fail & Fail & Fail & Pass & Pass & Fail & Fail & Pass & Fail & Fail & Pass & Pass &7.4$\times10^{-8}$ & 5.3$\times10^{36}$ & 1.4$\times10^{-2}$\\
0.20 & M & 0.0 & 10.0 & 40.0 & Pass & Fail & Fail & Fail & Pass & Pass & Fail & Fail & Pass & Fail & Fail & Pass & Pass &1.3$\times10^{-7}$ & 8.9$\times10^{36}$ & 1.4$\times10^{-2}$\\
0.20 & M & 0.0 & 10.0 & 160.0 & Pass & Fail & Fail & Fail & Pass & Pass & Fail & Fail & Pass & Fail & Fail & Pass & Fail &2.2$\times10^{-7}$ & 1.5$\times10^{37}$ & 1.4$\times10^{-2}$\\
0.20 & M & 0.0 & 30.0 & 1.0 & Fail & Fail & Fail & Fail & Pass & Pass & Fail & Fail & Pass & Fail & Fail & Pass & Pass &2.3$\times10^{-8}$ & 1.7$\times10^{36}$ & 1.4$\times10^{-2}$\\
0.20 & M & 0.0 & 30.0 & 10.0 & Pass & Pass & Fail & Fail & Pass & Pass & Fail & Fail & Pass & Fail & Fail & Fail & Pass &7.1$\times10^{-8}$ & 5.0$\times10^{36}$ & 1.4$\times10^{-2}$\\
0.20 & M & 0.0 & 30.0 & 40.0 & Pass & Pass & Fail & Fail & Pass & Pass & Fail & Fail & Pass & Fail & Fail & Fail & Pass &1.2$\times10^{-7}$ & 8.5$\times10^{36}$ & 1.4$\times10^{-2}$\\
0.20 & M & 0.0 & 30.0 & 160.0 & Pass & Pass & Fail & Fail & Pass & Pass & Fail & Fail & Pass & Fail & Fail & Fail & Fail &2.1$\times10^{-7}$ & 1.5$\times10^{37}$ & 1.4$\times10^{-2}$\\
0.20 & M & 0.0 & 50.0 & 1.0 & Fail & Fail & Fail & Fail & Pass & Pass & Fail & Pass & Pass & Fail & Fail & Pass & Pass &2.3$\times10^{-8}$ & 1.6$\times10^{36}$ & 1.4$\times10^{-2}$\\
0.20 & M & 0.0 & 50.0 & 10.0 & Pass & Fail & Fail & Fail & Pass & Pass & Pass & Fail & Pass & Fail & Fail & Fail & Pass &6.7$\times10^{-8}$ & 4.8$\times10^{36}$ & 1.4$\times10^{-2}$\\
0.20 & M & 0.0 & 50.0 & 40.0 & Pass & Fail & Fail & Fail & Pass & Pass & Pass & Fail & Pass & Fail & Fail & Fail & Pass &1.1$\times10^{-7}$ & 8.1$\times10^{36}$ & 1.4$\times10^{-2}$\\
0.20 & M & 0.0 & 50.0 & 160.0 & Pass & Pass & Fail & Fail & Pass & Pass & Pass & Fail & Pass & Fail & Fail & Fail & Fail &2.0$\times10^{-7}$ & 1.4$\times10^{37}$ & 1.4$\times10^{-2}$\\
0.20 & M & 0.0 & 70.0 & 1.0 & Fail & Fail & Fail & Fail & Pass & Pass & Pass & Fail & Pass & Fail & Fail & Pass & Pass &2.2$\times10^{-8}$ & 1.6$\times10^{36}$ & 1.4$\times10^{-2}$\\
0.20 & M & 0.0 & 70.0 & 10.0 & Pass & Fail & Fail & Fail & Pass & Fail & Fail & Fail & Pass & Fail & Fail & Fail & Pass &6.3$\times10^{-8}$ & 4.5$\times10^{36}$ & 1.4$\times10^{-2}$\\
0.20 & M & 0.0 & 70.0 & 40.0 & Pass & Fail & Fail & Fail & Pass & Pass & Fail & Fail & Pass & Fail & Fail & Fail & Pass &1.1$\times10^{-7}$ & 7.7$\times10^{36}$ & 1.4$\times10^{-2}$\\
0.20 & M & 0.0 & 70.0 & 160.0 & Pass & Fail & Fail & Fail & Pass & Pass & Fail & Fail & Pass & Fail & Fail & Fail & Fail &1.9$\times10^{-7}$ & 1.4$\times10^{37}$ & 1.4$\times10^{-2}$\\
0.20 & M & 0.0 & 90.0 & 1.0 & Pass & Fail & Fail & Fail & Pass & Fail & Fail & Fail & Pass & Fail & Fail & Pass & Pass &2.0$\times10^{-8}$ & 1.5$\times10^{36}$ & 1.4$\times10^{-2}$\\
0.20 & M & 0.0 & 90.0 & 10.0 & Pass & Fail & Fail & Fail & Pass & Fail & Pass & Fail & Pass & Fail & Fail & Fail & Pass &6.0$\times10^{-8}$ & 4.2$\times10^{36}$ & 1.4$\times10^{-2}$\\
0.20 & M & 0.0 & 90.0 & 40.0 & Pass & Fail & Fail & Fail & Pass & Fail & Pass & Fail & Pass & Fail & Fail & Fail & Pass &1.1$\times10^{-7}$ & 7.5$\times10^{36}$ & 1.4$\times10^{-2}$\\
0.20 & M & 0.0 & 90.0 & 160.0 & Pass & Fail & Fail & Fail & Pass & Pass & Fail & Fail & Pass & Fail & Fail & Pass & Fail &1.9$\times10^{-7}$ & 1.4$\times10^{37}$ & 1.4$\times10^{-2}$\\
0.20 & M & 0.5 & 10.0 & 1.0 & Fail & Pass & Fail & Fail & Pass & Pass & Fail & Fail & Pass & Fail & Fail & Pass & Pass &4.0$\times10^{-8}$ & 3.3$\times10^{37}$ & 1.6$\times10^{-1}$\\
0.20 & M & 0.5 & 10.0 & 10.0 & Fail & Pass & Fail & Fail & Pass & Pass & Pass & Fail & Pass & Fail & Fail & Fail & Pass &1.1$\times10^{-7}$ & 8.8$\times10^{37}$ & 1.6$\times10^{-1}$\\
0.20 & M & 0.5 & 10.0 & 40.0 & Fail & Pass & Fail & Fail & Pass & Pass & Pass & Fail & Pass & Fail & Fail & Pass & Pass &1.8$\times10^{-7}$ & 1.4$\times10^{38}$ & 1.6$\times10^{-1}$\\
0.20 & M & 0.5 & 10.0 & 160.0 & Fail & Pass & Fail & Fail & Pass & Pass & Pass & Fail & Pass & Fail & Fail & Pass & Pass &3.3$\times10^{-7}$ & 2.7$\times10^{38}$ & 1.6$\times10^{-1}$\\
0.20 & M & 0.5 & 30.0 & 1.0 & Fail & Fail & Fail & Fail & Pass & Pass & Fail & Fail & Pass & Fail & Fail & Pass & Pass &3.9$\times10^{-8}$ & 3.2$\times10^{37}$ & 1.6$\times10^{-1}$\\
0.20 & M & 0.5 & 30.0 & 10.0 & Fail & Pass & Fail & Fail & Pass & Pass & Pass & Pass & Pass & Pass & Fail & Fail & Pass &1.0$\times10^{-7}$ & 8.5$\times10^{37}$ & 1.6$\times10^{-1}$\\
0.20 & M & 0.5 & 30.0 & 40.0 & Fail & Pass & Fail & Fail & Pass & Pass & Pass & Pass & Pass & Pass & Fail & Fail & Pass &1.7$\times10^{-7}$ & 1.4$\times10^{38}$ & 1.6$\times10^{-1}$\\
0.20 & M & 0.5 & 30.0 & 160.0 & Pass & Pass & Fail & Fail & Pass & Pass & Pass & Pass & Pass & Pass & Fail & Fail & Pass &3.2$\times10^{-7}$ & 2.6$\times10^{38}$ & 1.6$\times10^{-1}$\\
0.20 & M & 0.5 & 50.0 & 1.0 & Fail & Fail & Fail & Fail & Pass & Pass & Pass & Pass & Pass & Pass & Fail & Pass & Pass &3.7$\times10^{-8}$ & 3.0$\times10^{37}$ & 1.6$\times10^{-1}$\\
0.20 & M & 0.5 & 50.0 & 10.0 & Pass & Pass & Fail & Fail & Pass & Pass & Pass & Pass & Pass & Pass & Fail & Fail & Pass &9.6$\times10^{-8}$ & 7.9$\times10^{37}$ & 1.6$\times10^{-1}$\\
0.20 & M & 0.5 & 50.0 & 40.0 & Pass & Pass & Fail & Fail & Pass & Pass & Pass & Pass & Pass & Pass & Fail & Fail & Pass &1.6$\times10^{-7}$ & 1.3$\times10^{38}$ & 1.6$\times10^{-1}$\\
0.20 & M & 0.5 & 50.0 & 160.0 & Pass & Pass & Fail & Fail & Pass & Pass & Pass & Pass & Pass & Pass & Fail & Fail & Pass &2.9$\times10^{-7}$ & 2.4$\times10^{38}$ & 1.6$\times10^{-1}$\\
0.20 & M & 0.5 & 70.0 & 1.0 & Fail & Fail & Fail & Fail & Pass & Pass & Pass & Pass & Pass & Pass & Fail & Pass & Pass &3.4$\times10^{-8}$ & 2.8$\times10^{37}$ & 1.6$\times10^{-1}$\\
0.20 & M & 0.5 & 70.0 & 10.0 & Pass & Pass & Fail & Fail & Pass & Fail & Pass & Fail & Fail & Fail & Fail & Fail & Pass &8.6$\times10^{-8}$ & 7.0$\times10^{37}$ & 1.6$\times10^{-1}$\\
0.20 & M & 0.5 & 70.0 & 40.0 & Pass & Pass & Fail & Fail & Pass & Fail & Pass & Fail & Fail & Fail & Fail & Fail & Pass &1.4$\times10^{-7}$ & 1.2$\times10^{38}$ & 1.6$\times10^{-1}$\\
0.20 & M & 0.5 & 70.0 & 160.0 & Pass & Pass & Fail & Fail & Pass & Pass & Pass & Fail & Pass & Fail & Fail & Fail & Pass &2.6$\times10^{-7}$ & 2.2$\times10^{38}$ & 1.6$\times10^{-1}$\\
0.20 & M & 0.5 & 90.0 & 1.0 & Fail & Fail & Fail & Fail & Pass & Fail & Pass & Fail & Pass & Fail & Fail & Pass & Pass &3.2$\times10^{-8}$ & 2.6$\times10^{37}$ & 1.6$\times10^{-1}$\\
0.20 & M & 0.5 & 90.0 & 10.0 & Pass & Pass & Fail & Fail & Pass & Fail & Pass & Fail & Fail & Fail & Fail & Fail & Pass &8.1$\times10^{-8}$ & 6.7$\times10^{37}$ & 1.6$\times10^{-1}$\\
0.20 & M & 0.5 & 90.0 & 40.0 & Pass & Pass & Fail & Fail & Pass & Fail & Pass & Fail & Fail & Fail & Fail & Fail & Pass &1.3$\times10^{-7}$ & 1.1$\times10^{38}$ & 1.6$\times10^{-1}$\\
0.20 & M & 0.5 & 90.0 & 160.0 & Pass & Pass & Fail & Fail & Pass & Fail & Pass & Fail & Pass & Fail & Fail & Fail & Pass &2.6$\times10^{-7}$ & 2.1$\times10^{38}$ & 1.6$\times10^{-1}$\\
0.20 & M & 0.94 & 10.0 & 1.0 & Pass & Pass & Fail & Fail & Pass & Pass & Fail & Fail & Pass & Fail & Fail & Pass & Pass &8.6$\times10^{-9}$ & 4.1$\times10^{37}$ & 9.3$\times10^{-1}$\\
0.20 & M & 0.94 & 10.0 & 10.0 & Pass & Pass & Fail & Fail & Pass & Pass & Pass & Pass & Pass & Pass & Fail & Fail & Pass &2.4$\times10^{-8}$ & 1.1$\times10^{38}$ & 9.3$\times10^{-1}$\\
0.20 & M & 0.94 & 10.0 & 40.0 & Pass & Pass & Fail & Fail & Pass & Pass & Pass & Pass & Pass & Pass & Fail & Pass & Pass &4.0$\times10^{-8}$ & 1.9$\times10^{38}$ & 9.3$\times10^{-1}$\\
0.20 & M & 0.94 & 10.0 & 160.0 & Pass & Pass & Fail & Fail & Pass & Pass & Pass & Pass & Pass & Pass & Fail & Pass & Pass &6.8$\times10^{-8}$ & 3.3$\times10^{38}$ & 9.3$\times10^{-1}$\\
0.20 & M & 0.94 & 30.0 & 1.0 & Pass & Fail & Fail & Fail & Pass & Pass & Pass & Fail & Pass & Fail & Fail & Pass & Pass &8.5$\times10^{-9}$ & 4.1$\times10^{37}$ & 9.3$\times10^{-1}$\\
0.20 & M & 0.94 & 30.0 & 10.0 & Pass & Pass & Fail & Fail & Pass & Pass & Pass & Pass & Pass & Pass & Fail & Fail & Pass &2.3$\times10^{-8}$ & 1.1$\times10^{38}$ & 9.3$\times10^{-1}$\\
0.20 & M & 0.94 & 30.0 & 40.0 & Pass & Pass & Fail & Fail & Pass & Pass & Pass & Pass & Pass & Pass & Fail & Fail & Pass &3.9$\times10^{-8}$ & 1.9$\times10^{38}$ & 9.3$\times10^{-1}$\\
0.20 & M & 0.94 & 30.0 & 160.0 & Pass & Pass & Fail & Fail & Pass & Pass & Pass & Pass & Pass & Pass & Fail & Fail & Pass &6.8$\times10^{-8}$ & 3.2$\times10^{38}$ & 9.3$\times10^{-1}$\\
0.20 & M & 0.94 & 50.0 & 1.0 & Pass & Fail & Fail & Fail & Pass & Pass & Pass & Pass & Pass & Pass & Fail & Pass & Pass &8.3$\times10^{-9}$ & 4.0$\times10^{37}$ & 9.3$\times10^{-1}$\\
0.20 & M & 0.94 & 50.0 & 10.0 & Pass & Pass & Fail & Fail & Pass & Fail & Pass & Pass & Fail & Fail & Fail & Fail & Pass &2.2$\times10^{-8}$ & 1.0$\times10^{38}$ & 9.3$\times10^{-1}$\\
0.20 & M & 0.94 & 50.0 & 40.0 & Pass & Pass & Fail & Fail & Pass & Pass & Pass & Pass & Fail & Fail & Fail & Fail & Pass &3.8$\times10^{-8}$ & 1.8$\times10^{38}$ & 9.3$\times10^{-1}$\\
0.20 & M & 0.94 & 50.0 & 160.0 & Pass & Pass & Fail & Fail & Pass & Pass & Pass & Pass & Pass & Pass & Fail & Fail & Pass &6.7$\times10^{-8}$ & 3.2$\times10^{38}$ & 9.3$\times10^{-1}$\\
0.20 & M & 0.94 & 70.0 & 1.0 & Pass & Fail & Fail & Fail & Pass & Fail & Pass & Pass & Pass & Fail & Fail & Pass & Pass &8.1$\times10^{-9}$ & 3.9$\times10^{37}$ & 9.3$\times10^{-1}$\\
0.20 & M & 0.94 & 70.0 & 10.0 & Pass & Pass & Fail & Fail & Pass & Fail & Pass & Fail & Fail & Fail & Fail & Fail & Pass &2.1$\times10^{-8}$ & 10.0$\times10^{37}$ & 9.3$\times10^{-1}$\\
0.20 & M & 0.94 & 70.0 & 40.0 & Pass & Pass & Fail & Fail & Pass & Fail & Pass & Fail & Fail & Fail & Fail & Pass & Pass &3.7$\times10^{-8}$ & 1.8$\times10^{38}$ & 9.3$\times10^{-1}$\\
0.20 & M & 0.94 & 70.0 & 160.0 & Pass & Pass & Fail & Fail & Pass & Fail & Pass & Fail & Pass & Fail & Fail & Fail & Pass &6.7$\times10^{-8}$ & 3.2$\times10^{38}$ & 9.3$\times10^{-1}$\\
0.20 & M & 0.94 & 90.0 & 1.0 & Pass & Fail & Fail & Fail & Pass & Fail & Pass & Fail & Pass & Fail & Fail & Fail & Pass &7.7$\times10^{-9}$ & 3.7$\times10^{37}$ & 9.3$\times10^{-1}$\\
0.20 & M & 0.94 & 90.0 & 10.0 & Pass & Pass & Fail & Fail & Pass & Fail & Pass & Fail & Fail & Fail & Fail & Fail & Pass &2.1$\times10^{-8}$ & 9.8$\times10^{37}$ & 9.3$\times10^{-1}$\\
0.20 & M & 0.94 & 90.0 & 40.0 & Pass & Pass & Fail & Fail & Pass & Fail & Pass & Fail & Fail & Fail & Fail & Fail & Pass &3.7$\times10^{-8}$ & 1.8$\times10^{38}$ & 9.3$\times10^{-1}$\\
0.20 & M & 0.94 & 90.0 & 160.0 & Pass & Pass & Fail & Fail & Pass & Pass & Pass & Fail & Pass & Fail & Fail & Pass & Fail &6.9$\times10^{-8}$ & 3.3$\times10^{38}$ & 9.3$\times10^{-1}$\\
\enddata
\end{deluxetable*}

\end{longrotatetable}
\startlongtable
\begin{deluxetable*}{cccc|cccc|c|ccccc|c|c}
\tabletypesize{\scriptsize}
\tablecaption{Pass/Fail Table, Frankfurt variable kappa models}
\label{tab:VKbhacPF}
\tablehead{ \colhead{M/S}  &  %
\colhead{Spin}  &  %
\colhead{$i$}  &  %
\colhead{$\Rh$}  &  %
\colhead{$F_{86}$}  &  %
\colhead{$\lambda_{maj,86}$}  &  %
\colhead{$F_{2\mu{\rm m}}$}  &  %
\colhead{non-EHT}  &  %
\colhead{$\lambda_{230}$}  &  %
\colhead{EHT}  &  %
\colhead{All}}
\startdata
S & -0.94 & 10.0 & 1.0 & Fail & Fail & Pass & Fail & Fail & Fail & Fail\\
S & -0.94 & 10.0 & 10.0 & Pass & Fail & Fail & Fail & Fail & Fail & Fail\\
S & -0.94 & 10.0 & 40.0 & Pass & Fail & Fail & Fail & Fail & Fail & Fail\\
S & -0.94 & 10.0 & 80.0 & Pass & Fail & Pass & Fail & Fail & Fail & Fail\\
S & -0.94 & 10.0 & 160.0 & Pass & Fail & Pass & Fail & Fail & Fail & Fail\\
S & -0.94 & 30.0 & 1.0 & Fail & Fail & Pass & Fail & Fail & Fail & Fail\\
S & -0.94 & 30.0 & 10.0 & Pass & Fail & Fail & Fail & Fail & Fail & Fail\\
S & -0.94 & 30.0 & 40.0 & Pass & Fail & Fail & Fail & Fail & Fail & Fail\\
S & -0.94 & 30.0 & 80.0 & Pass & Fail & Pass & Fail & Pass & Pass & Fail\\
S & -0.94 & 30.0 & 160.0 & Pass & Fail & Pass & Fail & Pass & Pass & Fail\\
S & -0.94 & 50.0 & 1.0 & Fail & Fail & Pass & Fail & Fail & Fail & Fail\\
S & -0.94 & 50.0 & 10.0 & Fail & Fail & Fail & Fail & Fail & Fail & Fail\\
S & -0.94 & 50.0 & 40.0 & Pass & Fail & Fail & Fail & Fail & Fail & Fail\\
S & -0.94 & 50.0 & 80.0 & Pass & Fail & Fail & Fail & Pass & Pass & Fail\\
S & -0.94 & 50.0 & 160.0 & Pass & Fail & Pass & Fail & Pass & Pass & Fail\\
S & -0.94 & 70.0 & 1.0 & Fail & Fail & Pass & Fail & Fail & Fail & Fail\\
S & -0.94 & 70.0 & 10.0 & Fail & Fail & Fail & Fail & Fail & Fail & Fail\\
S & -0.94 & 70.0 & 40.0 & Pass & Fail & Fail & Fail & Pass & Pass & Fail\\
S & -0.94 & 70.0 & 80.0 & Pass & Fail & Fail & Fail & Pass & Pass & Fail\\
S & -0.94 & 70.0 & 160.0 & Pass & Fail & Pass & Fail & Pass & Pass & Fail\\
S & -0.94 & 90.0 & 1.0 & Fail & Fail & Pass & Fail & Pass & Pass & Fail\\
S & -0.94 & 90.0 & 10.0 & Fail & Fail & Fail & Fail & Fail & Fail & Fail\\
S & -0.94 & 90.0 & 40.0 & Pass & Fail & Fail & Fail & Pass & Pass & Fail\\
S & -0.94 & 90.0 & 80.0 & Pass & Fail & Fail & Fail & Pass & Pass & Fail\\
S & -0.94 & 90.0 & 160.0 & Pass & Fail & Pass & Fail & Pass & Pass & Fail\\
S & -0.5 & 10.0 & 1.0 & Fail & Fail & Pass & Fail & Fail & Fail & Fail\\
S & -0.5 & 10.0 & 10.0 & Pass & Fail & Fail & Fail & Fail & Fail & Fail\\
S & -0.5 & 10.0 & 40.0 & Pass & Fail & Fail & Fail & Fail & Fail & Fail\\
S & -0.5 & 10.0 & 80.0 & Pass & Fail & Pass & Fail & Fail & Fail & Fail\\
S & -0.5 & 10.0 & 160.0 & Pass & Fail & Pass & Fail & Fail & Fail & Fail\\
S & -0.5 & 30.0 & 1.0 & Fail & Fail & Pass & Fail & Fail & Fail & Fail\\
S & -0.5 & 30.0 & 10.0 & Pass & Fail & Fail & Fail & Fail & Fail & Fail\\
S & -0.5 & 30.0 & 40.0 & Pass & Fail & Fail & Fail & Fail & Fail & Fail\\
S & -0.5 & 30.0 & 80.0 & Pass & Fail & Fail & Fail & Fail & Fail & Fail\\
S & -0.5 & 30.0 & 160.0 & Pass & Fail & Pass & Fail & Fail & Fail & Fail\\
S & -0.5 & 50.0 & 1.0 & Fail & Fail & Pass & Fail & Fail & Fail & Fail\\
S & -0.5 & 50.0 & 10.0 & Pass & Fail & Fail & Fail & Fail & Fail & Fail\\
S & -0.5 & 50.0 & 40.0 & Pass & Fail & Fail & Fail & Fail & Fail & Fail\\
S & -0.5 & 50.0 & 80.0 & Pass & Fail & Fail & Fail & Fail & Fail & Fail\\
S & -0.5 & 50.0 & 160.0 & Pass & Fail & Pass & Fail & Fail & Fail & Fail\\
S & -0.5 & 70.0 & 1.0 & Fail & Fail & Pass & Fail & Fail & Fail & Fail\\
S & -0.5 & 70.0 & 10.0 & Fail & Fail & Fail & Fail & Fail & Fail & Fail\\
S & -0.5 & 70.0 & 40.0 & Pass & Fail & Fail & Fail & Fail & Fail & Fail\\
S & -0.5 & 70.0 & 80.0 & Pass & Fail & Fail & Fail & Fail & Fail & Fail\\
S & -0.5 & 70.0 & 160.0 & Pass & Fail & Pass & Fail & Fail & Fail & Fail\\
S & -0.5 & 90.0 & 1.0 & Fail & Fail & Pass & Fail & Fail & Fail & Fail\\
S & -0.5 & 90.0 & 10.0 & Fail & Fail & Fail & Fail & Fail & Fail & Fail\\
S & -0.5 & 90.0 & 40.0 & Pass & Fail & Fail & Fail & Fail & Fail & Fail\\
S & -0.5 & 90.0 & 80.0 & Pass & Fail & Fail & Fail & Fail & Fail & Fail\\
S & -0.5 & 90.0 & 160.0 & Pass & Fail & Pass & Fail & Pass & Pass & Fail\\
S & 0.0 & 10.0 & 1.0 & Fail & Fail & Pass & Fail & Fail & Fail & Fail\\
S & 0.0 & 10.0 & 10.0 & Pass & Fail & Pass & Fail & Fail & Fail & Fail\\
S & 0.0 & 10.0 & 40.0 & Pass & Fail & Fail & Fail & Fail & Fail & Fail\\
S & 0.0 & 10.0 & 80.0 & Pass & Fail & Pass & Fail & Fail & Fail & Fail\\
S & 0.0 & 10.0 & 160.0 & Pass & Fail & Pass & Fail & Fail & Fail & Fail\\
S & 0.0 & 30.0 & 1.0 & Fail & Fail & Pass & Fail & Fail & Fail & Fail\\
S & 0.0 & 30.0 & 10.0 & Pass & Fail & Pass & Fail & Fail & Fail & Fail\\
S & 0.0 & 30.0 & 40.0 & Pass & Fail & Fail & Fail & Fail & Fail & Fail\\
S & 0.0 & 30.0 & 80.0 & Pass & Fail & Pass & Fail & Fail & Fail & Fail\\
S & 0.0 & 30.0 & 160.0 & Pass & Fail & Pass & Fail & Fail & Fail & Fail\\
S & 0.0 & 50.0 & 1.0 & Fail & Fail & Pass & Fail & Fail & Fail & Fail\\
S & 0.0 & 50.0 & 10.0 & Pass & Fail & Pass & Fail & Fail & Fail & Fail\\
S & 0.0 & 50.0 & 40.0 & Pass & Fail & Fail & Fail & Fail & Fail & Fail\\
S & 0.0 & 50.0 & 80.0 & Pass & Fail & Fail & Fail & Fail & Fail & Fail\\
S & 0.0 & 50.0 & 160.0 & Pass & Fail & Pass & Fail & Fail & Fail & Fail\\
S & 0.0 & 70.0 & 1.0 & Fail & Fail & Pass & Fail & Fail & Fail & Fail\\
S & 0.0 & 70.0 & 10.0 & Pass & Fail & Pass & Fail & Pass & Pass & Fail\\
S & 0.0 & 70.0 & 40.0 & Pass & Fail & Fail & Fail & Fail & Fail & Fail\\
S & 0.0 & 70.0 & 80.0 & Pass & Fail & Fail & Fail & Fail & Fail & Fail\\
S & 0.0 & 70.0 & 160.0 & Pass & Fail & Pass & Fail & Fail & Fail & Fail\\
S & 0.0 & 90.0 & 1.0 & Fail & Fail & Pass & Fail & Fail & Fail & Fail\\
S & 0.0 & 90.0 & 10.0 & Pass & Fail & Pass & Fail & Pass & Pass & Fail\\
S & 0.0 & 90.0 & 40.0 & Pass & Fail & Fail & Fail & Pass & Pass & Fail\\
S & 0.0 & 90.0 & 80.0 & Pass & Fail & Fail & Fail & Fail & Fail & Fail\\
S & 0.0 & 90.0 & 160.0 & Fail & Fail & Pass & Fail & Fail & Fail & Fail\\
S & 0.5 & 10.0 & 1.0 & Fail & Fail & Pass & Fail & Fail & Fail & Fail\\
S & 0.5 & 10.0 & 10.0 & Pass & Fail & Pass & Fail & Fail & Fail & Fail\\
S & 0.5 & 10.0 & 40.0 & Pass & Fail & Pass & Fail & Fail & Fail & Fail\\
S & 0.5 & 10.0 & 80.0 & Pass & Fail & Pass & Fail & Fail & Fail & Fail\\
S & 0.5 & 10.0 & 160.0 & Pass & Fail & Pass & Fail & Fail & Fail & Fail\\
S & 0.5 & 30.0 & 1.0 & Fail & Fail & Pass & Fail & Fail & Fail & Fail\\
S & 0.5 & 30.0 & 10.0 & Pass & Fail & Pass & Fail & Pass & Pass & Fail\\
S & 0.5 & 30.0 & 40.0 & Pass & Fail & Pass & Fail & Pass & Pass & Fail\\
S & 0.5 & 30.0 & 80.0 & Pass & Fail & Pass & Fail & Fail & Fail & Fail\\
S & 0.5 & 30.0 & 160.0 & Pass & Fail & Pass & Fail & Fail & Fail & Fail\\
S & 0.5 & 50.0 & 1.0 & Fail & Fail & Pass & Fail & Fail & Fail & Fail\\
S & 0.5 & 50.0 & 10.0 & Fail & Fail & Pass & Fail & Pass & Pass & Fail\\
S & 0.5 & 50.0 & 40.0 & Fail & Fail & Pass & Fail & Pass & Pass & Fail\\
S & 0.5 & 50.0 & 80.0 & Pass & Fail & Pass & Fail & Pass & Pass & Fail\\
S & 0.5 & 50.0 & 160.0 & Pass & Fail & Pass & Fail & Pass & Pass & Fail\\
S & 0.5 & 70.0 & 1.0 & Fail & Fail & Pass & Fail & Fail & Fail & Fail\\
S & 0.5 & 70.0 & 10.0 & Fail & Fail & Pass & Fail & Pass & Pass & Fail\\
S & 0.5 & 70.0 & 40.0 & Fail & Fail & Pass & Fail & Pass & Pass & Fail\\
S & 0.5 & 70.0 & 80.0 & Fail & Fail & Pass & Fail & Pass & Pass & Fail\\
S & 0.5 & 70.0 & 160.0 & Pass & Fail & Pass & Fail & Pass & Pass & Fail\\
S & 0.5 & 90.0 & 1.0 & Fail & Fail & Pass & Fail & Pass & Pass & Fail\\
S & 0.5 & 90.0 & 10.0 & Fail & Fail & Pass & Fail & Pass & Pass & Fail\\
S & 0.5 & 90.0 & 40.0 & Fail & Fail & Pass & Fail & Pass & Pass & Fail\\
S & 0.5 & 90.0 & 80.0 & Fail & Fail & Pass & Fail & Pass & Pass & Fail\\
S & 0.5 & 90.0 & 160.0 & Pass & Fail & Pass & Fail & Pass & Pass & Fail\\
S & 0.94 & 10.0 & 1.0 & Pass & Fail & Fail & Fail & Pass & Pass & Fail\\
S & 0.94 & 10.0 & 10.0 & Pass & Fail & Fail & Fail & Pass & Pass & Fail\\
S & 0.94 & 10.0 & 40.0 & Pass & Fail & Pass & Fail & Fail & Fail & Fail\\
S & 0.94 & 10.0 & 80.0 & Fail & Fail & Pass & Fail & Fail & Fail & Fail\\
S & 0.94 & 10.0 & 160.0 & Fail & Fail & Pass & Fail & Fail & Fail & Fail\\
S & 0.94 & 30.0 & 1.0 & Fail & Fail & Pass & Fail & Fail & Fail & Fail\\
S & 0.94 & 30.0 & 10.0 & Fail & Fail & Pass & Fail & Pass & Pass & Fail\\
S & 0.94 & 30.0 & 40.0 & Pass & Fail & Fail & Fail & Pass & Pass & Fail\\
S & 0.94 & 30.0 & 80.0 & Pass & Fail & Fail & Fail & Pass & Pass & Fail\\
S & 0.94 & 30.0 & 160.0 & Fail & Fail & Pass & Fail & Pass & Pass & Fail\\
S & 0.94 & 50.0 & 1.0 & Fail & Fail & Pass & Fail & Fail & Fail & Fail\\
S & 0.94 & 50.0 & 10.0 & Fail & Fail & Fail & Fail & Pass & Pass & Fail\\
S & 0.94 & 50.0 & 40.0 & Pass & Fail & Fail & Fail & Pass & Pass & Fail\\
S & 0.94 & 50.0 & 80.0 & Pass & Fail & Fail & Fail & Pass & Pass & Fail\\
S & 0.94 & 50.0 & 160.0 & Pass & Fail & Fail & Fail & Pass & Pass & Fail\\
S & 0.94 & 70.0 & 1.0 & Fail & Fail & Pass & Fail & Fail & Fail & Fail\\
S & 0.94 & 70.0 & 10.0 & Fail & Fail & Fail & Fail & Pass & Pass & Fail\\
S & 0.94 & 70.0 & 40.0 & Pass & Fail & Fail & Fail & Pass & Pass & Fail\\
S & 0.94 & 70.0 & 80.0 & Pass & Fail & Fail & Fail & Pass & Pass & Fail\\
S & 0.94 & 70.0 & 160.0 & Fail & Fail & Fail & Fail & Pass & Pass & Fail\\
S & 0.94 & 90.0 & 1.0 & Fail & Fail & Fail & Fail & Pass & Pass & Fail\\
S & 0.94 & 90.0 & 10.0 & Fail & Fail & Fail & Fail & Pass & Pass & Fail\\
S & 0.94 & 90.0 & 40.0 & Pass & Fail & Fail & Fail & Pass & Pass & Fail\\
S & 0.94 & 90.0 & 80.0 & Fail & Fail & Fail & Fail & Pass & Pass & Fail\\
S & 0.94 & 90.0 & 160.0 & Pass & Fail & Fail & Fail & Pass & Pass & Fail\\
M & -0.94 & 10.0 & 1.0 & Fail & Fail & Fail & Fail & Fail & Fail & Fail\\
M & -0.94 & 10.0 & 10.0 & Pass & Fail & Fail & Fail & Fail & Fail & Fail\\
M & -0.94 & 10.0 & 40.0 & Pass & Fail & Fail & Fail & Fail & Fail & Fail\\
M & -0.94 & 10.0 & 80.0 & Pass & Fail & Fail & Fail & Fail & Fail & Fail\\
M & -0.94 & 10.0 & 160.0 & Pass & Fail & Pass & Fail & Fail & Fail & Fail\\
M & -0.94 & 30.0 & 1.0 & Fail & Fail & Fail & Fail & Pass & Pass & Fail\\
M & -0.94 & 30.0 & 10.0 & Pass & Fail & Fail & Fail & Fail & Fail & Fail\\
M & -0.94 & 30.0 & 40.0 & Pass & Fail & Fail & Fail & Fail & Fail & Fail\\
M & -0.94 & 30.0 & 80.0 & Pass & Fail & Fail & Fail & Fail & Fail & Fail\\
M & -0.94 & 30.0 & 160.0 & Pass & Fail & Pass & Fail & Fail & Fail & Fail\\
M & -0.94 & 50.0 & 1.0 & Fail & Fail & Fail & Fail & Fail & Fail & Fail\\
M & -0.94 & 50.0 & 10.0 & Pass & Fail & Fail & Fail & Fail & Fail & Fail\\
M & -0.94 & 50.0 & 40.0 & Pass & Fail & Fail & Fail & Fail & Fail & Fail\\
M & -0.94 & 50.0 & 80.0 & Pass & Fail & Fail & Fail & Fail & Fail & Fail\\
M & -0.94 & 50.0 & 160.0 & Pass & Fail & Fail & Fail & Fail & Fail & Fail\\
M & -0.94 & 70.0 & 1.0 & Fail & Fail & Fail & Fail & Fail & Fail & Fail\\
M & -0.94 & 70.0 & 10.0 & Pass & Fail & Fail & Fail & Fail & Fail & Fail\\
M & -0.94 & 70.0 & 40.0 & Pass & Fail & Fail & Fail & Fail & Fail & Fail\\
M & -0.94 & 70.0 & 80.0 & Pass & Fail & Fail & Fail & Fail & Fail & Fail\\
M & -0.94 & 70.0 & 160.0 & Pass & Fail & Fail & Fail & Fail & Fail & Fail\\
M & -0.94 & 90.0 & 1.0 & Pass & Fail & Fail & Fail & Fail & Fail & Fail\\
M & -0.94 & 90.0 & 10.0 & Pass & Fail & Fail & Fail & Fail & Fail & Fail\\
M & -0.94 & 90.0 & 40.0 & Pass & Fail & Fail & Fail & Fail & Fail & Fail\\
M & -0.94 & 90.0 & 80.0 & Pass & Fail & Fail & Fail & Fail & Fail & Fail\\
M & -0.94 & 90.0 & 160.0 & Pass & Fail & Fail & Fail & Fail & Fail & Fail\\
M & -0.5 & 10.0 & 1.0 & Fail & Fail & Fail & Fail & Fail & Fail & Fail\\
M & -0.5 & 10.0 & 10.0 & Pass & Fail & Fail & Fail & Fail & Fail & Fail\\
M & -0.5 & 10.0 & 40.0 & Pass & Fail & Fail & Fail & Fail & Fail & Fail\\
M & -0.5 & 10.0 & 80.0 & Pass & Fail & Fail & Fail & Fail & Fail & Fail\\
M & -0.5 & 10.0 & 160.0 & Pass & Fail & Fail & Fail & Fail & Fail & Fail\\
M & -0.5 & 30.0 & 1.0 & Fail & Fail & Fail & Fail & Pass & Pass & Fail\\
M & -0.5 & 30.0 & 10.0 & Pass & Fail & Fail & Fail & Fail & Fail & Fail\\
M & -0.5 & 30.0 & 40.0 & Pass & Fail & Fail & Fail & Fail & Fail & Fail\\
M & -0.5 & 30.0 & 80.0 & Pass & Fail & Fail & Fail & Fail & Fail & Fail\\
M & -0.5 & 30.0 & 160.0 & Pass & Fail & Fail & Fail & Fail & Fail & Fail\\
M & -0.5 & 50.0 & 1.0 & Fail & Fail & Fail & Fail & Fail & Fail & Fail\\
M & -0.5 & 50.0 & 10.0 & Pass & Fail & Fail & Fail & Fail & Fail & Fail\\
M & -0.5 & 50.0 & 40.0 & Pass & Fail & Fail & Fail & Fail & Fail & Fail\\
M & -0.5 & 50.0 & 80.0 & Pass & Fail & Fail & Fail & Fail & Fail & Fail\\
M & -0.5 & 50.0 & 160.0 & Pass & Fail & Fail & Fail & Fail & Fail & Fail\\
M & -0.5 & 70.0 & 1.0 & Fail & Fail & Fail & Fail & Fail & Fail & Fail\\
M & -0.5 & 70.0 & 10.0 & Pass & Fail & Fail & Fail & Fail & Fail & Fail\\
M & -0.5 & 70.0 & 40.0 & Pass & Fail & Fail & Fail & Fail & Fail & Fail\\
M & -0.5 & 70.0 & 80.0 & Pass & Fail & Fail & Fail & Fail & Fail & Fail\\
M & -0.5 & 70.0 & 160.0 & Pass & Fail & Fail & Fail & Fail & Fail & Fail\\
M & -0.5 & 90.0 & 1.0 & Fail & Fail & Fail & Fail & Fail & Fail & Fail\\
M & -0.5 & 90.0 & 10.0 & Pass & Fail & Fail & Fail & Fail & Fail & Fail\\
M & -0.5 & 90.0 & 40.0 & Pass & Fail & Fail & Fail & Fail & Fail & Fail\\
M & -0.5 & 90.0 & 80.0 & Pass & Fail & Fail & Fail & Fail & Fail & Fail\\
M & -0.5 & 90.0 & 160.0 & Pass & Fail & Fail & Fail & Fail & Fail & Fail\\
M & 0.0 & 10.0 & 1.0 & Pass & Fail & Pass & Fail & Fail & Fail & Fail\\
M & 0.0 & 10.0 & 10.0 & Pass & Fail & Fail & Fail & Fail & Fail & Fail\\
M & 0.0 & 10.0 & 40.0 & Pass & Fail & Fail & Fail & Fail & Fail & Fail\\
M & 0.0 & 10.0 & 80.0 & Pass & Fail & Fail & Fail & Fail & Fail & Fail\\
M & 0.0 & 10.0 & 160.0 & Pass & Fail & Pass & Fail & Fail & Fail & Fail\\
M & 0.0 & 30.0 & 1.0 & Pass & Fail & Pass & Fail & Fail & Fail & Fail\\
M & 0.0 & 30.0 & 10.0 & Pass & Fail & Fail & Fail & Fail & Fail & Fail\\
M & 0.0 & 30.0 & 40.0 & Pass & Fail & Fail & Fail & Fail & Fail & Fail\\
M & 0.0 & 30.0 & 80.0 & Pass & Fail & Fail & Fail & Fail & Fail & Fail\\
M & 0.0 & 30.0 & 160.0 & Pass & Fail & Pass & Fail & Fail & Fail & Fail\\
M & 0.0 & 50.0 & 1.0 & Pass & Fail & Fail & Fail & Fail & Fail & Fail\\
M & 0.0 & 50.0 & 10.0 & Pass & Fail & Fail & Fail & Fail & Fail & Fail\\
M & 0.0 & 50.0 & 40.0 & Pass & Fail & Fail & Fail & Fail & Fail & Fail\\
M & 0.0 & 50.0 & 80.0 & Pass & Fail & Fail & Fail & Fail & Fail & Fail\\
M & 0.0 & 50.0 & 160.0 & Pass & Fail & Pass & Fail & Fail & Fail & Fail\\
M & 0.0 & 70.0 & 1.0 & Fail & Fail & Fail & Fail & Fail & Fail & Fail\\
M & 0.0 & 70.0 & 10.0 & Pass & Fail & Fail & Fail & Fail & Fail & Fail\\
M & 0.0 & 70.0 & 40.0 & Pass & Fail & Fail & Fail & Fail & Fail & Fail\\
M & 0.0 & 70.0 & 80.0 & Pass & Fail & Fail & Fail & Fail & Fail & Fail\\
M & 0.0 & 70.0 & 160.0 & Pass & Fail & Pass & Fail & Fail & Fail & Fail\\
M & 0.0 & 90.0 & 1.0 & Pass & Fail & Fail & Fail & Fail & Fail & Fail\\
M & 0.0 & 90.0 & 10.0 & Pass & Fail & Fail & Fail & Fail & Fail & Fail\\
M & 0.0 & 90.0 & 40.0 & Pass & Fail & Fail & Fail & Fail & Fail & Fail\\
M & 0.0 & 90.0 & 80.0 & Pass & Fail & Fail & Fail & Fail & Fail & Fail\\
M & 0.0 & 90.0 & 160.0 & Pass & Fail & Pass & Fail & Fail & Fail & Fail\\
M & 0.5 & 10.0 & 1.0 & Fail & Fail & Fail & Fail & Fail & Fail & Fail\\
M & 0.5 & 10.0 & 10.0 & Pass & Fail & Fail & Fail & Fail & Fail & Fail\\
M & 0.5 & 10.0 & 40.0 & Pass & Fail & Fail & Fail & Fail & Fail & Fail\\
M & 0.5 & 10.0 & 80.0 & Pass & Fail & Pass & Fail & Fail & Fail & Fail\\
M & 0.5 & 10.0 & 160.0 & Pass & Fail & Pass & Fail & Fail & Fail & Fail\\
M & 0.5 & 30.0 & 1.0 & Fail & Fail & Fail & Fail & Fail & Fail & Fail\\
M & 0.5 & 30.0 & 10.0 & Pass & Fail & Fail & Fail & Fail & Fail & Fail\\
M & 0.5 & 30.0 & 40.0 & Pass & Fail & Fail & Fail & Fail & Fail & Fail\\
M & 0.5 & 30.0 & 80.0 & Pass & Fail & Pass & Fail & Fail & Fail & Fail\\
M & 0.5 & 30.0 & 160.0 & Pass & Fail & Pass & Fail & Fail & Fail & Fail\\
M & 0.5 & 50.0 & 1.0 & Fail & Fail & Fail & Fail & Fail & Fail & Fail\\
M & 0.5 & 50.0 & 10.0 & Pass & Fail & Fail & Fail & Fail & Fail & Fail\\
M & 0.5 & 50.0 & 40.0 & Pass & Fail & Fail & Fail & Fail & Fail & Fail\\
M & 0.5 & 50.0 & 80.0 & Pass & Fail & Fail & Fail & Fail & Fail & Fail\\
M & 0.5 & 50.0 & 160.0 & Pass & Fail & Pass & Fail & Fail & Fail & Fail\\
M & 0.5 & 70.0 & 1.0 & Fail & Fail & Fail & Fail & Fail & Fail & Fail\\
M & 0.5 & 70.0 & 10.0 & Pass & Fail & Fail & Fail & Pass & Pass & Fail\\
M & 0.5 & 70.0 & 40.0 & Pass & Fail & Fail & Fail & Fail & Fail & Fail\\
M & 0.5 & 70.0 & 80.0 & Pass & Fail & Fail & Fail & Fail & Fail & Fail\\
M & 0.5 & 70.0 & 160.0 & Pass & Fail & Fail & Fail & Fail & Fail & Fail\\
M & 0.5 & 90.0 & 1.0 & Fail & Fail & Fail & Fail & Pass & Pass & Fail\\
M & 0.5 & 90.0 & 10.0 & Pass & Fail & Fail & Fail & Pass & Pass & Fail\\
M & 0.5 & 90.0 & 40.0 & Pass & Fail & Fail & Fail & Pass & Pass & Fail\\
M & 0.5 & 90.0 & 80.0 & Pass & Fail & Fail & Fail & Fail & Fail & Fail\\
M & 0.5 & 90.0 & 160.0 & Pass & Fail & Fail & Fail & Fail & Fail & Fail\\
M & 0.94 & 10.0 & 1.0 & Pass & Fail & Fail & Fail & Fail & Fail & Fail\\
M & 0.94 & 10.0 & 10.0 & Pass & Fail & Fail & Fail & Fail & Fail & Fail\\
M & 0.94 & 10.0 & 40.0 & Pass & Fail & Fail & Fail & Fail & Fail & Fail\\
M & 0.94 & 10.0 & 80.0 & Pass & Fail & Fail & Fail & Fail & Fail & Fail\\
M & 0.94 & 10.0 & 160.0 & Pass & Fail & Pass & Fail & Fail & Fail & Fail\\
M & 0.94 & 30.0 & 1.0 & Pass & Fail & Fail & Fail & Fail & Fail & Fail\\
M & 0.94 & 30.0 & 10.0 & Pass & Fail & Fail & Fail & Fail & Fail & Fail\\
M & 0.94 & 30.0 & 40.0 & Pass & Fail & Fail & Fail & Fail & Fail & Fail\\
M & 0.94 & 30.0 & 80.0 & Pass & Fail & Fail & Fail & Fail & Fail & Fail\\
M & 0.94 & 30.0 & 160.0 & Pass & Fail & Fail & Fail & Fail & Fail & Fail\\
M & 0.94 & 50.0 & 1.0 & Pass & Fail & Fail & Fail & Fail & Fail & Fail\\
M & 0.94 & 50.0 & 10.0 & Pass & Fail & Fail & Fail & Pass & Pass & Fail\\
M & 0.94 & 50.0 & 40.0 & Pass & Fail & Fail & Fail & Fail & Fail & Fail\\
M & 0.94 & 50.0 & 80.0 & Pass & Fail & Fail & Fail & Fail & Fail & Fail\\
M & 0.94 & 50.0 & 160.0 & Pass & Fail & Fail & Fail & Fail & Fail & Fail\\
M & 0.94 & 70.0 & 1.0 & Pass & Fail & Fail & Fail & Fail & Fail & Fail\\
M & 0.94 & 70.0 & 10.0 & Pass & Fail & Fail & Fail & Pass & Pass & Fail\\
M & 0.94 & 70.0 & 40.0 & Pass & Fail & Fail & Fail & Pass & Pass & Fail\\
M & 0.94 & 70.0 & 80.0 & Pass & Fail & Fail & Fail & Pass & Pass & Fail\\
M & 0.94 & 70.0 & 160.0 & Pass & Fail & Fail & Fail & Pass & Pass & Fail\\
M & 0.94 & 90.0 & 1.0 & Pass & Fail & Fail & Fail & Pass & Pass & Fail\\
M & 0.94 & 90.0 & 10.0 & Pass & Fail & Fail & Fail & Pass & Pass & Fail\\
M & 0.94 & 90.0 & 40.0 & Pass & Fail & Fail & Fail & Pass & Pass & Fail\\
M & 0.94 & 90.0 & 80.0 & Pass & Fail & Fail & Fail & Pass & Pass & Fail\\
M & 0.94 & 90.0 & 160.0 & Pass & Fail & Fail & Fail & Pass & Pass & Fail\\
\enddata
\end{deluxetable*}

\startlongtable
\begin{deluxetable*}{cccc|cccc|c|ccccc|c|c}
\tabletypesize{\scriptsize}
\tablecaption{Pass/Fail Table, Hamr variable kappa models}
\tablehead{ \colhead{M/S}  &  %
\colhead{Spin}  &  %
\colhead{$i$}  &  %
\colhead{$\rhigh$}  &  %
\colhead{$F_{86}$}  &  %
\colhead{$\lambda_{maj,86}$}  &  %
\colhead{$F_{2\mu{\rm m}}$}  &  %
\colhead{$L_X$}  &  %
\colhead{non-EHT}  &  %
\colhead{$\lambda_{230}$}  &  %
\colhead{EHT}  &  %
\colhead{All}}
\startdata
S & -0.93 & 10.0 & 1.0 & Fail & Fail & Pass & Pass & Fail & Fail & Fail & Fail\\
S & -0.93 & 10.0 & 10.0 & Fail & Pass & Pass & Pass & Fail & Pass & Pass & Fail\\
S & -0.93 & 10.0 & 40.0 & Fail & Fail & Pass & Pass & Fail & Pass & Pass & Fail\\
S & -0.93 & 10.0 & 160.0 & Fail & Pass & Pass & Pass & Fail & Pass & Pass & Fail\\
S & -0.93 & 30.0 & 1.0 & Fail & Pass & Pass & Pass & Fail & Pass & Pass & Fail\\
S & -0.93 & 30.0 & 10.0 & Fail & Pass & Pass & Pass & Fail & Pass & Pass & Fail\\
S & -0.93 & 30.0 & 40.0 & Fail & Fail & Pass & Pass & Fail & Pass & Pass & Fail\\
S & -0.93 & 30.0 & 160.0 & Fail & Pass & Pass & Pass & Fail & Pass & Pass & Fail\\
S & -0.93 & 50.0 & 1.0 & Fail & Pass & Pass & Pass & Fail & Pass & Pass & Fail\\
S & -0.93 & 50.0 & 10.0 & Fail & Pass & Pass & Pass & Fail & Pass & Pass & Fail\\
S & -0.93 & 50.0 & 40.0 & Fail & Pass & Pass & Pass & Fail & Pass & Pass & Fail\\
S & -0.93 & 50.0 & 160.0 & Fail & Fail & Pass & Pass & Fail & Pass & Pass & Fail\\
S & -0.93 & 70.0 & 1.0 & Fail & Pass & Pass & Pass & Fail & Pass & Pass & Fail\\
S & -0.93 & 70.0 & 10.0 & Fail & Fail & Pass & Pass & Fail & Pass & Pass & Fail\\
S & -0.93 & 70.0 & 40.0 & Fail & Pass & Pass & Pass & Fail & Pass & Pass & Fail\\
S & -0.93 & 70.0 & 160.0 & Fail & Fail & Pass & Pass & Fail & Pass & Pass & Fail\\
S & -0.93 & 90.0 & 1.0 & Pass & Fail & Pass & Pass & Fail & Pass & Pass & Fail\\
S & -0.93 & 90.0 & 10.0 & Fail & Fail & Pass & Pass & Fail & Pass & Pass & Fail\\
S & -0.93 & 90.0 & 40.0 & Fail & Pass & Pass & Pass & Fail & Pass & Pass & Fail\\
S & -0.93 & 90.0 & 160.0 & Fail & Fail & Pass & Pass & Fail & Pass & Pass & Fail\\
S & -0.5 & 10.0 & 1.0 & Fail & Pass & Pass & Pass & Fail & Pass & Pass & Fail\\
S & -0.5 & 10.0 & 10.0 & Pass & Fail & Pass & Pass & Fail & Pass & Pass & Fail\\
S & -0.5 & 10.0 & 40.0 & Fail & Fail & Pass & Pass & Fail & Pass & Pass & Fail\\
S & -0.5 & 10.0 & 160.0 & Pass & Fail & Pass & Pass & Fail & Pass & Pass & Fail\\
S & -0.5 & 30.0 & 1.0 & Fail & Fail & Pass & Pass & Fail & Pass & Pass & Fail\\
S & -0.5 & 30.0 & 10.0 & Pass & Fail & Pass & Pass & Fail & Pass & Pass & Fail\\
S & -0.5 & 30.0 & 40.0 & Fail & Pass & Pass & Pass & Fail & Pass & Pass & Fail\\
S & -0.5 & 30.0 & 160.0 & Fail & Pass & Pass & Pass & Fail & Pass & Pass & Fail\\
S & -0.5 & 50.0 & 1.0 & Fail & Fail & Pass & Pass & Fail & Pass & Pass & Fail\\
S & -0.5 & 50.0 & 10.0 & Pass & Fail & Pass & Pass & Fail & Pass & Pass & Fail\\
S & -0.5 & 50.0 & 40.0 & Pass & Pass & Pass & Pass & Pass & Pass & Pass & Pass\\
S & -0.5 & 50.0 & 160.0 & Pass & Pass & Pass & Pass & Pass & Pass & Pass & Pass\\
S & -0.5 & 70.0 & 1.0 & Pass & Fail & Pass & Pass & Fail & Pass & Pass & Fail\\
S & -0.5 & 70.0 & 10.0 & Fail & Fail & Pass & Pass & Fail & Pass & Pass & Fail\\
S & -0.5 & 70.0 & 40.0 & Pass & Pass & Pass & Pass & Pass & Pass & Pass & Pass\\
S & -0.5 & 70.0 & 160.0 & Pass & Fail & Pass & Pass & Fail & Pass & Pass & Fail\\
S & -0.5 & 90.0 & 1.0 & Pass & Fail & Pass & Pass & Fail & Pass & Pass & Fail\\
S & -0.5 & 90.0 & 10.0 & Pass & Pass & Pass & Pass & Pass & Pass & Pass & Pass\\
S & -0.5 & 90.0 & 40.0 & Pass & Pass & Pass & Pass & Pass & Pass & Pass & Pass\\
S & -0.5 & 90.0 & 160.0 & Pass & Fail & Pass & Pass & Fail & Pass & Pass & Fail\\
S & 0.0 & 10.0 & 1.0 & Fail & Pass & Pass & Fail & Fail & Pass & Pass & Fail\\
S & 0.0 & 10.0 & 10.0 & Pass & Fail & Fail & Pass & Fail & Pass & Pass & Fail\\
S & 0.0 & 10.0 & 40.0 & Fail & Fail & Fail & Pass & Fail & Pass & Pass & Fail\\
S & 0.0 & 10.0 & 160.0 & Pass & Fail & Pass & Pass & Fail & Pass & Pass & Fail\\
S & 0.0 & 30.0 & 1.0 & Fail & Fail & Fail & Fail & Fail & Pass & Pass & Fail\\
S & 0.0 & 30.0 & 10.0 & Pass & Fail & Fail & Pass & Fail & Pass & Pass & Fail\\
S & 0.0 & 30.0 & 40.0 & Pass & Fail & Fail & Pass & Fail & Pass & Pass & Fail\\
S & 0.0 & 30.0 & 160.0 & Fail & Fail & Pass & Pass & Fail & Pass & Pass & Fail\\
S & 0.0 & 50.0 & 1.0 & Fail & Fail & Fail & Fail & Fail & Pass & Pass & Fail\\
S & 0.0 & 50.0 & 10.0 & Pass & Pass & Fail & Pass & Fail & Pass & Pass & Fail\\
S & 0.0 & 50.0 & 40.0 & Pass & Fail & Fail & Fail & Fail & Pass & Pass & Fail\\
S & 0.0 & 50.0 & 160.0 & Pass & Fail & Pass & Pass & Fail & Pass & Pass & Fail\\
S & 0.0 & 70.0 & 1.0 & Fail & Fail & Fail & Fail & Fail & Pass & Pass & Fail\\
S & 0.0 & 70.0 & 10.0 & Fail & Pass & Fail & Fail & Fail & Pass & Pass & Fail\\
S & 0.0 & 70.0 & 40.0 & Pass & Fail & Fail & Fail & Fail & Pass & Pass & Fail\\
S & 0.0 & 70.0 & 160.0 & Pass & Fail & Pass & Pass & Fail & Pass & Pass & Fail\\
S & 0.0 & 90.0 & 1.0 & Fail & Fail & Fail & Fail & Fail & Pass & Pass & Fail\\
S & 0.0 & 90.0 & 10.0 & Fail & Pass & Fail & Fail & Fail & Pass & Pass & Fail\\
S & 0.0 & 90.0 & 40.0 & Pass & Fail & Fail & Fail & Fail & Pass & Pass & Fail\\
S & 0.0 & 90.0 & 160.0 & Pass & Fail & Pass & Pass & Fail & Pass & Pass & Fail\\
S & 0.5 & 10.0 & 1.0 & Fail & Fail & Pass & Pass & Fail & Pass & Pass & Fail\\
S & 0.5 & 10.0 & 10.0 & Pass & Fail & Pass & Pass & Fail & Pass & Pass & Fail\\
S & 0.5 & 10.0 & 40.0 & Fail & Fail & Pass & Pass & Fail & Pass & Pass & Fail\\
S & 0.5 & 10.0 & 160.0 & Fail & Fail & Pass & Pass & Fail & Pass & Pass & Fail\\
S & 0.5 & 30.0 & 1.0 & Fail & Fail & Pass & Pass & Fail & Pass & Pass & Fail\\
S & 0.5 & 30.0 & 10.0 & Pass & Fail & Pass & Pass & Fail & Pass & Pass & Fail\\
S & 0.5 & 30.0 & 40.0 & Fail & Pass & Pass & Pass & Fail & Pass & Pass & Fail\\
S & 0.5 & 30.0 & 160.0 & Fail & Pass & Pass & Pass & Fail & Pass & Pass & Fail\\
S & 0.5 & 50.0 & 1.0 & Fail & Fail & Pass & Pass & Fail & Pass & Pass & Fail\\
S & 0.5 & 50.0 & 10.0 & Fail & Fail & Pass & Pass & Fail & Pass & Pass & Fail\\
S & 0.5 & 50.0 & 40.0 & Fail & Pass & Pass & Pass & Fail & Pass & Pass & Fail\\
S & 0.5 & 50.0 & 160.0 & Fail & Fail & Pass & Pass & Fail & Pass & Pass & Fail\\
S & 0.5 & 70.0 & 1.0 & Pass & Fail & Pass & Pass & Fail & Pass & Pass & Fail\\
S & 0.5 & 70.0 & 10.0 & Fail & Fail & Pass & Pass & Fail & Pass & Pass & Fail\\
S & 0.5 & 70.0 & 40.0 & Fail & Fail & Pass & Pass & Fail & Pass & Pass & Fail\\
S & 0.5 & 70.0 & 160.0 & Pass & Fail & Pass & Pass & Fail & Pass & Pass & Fail\\
S & 0.5 & 90.0 & 1.0 & Pass & Fail & Pass & Pass & Fail & Pass & Pass & Fail\\
S & 0.5 & 90.0 & 10.0 & Fail & Fail & Pass & Pass & Fail & Pass & Pass & Fail\\
S & 0.5 & 90.0 & 40.0 & Fail & Fail & Pass & Pass & Fail & Pass & Pass & Fail\\
S & 0.5 & 90.0 & 160.0 & Pass & Fail & Pass & Pass & Fail & Pass & Pass & Fail\\
S & 0.93 & 10.0 & 1.0 & Fail & Fail & Pass & Pass & Fail & Pass & Pass & Fail\\
S & 0.93 & 10.0 & 10.0 & Pass & Fail & Pass & Pass & Fail & Pass & Pass & Fail\\
S & 0.93 & 10.0 & 40.0 & Fail & Fail & Pass & Pass & Fail & Pass & Pass & Fail\\
S & 0.93 & 10.0 & 160.0 & Fail & Fail & Pass & Pass & Fail & Pass & Pass & Fail\\
S & 0.93 & 30.0 & 1.0 & Fail & Fail & Pass & Pass & Fail & Pass & Pass & Fail\\
S & 0.93 & 30.0 & 10.0 & Fail & Fail & Pass & Pass & Fail & Pass & Pass & Fail\\
S & 0.93 & 30.0 & 40.0 & Fail & Pass & Pass & Pass & Fail & Pass & Pass & Fail\\
S & 0.93 & 30.0 & 160.0 & Fail & Pass & Pass & Pass & Fail & Pass & Pass & Fail\\
S & 0.93 & 50.0 & 1.0 & Fail & Fail & Pass & Pass & Fail & Pass & Pass & Fail\\
S & 0.93 & 50.0 & 10.0 & Fail & Fail & Pass & Pass & Fail & Pass & Pass & Fail\\
S & 0.93 & 50.0 & 40.0 & Fail & Fail & Pass & Pass & Fail & Pass & Pass & Fail\\
S & 0.93 & 50.0 & 160.0 & Fail & Fail & Pass & Pass & Fail & Pass & Pass & Fail\\
S & 0.93 & 70.0 & 1.0 & Fail & Fail & Pass & Pass & Fail & Pass & Pass & Fail\\
S & 0.93 & 70.0 & 10.0 & Fail & Fail & Pass & Pass & Fail & Pass & Pass & Fail\\
S & 0.93 & 70.0 & 40.0 & Fail & Fail & Pass & Pass & Fail & Pass & Pass & Fail\\
S & 0.93 & 70.0 & 160.0 & Fail & Fail & Pass & Pass & Fail & Pass & Pass & Fail\\
S & 0.93 & 90.0 & 1.0 & Pass & Fail & Pass & Pass & Fail & Pass & Pass & Fail\\
S & 0.93 & 90.0 & 10.0 & Fail & Fail & Pass & Pass & Fail & Pass & Pass & Fail\\
S & 0.93 & 90.0 & 40.0 & Fail & Fail & Pass & Pass & Fail & Pass & Pass & Fail\\
S & 0.93 & 90.0 & 160.0 & Fail & Fail & Pass & Pass & Fail & Pass & Pass & Fail\\
M & -0.93 & 10.0 & 1.0 & Pass & Pass & Fail & Fail & Fail & Pass & Pass & Fail\\
M & -0.93 & 10.0 & 10.0 & Pass & Fail & Fail & Fail & Fail & Pass & Pass & Fail\\
M & -0.93 & 10.0 & 40.0 & Pass & Fail & Fail & Fail & Fail & Pass & Pass & Fail\\
M & -0.93 & 10.0 & 160.0 & Pass & Fail & Fail & Fail & Fail & Pass & Pass & Fail\\
M & -0.93 & 30.0 & 1.0 & Pass & Pass & Fail & Fail & Fail & Pass & Pass & Fail\\
M & -0.93 & 30.0 & 10.0 & Pass & Fail & Fail & Fail & Fail & Pass & Pass & Fail\\
M & -0.93 & 30.0 & 40.0 & Pass & Fail & Fail & Fail & Fail & Pass & Pass & Fail\\
M & -0.93 & 30.0 & 160.0 & Pass & Pass & Fail & Fail & Fail & Pass & Pass & Fail\\
M & -0.93 & 50.0 & 1.0 & Pass & Pass & Fail & Fail & Fail & Pass & Pass & Fail\\
M & -0.93 & 50.0 & 10.0 & Pass & Pass & Fail & Fail & Fail & Pass & Pass & Fail\\
M & -0.93 & 50.0 & 40.0 & Pass & Pass & Fail & Fail & Fail & Pass & Pass & Fail\\
M & -0.93 & 50.0 & 160.0 & Pass & Pass & Fail & Fail & Fail & Pass & Pass & Fail\\
M & -0.93 & 70.0 & 1.0 & Pass & Pass & Fail & Fail & Fail & Pass & Pass & Fail\\
M & -0.93 & 70.0 & 10.0 & Pass & Pass & Fail & Fail & Fail & Pass & Pass & Fail\\
M & -0.93 & 70.0 & 40.0 & Pass & Pass & Fail & Fail & Fail & Pass & Pass & Fail\\
M & -0.93 & 70.0 & 160.0 & Pass & Pass & Fail & Fail & Fail & Pass & Pass & Fail\\
M & -0.93 & 90.0 & 1.0 & Pass & Pass & Fail & Fail & Fail & Pass & Pass & Fail\\
M & -0.93 & 90.0 & 10.0 & Pass & Pass & Fail & Fail & Fail & Pass & Pass & Fail\\
M & -0.93 & 90.0 & 40.0 & Pass & Pass & Fail & Fail & Fail & Pass & Pass & Fail\\
M & -0.93 & 90.0 & 160.0 & Pass & Pass & Fail & Fail & Fail & Pass & Pass & Fail\\
M & -0.5 & 10.0 & 1.0 & Pass & Pass & Fail & Fail & Fail & Pass & Pass & Fail\\
M & -0.5 & 10.0 & 10.0 & Pass & Fail & Fail & Fail & Fail & Pass & Pass & Fail\\
M & -0.5 & 10.0 & 40.0 & Pass & Fail & Fail & Fail & Fail & Pass & Pass & Fail\\
M & -0.5 & 10.0 & 160.0 & Pass & Fail & Fail & Fail & Fail & Pass & Pass & Fail\\
M & -0.5 & 30.0 & 1.0 & Pass & Pass & Fail & Fail & Fail & Pass & Pass & Fail\\
M & -0.5 & 30.0 & 10.0 & Pass & Pass & Fail & Fail & Fail & Pass & Pass & Fail\\
M & -0.5 & 30.0 & 40.0 & Pass & Pass & Fail & Fail & Fail & Pass & Pass & Fail\\
M & -0.5 & 30.0 & 160.0 & Pass & Pass & Fail & Fail & Fail & Pass & Pass & Fail\\
M & -0.5 & 50.0 & 1.0 & Pass & Pass & Fail & Fail & Fail & Pass & Pass & Fail\\
M & -0.5 & 50.0 & 10.0 & Pass & Pass & Fail & Fail & Fail & Pass & Pass & Fail\\
M & -0.5 & 50.0 & 40.0 & Pass & Pass & Fail & Fail & Fail & Pass & Pass & Fail\\
M & -0.5 & 50.0 & 160.0 & Pass & Pass & Fail & Fail & Fail & Pass & Pass & Fail\\
M & -0.5 & 70.0 & 1.0 & Pass & Fail & Fail & Fail & Fail & Pass & Pass & Fail\\
M & -0.5 & 70.0 & 10.0 & Pass & Pass & Fail & Fail & Fail & Pass & Pass & Fail\\
M & -0.5 & 70.0 & 40.0 & Pass & Pass & Fail & Fail & Fail & Pass & Pass & Fail\\
M & -0.5 & 70.0 & 160.0 & Pass & Pass & Fail & Fail & Fail & Pass & Pass & Fail\\
M & -0.5 & 90.0 & 1.0 & Pass & Fail & Fail & Fail & Fail & Pass & Pass & Fail\\
M & -0.5 & 90.0 & 10.0 & Pass & Pass & Fail & Fail & Fail & Pass & Pass & Fail\\
M & -0.5 & 90.0 & 40.0 & Pass & Pass & Fail & Fail & Fail & Pass & Pass & Fail\\
M & -0.5 & 90.0 & 160.0 & Pass & Pass & Fail & Fail & Fail & Pass & Pass & Fail\\
M & 0.0 & 10.0 & 1.0 & Fail & Pass & Fail & Fail & Fail & Pass & Pass & Fail\\
M & 0.0 & 10.0 & 10.0 & Pass & Fail & Fail & Fail & Fail & Pass & Pass & Fail\\
M & 0.0 & 10.0 & 40.0 & Pass & Fail & Pass & Pass & Fail & Pass & Pass & Fail\\
M & 0.0 & 10.0 & 160.0 & Pass & Fail & Pass & Pass & Fail & Pass & Pass & Fail\\
M & 0.0 & 30.0 & 1.0 & Fail & Pass & Fail & Fail & Fail & Pass & Pass & Fail\\
M & 0.0 & 30.0 & 10.0 & Pass & Pass & Fail & Fail & Fail & Pass & Pass & Fail\\
M & 0.0 & 30.0 & 40.0 & Pass & Pass & Fail & Pass & Fail & Pass & Pass & Fail\\
M & 0.0 & 30.0 & 160.0 & Pass & Pass & Pass & Pass & Pass & Pass & Pass & Pass\\
M & 0.0 & 50.0 & 1.0 & Fail & Fail & Fail & Fail & Fail & Pass & Pass & Fail\\
M & 0.0 & 50.0 & 10.0 & Pass & Pass & Fail & Fail & Fail & Pass & Pass & Fail\\
M & 0.0 & 50.0 & 40.0 & Pass & Pass & Fail & Pass & Fail & Pass & Pass & Fail\\
M & 0.0 & 50.0 & 160.0 & Pass & Pass & Pass & Pass & Pass & Pass & Pass & Pass\\
M & 0.0 & 70.0 & 1.0 & Pass & Fail & Fail & Fail & Fail & Pass & Pass & Fail\\
M & 0.0 & 70.0 & 10.0 & Pass & Pass & Fail & Fail & Fail & Pass & Pass & Fail\\
M & 0.0 & 70.0 & 40.0 & Pass & Pass & Fail & Pass & Fail & Pass & Pass & Fail\\
M & 0.0 & 70.0 & 160.0 & Pass & Pass & Pass & Pass & Pass & Pass & Pass & Pass\\
M & 0.0 & 90.0 & 1.0 & Pass & Fail & Fail & Fail & Fail & Pass & Pass & Fail\\
M & 0.0 & 90.0 & 10.0 & Pass & Pass & Fail & Fail & Fail & Pass & Pass & Fail\\
M & 0.0 & 90.0 & 40.0 & Pass & Pass & Fail & Pass & Fail & Pass & Pass & Fail\\
M & 0.0 & 90.0 & 160.0 & Pass & Pass & Pass & Pass & Pass & Pass & Pass & Pass\\
M & 0.5 & 10.0 & 1.0 & Pass & Fail & Pass & Pass & Fail & Pass & Pass & Fail\\
M & 0.5 & 10.0 & 10.0 & Pass & Fail & Pass & Pass & Fail & Pass & Pass & Fail\\
M & 0.5 & 10.0 & 40.0 & Pass & Fail & Pass & Pass & Fail & Pass & Pass & Fail\\
M & 0.5 & 10.0 & 160.0 & Pass & Pass & Pass & Pass & Pass & Pass & Pass & Pass\\
M & 0.5 & 30.0 & 1.0 & Pass & Pass & Pass & Pass & Pass & Pass & Pass & Pass\\
M & 0.5 & 30.0 & 10.0 & Pass & Pass & Pass & Pass & Pass & Pass & Pass & Pass\\
M & 0.5 & 30.0 & 40.0 & Pass & Pass & Pass & Pass & Pass & Pass & Pass & Pass\\
M & 0.5 & 30.0 & 160.0 & Pass & Pass & Pass & Pass & Pass & Pass & Pass & Pass\\
M & 0.5 & 50.0 & 1.0 & Pass & Pass & Pass & Pass & Pass & Pass & Pass & Pass\\
M & 0.5 & 50.0 & 10.0 & Pass & Pass & Pass & Pass & Pass & Pass & Pass & Pass\\
M & 0.5 & 50.0 & 40.0 & Pass & Pass & Pass & Pass & Pass & Pass & Pass & Pass\\
M & 0.5 & 50.0 & 160.0 & Pass & Pass & Pass & Pass & Pass & Pass & Pass & Pass\\
M & 0.5 & 70.0 & 1.0 & Pass & Pass & Pass & Pass & Pass & Pass & Pass & Pass\\
M & 0.5 & 70.0 & 10.0 & Pass & Pass & Pass & Pass & Pass & Pass & Pass & Pass\\
M & 0.5 & 70.0 & 40.0 & Pass & Pass & Pass & Pass & Pass & Pass & Pass & Pass\\
M & 0.5 & 70.0 & 160.0 & Pass & Pass & Pass & Pass & Pass & Pass & Pass & Pass\\
M & 0.5 & 90.0 & 1.0 & Pass & Pass & Pass & Pass & Pass & Pass & Pass & Pass\\
M & 0.5 & 90.0 & 10.0 & Pass & Pass & Pass & Pass & Pass & Pass & Pass & Pass\\
M & 0.5 & 90.0 & 40.0 & Pass & Pass & Pass & Pass & Pass & Pass & Pass & Pass\\
M & 0.5 & 90.0 & 160.0 & Pass & Pass & Pass & Pass & Pass & Pass & Pass & Pass\\
M & 0.93 & 10.0 & 1.0 & Pass & Fail & Pass & Pass & Fail & Pass & Pass & Fail\\
M & 0.93 & 10.0 & 10.0 & Pass & Fail & Pass & Pass & Fail & Pass & Pass & Fail\\
M & 0.93 & 10.0 & 40.0 & Pass & Fail & Pass & Pass & Fail & Pass & Pass & Fail\\
M & 0.93 & 10.0 & 160.0 & Pass & Fail & Pass & Pass & Fail & Pass & Pass & Fail\\
M & 0.93 & 30.0 & 1.0 & Pass & Fail & Pass & Pass & Fail & Pass & Pass & Fail\\
M & 0.93 & 30.0 & 10.0 & Pass & Fail & Pass & Pass & Fail & Pass & Pass & Fail\\
M & 0.93 & 30.0 & 40.0 & Pass & Fail & Pass & Pass & Fail & Pass & Pass & Fail\\
M & 0.93 & 30.0 & 160.0 & Pass & Fail & Pass & Pass & Fail & Pass & Pass & Fail\\
M & 0.93 & 50.0 & 1.0 & Pass & Fail & Pass & Pass & Fail & Pass & Pass & Fail\\
M & 0.93 & 50.0 & 10.0 & Pass & Fail & Pass & Pass & Fail & Pass & Pass & Fail\\
M & 0.93 & 50.0 & 40.0 & Pass & Fail & Pass & Pass & Fail & Pass & Pass & Fail\\
M & 0.93 & 50.0 & 160.0 & Pass & Fail & Pass & Pass & Fail & Pass & Pass & Fail\\
M & 0.93 & 70.0 & 1.0 & Pass & Fail & Pass & Pass & Fail & Pass & Pass & Fail\\
M & 0.93 & 70.0 & 10.0 & Pass & Fail & Pass & Pass & Fail & Pass & Pass & Fail\\
M & 0.93 & 70.0 & 40.0 & Pass & Fail & Pass & Pass & Fail & Pass & Pass & Fail\\
M & 0.93 & 70.0 & 160.0 & Pass & Fail & Pass & Pass & Fail & Pass & Pass & Fail\\
M & 0.93 & 90.0 & 1.0 & Pass & Fail & Pass & Pass & Fail & Pass & Pass & Fail\\
M & 0.93 & 90.0 & 10.0 & Pass & Fail & Pass & Pass & Fail & Pass & Pass & Fail\\
M & 0.93 & 90.0 & 40.0 & Pass & Fail & Pass & Pass & Fail & Pass & Pass & Fail\\
M & 0.93 & 90.0 & 160.0 & Pass & Fail & Pass & Pass & Fail & Pass & Pass & Fail\\
\enddata
\end{deluxetable*}
\begin{longrotatetable}
\startlongtable
\begin{deluxetable*}{cccc|cccc|c|cccc|c|c|cc|cccc}
\tabletypesize{\scriptsize}
\tablecaption{Pass/Fail Table, \hamr thermal models}
\label{tab:ThamrPF}
\tablehead{ \colhead{M/S}  &  %
\colhead{Spin}  &  %
\colhead{$i$}  &  %
\colhead{$\Rh$}  &  %
\colhead{$F_{86}$}  &  %
\colhead{$\lambda_{maj,86}$}  &  %
\colhead{$F_{2\mu{\rm m}}$}  &  %
\colhead{$L_X$}  &  %
\colhead{non-EHT}  &  %
\colhead{$\lambda_{230}$}  &  %
\colhead{Ring D}  &  %
\colhead{Ring W}  &  %
\colhead{Ring A}  &  %
\colhead{EHT}  &  %
\colhead{All}  &  %
\colhead{MI} & %
\colhead{4G$\lambda$} & %
\colhead{$\dot{M}/\dot{M}_{Edd}$}  &  %
\colhead{$L_{bol}/(\dot{M} c^{2})$}  &  %
\colhead{$P_{out}$(cgs)}  &  %
\colhead{$P_{out}/(\dot{M} c^2)$}}
\startdata
S & -0.94 & 10.0 & 1.0 & Fail & Fail & Pass & Pass & Fail & Fail & Pass & Fail & Pass & Fail & Fail & Fail & Pass &4.12$\times10^{-7}$ & 6.34$\times10^{-5}$ & 1.86$\times10^{36}$ & 8.74$\times10^{-4}$\\
S & -0.94 & 10.0 & 40.0 & Fail & Fail & Pass & Fail & Fail & Pass & Pass & Pass & Pass & Pass & Fail & Pass & Fail &2.77$\times10^{-5}$ & 2.18$\times10^{-5}$ & 1.25$\times10^{38}$ & 8.74$\times10^{-4}$\\
S & -0.94 & 10.0 & 160.0 & Fail & Fail & Pass & Fail & Fail & Pass & Pass & Pass & Pass & Pass & Fail & Fail & Fail &1.42$\times10^{-4}$ & 2.53 & 6.38$\times10^{38}$ & 8.74$\times10^{-4}$\\
S & -0.94 & 50.0 & 1.0 & Fail & Pass & Pass & Pass & Fail & Pass & Pass & Pass & Pass & Pass & Fail & Fail & Fail &3.62$\times10^{-7}$ & 5.66$\times10^{-5}$ & 1.63$\times10^{36}$ & 8.74$\times10^{-4}$\\
S & -0.94 & 50.0 & 40.0 & Fail & Fail & Pass & Fail & Fail & Pass & Fail & Pass & Pass & Fail & Fail & Fail & Fail &3.23$\times10^{-5}$ & 2.54$\times10^{-5}$ & 1.45$\times10^{38}$ & 8.74$\times10^{-4}$\\
S & -0.94 & 50.0 & 160.0 & Fail & Fail & Pass & Fail & Fail & Pass & Pass & Pass & Pass & Pass & Fail & Fail & Fail &2.14$\times10^{-4}$ & 9.45$\times10^{-2}$ & 9.65$\times10^{38}$ & 8.74$\times10^{-4}$\\
S & -0.94 & 90.0 & 1.0 & Pass & Fail & Pass & Pass & Fail & Pass & Pass & Pass & Pass & Pass & Fail & Fail & Fail &3.89$\times10^{-7}$ & 6.33$\times10^{-5}$ & 1.75$\times10^{36}$ & 8.74$\times10^{-4}$\\
S & -0.94 & 90.0 & 40.0 & Fail & Pass & Pass & Fail & Fail & Pass & Fail & Pass & Pass & Fail & Fail & Fail & Fail &3.98$\times10^{-5}$ & 3.05$\times10^{-5}$ & 1.80$\times10^{38}$ & 8.74$\times10^{-4}$\\
S & -0.94 & 90.0 & 160.0 & Pass & Pass & Pass & Fail & Fail & Pass & Pass & Pass & Pass & Pass & Fail & Fail & Fail &3.20$\times10^{-4}$ & 9.41$\times10^{-4}$ & 1.44$\times10^{39}$ & 8.74$\times10^{-4}$\\
S & -0.5 & 10.0 & 1.0 & Fail & Fail & Pass & Pass & Fail & Pass & Pass & Fail & Pass & Fail & Fail & Pass & Pass &1.84$\times10^{-7}$ & 2.06$\times10^{-4}$ & 2.84$\times10^{36}$ & 2.98$\times10^{-3}$\\
S & -0.5 & 10.0 & 40.0 & Fail & Fail & Pass & Pass & Fail & Pass & Pass & Pass & Pass & Pass & Fail & Fail & Pass &2.51$\times10^{-6}$ & 4.74$\times10^{-5}$ & 3.86$\times10^{37}$ & 2.98$\times10^{-3}$\\
S & -0.5 & 10.0 & 160.0 & Pass & Fail & Pass & Pass & Fail & Pass & Fail & Pass & Pass & Fail & Fail & Fail & Fail &4.77$\times10^{-6}$ & 2.10$\times10^{-5}$ & 7.35$\times10^{37}$ & 2.98$\times10^{-3}$\\
S & -0.5 & 50.0 & 1.0 & Fail & Fail & Pass & Pass & Fail & Pass & Pass & Pass & Pass & Pass & Fail & Fail & Pass &1.57$\times10^{-7}$ & 1.77$\times10^{-4}$ & 2.42$\times10^{36}$ & 2.98$\times10^{-3}$\\
S & -0.5 & 50.0 & 40.0 & Pass & Fail & Pass & Pass & Fail & Pass & Fail & Fail & Pass & Fail & Fail & Fail & Pass &2.15$\times10^{-6}$ & 4.10$\times10^{-5}$ & 3.32$\times10^{37}$ & 2.98$\times10^{-3}$\\
S & -0.5 & 50.0 & 160.0 & Fail & Fail & Pass & Pass & Fail & Pass & Fail & Fail & Pass & Fail & Fail & Fail & Fail &4.49$\times10^{-6}$ & 1.75$\times10^{-5}$ & 6.92$\times10^{37}$ & 2.98$\times10^{-3}$\\
S & -0.5 & 90.0 & 1.0 & Pass & Fail & Pass & Pass & Fail & Pass & Pass & Fail & Pass & Fail & Fail & Fail & Pass &1.47$\times10^{-7}$ & 1.66$\times10^{-4}$ & 2.26$\times10^{36}$ & 2.98$\times10^{-3}$\\
S & -0.5 & 90.0 & 40.0 & Pass & Fail & Pass & Pass & Fail & Pass & Pass & Pass & Pass & Pass & Fail & Pass & Fail &2.00$\times10^{-6}$ & 3.81$\times10^{-5}$ & 3.09$\times10^{37}$ & 2.98$\times10^{-3}$\\
S & -0.5 & 90.0 & 160.0 & Pass & Pass & Pass & Pass & Pass & Pass & Pass & Pass & Pass & Pass & Pass & Fail & Fail &4.40$\times10^{-6}$ & 1.69$\times10^{-5}$ & 6.78$\times10^{37}$ & 2.98$\times10^{-3}$\\
S & 0.0 & 10.0 & 1.0 & Fail & Fail & Pass & Pass & Fail & Pass & Pass & Fail & Pass & Fail & Fail & Fail & Pass &1.19$\times10^{-7}$ & 3.25$\times10^{-4}$ & 1.25$\times10^{36}$ & 2.03$\times10^{-3}$\\
S & 0.0 & 10.0 & 40.0 & Fail & Fail & Pass & Pass & Fail & Pass & Fail & Pass & Pass & Fail & Fail & Pass & Pass &1.56$\times10^{-6}$ & 3.80$\times10^{-5}$ & 1.64$\times10^{37}$ & 2.03$\times10^{-3}$\\
S & 0.0 & 10.0 & 160.0 & Pass & Fail & Pass & Pass & Fail & Pass & Fail & Pass & Pass & Fail & Fail & Pass & Pass &2.53$\times10^{-6}$ & 1.61$\times10^{-5}$ & 2.65$\times10^{37}$ & 2.03$\times10^{-3}$\\
S & 0.0 & 50.0 & 1.0 & Fail & Fail & Pass & Pass & Fail & Pass & Pass & Pass & Pass & Pass & Fail & Fail & Pass &1.06$\times10^{-7}$ & 2.90$\times10^{-4}$ & 1.11$\times10^{36}$ & 2.03$\times10^{-3}$\\
S & 0.0 & 50.0 & 40.0 & Fail & Fail & Pass & Pass & Fail & Pass & Fail & Pass & Pass & Fail & Fail & Pass & Pass &1.56$\times10^{-6}$ & 3.79$\times10^{-5}$ & 1.64$\times10^{37}$ & 2.03$\times10^{-3}$\\
S & 0.0 & 50.0 & 160.0 & Fail & Fail & Pass & Pass & Fail & Pass & Fail & Fail & Pass & Fail & Fail & Pass & Pass &2.60$\times10^{-6}$ & 1.67$\times10^{-5}$ & 2.73$\times10^{37}$ & 2.03$\times10^{-3}$\\
S & 0.0 & 90.0 & 1.0 & Pass & Fail & Pass & Pass & Fail & Pass & Pass & Fail & Pass & Fail & Fail & Fail & Pass &9.96$\times10^{-8}$ & 2.75$\times10^{-4}$ & 1.04$\times10^{36}$ & 2.03$\times10^{-3}$\\
S & 0.0 & 90.0 & 40.0 & Pass & Pass & Pass & Pass & Pass & Pass & Pass & Pass & Pass & Pass & Pass & Pass & Pass &1.69$\times10^{-6}$ & 4.07$\times10^{-5}$ & 1.77$\times10^{37}$ & 2.03$\times10^{-3}$\\
S & 0.0 & 90.0 & 160.0 & Pass & Pass & Pass & Pass & Pass & Pass & Pass & Fail & Pass & Fail & Fail & Pass & Pass &3.14$\times10^{-6}$ & 2.03$\times10^{-5}$ & 3.29$\times10^{37}$ & 2.03$\times10^{-3}$\\
S & 0.5 & 10.0 & 1.0 & Fail & Fail & Pass & Pass & Fail & Pass & Fail & Fail & Pass & Fail & Fail & Pass & Pass &7.02$\times10^{-8}$ & 7.42$\times10^{-4}$ & 1.34$\times10^{36}$ & 3.71$\times10^{-3}$\\
S & 0.5 & 10.0 & 40.0 & Fail & Fail & Pass & Pass & Fail & Pass & Pass & Pass & Pass & Pass & Fail & Pass & Pass &1.68$\times10^{-6}$ & 1.07$\times10^{-4}$ & 3.21$\times10^{37}$ & 3.71$\times10^{-3}$\\
S & 0.5 & 10.0 & 160.0 & Pass & Fail & Pass & Fail & Fail & Pass & Pass & Pass & Pass & Pass & Fail & Pass & Pass &3.43$\times10^{-6}$ & 3.16$\times10^{-5}$ & 6.56$\times10^{37}$ & 3.71$\times10^{-3}$\\
S & 0.5 & 50.0 & 1.0 & Fail & Fail & Pass & Pass & Fail & Pass & Pass & Pass & Pass & Pass & Fail & Pass & Pass &6.20$\times10^{-8}$ & 9.16$\times10^{-4}$ & 1.19$\times10^{36}$ & 3.71$\times10^{-3}$\\
S & 0.5 & 50.0 & 40.0 & Fail & Fail & Pass & Pass & Fail & Pass & Pass & Pass & Pass & Pass & Fail & Pass & Fail &1.79$\times10^{-6}$ & 1.14$\times10^{-4}$ & 3.42$\times10^{37}$ & 3.71$\times10^{-3}$\\
S & 0.5 & 50.0 & 160.0 & Fail & Pass & Pass & Fail & Fail & Pass & Pass & Pass & Pass & Pass & Fail & Pass & Pass &4.05$\times10^{-6}$ & 3.74$\times10^{-5}$ & 7.74$\times10^{37}$ & 3.71$\times10^{-3}$\\
S & 0.5 & 90.0 & 1.0 & Fail & Fail & Pass & Pass & Fail & Pass & Pass & Fail & Pass & Fail & Fail & Pass & Pass &5.84$\times10^{-8}$ & 6.19$\times10^{-4}$ & 1.12$\times10^{36}$ & 3.71$\times10^{-3}$\\
S & 0.5 & 90.0 & 40.0 & Fail & Fail & Pass & Pass & Fail & Pass & Pass & Pass & Pass & Pass & Fail & Fail & Fail &1.77$\times10^{-6}$ & 1.12$\times10^{-4}$ & 3.38$\times10^{37}$ & 3.71$\times10^{-3}$\\
S & 0.5 & 90.0 & 160.0 & Fail & Pass & Pass & Fail & Fail & Pass & Pass & Fail & Pass & Fail & Fail & Fail & Fail &3.94$\times10^{-6}$ & 3.60$\times10^{-5}$ & 7.54$\times10^{37}$ & 3.71$\times10^{-3}$\\
S & 0.94 & 10.0 & 1.0 & Fail & Fail & Pass & Pass & Fail & Pass & Pass & Pass & Pass & Pass & Fail & Fail & Pass &2.78$\times10^{-8}$ & 4.33$\times10^{-3}$ & 2.89$\times10^{36}$ & 2.01$\times10^{-2}$\\
S & 0.94 & 10.0 & 40.0 & Fail & Fail & Pass & Pass & Fail & Pass & Pass & Pass & Pass & Pass & Fail & Pass & Pass &7.53$\times10^{-7}$ & 6.41$\times10^{-4}$ & 7.80$\times10^{37}$ & 2.01$\times10^{-2}$\\
S & 0.94 & 10.0 & 160.0 & Fail & Fail & Pass & Fail & Fail & Pass & Pass & Pass & Pass & Pass & Fail & Pass & Pass &1.63$\times10^{-6}$ & 1.61$\times10^{-4}$ & 1.68$\times10^{38}$ & 2.01$\times10^{-2}$\\
S & 0.94 & 50.0 & 1.0 & Fail & Fail & Pass & Fail & Fail & Pass & Pass & Pass & Pass & Pass & Fail & Fail & Pass &2.47$\times10^{-8}$ & 3.66$\times10^{-3}$ & 2.56$\times10^{36}$ & 2.01$\times10^{-2}$\\
S & 0.94 & 50.0 & 40.0 & Fail & Fail & Fail & Fail & Fail & Pass & Pass & Pass & Pass & Pass & Fail & Pass & Pass &8.43$\times10^{-7}$ & 7.16$\times10^{-4}$ & 8.73$\times10^{37}$ & 2.01$\times10^{-2}$\\
S & 0.94 & 50.0 & 160.0 & Fail & Pass & Fail & Fail & Fail & Pass & Pass & Pass & Pass & Pass & Fail & Pass & Pass &1.93$\times10^{-6}$ & 1.90$\times10^{-4}$ & 2.00$\times10^{38}$ & 2.01$\times10^{-2}$\\
S & 0.94 & 90.0 & 1.0 & Fail & Fail & Fail & Fail & Fail & Pass & Pass & Fail & Pass & Fail & Fail & Fail & Pass &2.33$\times10^{-8}$ & 3.43$\times10^{-3}$ & 2.42$\times10^{36}$ & 2.01$\times10^{-2}$\\
S & 0.94 & 90.0 & 40.0 & Fail & Fail & Fail & Fail & Fail & Pass & Pass & Fail & Pass & Fail & Fail & Pass & Pass &8.20$\times10^{-7}$ & 6.97$\times10^{-4}$ & 8.50$\times10^{37}$ & 2.01$\times10^{-2}$\\
S & 0.94 & 90.0 & 160.0 & Fail & Pass & Fail & Fail & Fail & Pass & Pass & Fail & Pass & Fail & Fail & Pass & Fail &2.00$\times10^{-6}$ & 1.98$\times10^{-4}$ & 2.07$\times10^{38}$ & 2.01$\times10^{-2}$\\
M & -0.94 & 10.0 & 1.0 & Pass & Fail & Fail & Fail & Fail & Pass & Pass & Fail & Pass & Fail & Fail & Fail & Pass &4.76$\times10^{-8}$ & 1.00$\times10^{-2}$ & 4.77$\times10^{37}$ & 1.95$\times10^{-1}$\\
M & -0.94 & 10.0 & 40.0 & Pass & Fail & Pass & Pass & Fail & Pass & Pass & Fail & Pass & Fail & Fail & Fail & Fail &9.94$\times10^{-8}$ & 1.70$\times10^{-3}$ & 9.98$\times10^{37}$ & 1.95$\times10^{-1}$\\
M & -0.94 & 10.0 & 160.0 & Pass & Fail & Pass & Pass & Fail & Pass & Pass & Fail & Pass & Fail & Fail & Fail & Fail &1.70$\times10^{-7}$ & 6.64$\times10^{-4}$ & 1.70$\times10^{38}$ & 1.95$\times10^{-1}$\\
M & -0.94 & 50.0 & 1.0 & Pass & Pass & Fail & Fail & Fail & Pass & Fail & Fail & Pass & Fail & Fail & Fail & Pass &4.18$\times10^{-8}$ & 8.83$\times10^{-3}$ & 4.19$\times10^{37}$ & 1.95$\times10^{-1}$\\
M & -0.94 & 50.0 & 40.0 & Pass & Pass & Fail & Pass & Fail & Pass & Fail & Fail & Pass & Fail & Fail & Fail & Fail &8.72$\times10^{-8}$ & 1.50$\times10^{-3}$ & 8.75$\times10^{37}$ & 1.95$\times10^{-1}$\\
M & -0.94 & 50.0 & 160.0 & Pass & Pass & Pass & Pass & Pass & Pass & Pass & Fail & Pass & Fail & Fail & Fail & Fail &1.49$\times10^{-7}$ & 5.88$\times10^{-4}$ & 1.50$\times10^{38}$ & 1.95$\times10^{-1}$\\
M & -0.94 & 90.0 & 1.0 & Pass & Pass & Fail & Fail & Fail & Pass & Fail & Fail & Pass & Fail & Fail & Fail & Pass &3.90$\times10^{-8}$ & 8.21$\times10^{-3}$ & 3.92$\times10^{37}$ & 1.95$\times10^{-1}$\\
M & -0.94 & 90.0 & 40.0 & Pass & Pass & Fail & Pass & Fail & Pass & Pass & Fail & Pass & Fail & Fail & Fail & Fail &8.29$\times10^{-8}$ & 1.42$\times10^{-3}$ & 8.32$\times10^{37}$ & 1.95$\times10^{-1}$\\
M & -0.94 & 90.0 & 160.0 & Pass & Pass & Pass & Pass & Pass & Pass & Pass & Fail & Pass & Fail & Fail & Fail & Fail &1.44$\times10^{-7}$ & 5.66$\times10^{-4}$ & 1.44$\times10^{38}$ & 1.95$\times10^{-1}$\\
M & -0.5 & 10.0 & 1.0 & Pass & Fail & Fail & Fail & Fail & Pass & Fail & Fail & Pass & Fail & Fail & Fail & Fail &6.35$\times10^{-8}$ & 6.36$\times10^{-3}$ & 3.80$\times10^{37}$ & 1.16$\times10^{-1}$\\
M & -0.5 & 10.0 & 40.0 & Pass & Fail & Pass & Pass & Fail & Pass & Pass & Fail & Pass & Fail & Fail & Fail & Fail &1.46$\times10^{-7}$ & 1.48$\times10^{-3}$ & 8.74$\times10^{37}$ & 1.16$\times10^{-1}$\\
M & -0.5 & 10.0 & 160.0 & Pass & Fail & Pass & Pass & Fail & Pass & Fail & Fail & Pass & Fail & Fail & Fail & Fail &2.60$\times10^{-7}$ & 6.97$\times10^{-4}$ & 1.55$\times10^{38}$ & 1.16$\times10^{-1}$\\
M & -0.5 & 50.0 & 1.0 & Pass & Fail & Fail & Fail & Fail & Pass & Fail & Fail & Pass & Fail & Fail & Fail & Fail &5.64$\times10^{-8}$ & 5.57$\times10^{-3}$ & 3.37$\times10^{37}$ & 1.16$\times10^{-1}$\\
M & -0.5 & 50.0 & 40.0 & Pass & Fail & Pass & Pass & Fail & Pass & Pass & Fail & Pass & Fail & Fail & Fail & Fail &1.31$\times10^{-7}$ & 1.32$\times10^{-3}$ & 7.80$\times10^{37}$ & 1.16$\times10^{-1}$\\
M & -0.5 & 50.0 & 160.0 & Pass & Fail & Pass & Pass & Fail & Pass & Fail & Fail & Pass & Fail & Fail & Fail & Fail &2.32$\times10^{-7}$ & 6.24$\times10^{-4}$ & 1.39$\times10^{38}$ & 1.16$\times10^{-1}$\\
M & -0.5 & 90.0 & 1.0 & Pass & Fail & Fail & Fail & Fail & Pass & Fail & Fail & Pass & Fail & Fail & Fail & Fail &5.07$\times10^{-8}$ & 4.98$\times10^{-3}$ & 3.03$\times10^{37}$ & 1.16$\times10^{-1}$\\
M & -0.5 & 90.0 & 40.0 & Pass & Fail & Pass & Pass & Fail & Pass & Pass & Fail & Pass & Fail & Fail & Fail & Fail &1.20$\times10^{-7}$ & 1.21$\times10^{-3}$ & 7.16$\times10^{37}$ & 1.16$\times10^{-1}$\\
M & -0.5 & 90.0 & 160.0 & Pass & Fail & Pass & Pass & Fail & Pass & Fail & Fail & Pass & Fail & Fail & Fail & Fail &2.22$\times10^{-7}$ & 5.99$\times10^{-4}$ & 1.33$\times10^{38}$ & 1.16$\times10^{-1}$\\
M & 0.0 & 10.0 & 1.0 & Fail & Pass & Fail & Fail & Fail & Pass & Fail & Fail & Pass & Fail & Fail & Pass & Pass &3.74$\times10^{-8}$ & 4.82$\times10^{-3}$ & 1.41$\times10^{37}$ & 7.29$\times10^{-2}$\\
M & 0.0 & 10.0 & 40.0 & Pass & Fail & Pass & Pass & Fail & Pass & Pass & Fail & Pass & Fail & Fail & Fail & Fail &1.00$\times10^{-7}$ & 1.18$\times10^{-3}$ & 3.77$\times10^{37}$ & 7.29$\times10^{-2}$\\
M & 0.0 & 10.0 & 160.0 & Pass & Fail & Pass & Pass & Fail & Pass & Pass & Fail & Pass & Fail & Fail & Fail & Fail &1.74$\times10^{-7}$ & 5.55$\times10^{-4}$ & 6.54$\times10^{37}$ & 7.29$\times10^{-2}$\\
M & 0.0 & 50.0 & 1.0 & Fail & Fail & Fail & Fail & Fail & Pass & Fail & Pass & Pass & Fail & Fail & Fail & Pass &3.38$\times10^{-8}$ & 4.32$\times10^{-3}$ & 1.27$\times10^{37}$ & 7.29$\times10^{-2}$\\
M & 0.0 & 50.0 & 40.0 & Pass & Pass & Pass & Pass & Pass & Pass & Pass & Pass & Pass & Pass & Pass & Fail & Fail &8.52$\times10^{-8}$ & 1.01$\times10^{-3}$ & 3.21$\times10^{37}$ & 7.29$\times10^{-2}$\\
M & 0.0 & 50.0 & 160.0 & Pass & Pass & Pass & Pass & Pass & Pass & Pass & Fail & Pass & Fail & Fail & Fail & Fail &1.46$\times10^{-7}$ & 4.68$\times10^{-4}$ & 5.51$\times10^{37}$ & 7.29$\times10^{-2}$\\
M & 0.0 & 90.0 & 1.0 & Pass & Fail & Fail & Fail & Fail & Pass & Pass & Fail & Pass & Fail & Fail & Fail & Pass &2.96$\times10^{-8}$ & 3.76$\times10^{-3}$ & 1.11$\times10^{37}$ & 7.29$\times10^{-2}$\\
M & 0.0 & 90.0 & 40.0 & Pass & Pass & Pass & Pass & Pass & Pass & Fail & Fail & Pass & Fail & Fail & Fail & Fail &7.42$\times10^{-8}$ & 8.76$\times10^{-4}$ & 2.79$\times10^{37}$ & 7.29$\times10^{-2}$\\
M & 0.0 & 90.0 & 160.0 & Pass & Pass & Pass & Pass & Pass & Pass & Fail & Fail & Pass & Fail & Fail & Fail & Fail &1.29$\times10^{-7}$ & 4.15$\times10^{-4}$ & 4.84$\times10^{37}$ & 7.29$\times10^{-2}$\\
M & 0.5 & 10.0 & 1.0 & Pass & Fail & Pass & Fail & Fail & Pass & Pass & Pass & Pass & Pass & Fail & Pass & Pass &3.81$\times10^{-8}$ & 1.16$\times10^{-2}$ & 5.94$\times10^{37}$ & 3.02$\times10^{-1}$\\
M & 0.5 & 10.0 & 40.0 & Pass & Fail & Pass & Fail & Fail & Pass & Pass & Pass & Pass & Pass & Fail & Fail & Pass &1.22$\times10^{-7}$ & 2.73$\times10^{-3}$ & 1.91$\times10^{38}$ & 3.02$\times10^{-1}$\\
M & 0.5 & 10.0 & 160.0 & Pass & Fail & Pass & Pass & Fail & Pass & Pass & Pass & Pass & Pass & Fail & Fail & Fail &2.31$\times10^{-7}$ & 1.19$\times10^{-3}$ & 3.60$\times10^{38}$ & 3.02$\times10^{-1}$\\
M & 0.5 & 50.0 & 1.0 & Pass & Fail & Fail & Fail & Fail & Pass & Fail & Pass & Pass & Fail & Fail & Fail & Pass &3.48$\times10^{-8}$ & 1.06$\times10^{-2}$ & 5.42$\times10^{37}$ & 3.02$\times10^{-1}$\\
M & 0.5 & 50.0 & 40.0 & Pass & Fail & Pass & Fail & Fail & Pass & Pass & Pass & Pass & Pass & Fail & Fail & Fail &1.18$\times10^{-7}$ & 2.63$\times10^{-3}$ & 1.83$\times10^{38}$ & 3.02$\times10^{-1}$\\
M & 0.5 & 50.0 & 160.0 & Pass & Fail & Pass & Pass & Fail & Pass & Pass & Pass & Pass & Pass & Fail & Fail & Fail &2.19$\times10^{-7}$ & 1.12$\times10^{-3}$ & 3.41$\times10^{38}$ & 3.02$\times10^{-1}$\\
M & 0.5 & 90.0 & 1.0 & Pass & Fail & Fail & Fail & Fail & Pass & Pass & Fail & Pass & Fail & Fail & Fail & Fail &3.06$\times10^{-8}$ & 9.18$\times10^{-3}$ & 4.77$\times10^{37}$ & 3.02$\times10^{-1}$\\
M & 0.5 & 90.0 & 40.0 & Pass & Fail & Fail & Fail & Fail & Pass & Pass & Fail & Pass & Fail & Fail & Fail & Fail &1.06$\times10^{-7}$ & 2.39$\times10^{-3}$ & 1.66$\times10^{38}$ & 3.02$\times10^{-1}$\\
M & 0.5 & 90.0 & 160.0 & Pass & Fail & Pass & Pass & Fail & Pass & Fail & Fail & Pass & Fail & Fail & Fail & Fail &2.11$\times10^{-7}$ & 1.09$\times10^{-3}$ & 3.28$\times10^{38}$ & 3.02$\times10^{-1}$\\
M & 0.94 & 10.0 & 1.0 & Fail & Fail & Fail & Fail & Fail & Pass & Pass & Pass & Pass & Pass & Fail & Fail & Pass &1.84$\times10^{-8}$ & 2.68$\times10^{-2}$ & 1.14$\times10^{38}$ & 1.20\\
M & 0.94 & 10.0 & 40.0 & Pass & Fail & Pass & Pass & Fail & Pass & Pass & Pass & Pass & Pass & Fail & Fail & Pass &4.29$\times10^{-8}$ & 4.80$\times10^{-3}$ & 2.66$\times10^{38}$ & 1.20\\
M & 0.94 & 10.0 & 160.0 & Pass & Fail & Pass & Pass & Fail & Pass & Pass & Pass & Pass & Pass & Fail & Fail & Fail &7.62$\times10^{-8}$ & 1.75$\times10^{-3}$ & 4.72$\times10^{38}$ & 1.20\\
M & 0.94 & 50.0 & 1.0 & Fail & Fail & Fail & Fail & Fail & Pass & Pass & Pass & Pass & Pass & Fail & Fail & Pass &1.73$\times10^{-8}$ & 2.68$\times10^{-2}$ & 1.07$\times10^{38}$ & 1.20\\
M & 0.94 & 50.0 & 40.0 & Pass & Fail & Fail & Pass & Fail & Pass & Pass & Pass & Fail & Fail & Fail & Fail & Fail &4.05$\times10^{-8}$ & 4.70$\times10^{-3}$ & 2.51$\times10^{38}$ & 1.20\\
M & 0.94 & 50.0 & 160.0 & Pass & Fail & Pass & Pass & Fail & Pass & Pass & Pass & Pass & Pass & Fail & Fail & Fail &7.42$\times10^{-8}$ & 1.71$\times10^{-3}$ & 4.60$\times10^{38}$ & 1.20\\
M & 0.94 & 90.0 & 1.0 & Pass & Fail & Fail & Fail & Fail & Pass & Pass & Fail & Pass & Fail & Fail & Fail & Fail &1.55$\times10^{-8}$ & 2.64$\times10^{-2}$ & 9.63$\times10^{37}$ & 1.20\\
M & 0.94 & 90.0 & 40.0 & Pass & Fail & Fail & Fail & Fail & Pass & Pass & Fail & Pass & Fail & Fail & Fail & Fail &3.76$\times10^{-8}$ & 4.46$\times10^{-3}$ & 2.33$\times10^{38}$ & 1.20\\
M & 0.94 & 90.0 & 160.0 & Pass & Fail & Fail & Pass & Fail & Pass & Pass & Fail & Pass & Fail & Fail & Fail & Fail &7.22$\times10^{-8}$ & 1.67$\times10^{-3}$ & 4.47$\times10^{38}$ & 1.20\\
\enddata
\end{deluxetable*}
\end{longrotatetable}
\begin{longrotatetable}
\startlongtable
\begin{deluxetable*}{cccc|cccc|c|ccccc|c|c|cc}
\tabletypesize{\scriptsize}
\tablecaption{Pass/Fail Table}
\tablehead{ \colhead{M/S}  &  %
\colhead{Spin}  &  %
\colhead{$i$}  &  %
\colhead{$\Rh$}  &  %
\colhead{$F_{86}$}  &  %
\colhead{$\lambda_{maj,86}$}  &  %
\colhead{$F_{2\mu{\rm m}}$}  &  %
\colhead{non-EHT}  &  %
\colhead{$\lambda_{230}$}  &  %
\colhead{Nulls}  &  %
\colhead{Ring D}  &  %
\colhead{Ring W}  &  %
\colhead{Ring A}  &  %
\colhead{EHT}  &  %
\colhead{All}  &  %
\colhead{M$_3$} & %
\colhead{4G$\lambda$} & %
}
\startdata 
S & -0.94 & 10.0 & 1.0 & Fail & Fail & Pass & Fail & Fail & Pass & Pass & Fail & Pass & Fail & Fail & Pass & Pass\\
S & -0.94 & 10.0 & 40.0 & Fail & Fail & Pass & Fail & Fail & Pass & Pass & Fail & Pass & Fail & Fail & Pass & Pass\\
S & -0.94 & 10.0 & 160.0 & Fail & Fail & Pass & Fail & Fail & Pass & Pass & Fail & Pass & Fail & Fail & Pass & Fail\\
S & -0.94 & 50.0 & 1.0 & Fail & Fail & Pass & Fail & Pass & Pass & Pass & Fail & Pass & Fail & Fail & Fail & Pass\\
S & -0.94 & 50.0 & 40.0 & Fail & Fail & Pass & Fail & Fail & Pass & Pass & Fail & Pass & Fail & Fail & Pass & Pass\\
S & -0.94 & 50.0 & 160.0 & Fail & Fail & Pass & Fail & Fail & Pass & Pass & Fail & Pass & Fail & Fail & Pass & Pass\\
S & -0.94 & 90.0 & 1.0 & Fail & Fail & Pass & Fail & Pass & Pass & Pass & Fail & Pass & Fail & Fail & Fail & Pass\\
S & -0.94 & 90.0 & 40.0 & Fail & Fail & Pass & Fail & Pass & Pass & Fail & Fail & Pass & Fail & Fail & Pass & Pass\\
S & -0.94 & 90.0 & 160.0 & Fail & Fail & Pass & Fail & Pass & Pass & Pass & Fail & Pass & Fail & Fail & Pass & Pass\\
S & -0.5 & 10.0 & 1.0 & Fail & Pass & Pass & Fail & Pass & Pass & Pass & Fail & Pass & Fail & Fail & Pass & Pass\\
S & -0.5 & 10.0 & 40.0 & Pass & Pass & Pass & Pass & Pass & Pass & Pass & Fail & Pass & Fail & Fail & Pass & Pass\\
S & -0.5 & 10.0 & 160.0 & Pass & Pass & Pass & Pass & Pass & Pass & Pass & Fail & Pass & Fail & Fail & Fail & Fail\\
S & -0.5 & 50.0 & 1.0 & Fail & Fail & Pass & Fail & Pass & Pass & Pass & Fail & Pass & Fail & Fail & Pass & Pass\\
S & -0.5 & 50.0 & 40.0 & Pass & Pass & Pass & Pass & Pass & Pass & Pass & Fail & Pass & Fail & Fail & Fail & Pass\\
S & -0.5 & 50.0 & 160.0 & Pass & Pass & Pass & Pass & Pass & Pass & Pass & Fail & Pass & Fail & Fail & Pass & Pass\\
S & -0.5 & 90.0 & 1.0 & Pass & Fail & Pass & Fail & Pass & Pass & Pass & Fail & Pass & Fail & Fail & Fail & Pass\\
S & -0.5 & 90.0 & 40.0 & Pass & Pass & Pass & Pass & Pass & Pass & Pass & Fail & Pass & Fail & Fail & Fail & Pass\\
S & -0.5 & 90.0 & 160.0 & Pass & Pass & Pass & Pass & Pass & Pass & Pass & Fail & Pass & Fail & Fail & Pass & Fail\\
S & 0.0 & 10.0 & 1.0 & Fail & Pass & Pass & Fail & Pass & Pass & Fail & Fail & Pass & Fail & Fail & Pass & Pass\\
S & 0.0 & 10.0 & 40.0 & Pass & Fail & Pass & Fail & Pass & Pass & Pass & Fail & Pass & Fail & Fail & Pass & Pass\\
S & 0.0 & 10.0 & 160.0 & Pass & Fail & Pass & Fail & Pass & Pass & Pass & Fail & Pass & Fail & Fail & Pass & Pass\\
S & 0.0 & 50.0 & 1.0 & Fail & Fail & Pass & Fail & Pass & Pass & Pass & Fail & Pass & Fail & Fail & Fail & Pass\\
S & 0.0 & 50.0 & 40.0 & Pass & Pass & Pass & Pass & Pass & Pass & Pass & Fail & Pass & Fail & Fail & Pass & Pass\\
S & 0.0 & 50.0 & 160.0 & Pass & Pass & Pass & Pass & Pass & Pass & Pass & Fail & Pass & Fail & Fail & Pass & Pass\\
S & 0.0 & 90.0 & 1.0 & Pass & Fail & Pass & Fail & Pass & Pass & Pass & Fail & Pass & Fail & Fail & Pass & Pass\\
S & 0.0 & 90.0 & 40.0 & Pass & Fail & Pass & Fail & Pass & Pass & Pass & Fail & Pass & Fail & Fail & Pass & Pass\\
S & 0.0 & 90.0 & 160.0 & Fail & Fail & Pass & Fail & Pass & Pass & Pass & Fail & Pass & Fail & Fail & Pass & Pass\\
S & 0.5 & 10.0 & 1.0 & Fail & Fail & Pass & Fail & Pass & Pass & Pass & Fail & Pass & Fail & Fail & Pass & Pass\\
S & 0.5 & 10.0 & 40.0 & Pass & Pass & Pass & Pass & Pass & Pass & Pass & Fail & Pass & Fail & Fail & Fail & Pass\\
S & 0.5 & 10.0 & 160.0 & Pass & Pass & Pass & Pass & Pass & Pass & Pass & Fail & Pass & Fail & Fail & Pass & Pass\\
S & 0.5 & 50.0 & 1.0 & Fail & Fail & Pass & Fail & Pass & Pass & Pass & Fail & Pass & Fail & Fail & Pass & Pass\\
S & 0.5 & 50.0 & 40.0 & Pass & Pass & Pass & Pass & Pass & Pass & Pass & Fail & Pass & Fail & Fail & Pass & Pass\\
S & 0.5 & 50.0 & 160.0 & Pass & Fail & Pass & Fail & Pass & Pass & Fail & Fail & Pass & Fail & Fail & Pass & Pass\\
S & 0.5 & 90.0 & 1.0 & Fail & Fail & Pass & Fail & Pass & Pass & Pass & Fail & Pass & Fail & Fail & Pass & Pass\\
S & 0.5 & 90.0 & 40.0 & Pass & Pass & Pass & Pass & Pass & Pass & Pass & Fail & Pass & Fail & Fail & Pass & Fail\\
S & 0.5 & 90.0 & 160.0 & Pass & Fail & Pass & Fail & Pass & Pass & Pass & Fail & Pass & Fail & Fail & Pass & Pass\\
S & 0.94 & 10.0 & 1.0 & Fail & Fail & Pass & Fail & Pass & Pass & Pass & Fail & Pass & Fail & Fail & Fail & Pass\\
S & 0.94 & 10.0 & 40.0 & Pass & Fail & Pass & Fail & Pass & Pass & Pass & Fail & Pass & Fail & Fail & Pass & Pass\\
S & 0.94 & 10.0 & 160.0 & Pass & Pass & Pass & Pass & Pass & Pass & Pass & Fail & Pass & Fail & Fail & Pass & Pass\\
S & 0.94 & 50.0 & 1.0 & Fail & Fail & Pass & Fail & Pass & Pass & Pass & Pass & Pass & Pass & Fail & Fail & Pass\\
S & 0.94 & 50.0 & 40.0 & Pass & Pass & Pass & Pass & Pass & Pass & Pass & Pass & Pass & Pass & Pass & Pass & Pass\\
S & 0.94 & 50.0 & 160.0 & Pass & Fail & Pass & Fail & Pass & Pass & Pass & Fail & Pass & Fail & Fail & Pass & Pass\\
S & 0.94 & 90.0 & 1.0 & Fail & Fail & Pass & Fail & Pass & Pass & Pass & Pass & Pass & Pass & Fail & Fail & Pass\\
S & 0.94 & 90.0 & 40.0 & Pass & Pass & Pass & Pass & Pass & Pass & Pass & Fail & Pass & Fail & Fail & Pass & Pass\\
S & 0.94 & 90.0 & 160.0 & Pass & Fail & Pass & Fail & Pass & Pass & Pass & Fail & Pass & Fail & Fail & Pass & Pass\\
M & -0.94 & 10.0 & 1.0 & Pass & Pass & Fail & Fail & Pass & Pass & Pass & Fail & Pass & Fail & Fail & Fail & Pass\\
M & -0.94 & 10.0 & 40.0 & Pass & Fail & Pass & Fail & Pass & Pass & Pass & Fail & Pass & Fail & Fail & Fail & Fail\\
M & -0.94 & 10.0 & 160.0 & Pass & Fail & Pass & Fail & Pass & Pass & Pass & Pass & Pass & Pass & Fail & Fail & Fail\\
M & -0.94 & 50.0 & 1.0 & Pass & Pass & Fail & Fail & Pass & Pass & Fail & Fail & Pass & Fail & Fail & Fail & Pass\\
M & -0.94 & 50.0 & 40.0 & Pass & Pass & Pass & Pass & Pass & Pass & Pass & Fail & Pass & Fail & Fail & Fail & Fail\\
M & -0.94 & 50.0 & 160.0 & Pass & Pass & Pass & Pass & Pass & Pass & Pass & Fail & Pass & Fail & Fail & Fail & Fail\\
M & -0.94 & 90.0 & 1.0 & Pass & Pass & Fail & Fail & Pass & Pass & Pass & Fail & Pass & Fail & Fail & Fail & Pass\\
M & -0.94 & 90.0 & 40.0 & Pass & Pass & Pass & Pass & Pass & Pass & Pass & Fail & Pass & Fail & Fail & Fail & Fail\\
M & -0.94 & 90.0 & 160.0 & Pass & Pass & Pass & Pass & Pass & Pass & Pass & Fail & Pass & Fail & Fail & Fail & Fail\\
M & -0.5 & 10.0 & 1.0 & Fail & Pass & Pass & Fail & Pass & Pass & Fail & Fail & Pass & Fail & Fail & Fail & Pass\\
M & -0.5 & 10.0 & 40.0 & Pass & Fail & Pass & Fail & Pass & Pass & Pass & Fail & Pass & Fail & Fail & Fail & Fail\\
M & -0.5 & 10.0 & 160.0 & Pass & Fail & Pass & Fail & Pass & Pass & Pass & Fail & Pass & Fail & Fail & Fail & Fail\\
M & -0.5 & 50.0 & 1.0 & Pass & Pass & Pass & Pass & Pass & Pass & Pass & Fail & Pass & Fail & Fail & Fail & Pass\\
M & -0.5 & 50.0 & 40.0 & Pass & Pass & Pass & Pass & Pass & Pass & Fail & Fail & Pass & Fail & Fail & Fail & Fail\\
M & -0.5 & 50.0 & 160.0 & Pass & Pass & Pass & Pass & Pass & Pass & Pass & Fail & Pass & Fail & Fail & Fail & Fail\\
M & -0.5 & 90.0 & 1.0 & Pass & Fail & Pass & Fail & Pass & Pass & Fail & Fail & Pass & Fail & Fail & Fail & Fail\\
M & -0.5 & 90.0 & 40.0 & Pass & Pass & Pass & Pass & Pass & Pass & Pass & Fail & Pass & Fail & Fail & Fail & Fail\\
M & -0.5 & 90.0 & 160.0 & Pass & Pass & Pass & Pass & Pass & Pass & Pass & Pass & Pass & Pass & Pass & Fail & Fail\\
M & 0.0 & 10.0 & 1.0 & Fail & Pass & Pass & Fail & Pass & Pass & Pass & Fail & Pass & Fail & Fail & Pass & Pass\\
M & 0.0 & 10.0 & 40.0 & Pass & Fail & Pass & Fail & Pass & Pass & Pass & Fail & Pass & Fail & Fail & Fail & Pass\\
M & 0.0 & 10.0 & 160.0 & Pass & Fail & Pass & Fail & Pass & Pass & Pass & Fail & Pass & Fail & Fail & Fail & Fail\\
M & 0.0 & 50.0 & 1.0 & Fail & Fail & Pass & Fail & Pass & Pass & Pass & Pass & Pass & Pass & Fail & Fail & Pass\\
M & 0.0 & 50.0 & 40.0 & Pass & Pass & Pass & Pass & Pass & Pass & Pass & Fail & Pass & Fail & Fail & Fail & Fail\\
M & 0.0 & 50.0 & 160.0 & Pass & Pass & Pass & Pass & Pass & Pass & Pass & Fail & Pass & Fail & Fail & Fail & Fail\\
M & 0.0 & 90.0 & 1.0 & Pass & Fail & Pass & Fail & Pass & Pass & Pass & Pass & Pass & Pass & Fail & Fail & Pass\\
M & 0.0 & 90.0 & 40.0 & Pass & Pass & Pass & Pass & Pass & Pass & Pass & Fail & Pass & Fail & Fail & Fail & Fail\\
M & 0.0 & 90.0 & 160.0 & Pass & Pass & Pass & Pass & Pass & Pass & Pass & Fail & Pass & Fail & Fail & Fail & Fail\\
M & 0.5 & 10.0 & 1.0 & Pass & Fail & Pass & Fail & Pass & Pass & Pass & Fail & Pass & Fail & Fail & Pass & Pass\\
M & 0.5 & 10.0 & 40.0 & Pass & Fail & Pass & Fail & Pass & Pass & Pass & Fail & Pass & Fail & Fail & Fail & Pass\\
M & 0.5 & 10.0 & 160.0 & Pass & Pass & Pass & Pass & Pass & Pass & Pass & Fail & Pass & Fail & Fail & Fail & Fail\\
M & 0.5 & 50.0 & 1.0 & Pass & Pass & Pass & Pass & Pass & Pass & Pass & Pass & Pass & Pass & Pass & Pass & Pass\\
M & 0.5 & 50.0 & 40.0 & Pass & Pass & Pass & Pass & Pass & Pass & Pass & Fail & Pass & Fail & Fail & Fail & Fail\\
M & 0.5 & 50.0 & 160.0 & Pass & Pass & Pass & Pass & Pass & Pass & Pass & Fail & Pass & Fail & Fail & Fail & Fail\\
M & 0.5 & 90.0 & 1.0 & Pass & Pass & Pass & Pass & Pass & Pass & Pass & Pass & Pass & Pass & Pass & Fail & Pass\\
M & 0.5 & 90.0 & 40.0 & Pass & Pass & Pass & Pass & Pass & Pass & Pass & Fail & Pass & Fail & Fail & Fail & Fail\\
M & 0.5 & 90.0 & 160.0 & Pass & Pass & Pass & Pass & Pass & Pass & Fail & Fail & Pass & Fail & Fail & Fail & Fail\\
M & 0.94 & 10.0 & 1.0 & Fail & Fail & Fail & Fail & Pass & Pass & Fail & Fail & Pass & Fail & Fail & Fail & Pass\\
M & 0.94 & 10.0 & 40.0 & Pass & Fail & Pass & Fail & Pass & Pass & Pass & Fail & Pass & Fail & Fail & Fail & Pass\\
M & 0.94 & 10.0 & 160.0 & Pass & Fail & Pass & Fail & Pass & Pass & Pass & Fail & Pass & Fail & Fail & Fail & Fail\\
M & 0.94 & 50.0 & 1.0 & Fail & Fail & Fail & Fail & Pass & Pass & Fail & Fail & Pass & Fail & Fail & Fail & Pass\\
M & 0.94 & 50.0 & 40.0 & Pass & Fail & Pass & Fail & Pass & Pass & Pass & Pass & Pass & Pass & Fail & Fail & Fail\\
M & 0.94 & 50.0 & 160.0 & Pass & Fail & Pass & Fail & Pass & Pass & Pass & Pass & Fail & Fail & Fail & Fail & Fail\\
M & 0.94 & 90.0 & 1.0 & Pass & Fail & Fail & Fail & Pass & Pass & Pass & Pass & Fail & Fail & Fail & Fail & Fail\\
M & 0.94 & 90.0 & 40.0 & Pass & Fail & Pass & Fail & Pass & Pass & Pass & Fail & Pass & Fail & Fail & Fail & Fail\\
M & 0.94 & 90.0 & 160.0 & Pass & Fail & Pass & Fail & Pass & Pass & Pass & Fail & Pass & Fail & Fail & Fail & Fail\\
\enddata
\end{deluxetable*}
\end{longrotatetable}
\startlongtable
\begin{deluxetable*}{cccc|cccc|c|ccccc|c|c}
\tabletypesize{\scriptsize}
\tablecaption{Pass/Fail Table, Koral Thermal Models}
\label{koralPF}
\tablehead{ \colhead{M/S}  &  %
\colhead{Spin}  &  %
\colhead{$i$}  &  %
\colhead{$F_{86}$}  &  %
\colhead{$\lambda_{maj,86}$}  &  %
\colhead{$F_{2\mu{\rm m}}$}  &  %
\colhead{non-EHT}  &  %
\colhead{$\lambda_{230}$}  &  %
\colhead{Ring D}  &  %
\colhead{Ring W}  &  %
\colhead{Ring A}  &  %
\colhead{EHT}  &  %
\colhead{All}}
\startdata
M & -0.9 & 10.0 & Pass & Pass & Pass & Pass & Pass & Fail & Fail & Pass & Fail & Fail\\
M & -0.9 & 30.0 & Pass & Pass & Pass & Pass & Pass & Fail & Fail & Pass & Fail & Fail\\
M & -0.9 & 50.0 & Pass & Pass & Pass & Pass & Pass & Fail & Fail & Pass & Fail & Fail\\
M & -0.9 & 70.0 & Pass & Pass & Pass & Pass & Pass & Fail & Fail & Pass & Fail & Fail\\
M & -0.9 & 90.0 & Pass & Pass & Pass & Pass & Pass & Fail & Fail & Pass & Fail & Fail\\
M & -0.7 & 10.0 & Pass & Pass & Pass & Pass & Pass & Fail & Fail & Pass & Fail & Fail\\
M & -0.7 & 30.0 & Pass & Pass & Pass & Pass & Pass & Fail & Fail & Pass & Fail & Fail\\
M & -0.7 & 50.0 & Pass & Pass & Pass & Pass & Pass & Fail & Fail & Pass & Fail & Fail\\
M & -0.7 & 70.0 & Pass & Pass & Pass & Pass & Pass & Fail & Fail & Pass & Fail & Fail\\
M & -0.7 & 90.0 & Pass & Pass & Pass & Pass & Pass & Fail & Fail & Pass & Fail & Fail\\
M & -0.5 & 10.0 & Pass & Fail & Pass & Fail & Pass & Fail & Fail & Pass & Fail & Fail\\
M & -0.5 & 30.0 & Pass & Pass & Pass & Pass & Pass & Fail & Fail & Pass & Fail & Fail\\
M & -0.5 & 50.0 & Pass & Pass & Pass & Pass & Pass & Fail & Fail & Pass & Fail & Fail\\
M & -0.5 & 70.0 & Pass & Pass & Pass & Pass & Pass & Fail & Fail & Pass & Fail & Fail\\
M & -0.5 & 90.0 & Pass & Pass & Pass & Pass & Pass & Fail & Fail & Pass & Fail & Fail\\
M & -0.3 & 10.0 & Pass & Fail & Pass & Fail & Pass & Fail & Fail & Pass & Fail & Fail\\
M & -0.3 & 30.0 & Pass & Pass & Pass & Pass & Pass & Fail & Fail & Pass & Fail & Fail\\
M & -0.3 & 50.0 & Pass & Pass & Pass & Pass & Pass & Fail & Fail & Pass & Fail & Fail\\
M & -0.3 & 70.0 & Pass & Pass & Pass & Pass & Pass & Fail & Fail & Pass & Fail & Fail\\
M & -0.3 & 90.0 & Pass & Pass & Pass & Pass & Pass & Fail & Fail & Pass & Fail & Fail\\
M & 0.0 & 10.0 & Pass & Fail & Pass & Fail & Pass & Fail & Fail & Pass & Fail & Fail\\
M & 0.0 & 30.0 & Pass & Pass & Pass & Pass & Pass & Fail & Fail & Pass & Fail & Fail\\
M & 0.0 & 50.0 & Pass & Pass & Pass & Pass & Pass & Fail & Fail & Pass & Fail & Fail\\
M & 0.0 & 70.0 & Pass & Pass & Pass & Pass & Pass & Fail & Fail & Pass & Fail & Fail\\
M & 0.0 & 90.0 & Pass & Pass & Pass & Pass & Pass & Fail & Fail & Pass & Fail & Fail\\
M & 0.3 & 10.0 & Pass & Fail & Pass & Fail & Pass & Fail & Fail & Pass & Fail & Fail\\
M & 0.3 & 30.0 & Pass & Pass & Pass & Pass & Pass & Fail & Fail & Pass & Fail & Fail\\
M & 0.3 & 50.0 & Pass & Pass & Pass & Pass & Pass & Pass & Fail & Pass & Fail & Fail\\
M & 0.3 & 70.0 & Pass & Pass & Pass & Pass & Pass & Pass & Fail & Pass & Fail & Fail\\
M & 0.3 & 90.0 & Pass & Pass & Pass & Pass & Pass & Fail & Fail & Pass & Fail & Fail\\
M & 0.9 & 10.0 & Pass & Pass & Fail & Fail & Pass & Pass & Fail & Pass & Fail & Fail\\
M & 0.9 & 30.0 & Pass & Pass & Fail & Fail & Pass & Pass & Fail & Pass & Fail & Fail\\
M & 0.9 & 50.0 & Pass & Pass & Fail & Fail & Pass & Pass & Pass & Pass & Pass & Fail\\
M & 0.9 & 70.0 & Pass & Fail & Fail & Fail & Pass & Pass & Fail & Fail & Fail & Fail\\
M & 0.9 & 90.0 & Pass & Fail & Fail & Fail & Pass & Pass & Fail & Fail & Fail & Fail\\
\enddata
\end{deluxetable*}

\begin{deluxetable*}{cccc|cccc|c|ccccc|c|c}
\tabletypesize{\scriptsize}
\tablecaption{Pass/Fail Table}
\tablehead{ \colhead{$\beta$}  &  %
\colhead{$F_{86}$}  &  %
\colhead{$\lambda_{maj,86}$}  &  %
\colhead{non-EHT}  &  %
\colhead{$\lambda_{230}$}  &  %
\colhead{Ring D}  &  %
\colhead{Ring W}  &  %
\colhead{Ring A}  &  %
\colhead{EHT}  &  %
\colhead{All}}
\startdata
1E2 & Fail & Pass & Fail & Pass & Pass & Fail & Pass & Fail & Fail\\
1E6 & Fail & Pass & Fail & Pass & Fail & Fail & Pass & Fail & Fail\\
\enddata
\end{deluxetable*}
\begin{longrotatetable}
\startlongtable
\begin{deluxetable*}{cccc|ccc|c|cccc|c|c|cc}
\tabletypesize{\scriptsize}
\tablecaption{Pass/Fail Table, \hamr Tilted Models}
\label{tab:TiltedhamrPF}
\tablehead{ \colhead{Tilt}  &  %
\colhead{Spin}  &  %
\colhead{$i$}  &  %
\colhead{$\Rh$}  &  %
\colhead{$F_{86}$}  &  %
\colhead{$\lambda_{maj,86}$}  &  %
\colhead{$F_{2\mu{\rm m}}$}  &  %
\colhead{non-EHT}  &  %
\colhead{$\lambda_{230}$}  &  %
\colhead{Ring D}  &  %
\colhead{Ring W}  &  %
\colhead{Ring A}  &  %
\colhead{EHT}  &  %
\colhead{All}  &  %
\colhead{M$_3$} & %
\colhead{4G$\lambda$} %
}
\startdata
0 & 0.94 & 10.0 & 1.0 & Fail & Fail & Fail & Fail & Pass & Pass & Pass & Pass & Pass & Fail & Pass & Pass\\
0 & 0.94 & 10.0 & 40.0 & Pass & Fail & Fail & Fail & Pass & Pass & Pass & Pass & Pass & Fail & Pass & Pass\\
0 & 0.94 & 10.0 & 160.0 & Pass & Fail & Pass & Fail & Pass & Pass & Pass & Pass & Pass & Fail & Fail & Pass\\
0 & 0.94 & 50.0 & 1.0 & Fail & Fail & Fail & Fail & Pass & Pass & Pass & Pass & Pass & Fail & Pass & Pass\\
0 & 0.94 & 50.0 & 40.0 & Pass & Fail & Fail & Fail & Pass & Pass & Fail & Pass & Fail & Fail & Pass & Pass\\
0 & 0.94 & 50.0 & 160.0 & Fail & Fail & Fail & Fail & Pass & Pass & Fail & Pass & Fail & Fail & Fail & Pass\\
0 & 0.94 & 90.0 & 1.0 & Fail & Fail & Fail & Fail & Pass & Pass & Fail & Pass & Fail & Fail & Pass & Pass\\
0 & 0.94 & 90.0 & 40.0 & Fail & Fail & Fail & Fail & Pass & Pass & Fail & Pass & Fail & Fail & Pass & Pass\\
0 & 0.94 & 90.0 & 160.0 & Fail & Fail & Fail & Fail & Pass & Pass & Pass & Pass & Pass & Fail & Pass & Pass\\
30 & 0.94 & 10.0 & 1.0 & Fail & Fail & Fail & Fail & Pass & Fail & Fail & Pass & Fail & Fail & Pass & Pass\\
30 & 0.94 & 10.0 & 40.0 & Pass & Fail & Fail & Fail & Pass & Pass & Fail & Pass & Fail & Fail & Pass & Pass\\
30 & 0.94 & 10.0 & 160.0 & Pass & Fail & Pass & Fail & Pass & Pass & Pass & Pass & Pass & Fail & Pass & Pass\\
30 & 0.94 & 50.0 & 1.0 & Fail & Fail & Fail & Fail & Pass & Pass & Pass & Pass & Pass & Fail & Pass & Pass\\
30 & 0.94 & 50.0 & 40.0 & Pass & Fail & Fail & Fail & Pass & Pass & Fail & Pass & Fail & Fail & Pass & Pass\\
30 & 0.94 & 50.0 & 160.0 & Pass & Pass & Fail & Fail & Pass & Pass & Fail & Pass & Fail & Fail & Pass & Fail\\
30 & 0.94 & 90.0 & 1.0 & Fail & Fail & Fail & Fail & Pass & Pass & Pass & Pass & Pass & Fail & Pass & Pass\\
30 & 0.94 & 90.0 & 40.0 & Pass & Pass & Fail & Fail & Pass & Pass & Fail & Pass & Fail & Fail & Pass & Pass\\
30 & 0.94 & 90.0 & 160.0 & Pass & Pass & Pass & Pass & Pass & Pass & Fail & Pass & Fail & Fail & Fail & Fail\\
60 & 0.94 & 10.0 & 1.0 & Fail & Fail & Fail & Fail & Pass & Pass & Fail & Pass & Fail & Fail & Pass & Pass\\
60 & 0.94 & 10.0 & 40.0 & Pass & Fail & Fail & Fail & Pass & Pass & Fail & Pass & Fail & Fail & Pass & Pass\\
60 & 0.94 & 10.0 & 160.0 & Pass & Fail & Pass & Fail & Pass & Pass & Fail & Pass & Fail & Fail & Fail & Fail\\
60 & 0.94 & 50.0 & 1.0 & Fail & Fail & Fail & Fail & Pass & Pass & Pass & Pass & Pass & Fail & Pass & Pass\\
60 & 0.94 & 50.0 & 40.0 & Pass & Pass & Fail & Fail & Pass & Pass & Fail & Pass & Fail & Fail & Pass & Fail\\
60 & 0.94 & 50.0 & 160.0 & Pass & Pass & Fail & Fail & Pass & Pass & Fail & Pass & Fail & Fail & Fail & Fail\\
60 & 0.94 & 90.0 & 1.0 & Fail & Fail & Fail & Fail & Pass & Fail & Pass & Pass & Fail & Fail & Pass & Pass\\
60 & 0.94 & 90.0 & 40.0 & Pass & Pass & Fail & Fail & Pass & Pass & Fail & Pass & Fail & Fail & Fail & Fail\\
60 & 0.94 & 90.0 & 160.0 & Pass & Pass & Pass & Pass & Pass & Pass & Fail & Pass & Fail & Fail & Fail & Fail\\
\enddata
\end{deluxetable*}
\end{longrotatetable}

\clearpage


%==============================================================================
% Ensure that papers are listed in the references
\nocite{M87PaperI}
\nocite{M87PaperII}
\nocite{M87PaperIII}
\nocite{M87PaperIV}
\nocite{M87PaperV}
\nocite{M87PaperVI}
\nocite{M87PaperVII}
\nocite{M87PaperVIII}
\nocite{PaperI}
\nocite{PaperII}
\nocite{PaperIII}
\nocite{PaperIV}
\nocite{PaperV}
\nocite{PaperVI}

\bibliographystyle{yahapj}
\bibliography{main,refs,EHTCPapers}

%==============================================================================
\end{document}
