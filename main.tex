\documentclass[twocolumn,twocolappendix,tighten,dvipsnames,linenumbers]{aastex63}
\usepackage{eht}

\NewPageAfterKeywords

% Inline comments
\newcommand\note[1]{{\color{OliveGreen}[note: #1]}}
\newcommand\ckc[1]{{\color{MidnightBlue}[ckc: #1]}}
\newcommand\hyp[1]{{\color{Salmon}[HYP: #1]}}
\newcommand\monika[1]{{\color{Orange}[MM: #1]}}

%==============================================================================
\begin{document}

\title{First Sagittarius A* Event Horizon Telescope Results. V.\\
  Constraining the Astrophysical Properties of the Galactic Center Black Hole}
\shorttitle{First \sgra EHT Results on Astrophysics}

\author{Event Horizon Telescope Collaboration et al.}
\shortauthors{EHT Collaboration et al.}

\submitjournal{ApJL}

\received{\today}
\revised{\today}
%\accepted{\today}

\begin{abstract}
  We present theory interpretations of the accretion flow around the
  supermassive black hole Sagittarius A* (\sgra) using the Event
  Horizon Telescope (EHT) observation at 1.3\,mm collected during the
  2017 April 5--11 observing campaign.

  \color{BrickRed}

  [This is EHT's \sgra paper V.
  We are at the position to turn this from a sketchpad to an actual
  draft of the paper.
  The original draft is backed up in the \texttt{2018-08-01/}
  directory on Overleaf.

  For adding references, simply cite the ADS key, e.g.,
  \texttt{\textbackslash citep\{2019ApJ...875L...5E\}}.
  The bash script \texttt{tools/adsbib.sh} would automatically pull
  the full BibTeX database from ADS.
  See \texttt{Makefile} to learn how to use the scripts in
  \texttt{tools/}.]
\end{abstract}

\keywords{galaxies: individual: \sgra -- Galaxy: center -- black hole
  physics -- techniques: high angular resolution -- techniques: image
  processing -- techniques: interferometric}

\tableofcontents

\clearpage

%==============================================================================
\section{Introduction}

The scinetific goals and legacies of this paper may include:
\begin{itemize}
\item Describe a standard simulation library that is used in EHT's
  first \sgra papers.
  E.g., the ``GRMHD'' test set for imaging and calibration sets for
  MCFE and gravity.
\item Provide the first estimate on uncertainties in theoretical
  modelings by comparing models from different realizations/groups in
  similar parameter space.
\item Use the large number of models to extract theory-backed and
  statistical significant trends.
  Example includes edge-on images are more asymmetric.
\item Gather a standard observation data set that \emph{all future}
  theoretical studies should include.
\item Provide the first theoretical/numerical interpretation of the
  EHT \sgra VLBI observation, with helps from historical data.
  This should include ``scoring'' of the standard model sets, a few
  best bet models.
\item Summarize what models do not work and explain why.
\item Outlook into future theoretical research directions, i.e.,
  importance of variability and viscosity.
\end{itemize}

\clearpage

%==============================================================================
\section{Astrophysical Models}

\subsection{One-Zone Model and Estimate}

\subsection{Analytical Accretion Models}

\subsection{Numerical Models of the Inner Accretion Flows}

\subsection{Scattering Models}

\clearpage

%==============================================================================
\section{Observational Constraints}

\subsection{Standard Measurement and Assumptions}

\subsection{EHT Observations}
\subsubsection{230\,GHz Light Curve Variability}
\subsubsection{230\,GHz VLBI Null Locations}
\subsubsection{230\,GHz VLBI Pre-Image Size}
\subsubsection{230\,GHz VLBI Ring Size}
\subsubsection{230\,GHz VLBI Ring Thickness}
\subsubsection{230\,GHz VLBI Ring Asymmetry}

\subsection{Non-EHT Observations}
\subsubsection{86\,GHz Flux}
\subsubsection{86\,GHz Image Size}
\subsubsection{NIR (Non-Overproduction) Constraints}
\subsubsection{X-ray (Non-Overproduction) Constraints}

\clearpage

%==============================================================================
\section{Comparisons}

\subsection{EHT Constraints}

\subsection{Non-EHT Constraints}

\subsection{Discussions}

\subsection{Caveats and Limitations}

\clearpage

%==============================================================================
\section{Conclusions}

\subsection{MAD vs SANE}

\subsection{Jet vs Disk}

\subsection{Electron Temperature}

\subsection{Inclination}

\subsection{Black Hole Spin}

\clearpage

%==============================================================================
\facility{EHT} \software{\ehtim, \difmap, \smili, \dmc, \themis, \foci}

%==============================================================================
\appendix

\section{Pass/Fail Tables}

\clearpage

%==============================================================================
\section{Numerical Details}

\subsection{GRMHD Resolution and Convergence}

\subsection{Image Resolution and Field of View}

\subsection{Samping SED}

\subsection{Fast Light vs Light-Speed Light}

\clearpage

%==============================================================================
\bibliographystyle{yahapj}
\bibliography{main,EHTCPapers}

%==============================================================================
\end{document}
