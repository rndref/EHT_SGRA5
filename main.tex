\documentclass[twocolumn,tighten,dvipsnames]{aastex63}
%\documentclass[modern]{aastex63}
\usepackage{xspace}
\graphicspath{{./}{figures/}}

%==============================================================================
%% Text mode macros; should postfix with "\xspace"
\newcommand\sgra{Sgr~A*\xspace}

%% Math mode macros
\newcommand\msun{{M_\odot}}
\newcommand\yr{{\rm \, yr}}
\newcommand\erg{{\rm \, erg}}
\newcommand\cm{{\rm \, cm}}
\newcommand\pc{{\rm \, pc}}
\newcommand\gm{{\rm \, g}}
\newcommand\<{{\langle}}
\renewcommand\>{{\rangle}} %% "\>" is defined in the "tabbing" environment to advance to next tab stop

%% Inline comments
\newcommand\note[1]{{\color{OliveGreen}[note] #1}}
\newcommand\ckc[1]{{\color{MidnightBlue}[ckc] #1}}

%==============================================================================
\begin{document}

\title{\sgra Interpretation Notes}
\author{EHTC et al.}

\shorttitle{\sgra Interpretation}
\shortauthors{EHTC et al.}

%\received{June 1, 2019}
\revised{\today}
%\accepted{\today}
%\submitjournal{ApJ}

\begin{abstract}
  \note{This is a NOTES draft for the physical interpretation of EHTC 2017 campaign data for \sgra.}

  \note{Tentative title may be ``First \sgra Event Horizon Telescope Results. IV. Constraining the Physical Properties of the Galactic Center Black Hole''}

  \note{Please feel free to add your work and comment on existing notes, but do not delete anything without consulting the paper coordinators.}
\end{abstract}

\keywords{black hole}

%==============================================================================
\section{Introduction} \label{sec:intro}

\ckc{It may make sense to see the collection of EHT's first \sgra papers a book---and each paper is a chapter in the book.  Each chapter should be self-contained.  However, they shouldn't repeat each other too much.

It makes sense to leave the general introduction of \sgra to ``paper I''.  This paper should focus mainly on theoretical (and numerical) developments.}

\sgra is a well studied source.

\citep{
  1998ApJ...492..554N,
  1999ApJ...522..870M,
  2000ApJ...541..234O,
  2003ANS...324..445M,
  2006MNRAS.370..219M,
  2007MNRAS.379.1519M,
  2009A&A...508L..13M,
  2009ApJ...698..676D,
  2009ApJ...701..521C,
  2009ApJ...706..497M,
  2012MNRAS.421.1315Z,
  2013A&A...559L...3M,
  2014A&A...570A...7M,
  2014ApJ...790....1B,
  2015A&A...576A..41B,
  2015ApJ...799....1C,
  2015ApJ...802...69B,
  2015ApJ...812..103C,
  2015ApJ...814..115P,
  2015Sci...350.1242J,
  2016A&A...588A..57F,
  2016ApJ...817..173L,
  2016ApJ...818..121P,
  2016ApJ...824...40O,
  2016ApJ...826...77B,
  2016ApJ...831....4P,
  2016MNRAS.455.2187M,
  2016PhRvL.116c1101J,
  2017ApJ...837..180G,
  2017ApJ...844...35M,
  2017ApJ...851..148M,
  2017MNRAS.467.3604R,
  2018A&A...612A..34D,
  2018ApJ...856..163M,
  2018ApJ...859...60L,
  2018ApJ...863..148P,
  2018ApJ...865..104J,
  2018ApJ...868..101B,
  2018JCAP...07..015H,
  2018MNRAS.478.1875J,
  2018MNRAS.478.5209C,
  2019ApJ...871...30I,
  2019ApJ...881L...2B,
  2019ApJ...884..148B,
  2019ApJ...886...96H,
  2019GReGr..51..137P,
  2020ApJ...896L...6R,
  2020ApJ...897...99T,
  2020MNRAS.492.3272R,
  2020MNRAS.493.1404A,
  2020MNRAS.494.4168D,
  2020MNRAS.494.5923P,
  2020arXiv200514251B,
  2020arXiv200603657D,
  2020arXiv200603658P
}.

%==============================================================================
\section{Data}

%Abstract thinking: we are trying to connect theory and observation with a vector field $I$, $Q$, $U$, $V$ over 4 dimensions $t$, $\nu$, $\alpha$ (RA), and $\delta$ (dec).
%For (Fourier) time domain analysis, $t$ is transformed to $\omega$ (not to confused with photon frequency $\nu$).
%For visibility, the $\alpha$ and $\delta$ are transformed into $u$ and $v$.
%These limit what analyses we may do.

Summarize main characteristics that constrain the models.

%==============================================================================
\section{Estimates}

Assumed mass for \sgra is $M = 4.14 \times 10^6 \msun$, and distance $D = 8127\pc$ (based on GRAVITY results).

The implied characteristic length is $G M/c^2 = 6.1 \times 10^{11} \cm$, characteristic time is $G M/c^3 = 20.4\sec$, angular scale is $G M/(c^2 D) = 5.03\mu$as.  The radius of the shadow is $26.1\mu$as.

Then $\nu L_\nu = 4 \pi d^2 \nu F_\nu = 1.8 \times 10^{34} (d/8127 \pc)^2 (\nu/230 {\rm GHz})(F_\nu/{\rm Jy}) \erg \sec^{-1}$.

The Eddington luminosity for \sgra is $L_{Edd} = 4\pi G M c/\kappa_{es} = 5.2 \times 10^{44} \erg\sec^{-1}$. The Eddington accretion rate is $\dot{M}_{Edd} \equiv  L_{Edd}/(0.1 c^2) = 5.8 \times 10^{24} \gm \sec^{-1} = 0.09 \msun \yr^{-1}$.  The Eddington ratio is $L_{bol}/L_{Edd} = 1.9 \times 10^{-10} (L_{bol}/10^{35})$.

Next we consider a one zone model for \sgra.  The model consists of a single temperature sphere with magnetic field oriented at $\pi/3$ to the line of sight.  Assumed flux is $2$Jy, which is the average flux for Sgr A* measured by ALMA during the 2017 campaign.  The sphere radius is $5 G M/c^2$.  The ratio of ion to electron temperature $R \equiv T_i/T_e = 3$.  Assumed plasma $\beta \equiv  P_{gas}/(B^2/(8\pi)) = 1$, electron temperature $\Theta_e \equiv k T_e/(m_e c^2) = 10$.

The one zone model flux density is
\begin{equation}
    F_\nu = \int I_\nu \, d\Omega = \frac{2 k T_e \nu^2}{c^2 D^2} \int \, dx dy\,  (1 - \exp(-\tau_\nu(x,y))
\end{equation}
where $d\Omega$ is a differential solid angle, $x$ and $y$ are coordinates on the source plane in the same units as $D$, and $\tau_\nu(x,y) = 2 \alpha_\nu (G M/c^2) \sqrt{X_0^2 - X^2}$, where $X$ is the offset on the sky from the center of the sphere in units of $GM/c^2$, and $X_0 = 5 G M/c^2$.  Finally, $\alpha_\nu$ is the thermal synchrotron absorption coefficient (units: $\cm^{-1}$).  The density is adjusted until the flux density is $2$Jy.

For a pure hydrogen model, resulting parameters are: $n_e = 1.1 \times 10^6$, $B = 30$G, $\tau_I = 0.48$, $\tau_Q = 0.35 $, $\tau_V = 0.27$, Thomson depth $\tau_e = 5 G M/c^2 \sigma_T = 2.2 \times 10^{-6}$, Compton $y = \tau_e 16 \Theta_e^2 = 3.6 \times 10^{-3}$, synchrotron cooling time is $t_{cool} = 3.1 \times 10^4\sec = 1.5 \times 10^3 G M/c^3$.

There is reason to believe that, if Sgr A* is fed by stellar winds as in Ressler et al. 2019, the inflowing plasma is almost pure helium.  In this case the results change only slightly; the cooling time drops slightly because the rest-mass density (and therefore field strength) per electron increases.

%==============================================================================
\section{Models}

...

%------------------------------------------------------------------------------
\subsection{Phenomenological Models}

...

%------------------------------------------------------------------------------
\subsection{Analytical Models}

...

%------------------------------------------------------------------------------
\subsection{Semi-Analytical Models}

...

%------------------------------------------------------------------------------
\subsection{Numerical Models}

%Propose process:
%\begin{itemize}
%    \item Use one zone model to estimate parameters.
%    \item Use x-ray flux to set bound and rule out part of the parameter space, especially for amount of non-thermal electrons, density normalization/accretion rate.
 %   \item Use mm--cm flux to constraint electron temperature.
%    \item ...
%    \item Use image/ring size to rule score models.
%\end{itemize}

Only study quiescent state for now and leave flare/variability to different papers (time domain, MWL, etc)?

...

%==============================================================================
\section{EHT Derived Constraints}

Consider two measurements from Sgr A* imaging: the asymmetry of the image and the size of the image.

%------------------------------------------------------------------------------
\subsection{Image asymmetry constraint}

Current results from the imaging group suggest Sgr A* is a surprisingly symmetric ring.   What level of asymmetry is expected in the model library?

To answer this we must first define an asymmetry statistic.  To start, we defined
\begin{equation}
    A = \frac{1}{F_\nu} \int \, d\alpha d\beta \, | I_\nu - {\rm AXI}(I_\nu) |
\end{equation}
where AXI indicates an axisymmetrized version of the image.  We use a crude centering algorithm, and studied only the average image from each model.  This statistic measures how much of the flux is in nonaxisymmetric piece of the image.   The result from measuring $A$ for the average image of the Illinois segment of the Sgr A* model library is shown in Figure \ref{fig:asymm}.

\begin{figure}\label{fig:asymm}
    \plotone{asymm.pdf}
    \caption{Asymmetry statistic $A$ for models in the Sgr A* image library.  Lower $A$ is more symmetric.  All models are more symmetric when viewed nearly face on (or anti-face-on).  The $R_{high} = 1$ models exhibit distinct trends and the models will likely be eliminated by other considerations.  For $R_{high} \ge 10$ SANE edge-on models are most asymmetric at high positive spin, while the MAD edge-on models are most asymmetric at moderate negative spin.}
\end{figure}

Improvements in analysis: a better approach to this analysis might be to generate multiple realizations of synthetic data from models, reconstruct an image using one of the imaging pipelines, and measure an asymmetry from the image.  This would produce a distribution $f(A)$ for each model from which a likelihood could be generated.

This should be done for all models in the Sgr A* image library.

Alternative analysis: one could fit a slashed crescent model to the data and estimate an asymmetry parameter $h$.

%------------------------------------------------------------------------------
\subsection{Image size constraints}

M. Johnson (private communication) reports that the best-fit size of Sgr A* obtained by measuring the FWHM of a Gaussian fit along a direction 45 deg E of N is approximately $57 \pm 11 \mu$as.  This implies $\sigma = 24 \pm 4.7 \mu$as.  What is expected from the models?

As a first step, consider the following statistic measured from the image library.  For each image in each model, compute the trace of the second moment tensor, i.e.
\begin{equation}
    R^2 = \frac{1}{F_\nu} \int \, d\alpha d\beta \, I_\nu \,\, (\alpha^2 + \beta^2).
\end{equation}
This produces a distribution of characteristic isotropic size $R$ for each model.

We have done a preliminary analysis of $\langle R \rangle$ and $\sigma_R$ for the Illinois segment of the image library, 360 models in all.  The models have  $18.3 \le \langle R \rangle 64.6 \mu$as; the smallest is an edge-on SANE, $a = 0.94$ model while the largest is a nearly face-on SANE, $a = -0.94$ model.  Only 22 models have $\langle R \rangle < 30\mu$as.  All of these are SANE models, and most have positive spin.  This is in tension with a possible low  asymmetry constraint because this is precisely the set of models with largest asymmetry.

Improvement in analysis: measure the full second moment tensor for each image and produce a distribution of $R$ measured for a random chord drawn across the model.  Since the major and minor axes are likely quite different, at least for some models (Issaoun et al. 2019) this likely relaxes the constraint.  Since the observed size is rather small, this will favor either small models or larger models with a large axis ratio and a particular orientation.

Comment: model size may be sensitive to the presence of a nonthermal tail in the eDF.  The likely sense of the effect is that a nonthermal tail adds millimeter emission at larger radii, making the model larger.  Since most models are already too large, this may constrain the nonthermal tail.

%==============================================================================
\section{Non-EHT Constraints}

Sgr A* is one of the most well-characterized objects on the sky. Simultaneous and non-simultaneous observations at $230$GHz and  other frequencies have the potential to sharply constrain models.

%------------------------------------------------------------------------------
\subsection{ALMA light curve properties}

Because the ALMA light curve is tightly sampled with intervals  between measurements as little as a second and high signal to noise (typically $\sim 60$), it can provide tight new constraints on the high temporal frequency variability of Sgr A*.  This high frequency variability can also be computed from GRMHD models and compared to the data using various statistics.

An example light curve for 3598 is shown in Figure \ref{fig:LC3598}.

\begin{figure}
    \centering
    \plotone{LC_3598.pdf}
    \caption{ALMA lo band light curve for experiment 3598, with Gaussian interpolation and error bars.}
    \label{fig:LC3598}
\end{figure}

V. Ramakrishnan is carrying out an analysis of correlation times and power spectra in the ALMA light curves.

D. Lee and C. Gammie fit the covariance of the ALMA light curve to a Mat\'ern covariance function, which has the form
\begin{equation}
    C(\Delta t) = const. \, \times \, q^\nu K_\nu(q).
\end{equation}
Here $q \equiv \sqrt{2\nu} |\Delta t|/\tau$, $\tau$ is a characteristic timescale, $K_\nu$ is a modified Bessel function, and $\nu$ is a parameter that determines the high frequency falloff of the power spectrum, with $P_\omega \propto \omega^{-2}$ for $\nu = 1/2$.  In general $P_\omega \propto \omega^{-2\nu - 1}$. The special case $\nu = 1/2$ is equivalent to the damped random walk that is commonly used to model AGN light curves, and has been used to model Sgr A* and M87's submm light curve. A maximum likelihood fit to the ALMA data yields $\tau \sim 0.9$hr and $\nu \sim 1.5$.

It remains to be seen whether this provides a strong constraint on the models; a preliminary study of model light curves suggests that at least some models are also well fit by a Mat\'ern covariance with similar parameters.

%------------------------------------------------------------------------------
\subsection{ALMA polarization constraints}

A typical linear polarization fraction for ALMA is $P \sim 6\%$ during the 2017 campaign, with EVPA $\sim -60$deg.  A typical circular polarization fraction is $C \sim 1\%$.  Historical data may also be useful.

GRMHD models with fixed eDFs make definite predictions for the distribution of $P$ and $C$.  An example PDF for linear polarization fraction is shown in Figure \ref{fig:lpexamp}.

\begin{figure}
    \centering
    \plottwo{flin_distr.pdf}{fcirc_distr.pdf}
    \caption{Left: distribution of linear polarization fraction from a MAD model with $a = 0$, $i = 30$deg, $R_{high} = 10$. Right: distribution of circular polarization fraction for the same model.}
    \label{fig:lpexamp}
\end{figure}

For the Illinois segment of the Sgr A* GRMHD models $\< F\>$ varies widely, $\sigma_F/\<F\>$ is typically small or of order $1$, and therefore linear polarization has considerable power to discriminate between models.  The linear polarization fraction varies from $0.08 < \<F\> < 42 \%$.  The smallest $\<F\>$ is in a SANE model with $a = 0$, $i = 90$deg, $R_{high} = 10$; $\sigma_F$ for this model is $0.05\%$. The largest $\<F\>$ is in a SANE model with $a = 0.94$, $i = 90$deg, $R_{high} = 1$; $\sigma_F$ for this model is $5.2\%$.  All of the highest polarization models have $R_{high} = 1$.  Other groups' models should be checked for consistency.

The circular polarization in GRMHD models also varies widely, with  $0.09 < \<|C|\> < 3.8 \%$.  The smallest $\<|C|\>$ is in a MAD model with $a = -0.5$, $i = 90$, and $R_{high} = 1$.  The largest $\<|C|\>$ is in a SANE model with $a = 0.94$, $i = 50$, and $R_{high} = 10$.

Comment: this is a very limited set of models.  We may want to consider $R_{low} = 10$ models, as in the M87 polarization theory effort.   We do not know if the distribution of polarizations is converged, and we do not fully understand sensitivity to all the parameters: this analysis has many of the same issues as the M87 polarization analysis, except that we know that almost all the flux is compact.

%------------------------------------------------------------------------------
\subsection{NIR constraints}

Although we have relatively little simultaneous NIR data from the 2017 campaign, the NIR lightcurve is observationally well characterized, with NIR flares peaking at $\sim 30$mJy and quiescent flux of order a few mJy.  NIR emission is produced by synchrotron emission in all models that we are aware of.  The NIR/mm color of models is sensitive to the eDF, and in particular nonthermal tails on the eDF, which tend to increase the NIR flux (this point needs to be thoroughly tested).  Thermal eDF models that consistently produce flux densities above a few $mJy$ at K band can therefore be ruled out.

An example analysis based on preliminary NIR images from the Illinois segment of the simulations is presented in Figure \ref{fig:NIRmodels}.  Evidently $R_{high} = 1$ MAD models are strongly disfavored because they overproduce NIR, presumably because they contain more high energy electrons.  A few other models are ruled out, they tend to have large spin and/or be viewed edge on.

\begin{figure}
    \centering
    \plottwo{NIR_MAD.pdf}{NIR_SANE.pdf}
    \caption{Left: MAD models, Right: SANE models.  Models marked with a red dot consistently overproduce NIR emission, green dots indicate models that do not overproduce NIR.  The inclination is indicated by the position of the dot, with face-on models near twelve o'clock and edge-on models near three o'clock.}
    \label{fig:NIRmodels}
\end{figure}

%------------------------------------------------------------------------------
\subsection{X-ray constraints}

We are fortunate to have good X-ray coverage on 6 April 2017 from Chandra and NuSTAR.

Rehearse estimates of one-zone emission from thermal synchrotron, Compton scattering, and nonthermal synchrotron.

Model comparison requires computation of SEDs from the models, which are both expensive (they use Monte Carlo to capture Compton scattering) and uncertain (there is evidence that X-ray arises via direct synchrotron from a long tail on the eDF).  C.-K. Chan will compute spectra using {\tt grmonty} and possibly {\tt kmonty} on the OSG.  An example SED is shown in Figure \ref{fig:SEDexamp}.

\begin{figure}
    \centering
    \plotone{img_Ma+0.94_1900.spectrum_1.0e16_avg.pdf}
    \caption{Caption}
    \label{fig:SEDexamp}
\end{figure}

Nevertheless, it is possible to apply a not-to-exceed threshold constraint similar to that used in the NIR: if a thermal eDF model consistently produces too much X-ray then it can be ruled out.  Early work by Moscibrodzka et al. 2009 finds that X-ray emission provides a powerful constraint on constant temperature ratio models.

%------------------------------------------------------------------------------
\subsection{Low frequency SED}

Purely thermal $R_{high}$ eDF models do not match the low frequency SED.

Needed: nonthermal models, or thermal models with an extended isothermal component, that match the SED.

%------------------------------------------------------------------------------
\subsection{Low frequency image size}
...

%==============================================================================
\section{Joint analysis of constraints}
%------------------------------------------------------------------------------
\subsection{Unsupervised Learning}

Apply k-mean clustering (or so) to classify different phenomenological models that satisfy most constraints.

Constraints may be multi-messenger, i.e., morphology (asymmetry, compactness, image moments), spectrum (mm, IR, X-Ray), variability, etc. to ensure robustness of the best-bet models selected.

Score each category independently and study top sets from each category.  May learn something interesting, e.g., low spin MAD and model X, while show very different images, fit visibility data equally well.

%==============================================================================
\section{Caveats/Limitations}

%==============================================================================
\section{Conclusions}

%==============================================================================
\acknowledgments

Standard EHT acknowledgments.

\vspace{5mm}
\facilities{HST(STIS), Swift(XRT and UVOT), AAVSO, CTIO:1.3m,
CTIO:1.5m,CXO}

\software{astropy \citep{2013A&A...558A..33A},
          Cloudy \citep{2013RMxAA..49..137F},
          SExtractor \citep{1996A&AS..117..393B}
          }

%==============================================================================
\appendix
%------------------------------------------------------------------------------
\section{Details}

%------------------------------------------------------------------------------
\section{Pass-Fail Table }

%==============================================================================
%\bibliography{sample63}{}
\bibliographystyle{aasjournal}
\bibliography{main}

%==============================================================================
\end{document}
