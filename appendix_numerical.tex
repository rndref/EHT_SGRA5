\section{Numerical Methods}\label{app:numerical}

%==============================================================================
\subsection{Consistency of Radiative Transfer Simulations}\label{app:radtrans}

Two studies have been undertaken within the EHT Collaboration to evaluate the consistency of radiative transfer codes.

The first, \citet{2020ApJ...897..148G}, evaluated the consistency between general relativistic ray-traced radiative transfer (GRRT) codes when tracing geodesics and when integrating the unpolarized radiative transfer equation.  \citeauthor{2020ApJ...897..148G} compares \bhoss and \ipole, which are the two transfer codes used in this paper, and  also compares to {\tt grtrans}, {\tt raptor}, {\tt odyssey}, {\tt gray2} and {\tt raikou}.  Code consistency was found to be excellent, with sub-percent level variations between codes when run with standard numerical parameters, i.e. without accuracy parameters tuned for consistency.

The second, \citet{Prather_et_al_2022}, evaluates code performance when imaging GRMHD simulation output and when integrating the equations of polarized radiative transfer.  \citeauthor{Prather_et_al_2022} includes \ipole, {\tt grtrans}, {\tt odyssey}, and {\tt raptor}.  Code consistency was also found to be excellent.

Uncertainty in the radiative transfer calculation is therefore unlikely to contribute significantly to the model error budget.

%==============================================================================
\subsection{GRMHD Simulations Consistency and Convergence}\label{app:resolution_study}

As evident in Table~\ref{tab:GRMHDmodels} the thermal models have been calculated for an identical parameter space from two different codes, namely \kharma and \bhac for the GRMHD simulations and \ipole and \bhoss codes for the GRRT calculations.
This allows us to perform an in depth comparison between the different numerical methods used in this work in addition to the EHTC code comparison projects \citep{2019ApJS..243...26P,2020ApJ...897..148G}.

\begin{figure*}
  \centering
  \includegraphics[width=0.8\textwidth]{./figures/BHAC_iharm_correlationNew}
  \caption{Correlation between \bhac and \kharma models for 9 model constraints.
The horizontal axis is the constraint value from \bhac/\bhoss, and the vertical axis shows the constraint value from \kharma/\ipole.
Each point corresponds to a single model, with the width of the distribution shown by the error bars.
See text for details.}
  \label{fig:modelcorrelation}
\end{figure*}

In Figure \ref{fig:modelcorrelation} we show the correlation between the thermal \kharma and \bhac models for constraints where we have predictions from both models.
The top row shows from left to right the 230\,GHz flux density, \mi{3}, and the 230\GHz image size obtained from image moments.
Since we normalize the 230\GHz images to an average flux of 2.4\,Jy within a time window of 5000\,M (28.5 h for \sgra), the scatter around this value is small.
The deviation from an ideal correlation reflects the precision and number of GRMHD snapshots included during normalization procedure.

The correlation in $\mi{3}$ spreads over $\Delta \mi{3}=0.75$, which serves as a measure of intra-code (e.g., MAD vs. SANE accretion) and inter-code (\bhac vs. \kharma) differences.
Despite these differences the models show a strong correlation throughout the investigated models and parameter space.

We also find a strong correlation between models and codes for the image size computed from image moments, i.e. second moments analysis.

The middle row presents the correlation plots for the 86\GHz flux density (left), the 86\,GHz image size using second moments (middle), and the NIR flux (right).
The 86\GHz flux and 86\GHz image size exhibit a shift toward larger values for the \bhac models.
This difference can be explained by the larger field of view used for the \bhac models at 86\GHz during the radiative transfer calculations.
Thus, more extended structure and therefore a larger total flux is included in the \bhac models.
This affects mainly models with large inclinations $i\geq70^\degree$ and jet dominated emission models ($\Rh \geq 40$).

The NIR fluxes show a tight correlation over four orders of magnitude and systematically larger flux for the \bhac models for low NIR fluxes ($\log_{10}({\rm NIR}/{\rm Jy}) < -7$).
These fluxes are far below the NIR constraints of $\sim 1\,\mathrm{mJy}$, and therefore they do not affect the passing or failing of the models.
In the thermal models the NIR flux is generated from the tail of the electron distribution function and is thus very sensitive to the electron temperature.
Small differences in the distribution and value of the electron temperature between the two codes explain the observed de-correlation at very low NIR flux.

The correlation between models for the m-ring parameters is presented in the third row of Figure~\ref{fig:modelcorrelation}.
The correlation of the diameter of the m-ring is plotted in the left panel.
The spread covers nearly the same extent as the 230\GHz image size (top row, right panel) however the scatter in the correlation is larger.
The same is true for the width of the m-ring (middle panel in the last row of Figure~\ref{fig:modelcorrelation}).
Compared to the diameter and width of the m-ring, the asymmetry of the m-ring is less correlated (right panel).
Notice that horizontal and vertical limits in the asymmetries occur because the parameter hits the boundary of the allowed range, which is $0.5$.

The smaller correlation of the m-ring parameters as compared to the other parameters presented in Figure~\ref{fig:modelcorrelation} is a consequence of the noisy nature of the m-ring fits.
Still, the distributions are quite symmetric under reflection across the diagonal, so the models are at least not biased with respect to each other.
Notice also that these plots do not capture all the information that is contained in the distribution of m-ring parameters, just the central value.

We are somewhat surprised by the strength of the correlations seen in Figure~\ref{fig:modelcorrelation}.
The range of each constraint is substantially larger than the width of the correlation, so the variations between models are real, detectable, and reproducible with independent codes.
The question of the origin of the systematic offsets between models for some constraints (for example, in the NIR) is interesting but beyond the scope of this paper.
