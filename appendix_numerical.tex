\section{Numerical Methods}\label{app:numerical}

%==============================================================================
\subsection{GRMHD Consistency and Convergence}\label{app:resolution_study}

\note{Hector to provide first draft.}

\note{Brief general discussion of Porth et al. and Olivares et al.}

\note{Comparison of GRMHD output from Illinois/Frankfurt/HAMR}

\note{Comparison of GRMHD output from Illinois at multiple resolutions}

%==============================================================================
\subsection{Radiative Transfer Consistency and Convergence}
\label{app:radtrans}

\note{Ben to provide first draft.}

Brief discussion of Gold et al., Prather et al.

Field of View

Resolution

%==============================================================================
\subsection{Spectral Energy Distribution Consistency and Convergence}

\begin{figure*}
    \centering
    %\includegraphics{}
    \note{altex: plot GRRT flux vs SED flux at different wavelenghts.  Demostrates that we have good agreements for 230GHz.  Demostrate that NIR may require monty carlo calculations due to inverse Compton.  Demostrate some 86GHz images require larger FoV but they are ruled out anyway and would not affect the results.}
    \caption{Comparing GRRT flux from monte carlo calculations.  The three columns are 86GHz, 230GHz, and NIR, respectively.  GRRT is only used to spot check x-ray and does not have a corresponding scatter plot.}
    \label{fig:sed_vv}
\end{figure*}

%==============================================================================
\subsection{Cross-Validations}

\begin{figure*}
    \centering
    %\includegraphics{}
    \note{altex: scatter plot between, e.g., Illinois and Frankfurt models for the different measurable.}
    \caption{Comparing model predictions from different modeling pipelines.  ...}
    \label{fig:xv}
\end{figure*}
