\section{Astrophysical Models}
\label{sec:models}

%==============================================================================
\subsection{Basic Assumptions}
\label{sec:basic}

We assume the mass of and distance to \sgra are
\begin{align}
  \mbh &= 4.14  \times 10^6 \msun, \label{eq:mass} \\
  D    &= 8.127 \kpc,              \label{eq:dist}
\end{align}
which are approximately the mean of the values reported by \citet{2019Sci...365..664D} and \citet{2019A&A...625L..10G}, which differ from each other by about 4\%.
The distance is consistent with maser parallax measurements \citep{2019ApJ...885..131R}.

We also assume that \sgra is a black hole described by the Kerr metric.
The dimensionless spin, $\abh \equiv Jc/G\mbh^2$, is a free parameter with $-1 < \abh < 1$, where $J$, $G$, and $c$ are the black hole spin angular momentum, gravitational constant, and speed of light, respectively.
Following \citetalias{M87PaperV}, we use
$\abh > 0$ to indicate that the angular momentum of the accretion flow and black hole are parallel (the accretion flow is ``prograde'') and
$\abh < 0$ to indicate that the angular momentum of the accretion flow and black hole are antiparallel (``retrograde'')\footnote{For tilted disks the sign of $\abh$ is the sign of ${\bf J}\cdot{\bf L}$ where ${\bf J}$ is black hole spin angular momentum and ${\bf L}$ is accretion flow orbital angular momentum.}.

The implied characteristic length 
\begin{equation}
  r_\mathrm{g}         \equiv G\mbh/ c^2    \simeq 6.1\times10^{11}\cm,
\end{equation}
the characteristic time
\begin{equation}
  t_\mathrm{g}         \equiv G\mbh/ c^3    \simeq 20.4\sec,
\end{equation}
and the angular scale
\begin{equation}
  \vartheta_\mathrm{g} \equiv G\mbh/(c^2 D) \simeq 5.03\uas.
\end{equation}
The expected diameter of the black hole shadow is $2\sqrt{27} G\mbh/(c^2 D)$ for $\abh = 0$.
For $|\abh| > 0$ the shadow is noncircular and its size and shape depend on $\abh$ and inclination $i$ (the angle between the line of sight and the spin axis); its width can be as small as $9 G\mbh/(c^2 D)$ for $\abh = 1$ and $i = 90\degree$ \citep{1973blho.conf..215B}.

If the emitting plasma is ionized hydrogen then the Eddington luminosity 
\begin{align}
  L_\mathrm{Edd}
  \equiv 4\pi G\mbh c m_p/\sigma_\mathrm{T}
  = 5.2 \times 10^{44}\ergps,
\end{align}
where symbols have their usual meaning.
The corresponding Eddington accretion rate
\begin{align}
  \dot\mbh_\mathrm{Edd}
  \equiv L_\mathrm{Edd}/(0.1 c^2)
  &= 5.8 \times 10^{24} \gm \sec^{-1} \nonumber\\
  &= 0.09 \msun \yr^{-1},
\end{align}
where the nominal efficiency is 10\%.
The bolometric luminosity of \sgra is $L_\mathrm{bol} \sim 10^{35}\ergps$ in a quiescent, non-flaring state, so that
\begin{align}
 \frac{L_\mathrm{bol}}{L_\mathrm{Edd}} = 1.9 \times 10^{-10} \left(\frac{L_\mathrm{bol}}{10^{35} \ergps}\right),
\end{align}
an extremely small Eddington ratio.  
% CFG 2/5: commented out as out of place.
%In what follows, we assume that radiative cooling of plasma around the black hole can be neglected (which is partially justified later) and that the emergent radiation can be calculated in post-processing.

%==============================================================================
\subsection{One-Zone Model}

Here we motivate the more complicated models that follow using  a simple one-zone model, following \citetalias{M87PaperV}
and one-zone models developed in the literature over many decades \citep[e.g.][]{1996IAUS..169..169F}.

Consider a uniform sphere of plasma with radius $r = 5\rg$, comparable to the observed size of \sgra at $230\GHz$ (\citetalias{PaperIII}, \citetalias{PaperIV}), with uniform magnetic field oriented at $\pi/3$ to the line-of-sight.
In turbulent astrophysical plasmas, it is common for the gas pressure to be comparable to the magnetic pressure, so we set $n_i \kB T_i + n_e \kB T_e = B^2/(8\pi)$, where $T_i \equiv$ ion temperature, $T_e \equiv$ electron temperature, $\kB \equiv$ Boltzmann constant, and $B \equiv$ magnetic field strength.
The plasma is collisionless (checked below), and it is plausible that the ions are preferentially heated, so we assume $T_i = 3 T_e$.
If the ions are sub-virial by a factor of $3$, commonly seen in relativistic MHD simulations, i.e., $(3/2) k T_i \sim (1/3) (1/2) (G M m_p/r)$, then the ions are nonrelativistic and the electrons are relativistic, with $\Theta_e \equiv  \kB T_e / m_e c^2 \sim 10$.

Assuming a thermal distribution of electron energies (eDF) and therefore a thermal synchrotron emissivity $j_\nu$ \citep[e.g.,][]{2011ApJ...737...21L} and assuming optically thin emission, the flux density from a uniform sphere, $F_\nu = (4/3)\pi r^3 j_\nu D^{-2} 10^{23}\,\mathrm{Jy}$.
Requiring $F_\nu = 2.4\,\mathrm{Jy}$, the average measured by ALMA during the 2017 campaign \citep{Wielgus2022}, yields a nonlinear equation for the electron density $n_e$ with solution
\begin{align}
  n_e &\simeq 1.1\times10^6\cm^{-3},\\
  B   &\simeq 30\,\mathrm{Gauss}.
  \label{eq:onezone}
\end{align}
This is consistent with a similar one-zone model fit to archival \sgra millimeter spectrum \citep{2019ApJ...881L...2B}.
The synchrotron optical depth $\tau_\mathrm{sync} = r j_\nu/B_\nu \simeq 0.4$, where $B_\nu$ is the Planck function, so the optically thin approximation is marginal.

The one-zone model has electron scattering optical depth  $\tau_e = \sigma_T n_e r \simeq 2\times10^{-5}$ and thus the Compton parameter $y = 16 \Theta_e^2 \max(\tau_e,\tau_e^2) \simeq 0.03$ is small.
Synchrotron cooling therefore dominates Compton cooling.

The synchrotron cooling timescale for electrons $t_\mathrm{cool} \equiv u_e/\Lambda$ where $u_e = 3 n_e k T_e$ is the electron internal energy and $\Lambda \simeq 5.4 B^2 e^4 n_e \Theta_e^2 /(c^3 m_e^2)$ is the synchrotron cooling rate for a thermal population of electrons with $\Theta_e \gtrsim 1$ (see Appendix~A in \citealt{2011ApJ...735....9M}; finite optical depth reduces $\Lambda$).
Therefore $t_\mathrm{cool}=2.3 \times 10^4\sec \simeq 1.1 \times 10^3 \tg$, which is longer than the inflow timescale $r/v^r \sim r^{3/2}$.
This suggests that radiative cooling can be neglected in the plasma models \citep[more detailed calculations confirm this estimate][]{2012MNRAS.426.1928D,2020MNRAS.499.3178Y}.\footnote{If \sgra is fed by stellar winds then the inflowing plasma may be mainly helium \citep{2019MNRAS.482L.123R}; this changes the one-zone model slightly.
Helium accretion is discussed in \cite{Wong_2022}.}

The one-zone model solution implies that the mean free path to Coulomb scattering is large compared to $\rg$, i.e. the source plasma is collisionless. 
At $\Theta_e \sim 1$, for example, the electron-electron Coulomb scattering cross section is comparable to the Thomson cross section, and the mean free path is therefore $\sim \tau_e^{-1} \rg$.
The electron-ion Coulomb scattering timescale is also long, and the electrons and ions are therefore poorly coupled.
This is consistent with our assumption that the ions and electrons can have different temperatures  \citep{1976ApJ...204..187S,1977ApJ...214..840I, 1982Natur.295...17R} and motivates consideration of nonthermal (unrelaxed) electron distribution functions \citep[see][]{2000ApJ...541..234O, 2009ApJ...701..521C, 2014A&A...570A...7M, 2018A&A...612A..34D, 2021arXiv211102518F, 2021NatAs.tmp..218C, Chatterjee2021, 2021arXiv211203933E, Scepi2021}.

%==============================================================================
\subsection{Numerical Models}

\begin{deluxetable*}{cccccccccc}
  \label{tab:GRMHDmodels}
  \tablecaption{EHT GRMHD Simulation Library}
  \tablehead{
    \colhead{Setup}                &
    \colhead{Code}                 &
    \colhead{$\abh$}               &
    \colhead{Mode}                 &
    \colhead{$\Gamma_\mathrm{ad}$} &
    \colhead{$t_\mathrm{final}$ }  &
    \colhead{$r_{\rm in}$}         &
    \colhead{$r_{\rm max}$}        &
    \colhead{$r_{\rm out}$}        &
    \colhead{Resolution}%
  }
  \startdata
  torus    & {\kharma}$^a$   & 0, $\pm 0.5$, $\pm 0.94$   & MAD/SANE & $\frac{4}{3}/\frac{4}{3}$  & \no{30000}  & --- & --- & \no{1000}     & $288\times128\times128$     \\
  torus    & {\bhac}$^b$     & 0, $\pm 0.5$, $\pm 0.94$   & MAD/SANE & $\frac{4}{3}/\frac{4}{3}$  & \no{30000}  & --- & --- & \no{3333}     & $512\times192\times192$     \\
  torus    & {\hamr}$^c$     & 0, $\pm 0.5$, $\pm 0.94$   & MAD/SANE & $\frac{13}{9}/\frac{5}{3}$ & \no{35000}  & --- & --- & \no{1000}/200 & $348/240\times192\times192$ \\
  torus    & {\koral}$^d$    & \!\!\!\!\!\!0, $\pm 0.3$,  %
  $\pm 0.5$, $\pm 0.7$, $\pm 0.9$\!\!\!\!\!\!\!\!\!\!\!\! & MAD      & $\frac{13}{9}$             & \no{101000} & --- & --- & \no{100000}  & $288\times192\times144$     \\
  tilted   & {\hamr}$^e$     & $0.94$                     & IN-SANE  & $\frac{5}{3}$              & \no{105000} & --- & --- & \no{100000}  & $448\times144\times240$     \\
  wind-fed & {\athenapp}$^f$ & 0                          & ILAF     & $\frac{5}{3}$              & \no{20000}  & --- & --- & \no{2400}    & $356\times128\times128$
  \enddata
  \tablecomments{Summary of the EHT \sgra GRMHD simulation library.
    The last column is $N_1 \times N_2 \times N_3$, with coordinate
    $x_1$ monotonic in radius, $x_2$ monotonic in colatitude $\theta$,
    and $x_3$ proportional to longitude $\phi$.
    The first four entries use aligned torus initial conditions.
    The last two entries are tilted accretion models and two
    realizations of the wind-fed accretion model which differ in
    stellar wind magnetization.
    Times are given in units of $G M/c^3 = 20.4\sec$ and radii in units
    of $G M/c^2$.%
  }
  \tablerefs{%
    $^a$see \citet{2021JOSS....6.3336P}; \kharma is a GPU-enabled version of the {\tt iharm3d} code.
    $^b$\citet{2017ComAC...4....1P, Olivares2019, 2021MNRAS.506..741M, 2021NatAs.tmp..218C}.
    $^c$\citet{2021arXiv210812380N}.
    $^e$\citet{Liska2019, Chatterjee2020}.
    $^f$\citet{2016ApJS..225...22W, 2020ApJ...896L...6R}.%
  }
\end{deluxetable*}

The one-zone model is too simple for comparison with EHT data.
Steady spherical accretion models \citep[e.g.,][]{2019ApJ...885L..33N, 2021arXiv211102178B} go one  step beyond the one zone model, incorporating relativistic gravity.
Steady, disk-like (RIAF) accretion models in the Kerr metric go still further and include rotation and departures from spherical symmetry \citep[e.g.,][]{2009ApJ...697...45B, 2009ApJ...706..960H,2018ApJ...863..148P}.
Minor edit to retain that meaning.
Steady phenomenological models do not, however, self-consistently capture fluctuations in the flow.
That requires either a statistical model \citep{2021ApJ...906...39L} or a time-dependent numerical simulation.
Here we
use numerical simulations,
adopt an ideal GRMHD model for the flow,
employ simple parameterized models to assign an electron distribution function, and
solve the radiative transfer equation along geodesics to produce simulated images.

%------------------------------------------------------------------------------
\subsubsection{Plasma Flow Model}

\begin{figure*}
  \centering
  \includegraphics[width=0.425\textwidth]{figures/sane_3D_corrected.png}\hspace{1.5pt}%
  \includegraphics[width=0.425\textwidth]{figures/mad_3D_corrected.png}\\
  \includegraphics[width=0.425\textwidth]{figures/tilted_3D_corrected.png}\hspace{1.5pt}%
  \includegraphics[width=0.425\textwidth]{figures/ressler_3D_corrected.png}
  \caption{3-D overview of selected GRMHD simulations of \sgra in our library.
    The color marks constant dimensionless density surfaces and lines follow magnetic field lines.  The magnetic field lines shown are only those which are attached to the inner part of the accretion flow, at $r\approx 5~\rg$.
    Two top panels show accretion simulations with default torus initial condition and
    two bottom panels show non-standard accretion models.
    The spin is aligned with z-axis.}
  \label{fig:GRMHD}
\end{figure*}

\begin{figure*}
  \centering
  \includegraphics[width=0.9\textwidth]{figures/grmhd_temp.pdf}
  \caption{Time- and azimuth-averaged profiles of midplane dimensionless gas temperature $P/(\rho c^2)$ in \kharma fiducial GRMHD simulations.
    Evidently MAD models are hotter than SANE, and both MADs and SANEs grow hotter as the black hole spin $\abh$ increases.
    The hottest models are $\abh = 0.94$ MAD models.}
  \label{fig:grmhd_temp}
\end{figure*}

We model the plasma flow around \sgra using ideal, non-radiative GRMHD in the Kerr metric, 
%.
%We assume the gravitational field
%is described by the Kerr metric, with mass from Equation~(\ref{eq:mass}) and 
with $\abh$ a free parameter \citep[see e.g.,][]{1999ApJ...522..727K,2003ApJ...589..444G, 2003ApJ...589..458D, 2005ApJ...635..723A, 2007A&A...473...11D}.

We integrate the GRMHD equations in three spatial dimensions using multiple algorithms:
\kharma   \citep{2021JOSS....6.3336P},
\bhac     \citep{2017ComAC...4....1P},
\hamr     \citep{Liska2019},
\koral    \citep{2013MNRAS.429.3533S}, and
\athenapp \citep{2016ApJS..225...22W};
see \citet{2019ApJS..243...26P} and \citet{Olivares_et_al} for comparisons of GRMHD codes.
All simulations assume constant adiabatic index  $\Gamma_\mathrm{ad}$.

Unless stated otherwise the initial conditions for the GRMHD simulations are constant-angular-momentum hydrodynamic equilibrium tori \citep{1976ApJ...207..962F}, with orbital angular momentum that is parallel or antiparallel to the black hole spin.
The tori are seeded with a weak, poloidal magnetic field.  The simulations use varying torus pressure maximum radius 
(from $\sim 15\,\rg$ to $40\,\rg$), peak temperature, adiabatic index, and initial field configurations.  These variations permit us to test the robustness of our results (see Appendix \ref{app:numerical}).  
%Despite these variations the radiative models based on these simulations are in broad agreement, demonstrating the robustness of our results.

The torus initial conditions are motivated by the notion that the near-horizon flow, where most of the emission is generated (\citetalias{M87PaperV}), relaxes to a statistically steady state that is nearly independent of the flow at larger radius.
This notion is challenged in the stellar wind-fed models of \cite{2020ApJ...896L...6R}, which are included in our study.

All simulations are run in Kerr-Schild-like coordinates, which are regular on the horizon.
%Specifically, most of the simulations use variations of the spherical polar form of Kerr-Schild coordinates to respect the symmetry of the spacetime.
Unless stated otherwise, boundary conditions are outflow at the inner boundary, which is located inside the event horizon, and outflow at the outer boundary, which is located at $r_{\rm max} \ge \no{1000}\,\rg$.
Most simulations are evolved to $t_\mathrm{final} \ge  \no{30000}\,\tg$.

Once the evolution has started, a combination of instabilities including the magnetorotational instability \citep[MRI,][]{1992ApJ...400..610B} drives the torus to a turbulent, fluctuating state.
Defining $P_\mathrm{gas} \equiv$ gas pressure and $P_\mathrm{mag} \equiv B^2 / (8\pi) \equiv$ magnetic pressure, the standard accretion flow models can be divided by latitude into three zones:
\emph{i})~an equatorial inflow,
\emph{ii})~a mid-latitude disk wind/corona with  $\beta  \equiv P_\mathrm{gas} / P_\mathrm{mag} \sim 1$, and
\emph{iii})~a polar ``funnel'' with $\sigma \equiv B^2/(4\pi \rho c^2) \gg 1$.

The magnetic flux through the horizon, characterized by $\phi \equiv \Phi_{\mathrm{BH}} (\dot{M} r_\mathrm{g}^2 c)^{-1/2}$ ($\Phi_{\rm BH} \equiv$ magnetic flux interior to the black hole equator, $\dot{M} \equiv$ mass accretion rate) divides the outcome into two states:
% cfg 2/5: these references seemed extraneous and occupy a lot of space
%(see, e.g, \citealt{M87PaperV} and \citealt{M87PaperVIII}, hereafter \citetalias{M87PaperVIII}, and references therein):
the magnetically arrested disk (MAD) state \citep[e.g.,][]{1974Ap&SS..28...45B, 2003ApJ...592.1042I, 2003PASJ...55L..69N, 2011MNRAS.418L..79T}, 
%in which the magnetic flux on the horizon saturates 
%and substantially affects the dynamics of the flow
and the Standard and Normal Evolution (SANE) state \citep[e.g.,][]{2003ApJ...589..444G, 2003ApJ...599.1238D, 2012MNRAS.426.3241N}.
%The dimensionless magnetic flux $\phi \equiv \Phi_{\mathrm{BH}} (\dot{M} r_\mathrm{g}^2 c)^{-1/2}$, where $\Phi_{\rm BH}$ is the magnetic flux interior to the black hole equator and $\dot{M}$ is the mass accretion rate through the horizon.
MAD models have $\phi \sim \phi_{\rm crit} \sim 60$.\footnote{In the Lorentz-Heaviside units commonly used in GRMHD simulations $\phi_\mathrm{crit}$ is smaller by a factor of $(4\pi)^{1/2} \simeq 3.545$.}
In MAD models, magnetic flux accretes onto the hole until $\phi \gtrsim \phi_\mathrm{crit}$.  Accretion of additional flux leads to flux expulsion events so that the flow maintains $\phi \sim \phi_\mathrm{crit}$.  Our SANE models, in contrast, typically have $\phi \sim 1$.

We consider two GRMHD simulations with initial conditions that differ from the fiducial aligned torus: strongly magnetized non-MAD tilted torus simulations \citep{Liska2018, Chatterjee2020} and a simulation in which \sgra is fed directly by winds from stars in its vicinity \citep{2020ApJ...896L...6R}.
The wind-fed simulations result in a mode of accretion that is similar to MAD but typically has lower mean angular momentum and is less well organized.
The wind-fed models have $\abh = 0$.

The GRMHD simulation library is summarized in Table~\ref{tab:GRMHDmodels}.
Figure~\ref{fig:GRMHD} shows a few examples of GRMHD simulations for an aligned SANE, an aligned MAD, a tilted torus, and a wind-fed simulation.
These simulations vary in numerical method and in numerical resolution.
We present more information on numerical methods in Appendices~\ref{app:numerical} and \ref{app:variability}.

The gas temperature profile is a critical feature of the GRMHD simulations.
Figure \ref{fig:grmhd_temp} shows the time- and azimuth-averaged profiles of the midplane dimensionless gas temperature $P/(\rho c^2)$ in a set of aligned GRMHD simulations.
The temperature profiles exhibit trends with spin and magnetic state (MAD or SANE) that drive many of the trends seen in the radiative models: MAD models are factor of several hotter than SANE models and both MAD and SANE become hotter as $\abh$ increases.

%------------------------------------------------------------------------------
\subsubsection{Radiative Transfer Model}

Synthetic images are generated from the GRMHD simulations in a radiative transfer step.  The transfer step requires
\emph{i})~a model for the electron distribution function (hereafter eDF);
\emph{ii})~assignment of a density scale to the GRMHD simulation;
\emph{iii})~the inclination $i$ (angle between the torus angular momentum and the line of sight)
\emph{iv})~a numerical integration performed as a post-processing step that assumes that the plasma evolution is unaffected by radiation.

% cfg 2/5: this was reduandant 
%The transfer step requires that we specify parameters associated with the eDF (e.g. $\Rh$), the inclination $i$ (angle between the torus angular momentum and the line of sight), and an accretion rate or equivalently a density scale, which is not determined by the GRMHD simulation.

%..............................................................................
\subsubsubsection{Electron Distribution Function}
\label{sec:eDF}

\emph{Thermal} models have electron energies distributed according to the Maxwell-J{\"u}ttner distribution function:
\begin{align}\label{eq:thermaleDF}
  \frac{1}{n_e}\frac{dn_e}{d\gamma} = \frac{\gamma^2 \sqrt{1-1/\gamma^2}} {\Theta_e K_2(1/\Theta_e)} \exp\left(-\frac{\gamma}{\Theta_e}\right);
\end{align}
where $K_2$ is a modified Bessel function of the second kind and $\gamma$ is the electron Lorentz factor.
Recall $\Theta_e = \kB T_e/(m_e c^2)$, which is determined by the ion-electron temperature ratio $R \equiv T_i/T_e$:
\begin{align}\label{eq:te_vs_R}
  T_e=\frac{2 m_p u}{3 \kB \rho (2+R)}.
\end{align}
Here $u$ and $\rho$ are the internal energy density and rest-mass density from the GRMHD simulation, and we have assumed that the ions are nonrelativistic with adiabatic index $5/3$ and the electrons are relativistic with  adiabatic index $4/3$.
Thermal models are motivated by the idea that wave-particle scattering drives partial relaxation of the eDF, even though Coulomb scattering is ineffective.

The temperature ratio depends on a balance between microphysical dissipation, radiative cooling, and fluid transport.
Models for collisionless dissipation vary widely in their predictions for the ratio of heat deposited in ions and electrons, but depend most strongly on the local magnetic field strength.
For simplicity, we adopt a prescription in which the temperature ratio is only a function of the plasma beta.
This motivates a prescription in which the temperature ratio depends solely on the plasma $\beta \equiv P_\mathrm{gas}/P_\mathrm{mag}$ \citep{2015ApJ...799....1C}.
We adopt the same model as \citetalias{M87PaperV} and \citetalias{M87PaperVIII}, where $R$ is a smooth function adopted from \cite{2016A&A...586A..38M}:
\begin{equation}\label{eq:rhigh_prescription}
  R = \frac{T_i}{T_e} = \Rh \frac{b^2}{b^2+1} + \Rl \frac{1}{b^2+1},
\end{equation}
where $b \equiv \beta/\beta_\mathrm{crit}$.
This model has three free parameters: $\beta_\mathrm{crit}$, $\Rl$, and $\Rh$.
We fix $\Rl = 1$ (consistent with the long cooling time in \sgra; see discussion in \citealt{M87PaperVIII}) and $\beta_\mathrm{crit} = 1$, but allow $\Rh$ to vary from 1 to 160.  When $\Rh \gg 1$ emission is shifted away from the midplane and toward the poles.

In \emph{non-thermal} models, the eDF has a power-law tail extending to high energy.
We explore two implementations:
\emph{i}) a power-law distribution function
\begin{align}
  \frac{1}{n_e} \frac{d n_e}{d\gamma} &=
  \frac{p-1}{\gamma_{\min}^{1-p} - \gamma_{\vphantom{i}\max}^{1-p}} \gamma^{-p},
  \label{eq:nonthermaleDF}
\end{align}
which has power-law index $p$ and upper and lower limits $\gamma_{\min}$ and $\gamma_{\vphantom{i}\max}$; and
\emph{ii}) a so-called $\kappa$ distribution function, inspired by observations of the solar wind and by results of collisionless plasma simulations \citep[e.g.,][and references therein]{2015JPlPh..81e3201K}
\begin{align}
  \frac{1}{n_e} \frac{d n_e}{d\gamma} =
  \gamma \sqrt{\gamma^2-1} \left(1+\frac{\gamma-1}{\kappa w}\right)^{-(\kappa+1)},
  \label{eq:kappaeDF}
\end{align}
which has width parameter $w$ and power-law index parameter $\kappa$.

Evidently, any eDF assignment scheme is an approximation since the eDF depends in general on both local conditions and particle histories.
Notice that we also assume the eDF is isotropic and neglect electron-positron pairs.

Once the eDF is specified, the radiative transfer coefficients (emissivities, absorptivities, and rotativities) can be readily calculated; see \cite{2021ApJ...921...17M} for a recent summary.

%..............................................................................
\subsubsubsection{Model Scaling}

With the exception of the stellar wind-fed simulations, the GRMHD simulations considered in this work contain a characteristic speed, $c$, but are otherwise scale-free; they set $GM = c = 1$.
Physical scales are assigned during the radiative transfer step.
The black hole mass $\mbh$ fixes the length unit $\rg$ and time unit $\tg$.
Because the GRMHD simulations are non-selfgravitating, one is free to set a density scale, or equivalently the accretion rate $\dot{M}$ or plasma mass scale $\Munit$. 
%\citep[see, e.g.,][for a full discussion]{Wong_2022}.

The plasma mass scale parameter $\Munit$ controls the plasma emissivity and the plasma optical depth and thus the source brightness.
We adjust $\Munit$ iteratively until the time-averaged 230\GHz flux densities of the models are within a few percent of the $2.4\,\mathrm{Jy}$ mean observed during the 2017 campaign.
Notice that, in this work, model parameters are always varied at constant time-averaged millimeter flux density.

%..............................................................................
\subsubsubsection{Radiative Transfer Calculation}

\begin{figure*}
  \centering
  \includegraphics[width=\textwidth]{figures/example_imgs.pdf}
  \caption{Example images from the model library.
    Left column: thermal MAD from the best-bet region of parameter space; middle column: nonthermal variable $\kappa$ MAD; right column: thermal SANE model.
    Top row: 86\GHz images; bottom row: 230\GHz images.
    Color represents intensity, or equivalently brightness temperature.
    Angular momentum of the accretion flow projected onto the image points up.
    These are relatively successful models satisfying most of the observational  constraints.}
  \label{fig:example_imgs}
\end{figure*}

\begin{figure*}
  \centering
  \includegraphics[width=\textwidth]{figures/example_vas_seds.pdf}
  \caption{Visibility amplitudes (VAs; left) and SEDs (right) of the three examples models compared with the calibrated EHT~2017 data.
    Black symbols show observations.
    Blue, orange, and green are the models shown in Figure~\ref{fig:example_imgs}.
    Observed VAs are 1\,minute incoherently averaged data from the HOPS pipeline on \aprilvii.
    Model VAs for a single snapshot are shown as a solid line for a section in the $(u,v)$ plane at position angle $0\degree$. The band shows the 1st through 99th percentile over all position angles and all times.  No noise is included in the model VAs in this figure.
    Model SEDs (right) show a solid line for the mean SED and a band for the range across snapshots.}
  \label{fig:example_vas_seds}
\end{figure*}

Given an eDF, density scale $\Munit$, inclination $i$, and radiative transfer coefficients based on local properties of the plasma, the emergent radiation is obtained by integrating the radiative transfer equation.
We use two classes of numerical methods: observer-to-emitter ray tracing to generate synthetic images ({\tt ipole}, \citealt{2018MNRAS.475...43M}, {\tt BHOSS}, \citealt{2012A&A...545A..13Y}), and emitter-to-observer Monte Carlo to generate spectral energy distributions (SEDs, using {\tt grmonty}, \citealt{2009ApJS..184..387D}).

All radiative transport calculations are carried out using the fast light approximation, in which plasma variables are read from a GRMHD output file at constant Kerr-Schild time and are assumed not to change during ray tracing.
Including light travel time effects in the model introduces minor changes to light curves and images \citet{2010ApJ...717.1092D} (and recently by \citealt{2021MNRAS.508.4282M}).
%It has been demonstrated by \citet{2010ApJ...717.1092D} (and recently by \citealt{2021MNRAS.508.4282M}) that including light travel effects in the model introduces minor changes to light curves and images.
Further detail on numerical methods is given Appendix~\ref{app:radtrans}.
Comparisons of radiative transfer codes \citep{2020ApJ...897..148G, Prather_et_al_2022} show that differences between codes do not contribute substantially to the error budget.

The images are produced at 86\GHz, 230\GHz and 2.2\um (near infrared, hereafter NIR).
Direct imaging includes synchrotron and bremsstrahlung \citep[both ion-electron and electron-electron; see][for a recent review]{2020ApJ...898...50Y}.
Unless stated otherwise the image library has a field of view (full width), resolution (pixel count), and half-width angular size of: $800\uas$, $200 \times 200$, $80 \vartheta_\mathrm{g}$ at 86\GHz; $200\uas$, $400 \times 400$, $20 \vartheta_\mathrm{g}$ at 230\GHz; and $100\uas$, $200\times 200$, $10 \vartheta_\mathrm{g}$ at $2.2\um$.
%A few models are imaged with larger field of view or higher resolution for validation.

SEDs are produced for a set of narrow bins in inclination angle.
At each inclination, the SED is averaged over azimuth.
The SED includes synchrotron, bremsstrahlung, \emph{and} Compton scattering.

We find that $2.2\um$ emission is usually dominated by synchrotron, but occasionally $2.2\um$ synchrotron is so weak that Compton scattering dominates.
We also find that the X-ray can be dominated by either Compton scattering or bremsstrahlung, with the latter dominating in models with a large population of cold electrons at large radius.
Figures~\ref{fig:example_imgs} and \ref{fig:example_vas_seds} show examples of model images and multiwavelength SEDs from our library.

The GRMHD simulation-derived temperatures are unreliable in regions where $\sigma \equiv B^2/(8\pi\rho c^2)$ is large, because truncation error in integration of the total energy equation produces large fractional errors in temperature.
All radiative transfer models therefore set the emissivity, absorptivity, and inverse-Compton scattering cross-sections to $0$ for the regions with $\sigma > \sigma_\mathrm{cut} = 1$.

%==============================================================================
\subsection{Summary of \texorpdfstring{\sgra}{Sgr A*} Model Library}

%You can check which data is available here:
%https://docs.google.com/spreadsheets/d/1gw9ichvvYGHLFsZl2wlxqu-O03qEULrwcw3Wixd8BhQ/edit#gid=930351969
%not sure that gives complete answer, but it will help.

A summary of radiative transfer calculations is given in Table~\ref{tab:radiativemodels}.
The entire image library contains $6$ simulation sets,  $\sim 1.8$\,million images at each of 86\GHz, 230\GHz, and $2.2\um$, and $\sim 1.3$\,million SEDs.
The images and SEDs together occupy about $50$\,terabytes.

We refer to the thermal, $\Rh$ models as ``fiducial'' models, and the remainder as ``exploratory'' models that test the effect of incorporating changes in the eDF or initial conditions.
Nearly all the exploratory models (exceptions are described in the discussion) are imaged over $5 \times 10^3 G M/c^3$, in comparison to $\ge 10^4 G M/c^3$ for the fiducial models.
The sampling noise in the exploratory models is therefore larger than in the fiducial models and they cannot be tested as rigorously.

The library contains multiple, redundant models for the fiducial models and variable $\kappa$ models.
This provides some control over the systematic uncertainties associated with variations in GRMHD simulation setup and algorithms.

\begin{deluxetable*}{ccccccc}\label{tab:radiativemodels}
\tablecaption{EHT Model Library}
\tablehead{
  \colhead{Simulation}           &%
  \colhead{Transfer Code}        &%
  \colhead{$\Rh$}                &%
  \colhead{Inclination}          &%
  \colhead{SED}                  &%
  \colhead{$\Delta t/(10^3\tg)$} &%
  \colhead{Notes}%
  }
\startdata
\multicolumn{7}{c}{\bf Fiducial models}\\
\hline
\multicolumn{4}{l}{\it Thermal $\Rh$ models} & & &\\
\kharma& \ipole & 1, 10, 40, 160 &  10, 30, ..., 170 &  Yes & 15--30 & \\
\bhac  & \bhoss & 1, 10, 40, 160 &  10, 30, ..., 90  &  Yes & 20--30 & \\
\hamr  & \bhoss & 1, 40, 160     &  10, 50, 90       &  Yes & 20--35 & \\
\koral & \ipole & 20             &  10, 30, ..., 170 &  No  & 5--100 & \\
\hline
\multicolumn{7}{c}{\bf Exploratory models}\\
\hline
\multicolumn{4}{l}{\it Thermal $\Rh$ models} & &  &\\
\hamr Tilted   & \bhoss & 1, 40, 160 & 10, 50, 90 &  Yes & 100--103 & \\
Wind Accretion & \ipole & 13, 28     & N/A        &  No  & 10       &  \\
\hline
\multicolumn{4}{l}{\it Thermal critical $\beta$ model} & & & \\
\kharma & \ipole & N/A &  10, 50, 90 &  No & 30--35 &  \\
\hline
\multicolumn{4}{l}{\it Thermal + power-law models} & & & \\
\hamr &  \bhoss & 1, 40, 160 &  10, 50, 90 &  No & 30--35 & $p = 4$ \\
\hline
\multicolumn{4}{l}{\it Thermal + $\kappa$ models} & & & \\
\bhac & \bhoss & 1, 10, 40, 160 &  10, 30, ..., 90     &  No  & 25--30 & $\kappa = 5$ \\
\bhac & \bhoss & 1, 10, 40, 160 &  10, 30, ..., 90     &  No  & 25--30 & $\kappa = 3.5 (\epsilon_0 = 0.05)$\\
\bhac & \bhoss & 1, 10, 40, 160 &  10, 30, ..., 90     &  No  & 25--30 & $\kappa = 3.5 (\epsilon_0=0.10)$ \\
\bhac & \bhoss & 1, 10, 40, 160 &  10, 30, ..., 90     &  No  & 25--30 & $\kappa = 3.5 (\epsilon_0=0.20)$ \\
\bhac & \bhoss & 1, 10, 40, 80, 160 &  10, 30, ..., 90 &  No  & 25--30 & variable $\kappa=\kappa(\beta, \sigma)$ \\
\hamr & \ipole & 1, 10, 40, 160 &  10, 30, ..., 90     &  Yes & 30--35 & variable $\kappa=\kappa(\beta, \sigma)$ \\
\enddata
\tablecomments{%
  Summary of the EHT \sgra model library.
  All models are imaged at $86\GHz$, $230\GHz$, and $2.2\um$ and some (see column 5) also have spectral energy distributions.
  For the wind-fed accretion model the viewing angle is set by the stellar orbits and $\Rh$ is set so the model matches the observed 230\GHz flux; $\Rh = 13, 28$ for models with weak and strong stellar wind magnetizations, respectively \citep{2020ApJ...896L...6R}.
}

\end{deluxetable*}
