\section{Astrophysical Models}\label{sec:models}

\color{red}
[{\bf Monika's first pass}]
\color{black}

%==============================================================================
\subsection{Basic Assumptions}\label{sec:basic}

Throughout the paper we assume a mass and distance for \sgra:
\begin{align}
  \mbh &= (4.14 \pm 0.014) \times 10^6 \msun,\\
  D    &= (8.127 \pm 0.023) \kpc.
\end{align}
which is approximately the mean of \citet{2019Sci...365..664D} and
\citet{2019A&A...625L..10G}.

We will also assume that \sgra is a black hole so that the spacetime around it is described by the Kerr metric. The black hole dimensionless spin, $\abh$, is a free parameter with $-1 < \abh < 1$, where $\abh \equiv Jc/G\mbh^2$, where $J$, $G$, and $c$ are the black
hole angular momentum, gravitational constant, and speed of light,
respectively.  Following \citetalias{M87PaperV}, 
$\abh < 0$ indicates the angular momentum of the accretion flow and black hole are antiparallel (the disk is ``retrograde'').

Using the above mass and distance, the implied characteristic length
is $\rg = 6.1 \times 10^{11}\cm$, characteristic time is
$\tg = 20.5 \sec$, and angular scale is $G\mbh/(c^2 D) = 5.03\uas$.
The expected diameter of the black hole shadow is $2\sqrt{27} G\mbh/(c^2 D) = (52.2 \pm x) \uas $, where errorbars indicate uncertainty in the black hole spin and viewing angle.

Assuming that the emitting plasma is made of ionized hydrogen (electron-proton plasma), the adopted black hole mass yields the \sgra Eddington luminosity:
$ L_\mathrm{Edd}
= 4\pi G\mbh c m_p/\sigma_{T}
= 5.2 \times 10^{44}\allowbreak \ergsps$
% CG: no effficiency?  usually use 0.1
%The corresponding Eddington accretion rate is $ \dot\mbh_\mathrm{Edd}
%\equiv L_\mathrm{Edd}/(0.1 c^2)
%= 5.8 \times 10^{24} \gm \sec^{-1}
%= 0.09 \msun \yr^{-1}$ where efficiency of energy conversion is 10\%.
and the Eddington ratio:
$ L_\mathrm{bol}/L_\mathrm{Edd}
= 1.9 \times 10^{-10} (L_\mathrm{bol}  /10^{35}) \erg\sec^{-1}$. In a quiescent state the bolometric luminosity of \sgra is $L_{bol} \sim 10^{35}\erg\sec^{-1}$ resulting in extremely small Eddington ratio. In what follows, we will assume that the radiative cooling of plasma around the black hole can be neglected and that model emission can be calculated in a post-processing step.

% below probably not necessary:
% \footnote{Translation of the source flux to luminosity is given by: $ \nu L_\nu
% = 4 \pi D^2 \nu F_\nu
% = 1.8 \times 10^{34} (D/8127 \pc)^2 \times\allowbreak
%   (\nu/230 \GHz)(F_\nu/{\rm Jy}) \erg \sec^{-1}$.}

%==============================================================================
\subsection{One-Zone Model and Estimates}

The development of complex models is guided by simple estimates. Following \citetalias{M87PaperV} we consider a one-zone model for \sgra. The model consists of a uniform plasma sphere of radius $5 \rg$ with magnetic field oriented at $\pi/3$ to the line-of-sight. The magnetic field in the sphere is in pressure equilibrium with the plasma: $n_i k T_i + n_e k T_e = \beta B^2/(8\pi)$, where we assume $\beta=1$ and $T_i = 3 T_e$.  Assuming the dimensionless electron temperature $\Theta_e \equiv  kT_e / m_e c^2 = 10$, using the thermal emissivity of \cite{2011ApJ...737...21L}, and assuming optically thin emission, the values of $B$ and $n_e$ required to produce observed flux of 2.4 Jy (average flux of \sgra measured by ALMA during the 2017 campaign) are : \color{green}\begin{eqnarray}
\label{eq:onezone}
    n_e &\simeq& 1.1\times 10^6 \,\rm cm^{-3},\\
    B &\simeq& 30 \rm\, G.
\end{eqnarray}\color{black}
These values are consistent with $n_e$ and $B$ of a similar one-zone model fitted to archival Sgr~A* spectrum reported in \citet{2018ApJ...868..101B}.
% CFG: I've placed a mathematica script implementing this model in 

The one-zone model density and size translate into optical thickness for Compton scatterings, $\tau_e \approx \sigma_T n_e 5 \rg \approx 2 \times 10^{-5}$, which yields a small Compton parameter: $y =  16 \Theta_e^2 max(\tau_e,\tau_e^2) = 3 \times 10^{-2}$. This implies that the synchrotron emission is the main plasma cooling process. We can calculate the synchrotron cooling timescale $t_{cool} \equiv u/\Lambda$ where $u=?$ is the electron energy and $\Lambda=16 B^2 e^4 n_e \Theta_e^2 /(3 c^3 m_e^2)$ is the total synchrotron cooling rate from thermal population of electrons (for details see Appendix~A in \citealt{2011ApJ...735....9M}). For $n_e = 1.1 \times
10^6$, $B = 30\,\mathrm{G}$ and $\Theta_e=10$, the estimated $t_{cool}=3.1 \times 10^4\sec
= 1.5 \times 10^3 \tg$ which is much longer than the characteristic time scale in the system. One-zone models confirms that the cooling effects can be neglected in the complex models (e.g., \citealt{2012MNRAS.426.1928D}).

There is a reason to believe that, if \sgra is fed by stellar winds as
in \citet{2019MNRAS.482L.123R}, the inflowing plasma is almost pure
helium. In this case the results change only slightly; the cooling time drops
slightly because the rest-mass density (and therefore field strength)
per electron increases.

Simple considerations also suggest that plasma around \sgra is mostly collisionless (mean free path for electron-ion collisions, $\lambda_F=x$ cm, is much longer than the characteristics length of the systems defined in Section~\ref{sec:basic}).
This allows electron heating (e.g., accelerated by reconnection or turbulence) and
cooling (e.g., radiative cooling) mechanism to drive the electrons
away from thermodynamic equilibrium with ions.
These findings motivate considering \emph{i}) two-temperature plasma models where electrons are cooler than the
ions, and \emph{ii}) the non-thermal electron distribution functions.
As shown in early works by
\citet{1998ApJ...492..554N} and \citet{2000ApJ...541..234O} both
effects may significantly change the predicted properties of \sgra.

\monika{ need to revise the numbers above}

%==============================================================================
% \subsection{Semi-Analytical Accretion Models}

% ...

% \subsubsection{RIAF Model}

% ...

% \subsubsection{RIAF+Jet Model}

% ...

%==============================================================================
\subsection{Numerical Models of the Inner Accretion Flows}


\begin{deluxetable*}{cccccccc}
\tabletypesize{\footnotesize}
\renewcommand{\arraystretch}{1.1}
\tablehead{
  & \colhead{Spacetime} & \multicolumn{2}{c}{Fluid} & \multicolumn{3}{c}{Numerical} &  \colhead{Note} \\
  \colhead{Name/Code} &
  \colhead{$\abh$} &
  \colhead{Mode} & \colhead{$\Gamma$} & \colhead{$t_\mathrm{final}$ [$M$]}
 & \colhead{Size [$M$]} & \colhead{Resolution} &
  \colhead{Reference}
}
\startdata
iHARM/Illinois & 0,$\pm1/2$,$\pm15/16$ & MAD, SANE  & $4/3$ & 30,000 & ? & [1x1x1] & This work/patoka? \\
BHAC/Frankfurt/Radboud & 0,$\pm1/2$,$\pm15/16$ & MAD, SANE  & $4/3$ & 30,000 & ?& AMR? & This work \\
HAMR & 0,$\pm1/2$,$\pm15/16$ & MAD, SANE  & $13/9,5/3$ & 35,000 & 1000,200 & [348/240×192×192] &This work \\
Koral Long &0,0.3,0.7,0.9 & MAD & ? & 100,000 & ? & ?&?\\
HAMR-Tilted (Tilt=$?^\circ$) & $15/16$ & IN-SANE & $5/3$ & 100,000 & 100,000 & [448x144x240] & Chatterjee+20, Liska+18 \\
Self-consistent Wind Feeding & 0 & MAD? & ? & ? & ? &?& \citet{2019MNRAS.482L.123R}
\enddata
\caption{Summary of GRMHD simulations in \sgra EHT GRMHD model library.}
\label{tab:GRMHDmodels}
\end{deluxetable*}

%------------------------------------------------------------------------------
\subsubsection{Gas Dynamics Simulations}

\begin{figure*}
  \centering
  [altext: 3d plots of MAD, SANE, tilted disk, Resseler]
  \caption{Flow properties of fiducial GRMHD models.
    ...}
  \label{fig:GRMHD}
\end{figure*}

% \paragraph{Methods, initial and boundary conditions:}
We model complex dynamics of RIAF/ADAF around a black hole by integrating standard relativistic equations of  non-radiative magnetohydrodynamics in Kerr metric where the black hole spin is a parameter (ref). We integrate the equations in three dimensions using several distinct numerical codes: iHARM (ref), BHAC (ref), HAMR (ref), Koral (ref) and Athena (ref). (For general comparison of GRMHD codes we refer the reader to the code comparison papers by Porth et al and Olivares et al. ).
All simulations use adiabatic equation of state with adiabatic index, $\Gamma_{\rm ad}$, equal 5/3, 4/3 or 13/9.
In our standard simulation setup the plasma is initially confined into a Fishbone-Moncrief torus on co-rotating or counter-rotating orbit around the black hole. Only in one model, HARM-Tilted, started with torus angular momentum tilted with respect to black hole spin. The torus initial size is parametrized by the inner radius $R_{\rm in}=(6M,12M)$ and radius of the pressure maximum $R_{\rm max}=(12M,24M)$, where typically adopted values are given in parenthesis.  MAGNETIC FIELD AND FLUX MAD AND SANE
All simulations in our library are evolved for at least 30000 M and up to 100,000M. In our standard models, the boundary conditions are set to "outflow" so the torus is a finite reservoir of matter.
We also consider one non-standard GRMHD simulations in which \sgra black hole is feed directly by stellar winds produced by .... ARE THOSE SANE OR MADS?
The content of the model library is summarized in Table~\ref{tab:GRMHDmodels}. Our simulations of \sgra are run at different numerical resolutions and using different numerical methods. We present convergence study between models and codes in Appendix~\ref{app:resolution_study}. We find that all solutions relax to quasitationary state and coverge to the same state until what radius? Our models


%------------------------------------------------------------------------------
\subsubsection{Emission Simulations}


Most of our GRMHD models are carried out in dimensionless units and
they are single fluid models meaning that they track total gas
pressure or gas internal energy. To calculate model emission one has
to make an assumption about electron distribution function (hereafter
eDF). In this work we consider thermal and non-thremal electron
distribution functions. In what follows we describe the details of our
electron models.

\begin{figure*}
  \centering
  [altex: emissivity from different models]
  \caption{Emissivity from thermal, critical beta, kappa, and variable
    kappa models.}
  \label{fig:jnu}
\end{figure*}

\paragraph{Parametrized thermal electron distribution function}
In thermal models electron energies are distributed
- thermal models, which parameters and which simulations, hilighting disk vs jet emission
\begin{equation}
\frac{T_i}{T_e} = R_{\rm high} \frac{b^2}{b^2+1} + R_{\rm low} \frac{1}{b^2+1}
\end{equation}
where $b\equiv\beta/\beta_{crit}$, $\beta\equiv=P_{gas}/P_{mag}$.


\paragraph{Non-thermal electron distribution functions}
- nonthermal models, which parameters and which simulations, motivation and references to previous work

\begin{itemize}
\item $\kappa$-model with $\kappa = 5$ everywhere.
\item $\kappa$-model with $\kappa = \kappa(\sigma)$ prescription
  \citep{2016ApJ...826...77B}.
\item Mixed thermal-non-thermal models $\kappa = 3.5$ with $\eta =
  \eta(\sigma)$ using $\Rh$ like prescription; $\eta$ range from 0\%
  to 20\%.
\end{itemize}

%------------------------------------------------------------------------------

\begin{figure*}
  \centering
  [altex: figures showing images for a few fiducial models?
    Maybe SANE and MAD thermal, Ressel and a nonthermal model?]
  \caption{Fiducial images from the simulation library.
    (a) thermal SANE, (b) thermal MAD, (c) nonthermal SANE, (d)
    Resseler.
    \monika{no, no images here, I want a 3D figure of
      MAD/SANE/TILTED DISK/Ressler model. we discussed that we will
      show only best bet images later on.}
    \ckc{updated fig.~\ref{fig:GRMHD} to show 3d plots.  It will be odd to show emissivity (fig.~\ref{fig:jnu}) but not the GRRT images.  Maybe skip fig.~\ref{fig:jnu} as well?}}
  \label{fig:fiducial_imgs}
\end{figure*}

\paragraph{Model Images}
Model images are ray-traced at two frequencies: at the GMVA 86 GHz and at EHT frequency of 230 GHz. 
OPTICALLY THICKER AT 86 GHz BUT RELATIVISTIC EFFECTS STILL MATTER
Ray-tracing simulations are carried out using several numerical codes (ipole, BHOSS, ? list codes are references). A detailed comparison between numerical radiative transfer methods presented in Gold et al. 2021 and in Prather et al. 2021 shows agreement between codes sufficient to make no difference. Our images are generated with different field-of-view, resolution (see discussion in Appendix B2).
LIST ALL IMAGING PARAMETERS. REFERE TO TABLE 2.

\begin{figure*}
  \centering
  [altex: figures showing SEDs for a few fiducial models?
    Maybe SANE and MAD thermal, Ressel and a nonthermal model?]
  \caption{Fiducial SEDs from the simulation library.
    (a) thermal SANE, (b) thermal MAD, (c) nonthermal SANE, (d)
    Resseler.\monika{no, no images here, I want a 3D figure of
      MAD/SANE/TILTED DISK/Ressler model. we discussed that we will
      show only best bet images later on.}
    \ckc{updated fig.~\ref{fig:GRMHD} to show 3d plots.  It will be odd to show emissivity (fig.~\ref{fig:jnu}) but not the SEDs.  Maybe skip fig.~\ref{fig:jnu} as well?}\monika{how about showing model SEDs for all models in the appendix? (a big grid of seds showing convergence between models/codes etc); in the main text we only show the best bet models SEDs but in the "result" section or "data" section. we can do the same for images.what do you think?}}
  \label{fig:fiducial_SEDs}
\end{figure*}

\paragraph{Model Spectra}
what code and what processes are included...
HOW DO WE COMPUTE SEDS? mention that we do not attempt to recover entire radio spectrm because some of the models are not relaxed at large radii, maybe some of them are. refer to appendix for numerical details/


%==============================================================================
\subsection{Summary of astrophysical models}

summary of how many image templates we have and are they consistent with single zone model estimates?

\monika{edited until here, but not finished}

\begin{deluxetable*}{cccccccccccccc}
\tabletypesize{\footnotesize}
\renewcommand{\arraystretch}{1.5}
\caption{Summary of emission simulations in \sgra EHT model library.}~\label{tab:radiativemodels}
\tablehead{
  \colhead{$R_{\rm low}$}          &%
  \colhead{$R_{\rm high}$}         &%
  \colhead{$\beta_{\rm crit}$}     &%
  \colhead{$p$}                    &%
  \colhead{$\gamma_{\rm min/max}$} &%
  \colhead{$\kappa$}               &%
  \colhead{$i^\circ$}              &%
  \colhead{$\rho_{\rm unit}$}      &%
  \colhead{$\nu$[GHz]}             &%
  \colhead{SED}                    &%
  \colhead{$\Delta t$ [1000 M]}    &%
  \colhead{\#snapshots}            &%
  \colhead{Name}}
\startdata
\multicolumn{13}{c}{Thermal models}\\
1 & [1;10;40;160] & 1 & - &  - & - & [10,30,...,170] & 3 & [86,230] & yes & [15,20) & x & iHARM-Thermal1\\
1 & [1;10;40;160] & 1 & - &  - & - & [10,30,...,170] & 3 & [86,230] & yes & [20-25) & x & iHARM-Thermal2\\
1 & [1;10;40;160] & 1 & - &  - & - & [10,30,...,170] & 3 & [86,230] & yes & [25-30) & x & iHARM-Thermal3\\
1 & [1;2.5;5;10;40;160] & 1 &  - & - & - & [10,30,...,90] & 3 & [86,230] & yes & [10-15) & x & BHAC-Thermal1\\
1 & [1;2.5;5;10;40;160] & 1 & - &  - & - & [10,30,...,90] & 3 & [86,230] & yes & [20-25) & x & BHAC-Thermal2\\1 & [1;2.5;5;10;40;160] & 1 & - &  - & - & [10,30,...,90] & 3 & [86,230] & yes & [25-30) & x & BHAC-Thermal3\\
1 & [1,40,160] & 1 & - & - & - & [10,50,90] & 1 & [230] & ? & ? & x & HAMR-Thermal\\
\hline
\multicolumn{13}{c}{Non-thermal power-law models}\\
- & - & - & ?  & ? & - & & & & & \\
\hline
\multicolumn{13}{c}{Non-thermal $\kappa$ models}\\
- & - & - & -  & - & 5 & & & & & \\
1 & [1;2.5;5;10;40;160]  & 1 & -  & - & 3.5 (\epsilon_0=0.05) & [10,30,...,90]  & 1 & [86,230] & no & [25-30) & x & BHAC-kappa005 \\
1 & [1;2.5;5;10;40;160]  & 1 & -  & - & 3.5 (\epsilon_0=0.10) & [10,30,...,90]  & 1 &  [86,230] & no &[25-30) & x & BHAC-kappa010 \\
1 & [1;2.5;5;10;40;160]  & 1 & -  & - & 3.5 (\epsilon_0=0.20) & [10,30,...,90]  & 1 & [86,230] & no & [25-30) & x & BHAC-kappa020 \\
- & - & - & -  & - & $\kappa(\beta,\sigma)$ & & & & &
\enddata
\end{deluxetable*}


% ----------------------------

% %..............................................................................
% \subsubsection{Thermal Models}

% ...

% \paragraph{Illinois Models}

% % Please fill in basic information of the models in the following list.
% % Please add more details if necessary. A full paragraph description
% % of the model is welcome, but not required at this point.
% \begin{itemize}[noitemsep]
% \item $a_\mathrm{spin}$: 0, $\pm1/2$, $\pm15/16$
% \item Magnetic Flux: MAD, SANE
% \item Adiabatic Index $\Gamma$: 4/3
% \item Time $t_\mathrm{final}$: 30,000$M$
% \item $\rho_0$: 3 different density normalization chosen for each parameter set for $t \in [15,000, 20,000), [20,000, 25,000), [25,000, 30,000)$.
% \item $R_\mathrm{high}$: 1, 10, 40, 160
% \item Inclination $i$: 10$^\circ$, 30$^\circ$, 50$^\circ$, ..., 170$^\circ$
% \item Resolution:
% \item Initial conditions:
% \item Reference: this work
% \item Status: w4 and w5 all done; w3 in progress
% \end{itemize}

% \paragraph{Frankfurt Models}

% % Please fill in basic information of the models in the following list.
% % Please add more details if necessary. A full paragraph description
% % of the model is welcome, but not required at this point.
% \begin{itemize}[noitemsep]
% \item $a_\mathrm{spin}$: 0, $\pm1/2$, $\pm15/16$
% \item Magnetic Flux: MAD, SANE
% \item Adiabatic Index $\Gamma$: 4/3
% \item Time $t_\mathrm{final}$: 30000
% \item $\rho_0$: 3 different density normalizations chosen for each parameter set for $t \in [10,000, 15,000), [20,000, 25,000), [25,000, 30,000)$
% \item $R_\mathrm{high}$: 1, 2.5, 5, 10, 40, 160
% \item Inclination $i$: 10$^\circ$, 30$^\circ$, 50$^\circ$,..., 90$^\circ$
% \item Resolution:
% \item Initial conditions:
% \item Reference: this work
% \item Status: all done except for SANE a=-15o16
% \end{itemize}

% \paragraph{HAMR Models}

% % Please fill in basic information of the models in the following list.
% % Please add more details if necessary. A full paragraph description
% % of the model is welcome, but not required at this point.
% \begin{itemize}[noitemsep]
% \item $a_\mathrm{spin}$: 0, $\pm1/2$, $\pm15/16$
% \item Magnetic Flux: MAD, SANE
% \item Adiabatic Index $\Gamma$: 13/9, 5/3
% \item Time $t_\mathrm{final}$: $35,000M$
% \item $\rho_0$: 1 density normalization for $[30,000-35,000)M$
% \item $R_\mathrm{high}$: 1, 40, 160
% \item Inclination $i$: 10$^\circ$, 50$^\circ$, 90$^\circ$
% \item Resolution: $348\times 192\times 192$, $240\times 192\times 192$
% \item Initial conditions: FM: $r_{\rm in}=6, 20M$; $r_{\rm pmax}=12, 41M$
% \item Grid outer radius: $1000M$, $200M$
% \item Reference: this work
% \item Status: GRMHD simulations done
% \end{itemize}

% %..............................................................................
% \subsubsection{Non-thermal (power-law )Models}

% ...

% \paragraph{Frankfurt Models}

% % Please fill in basic information of the models in the following list.
% % Please add more details if necessary. A full paragraph description
% % of the model is welcome, but not required at this point.
% \begin{itemize}[noitemsep]
% \item $a_\mathrm{spin}$: 0, $\pm1/2$, $\pm15/16$
% \item Magnetic Flux: MAD, SANE
% \item Adiabatic Index $\Gamma$:
% \item Time $t_\mathrm{final}$:
% \item $\rho_0$:
% \item Power law fraction $f$:
% \item Power law index $p$:
% \item Inclination $i$:
% \item Reference:
% \item Status: no power-law model so far
% \end{itemize}

% \paragraph{HAMR Models}

% % Please fill in basic information of the models in the following list.
% % Please add more details if necessary. A full paragraph description
% % of the model is welcome, but not required at this point.
% \begin{itemize}[noitemsep]
% \item $a_\mathrm{spin}$: 0, $\pm1/2$, $\pm15/16$
% \item Magnetic Flux: MAD, SANE
% \item Adiabatic Index $\Gamma$: 13/9, 5/3
% \item Time $t_\mathrm{final}$: $35,000M$
% \item $\rho_0$: 1 density normalization for $[30,000-35,000)M$
% \item $R_\mathrm{high}$: 1, 40, 160
% \item Inclination $i$: 10$^\circ$, 50$^\circ$, 90$^\circ$
% \item Resolution: $348\times 192\times 192$, $240\times 192\times 192$
% \item Initial conditions: FM: $r_{\rm in}=6, 20M$; $r_{\rm pmax}=12, 41M$
% \item Grid outer radius: $1000M$, $200M$
% \item Reference: this work
% \item Status: GRMHD simulations same as for thermal models
% \end{itemize}

% %..............................................................................
% \subsubsection{Non-thermal ($\kappa$) Models}

% ...

% % Please fill in basic information of the models in the following list.
% % Please add more details if necessary. A full paragraph description
% % of the model is welcome, but not required at this point.
% \begin{itemize}[noitemsep]
% \item $a_\mathrm{spin}$: 0, $\pm1/2$, $\pm15/16$
% \item Magnetic Flux: MAD, SANE
% \item Adiabatic Index $\Gamma$:
% \item Time $t_\mathrm{final}$:
% \item $\rho_0$:
% \item $\kappa$:
% \item Inclination $i$:
% \item Reference:
% \item Status:
% \end{itemize}

% \paragraph{Frankfurt Models}

% % Please fill in basic information of the models in the following list.
% % Please add more details if necessary. A full paragraph description
% % of the model is welcome, but not required at this point.
% \begin{itemize}[noitemsep]
% \item $a_\mathrm{spin}$: 0, $\pm1/2$, $\pm15/16$
% \item Magnetic Flux: MAD, SANE
% \item Adiabatic Index $\Gamma$: 4/3
% \item Time $t_\mathrm{final}$: 30000
% \item $\rho_0$: individual normalisation for each kappa model; only  for $t \in [25,000, 30,000)$
% \item $\kappa$: variable $\kappa(\beta, \sigma)$, fixed $\kappa=3.5$ with $\epsilon=\epsilon_{0} f(\beta,\sigma)$ for $\epsilon_{0}=0.05,0.10,0.20$
% \item $R_\mathrm{high}$: 1, 2.5, 5, 10, 40, 160
% \item Inclination $i$: 10$^\circ$, 30$^\circ$, 50$^\circ$,..., 90$^\circ$
% \item Reference: this work
% \item Status: in production
% \end{itemize}

% \paragraph{HAMR Models}

% % Please fill in basic information of the models in the following list.
% % Please add more details if necessary. A full paragraph description
% % of the model is welcome, but not required at this point.
% \begin{itemize}[noitemsep]
% \item $a_\mathrm{spin}$: 0, $\pm1/2$, $\pm15/16$
% \item Magnetic Flux: MAD, SANE
% \item Adiabatic Index $\Gamma$: 13/9, 5/3
% \item Time $t_\mathrm{final}$: $35,000M$
% \item $\rho_0$: 1 density normalization for $[30,000-35,000)M$
% \item $\kappa$: variable $\kappa (\beta, \sigma)$
% \item $R_\mathrm{high}$: 1, 40, 160
% \item Inclination $i$: 10$^\circ$, 50$^\circ$, 90$^\circ$
% \item Resolution: $348\times 192\times 192$, $240\times 192\times 192$
% \item Initial conditions: FM: $r_{\rm in}=6, 20M$; $r_{\rm pmax}=12, 41M$
% \item Grid outer radius: $1000M$, $200M$
% \item Reference: this work
% \item Status:
% \end{itemize}

% %..............................................................................
% \subsubsection{Critical $\beta$ Models}

% % Please fill in basic information of the models in the following list.
% % Please add more details if necessary. A full paragraph description
% % of the model is welcome, but not required at this point.
% \begin{itemize}[noitemsep]
% \item $a_\mathrm{spin}$:
% \item Magnetic Flux: MAD, SANE
% \item Adiabatic Index $\Gamma$:
% \item Time $t_\mathrm{final}$:
% \item $\rho_0$:
% \item Power law fraction $f$:
% \item Power law index $p$:
% \item Inclination $i$:
% \item Reference:
% \item Status:
% \end{itemize}

% %..............................................................................
% \subsubsection{Stellar Wind Accretion Models}

% % Please fill in basic information of the models in the following list.
% % Please add more details if necessary. A full paragraph description
% % of the model is welcome, but not required at this point.
% \begin{itemize}[noitemsep]
% \item $a_\mathrm{spin}$: 0
% \item Magnetic Flux: MAD, SANE
% \item Adiabatic Index $\Gamma$:
% \item Time $t_\mathrm{final}$:
% \item $\rho_0$:
% \item Power law fraction $f$:
% \item Power law index $p$:
% \item Inclination $i$:
% \item Reference:
% \item Status:
% \end{itemize}

% %..............................................................................
% \subsubsection{Koral Long MAD Models}

% % Please fill in basic information of the models in the following list.
% % Please add more details if necessary. A full paragraph description
% % of the model is welcome, but not required at this point.
% \begin{itemize}[noitemsep]
% \item $a_\mathrm{spin}$: 0, $\pm0.3$, $\pm0.5$, $\pm0.7$, $\pm0.9$
% \item Magnetic Flux: MAD
% \item Adiabatic Index $\Gamma$:
% \item Time $t_\mathrm{final}$: 100,000$M$
% \item $\rho_0$:
% \item Power law fraction $f$:
% \item Power law index $p$:
% \item Inclination $i$:
% \item Reference:
% \item Status:
% \end{itemize}

% %..............................................................................
% \subsubsection{Tilted Models}

% % Please fill in basic information of the models in the following list.
% % Please add more details if necessary. A full paragraph description
% % of the model is welcome, but not required at this point.
% \begin{itemize}[noitemsep]
% \item $a_\mathrm{spin}$: $+15/16$
% \item Magnetic Flux: INSANE
% \item Adiabatic Index $\Gamma$: 5/3
% \item Time $t_\mathrm{final}$: $>100,000M$
% \item $\rho_0$: 1 density normalization for $[100,000-103,000)M$
% \item $R_\mathrm{high}$: 1, 40, 160
% \item Inclination $i$: 10$^\circ$, 50$^\circ$, 90$^\circ$
% \item Resolution: $448\times 144\times 240$,
% \item Initial conditions: FM: $r_{\rm in}=12.5M$; $r_{\rm pmax}=25M$
% \item Grid outer radius: $100,000M$
% \item Reference: Chatterjee+20, Liska+18
% \end{itemize}
