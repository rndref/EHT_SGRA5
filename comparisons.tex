\section{Model Comparison}\label{sec:comparisons}

We now apply the observational constraints from Section~\ref{sec:observations} to the models described in Section~\ref{sec:models}.

%==============================================================================
\subsection{Thermal, Aligned Models}\label{subsec:thermal}

We begin with a set of ``standard'' models with aligned (prograde or retrograde) GRMHD simulations, thermal eDFs, and electron temperature assigned according to the $\Rh$ model, as in \citetalias{M87PaperV}.  This includes the \kharma, \bhac, and \hamr model sets listed in Table~\ref{tab:radiativemodels}.

%------------------------------------------------------------------------------
\subsubsection{EHT Constraints}

How do the models fare when compared to the EHT data alone, using the null location, size, and m-ring fitting constraints?  The null location test is informative and tends to rule out edge-on models.  The pre-image size constraint is simple but uninformative: only two face-on, $\Rh = 1$ models fail the test.   The m-ring fitting is noisy but highly informative.  Many thermal model distributions look like the data, but edge-on models are strongly disfavored.

%..............................................................................
\subsubsubsection{Null Location}

% Statement about consistency between overlapping model sets.

\begin{figure*}
  \centering
  %\includegraphics[width=0.5\textwidth]{figures/SANE_snapshot.pdf}%
  %\includegraphics[width=0.5\textwidth]{figures/MAD_snapshot.pdf}\\
  \includegraphics[width=0.5\textwidth]{figures/SANE_va.pdf}%
  \includegraphics[width=0.5\textwidth]{figures/MAD_va.pdf}
  \caption{The left panels show two snapshots from a GRMHD simulation
    with SANE field configuration and black hole spin $a=0.5$ and the
    right panels the corresponding visibility amplitudes for a
    horizontal and a vertical cross section through the images.
    The snapshot in the top row obeys both selection criteria: the
    minima are in the 2.5-3.5 G$\lambda$ range and the amplitude in
    the $6-8$G$\lambda$ is below 6\%.
    The image in the bottom row, on the other hand, is an example that
    has no minimum in one cross section and too much power at long
    baselines, due to the asymmetry introduced by a transient bright
    structure in the flow.
    \ckc{The above plot looks odd because it's just one of the snapshot.  Do we want to show the mean?  The range?}
    \aeb{A few very minor pedantic points: 1. Can we fix the plot domain to avoid the sudden ending of lines?  They are misleading in that they seem to show an end to a function that certainly continue. 2. The fonts are wrong and the font size is too small.  3. Colors are not being used effectively to indicate conceptual closeness.  4. This plot fails the Jon Aarons test: if it were printed in grayscale or seen by a colorblind individual it would be impenetrable.}}
  \label{fig:cmp_VA}
\end{figure*}

%\begin{figure}
%  \centering
%  \includegraphics[width=0.5\textwidth]{figures/va_compare.pdf}
%  \caption{Visibility amplitude as a function of baseline length
%    observed on 2017 April 7.
%    The pink band marks the location of the first minima in the
%    visibility amplitude along different orientations.
%    The horizontal red line marks our conservative upper limit for the
%    observed visibility amplitude between $6-8$G$\lambda$.}
%  \label{fig:cmp_null}
%\end{figure}

%\begin{figure}
%  \centering
%  \includegraphics[width=\columnwidth]{./figures/Null_loc_Constraints.png}
%  \caption{Null Location Constraint\ckc{Texts/labels in pizza plots too %small to read.}}
%  \label{fig:cmp_ozel}
%\end{figure}

The null location constraint is a simple comparison between the model and observed variability amplitude.
It disfavors edge-on MAD models at positive spin and a few large $\Rh$ SANE models.
The null location constraint passes 77\% of models.

%..............................................................................
\subsubsubsection{Second Moment}

%\begin{figure}
%  \centering
%  \includegraphics[width=\columnwidth]{./figures/230GHz_size_Constraints.png}
%  \caption{2nd moment plots}
%  \label{fig:cmp_2nd_moment}
%\end{figure}

The second moment constraint passes 99\% of models, that is, the models are all crudely the right size.  The models that fail are retrograde, face-on, SANE models with $\Rh = 1$. These models have extended rings of emission with angular extent that is large compared to the critical impact parameter.

%..............................................................................
\subsubsubsection{M-ring Fits}
\label{sec:mring}

The m-ring fits pass 94\%, 65\%, and 36\% of models for the ring asymmetry, diameter, and width respectively.

The asymmetry parameter is typically not very well constrained.  The models that fail are almost all high inclination models with positive spin.  The failing models have asymmetries that are large and detectable because Doppler boosting concentrates emission in an equatorial spot on the approaching side of the disk.

%\begin{figure*}
%  \centering
 % \includegraphics[width=\columnwidth]{./figures/Mring_f1_Constraints.png}
%  \caption{m-ring asymmetry}
%  \label{fig:cmp_m-ring_asymm}
%\end{figure*}

The ring diameter is better constrained than the asymmetry parameter.  It also varies systematically from model to model.  A much larger fraction of models therefore fails the ring diameter test.

Most of the models that fail are low inclination models with ring diameters that are too large (only one model fails because the ring diameter is too small).  For example, the face-on, $\Rh = 10$ SANE models fail for all spins except $\abh = 0.94$ because the ring is too large.  The same is true for all face-on MAD models with $\Rh = 1$.

\begin{figure*}
  \centering
  \includegraphics[width=\textwidth]{./figures/Mring_d_Constraints.png}
  \caption{M-Ring diameter.}
  \label{fig:cmp_m-ring_diam}
\end{figure*}

The m-ring width is best constrained and therefore most constraining.  Although the closure phases constrain the m-ring width as well, it is easy to see how the visibility amplitudes are affected by m-ring width: the width controls long-baseline amplitudes. Smaller width for a simple, symmetric ring corresponds to larger visibility amplitudes on long baselines.

All models that fail have m-ring widths that are too small.  This includes all but 3 MAD models at $\abh \le 0$ and all MAD models at $i = 90\degree$.

\begin{figure*}
 \centering
 \includegraphics[width=\textwidth]{./figures/Mring_w_Constraints.png}
  \caption{m-ring widths}
% \label{fig:cmp_m-ring_width}
\end{figure*}

\subsubsubsection{EHT Constraint Summary}

Constraints based on EHT data are summarized in Figure \ref{fig:all_EHT_constraints}.  The cuts favor $\abh > 0$ models.  We are left with $31/100$ SANE models and $20/100$ MAD models.

\begin{figure*}
  \centering
    \includegraphics[width=\textwidth]{./figures/Interferometric_Constraints.png}
  \caption{Combined EHT constraints (logical {\em and}) including the second moment, null location, and m-ring fit constraints.}
  \label{fig:all_EHT_constraints}
\end{figure*}

\subsubsection{Non-EHT Constraints}

Now consider constraints from 86GHz, NIR, and X-ray observations.  Most or all of the emission in these bands is believed to originate in the compact source from plasma that is close to or overlaps the plasma the 230GHz-emitting plasma observed by EHT.

%..............................................................................
\subsubsubsection{NIR Median Flux}

NIR photons are produced by synchrotron process from photons at the high energy end of the distribution function.  For a one-zone model with $B = 30$G, the  critical frequency $\nu_{crit} \simeq \gamma^2 e B/(m_e c) \simeq 80$GHz and emission at $2.2\mu$m is therefore produced by electrons with $\gamma \simeq 10^3$, compared to a mean Lorentz factor of $30$ for plasma with $\Theta_e = 10$.  NIR flux density will therefore be sensitive to $\Theta_e$ and therefore to $\Rh$.

Interestingly, we find that some models are synchrotron-weak and Compton-strong in the NIR.  \note{Discussion of which models fall in this category}

Models that pass the NIR flux limit are shown in Figure \ref{fig:cmp_2um_flux}.

All but $6\%$ of the SANE models pass; the exceptions are high spin, high inclination models where Doppler boosting increases the NIR flux from the bright spot on the approaching side of the disk.  Considering MAD models alone, $68\%$ fail the NIR test, including all but 1 model at $\Rh = 1$ and $10$.

%\begin{figure}
%  \centering
%  \includegraphics[width=\columnwidth]{./figures/2um_flux_Constraints.png}
%  \caption{NIR flux limit}
%  \label{fig:cmp_2um_flux}
%\end{figure}

%..............................................................................
\subsubsubsection{X-ray Luminosity}

Most thermal models produce X-ray emission through Compton upscattering of thermal synchroton photons.  Typically the X-ray band lies in the first Compton bump, although that is sensitive to the temperature of the electrons doing the upscattering.  In the first Compton bump $\nu L_\nu$ is proportional to the y-parameter $y \sim 16 \Theta_e^2 \tau_e$ where $\tau_e$ is a characteristic electron-scattering optical depth and $\Theta_e$ is a typical electron temperature.

In many large $\Rh$ SANE models, however, X-ray emission is dominated by bremsstrahlung.  Since the bremsstrahlung emissivity $\propto n^2$, bremsstrahlung comes from the midplane where the density is largest, at larger radius than the synchrotron and Compton-upscattered X-ray emission.  It dominates in high accretion rate models (this turns out to mean large $\Rh$ models; see \S 5) where $\Theta_e < 1$, and is significant only when the midplane is cool and $r: \Theta_e = 1 < YY$, i.e. at large $\Rh$.  The resurgence of bremsstrahlung in cool disks occurs because at $\Theta_e < 1$, $j_\nu \propto n^2 \Theta_e^{-1/2}$.  When the disk is cool and dense the latter is large.

The X-ray cut results are shown in Figure \ref{fig:cmp_xray_flux}.

Many large $\Rh$ SANE models fail the X-ray test: all but 3/25 at $\Rh = 160$ and all but 6/25 at $\Rh = 40$.  These models fail due to excess bremsstrahlung.

Many MAD models that fail have large absolute spin and low $\Rh$.  These models are Compton-dominated.  The midplane temperature is highest at low $\Rh$.  Since the midplane contributes most of the electron scattering optical depth, low $\Rh$ models have the largest $y$ parameter and are most at risk of overproducing X-rays.

%\begin{figure}
%  \centering
%  \includegraphics[width=\columnwidth]{./figures/Xray_flux_Constraints.png}
%  \caption{X-ray flux limits}
%  \label{fig:cmp_xray_flux}
%\end{figure}

%..............................................................................
\subsubsubsection{86 GHz Median Flux}

In a naive picture \sgra's millimeter flux is produced at a photosphere that decreases in size as frequency increases.  Because of the marginal optical depth at $1.3$mm ($\sim 0.3$ in the one-zone model) and the complicated source structure (the optical depth varies across the image; the $\tau = 2/3$ surface is non-spherical, folded, not even simply connected) this picture does not hold exactly.  Nevertheless 86GHz photons are on average produced at larger radius than 230GHz photons, and the 86GHz source size is therefore larger than the 230GHz source size; see Ricarte et al. 2022 for a discussion.

The 86GHz/230GHz color therefore probes the radial structure of the source plasma.  Figure~\ref{fig:cmp_86ghz_flux} shows the result of requiring that the 86GHz flux match the observed flux 3 days before the beginning of the EHT 2017 campaign.

Many $\Rh = 1$ models, both MAD and SANE, fail the $86$GHz flux density test: 23/25 SANE and 9/25 MAD.  These models overproduce $86$GHz emission.
There is also a substantial set of SANE models, 19/100 in all, that underproduce $86$GHz emission.  These models have larger $\Rh$.

%\begin{figure}
%  \centering
%  \includegraphics[width=\columnwidth]{./figures/86GHz_flux_Constraints.png}
%  \caption{86GHz flux limits}
%  \label{fig:cmp_86ghz_flux}
%\end{figure}

%..............................................................................
\subsubsubsection{86 GHz Major Axis}

As for the $86$GHz flux, the $86$GHz size is sensitive to optical depth as a function of radius in the source plasma. Models that pass and fail are shown in Figure \ref{fig:cmp_86ghz_size}.

Many face-on models fail because they are too small (purple in the figure), while a few other high inclination models fail because they are too large.  Only $52\%$ of models pass, making this one of the tightest constraints.

The physical picture for 86GHz source size is complicated, as is the extraction of the constraint itself from observations.  Notice that (1) two different values for the 86GHz intrinsic source size have been reported in the literature; (2) scattering is $7$ times stronger at $86$GHz than at $230$GHz; (3) scattering must be subtracted accurately to obtain the intrinsic source size; (4) the error bars for the 86GHz source size are narrow and this determines the strength of the constraint.

%\begin{figure}
%  \centering
%  \includegraphics[width=\columnwidth]{./figures/86GHz_size_Constraints.png}
%  \caption{86GHz size}
%  \label{fig:cmp_86ghz_size}
%\end{figure}

%..............................................................................
\subsubsubsection{Summary of Non-EHT constraints}

Applying only non-EHT constraints, we are left with the 9/100 SANE models and 25/100 MAD models shown in Figure~\ref{fig:non_eht_cuts}.

The surviving models include a set of SANE models at intermediate $\Rh$ and modest inclination, as well as MAD models at large $\Rh \ge 40$. No $\Rh = 1$ models survive the non-EHT cuts.

\begin{figure*}
  \centering
  \includegraphics[width=\textwidth]{./figures/Non_Interferometric_Constraints.png}
  \caption{Combined non-EHT constraints}
  \label{fig:non_eht_cuts}
\end{figure*}

%------------------------------------------------------------------------------
\subsubsection{Variability}

Variability is central to interpretation of \sgra: the small black hole size means that observations considered here are taken over intervals when the source is expected to vary significantly.  This distinguishes \sgra from \m87, where the source is expected to vary only over timescales that are long compared to a single track.\footnote{This does not mean that \m87 is less variable than \sgra.  In 2017 EHT observed \m87 over only $\sim 15\tg$\monika{1 week is 168h and \tg for m87 is 8.5h, this gies 19 \tg not 15...}, so it is not possible to characterize \m87 variability using EHT data alone.  To obtain variability information for \m87 similar to what we present here for \sgra would require multi-year observations.}

Variability is a strong constraint on the models.  Although models differ in their degree of variability, both in an integrated sense and on 4 $G\lambda$ baselines, only a small fraction of models are as quiet as the data.  For the light curve variability, this remains true whether we use data from 2017 April 7, all days from the 2017 observing campaign, or from historical monitoring of \sgra.   In general, we find that SANE models are quieter than MAD models, and (less strongly) face-on models are quieter than edge-on models.

If we were to apply the variability constraints directly to the models, there would be 30/200 successful models left using 1\% cuts (47/200 for the ALMA constraint and 95/200 for the visibility amplitude constraint).  One interpretation of this result is that the surviving models are the correct description of the source (although we would expect some misclassification of models as consistent or inconsistent when using 1\% cuts on such a large model set).  Another interpretation is that there is a missing physical ingredient in the models, see Section \ref{sec:discussions} for a discussion.

%..............................................................................
\subsubsubsection{ALMA Light Curve}

The distributions of 3 hour modulation index (rms \%) across all SANE models, across all MAD models, and across the historical dataset are shown in Figure \ref{fig:cmp_ALMA_var}, along with individual distributions for the models with the lowest and highest MI for SANEs and MADs. Although some individual models  pass (particularly SANE models), the distributions for the SANEs and the MADs are noticeably offset from the data, with the MADs in particular being more variable. As can be seen, even the quietest MAD model lies above the historical distribution.

If we compare each individual model to the three segments from the 7 April 2017 ALMA observation using a 2-sample KS test, eliminating models with $p < 1\%$, we are left with 56 SANE models and 9 MAD models.

If we instead compare the models to the full historical distribution (40 measurements of $\mi{3}$ in all, including ALMA), we find that 47 SANE models and no MAD models pass. This is more stringent than the comparison with just 7 April, since the historical distribution has more samples and thus disparate models can be eliminated with higher confidence.

\begin{figure}
  \centering
  \includegraphics[width=\columnwidth]{./figures/mi_hist.pdf}
  \caption{Distributions of $\mi{3}$ for \kharma models (black), compared to distributions from historical observations (gray). The distributions for models with the lowest (blue) and highest (red) average $\mi{3}$ for SANEs and MADs are also shown. The heights of these distributions have been scaled down for visual clarity.
  }
  \label{fig:cmp_ALMA_var}
\end{figure}

%..............................................................................
\subsubsubsection{4 $G\lambda$ Visibility Amplitude Variability}

The power-law indexes of the variance $\sigma_\text{var}^2 (|u|)$ at $4~{\rm G}\lambda$ of the GRMHD models is generally in good agreement with the measured value of $b$ from the 2017 EHT campaign (excluding April 11). The amplitude $\afour^2$, however, varies depending on the model and the code.

Figure~\ref{fig:cmp_VLBI_var} shows the distribution of $(\afour^2, b)$ from the EHT observation, along with the distribution across all \kharma models. The GRMHD models are shown as an aggregate whole, but each model consists of only three measurements of $\afour^2$, one on each window. This makes a direct comparison with the measured value difficult, as the distribution for a given model is poorly constrained.

\citet{Georgiev_2022} gives an estimate for the width of the distribution as $\log_{10}(\afour^2) \pm 0.1$. We can get a rough estimate for how the GRMHD models fare compared to the measurement by taking the mean across windows and the estimate for the width, and comparing this with the measurement distribution under the assumption that both are distributed normally. Under this, 95/200 models (50 SANE and 45 MAD) agree with the data within 1\%, although we caution against interpreting this number as the number of passing or failing models, since the uncertainties in the model distribution are so large.

Overall, the GRMHD models tend to be slightly more variable than the measured value, with face-on models performing better than edge-on models. For SANE models, $\Rh = 10$ tend to be more variable than others. For MAD models, there is a slight preferance for lower $\Rh$.

We have also considered a set of thermal, $\Rh$, MAD models run with the \koral code out to $\sim 100,000\tg$.  These models permit us to assess the importance of integration time for application of the constraints.  They permit us to obtain more accurate distributions for the constraint quantities, and to assess whether the constraints evolve from the beginning to end of the integrations. We do not see evidence for evolution in the \koral model set (a full pass/fail table is given in Appendix~\ref{app:tables}, Table~\ref{tab:koralPF} and more detailed discussion in Appendix~\ref{app:variability}). The \koral pass/fail results are similar to those for comparable models in the \kharma thermal model set. Moreover the constraints measured at the beginning of the evolution are similar to those measured at the end.

% The distribution of model $4G\lambda$ lightcurve-normalized PSDs are shown in Figure \ref{fig:cmp_VLBI_var}.  The best fit PSD from the 2017 EHT campaign (excluding April 11) is shown as a solid vertical line, with the other vertical lines showing percentiles in the posterior.  Evidently the observations are quieter than both SANEs and MADs as a group.

% The PSD estimates for the models are broken up into $5000\tg$ windows for each model. To compare the models to the observation, we take the mean value across all windows and assume the width of the distribution is of $\log_{10} a_{4G\lambda} \pm 0.1$. A model is considered passing if this estimated distribution overlaps with the median observed value. \note{refer GRMHD variability paper appendix} \dl{passing criterion subject to change}

% With this approach only 6\% of the models pass, and all SANE. While the quietest models tend to be SANEs, the general distribution across all MADs is not as offset from the SANEs as in the MI distributions.

% \begin{figure}
%   \centering
%   \includegraphics[width=\columnwidth]{./figures/va_hist.pdf}
%   \caption{Distributions of $PSD(4\,\mathrm{G}\lambda)$ for thermal models, compared to the observed distribution from the 2017 EHT campaign.
%   \dl{HAMR distribution will be updated later.}
%   \aeb{The quantity on the horizontal axis has been called $a_4$ in \citetalias{PaperIV}.  I have a similar concern regarding the abrupt end of the curves in this plot as well.  What do they mean?  How should these be interpreted?  There are the standard presentation issues (see \autoref{fig:cmp_VA}).}
%   \ckc{Nice new plot!  Are you using matplotlib fill\_between?  I suggest setting edgecolor to None and only set facecolor.  (Or setting linewidth=0 should also remove the lines in the orange shade.)}}
%   \label{fig:cmp_VLBI_var}
% \end{figure}

\begin{figure}
  \centering
  \includegraphics[width=\columnwidth]{./figures/grmhd_fit.png}
  \caption{\dl{placeholder figure}}
  \label{fig:cmp_VLBI_var}
\end{figure}

% At a 1\% cut, the models that pass the PSD constraint also all pass the MI constraint. It should be noted that this is partially because these constraints are weakly correlated \dl{also partially because we're imposing constraints differently between VLBI and light curve, and the light curve constraint is more lenient} None of the 12 models with acceptable variability pass the other tests.

%------------------------------------------------------------------------------
\subsubsection{Summary of Constraints on Thermal Models}

If we set aside variability but use all the remaining constraints we are left with the models shown in Figure \ref{fig:all_cuts}.  Only 1/100 SANE models and 8/100 MAD models survive. The passing models cluster at low inclination ($i \le 50$deg) MAD models with $\Rh > 10$.  A full set of pass/fail values is provided in Appendix \ref{app:tables}.

It is remarkable that so many of the models look like the EHT data, which lends some  confidence to the models, and that EHT data alone are capable of distinguishing between models in the model set with only $N = 6$ antennas.  Future experiments with more antennas will contain much more information and provide even tighter constraints.

All $\Rh = 1$ models have been eliminated, most by multiple constraints, and we conclude that models with equal ion and electron temperatures are unacceptable.

All models with $i > 50$deg are eliminated, most by multiple constraints, and we conclude that high inclination models are disfavored.  In the SED edge-on models have a clear signature derived from Doppler boosting, with increased NIR and X-ray flux density.  In EHT constraints many edge-on models have a clear signature derived from low m-ring widths, strong asymmetry (although only for a few models), and failed null location constraint.

For thermal model sets both EHT and non-EHT constraints individually eliminate many models, but together they eliminate all but 5\%.  The success of the models for EHT constraints (apart from variability) supports the use of the models for applying non-EHT constraints and highlights the importance of contemporaneous multiwavelength monitoring of \sgra.

\begin{figure*}
  \centering
  \includegraphics[width=\textwidth]{./figures/All_Constraints.png}
  \caption{Combined EHT and non-EHT constraints.}
  \label{fig:all_cuts}
\end{figure*}

Variability, when included in the constraint map, would eliminate all thermal models.  Although a few models pass the variability constraints alone this does not mean that we should regard them as favored, since we expect to eliminate at least a couple of models incorrectly when using a $1\%$ cut for $200$ models.

% this is now moved to appendix as discussed on Thursday Dec 9 2021 among coordinators
% %------------------------------------------------------------------------------
% \subsubsection{Inter-Model Comparison: $\Rh$ Thermal Models}

% % note: passfail tables are consistently, e.g., tab:illinoisPF or tab:VKhamrPF

% \subsubsubsection{Frankfurt Thermal $\Rh$ Models}

% As can be seen in Table \ref{tab:GRMHDmodels} the thermal models have been calculated for an identical parameter space from two different codes, namely KHARMA and BHAC for the GRMHD simulations and iPOLE and BHOSS for the GRRT calculations. This allows us for the first time to perform an in depth comparison between the different numerical methods used in this work in addition to the EHTC code comparison projects \citep{2019ApJS..243...26P,2020ApJ...897..148G}.

% \begin{figure*}
%   \centering
%   \includegraphics[width=0.8\textwidth]{./figures/BHAC_iharm_correlation}
%   \caption{Correlation between BHAC and KHARMA models for 9 model constraints.  The horizontal axis is the constraint value from \bhac/\bhoss, and the vertical axis shows the constraint value from \kharma/\ipole.  Each point corresponds to a single model, with the width of the distribution shown by the error bars.  See text for details.}
%   \label{fig:modelcorrelation}
% \end{figure*}

% In Figure \ref{fig:modelcorrelation} we show the correlation between the thermal KHARMA and BHAC models for constraints where we have predictions from both models.

% The top row shows from left to right the 230\,GHz flux density, the 230\,GHz modulation index, MI, computed for a time window of 3 hours, and the 230\,GHz image size obtained from image moments. Since we normalise the 230\,GHz images to an average flux of 2.4\,Jy within a time window of 5000\,M (corresponds to 28.5 h for SgrA a mass of $4.14\times 10^6\,\msun$), the scatter around this values is small. The deviation from an ideal correlation reflects the precision and number of GRMHD snapshots included during normalization procedure.

% The correlation in the 3\,hour modulation index spreads over $\Delta \mi{3}=0.75$ which serves as a measure of intra-code ( e.g., MAD vs. SANE accretion) and inter-code (BHAC vs. KHARMA) differences. Despite these differences the models show a strong correlation throughout the investigated models and parameter space.

% We found a strong correlation between models and codes for the image size computed from image moments, i.e. second moments analysis.

% \cfg{This may change if we recompute the major axis, etc., for the IL 86 GHz large fov images}
% The middle row presents the correlation plots for the 86\,GHz flux density (left), the 86\,GHz image size using second moments (middle) and the NIR flux (right). The 86\,GHz flux and 86\,GHz image size exhibit a shift toward larger values for the BHAC models. This difference can be explained by the larger field of view used for the BHAC models at 86\,GHz during the radiative transfer calculations. Thus, more extended structure and therefore a larger total flux is included in the BHAC models. This affects mainly models with large inclinations $i\geq70^\circ$ and jet dominated emission models ($\rm{R}_{\rm high}\geq 40$).

% The NIR fluxes show a tight correlation over four orders of magnitude and systematically larger flux for the BHAC models for low NIR fluxes ($\log_{10}(NIR)<-7$). These fluxes are far below the NIR constraints of $\sim 1mJy$, and therefore they do not affect the passing or failing of the models. In the thermal models the NIR flux is generated from the tail of the electron distribution function and is thus very sensitive to the electron temperature. Small differences in the distribution and value of the electron temperature between the two codes explain the observed de-correlation at very low NIR flux.

% The correlation between models for the m-ring parameters is presented in the third row of Fig.~\ref{fig:modelcorrelation}. The correlation of the diameter of the m-ring is plotted in the left panel. The spread covers nearly the same extent as the 230\,GHz image size (top row, right panel) however the scatter in the correlation is larger.  The same is true for the width of the m-ring (middle panel in the last row of Fig.~\ref{fig:modelcorrelation}). Compared to the diameter and width of the m-ring, the asymmetry of the m-ring is less correlated (right panel). Notice that horizontal and vertical limits in the asymmetries occur since the parameter hits the boundary of the allowed range.

% The smaller correlation of the m-ring parameters as compared to the other parameters presented in Fig.~\ref{fig:modelcorrelation} is a consequence of the noisy nature of the m-ring fits.  Still, the distributions are quite symmetric under reflection across the diagonal, so the models are at least not biased with respect to each other.  Notice also that these plots do not capture all the information that is contained in the distribution of m-ring parameters, just the central value.

% We find ourselves somewhat surprised by the strength of the correlations seen in Figure \ref{fig:modelcorrelation}.  The range of each constraint is significantly larger than the width of the correlation, so the variations between models are real, detectable, and reproducible with independent codes.  The question of the origin of the systematic offsets between models for some constraints (for example, in the NIR) is interesting but beyond the scope of this paper.

% \subsubsubsection{\hamr Models}

% %\note{Doosoo, Koushik to write here about HAMR thermal models.}
% Along the KHARMA/iPOLE and BHAC/BHOSS models, we produced a set of thermal models out to $35,000\tg$ using the GRMHD code H-AMR and the GRRT code BHOSS (see Table~\ref{tab:GRMHDmodels}). These models consider a gas adiabatic index of $\Gamma_{\rm ad}=5/3$ for the SANE models and $\Gamma_{\rm ad}=13/9$ for the MAD models (Table~\ref{tab:radiativemodels}), allowing us to understand how sensitive the images are to the GRMHD fluid properties, in addition to code numerics.

% We are glad to note that overall, the H-AMR/BHOSS thermal eDF images perform similarly to the KHARMA/iPOLE and BHAC/BHOSS models. \kc{needs verification from Michi}

% \kc{is the plan to do H-AMR and KHARMA correlations similar to figure 18?}

%\subsubsubsection{\koral Long Models}
% moved up to variability section
% We have also considered a set of thermal, $\Rh$, MAD models run with the \koral code out to $\sim 100,000\tg$.  These models permit us to assess the importance of integration time for application of the constraints.  They permit us to obtain more accurate distributions for the constraint quantites, and to assess whether the constraints evolve from the beginning to end of the integrations. We do not see evidence for evolution in the \koral  model set (a full pass/fail table is given in Appendix \ref{app:tables}, Table~\ref{tab:koralPF}). The \koral pass/fail results are similar to those for comparable models in the \kharma thermal model set. Moreover the constraints measured at the beginning of the evolution are similar to those measured at the end.

%------------------------------------------------------------------------------
%\subsubsection{Critical Beta Model}

Finally, the $\Rh$ eDF model has been the default parameterization used to span the uncertainties in emitting particle thermodynamics in EHT analyses. The $\Rh$ models  are compatible with the well-motivated turbulent plasma heating ADAF models of \cite{1999ApJ...520..248Q} in asymptotic behavior of electron-to-total heating ratio as a function of plasma $\beta$. There is, however, a vast function space of possible alternative parameterizations. Here we have also considered the Critical Beta model \cite{2020MNRAS.493.1404A}, which sets $T_e = f (T_e + T_i) \exp(-\beta/\beta_c)$.  The model is motivated by the observation that when the transition between electron heating domination and proton heating domination is smoothed (controlled by increasing exponent parameter $\beta_c$), the 230 GHz emitting region profile tends to be less coronally dominated and more compact and asymmetric. This is a consequence, when fixing synthetic image flux, of higher critical beta values shifting the locus of electrons dominating the emission profile towards the colder, higher density equatorial inflow.  We consider only a single point in the critical beta parameter space, with $f = \beta_c = 1$. We have applied a subset of tests to the critical beta models (all except X-ray; NIR is calculated with imaging only and therefore does not include Compton scattering).  The full set of results is given in Appendix~\ref{app:tables}, Table~\ref{tab:betacritPF}. All Critical beta models with high inclinations fail EHT constraints, which is consistent with our main conclusion about the thermal models. We also find that all critical beta models fail all considered non-EHT constraints.


% CFG: we shouldn't report preliminary indications here
%A second motivation is that there are preliminary indications that the exponential electron cooling in the Critical Beta Model suppressed the SANE bremsstrahlung spectral component allowing more to pass the X-ray constraint.

%==============================================================================

\subsection{Nonthermal, Aligned Models}

%\note{Koushik to write here about powerlaw nonthermal HAMR models.  Define a subsection, describe the results and how they differ from the thermal results.][Maybe add Tomohisa's models here as well.] [by including power-law, we see this and that change (only include significant changes from the thermal models)}

So far we have assumed a Maxwell-J{\"u}ttner distribution function that describes a thermal population of electrons. Next we assume that these electrons are accelerated to a non-thermal tail. We model such the energy distribution of such a population of electrons using either a mixed thermal-powerlaw eDF or a pure/mixed kappa eDF. Similar to the thermal models, the accretion rates of all model sets in this section are normalized such that the time-averaged 230\,GHz compact flux is 2.4 Jy over 5,000M. 

\subsubsection{Thermal Plus Constant power-law models with $p_\mathrm{pl} = 4$}

In this section, we employ a hybrid thermal/power-law tail distribution using \hamr/\bhoss, and assume a power-law index of $p=4$ with a constant non-thermal acceleration efficiency $\epsilon=n_{\rm e, power-law}/n_{\rm e, thermal}=0.1$. Following the method given in \citet{Chatterjee2021}, the power-law tail is stitched to the thermal core by choosing the minimum Lorentz factor limit of the power-law $\gamma_{\rm min}$ to be at the peak of the Maxwellian component. The maximum limit of the power-law is taken to $10^5 \gamma_{\rm min}$. Further, we note that the normalization value for the accretion rate is slightly smaller than that of the corresponding \hamr/\bhoss thermal models, suggesting that the power-law emission contributes to the 230\,GHz total intensity.

\subsubsubsection{230\,GHz VLBI pre-image size}

Hybrid thermal/power-law models have similar 230\,GHz VLBI pre-image sizes to their purely thermal counterparts. This is because the power-law index is large enough 

\subsubsubsection{86\,GHz flux} 

In general, the $R_{\rm high}=1$ models produce too much 86\,GHz flux. Since the lower limit of the power-law $\gamma_{\rm min}$ is directly affected by the local electron temperature, the highest energy electrons are located in the jet sheath where the ion and electron temperatures are similar, i.e. $T_i\approx T_e$. Indeed this is why SANE models produce more 86\,GHz flux when non-thermal electrons are introduced, especially at larger $R_{\rm high}$ values. On the other hand, MAD pure thermal and mixed thermal/non-thermal models behave similarly as the bulk of the emission is produced in the inner disk.

\subsubsubsection{86\,GHz image size} 



\subsubsubsection{NIR constraints}

The addition of the power-law tail primarily increases the flux in the NIR and thus, the NIR GRAVITY flux of 1.0 mJy could provide a constraint on the power-law index and the acceleration efficiency. In brief, $R_{\rm high}=1$ and $40$ MAD models are ruled out by the NIR constraint.

\subsubsubsection{ALMA Light curves}
\subsubsubsection{m-ring}

\note{Razi to write here about variable kappa models.  Define a subsection, describe the results and how they differ from the thermal results.}\kc{might be better to move this to 4.2.3 along with bhac/bhoss results}

\note{Christian to write here about constant kappa models.  Define a subsection, describe the results and how they differ from the thermal results.}\cmf{done}

%------------------------------------------------------------------------------
\subsubsection{Thermal plus Constant kappa models $\kappa=3.5$ with variable efficiency, $\varepsilon(\sigma,\beta)$}

To investigate the effect of mixed thermal/nonthermal distribution functions, we combine a thermal electron distribution function with a kappa electron distribution with $\kappa=3.5$. The value of $\kappa=3.5$ is motivated from the spectral slope in the NIR during a quiescent state \cmf{add reference here} \kc{suggest to say that the slope is a limit as $\alpha=0.6$ where $F_{\nu}\propto \nu^{\alpha}$ is seen in the flaring state, e.g., Hornstein+2006}.  In addition to the fixed kappa value we assume that the fraction between thermal and non-thermal particles depends on the local plasma properties, e.g, the magnetisation, $\sigma$, and the plasma beta parameter, $\beta_{\rm p}$. Given this assumption we can write the total emissivity as:
\begin{equation}
j_{\nu,\rm{tot}}=(1-\epsilon) j_{\nu,\rm{thermal}} + \epsilon j_{\nu, \kappa},
\label{eq:kappaeff}
\end{equation}
where the nonthermal efficiency $\epsilon( \varepsilon, \beta, \sigma)$ is
\begin{equation}
    \epsilon(\varepsilon,\beta,\sigma)=\varepsilon\,
    \left[1 - e^{-\beta_{\rm p}^{-2}}\right]
    \left[1-e^{-(\sigma/\sigma_{\rm min})^2}\right].
    \label{eq:efficiencybetasigma}
\end{equation}
Evidently the nonthermal emissivity is non-negligible only when $\beta_p \lesssim 1$ and $\sigma \gtrsim \sigma_{\rm min}$.  We fix $\sigma_{\rm min}=0.01$ and vary the base efficiency, $\varepsilon$, between 0.05, 0.1 and 0.2. For each base efficiency we generate a set of models spanning the same parameter space as the thermal models (see Table \ref{tab:radiativemodels} for details). For each model we iterate the mass accretion rate to obtain an average flux of 2.4\,Jy at 230\,GHz across a time interval of 5000\,M. In order to explore several values of $\varepsilon$ efficiently we increased the model snapshot cadence to 50\,M. This allows us to keep the numerical costs for this parameter sweep low while still being within the correlation time of the GRMHD simulations.
%($t_{\rm corr}\approx 50-100\,M$) \cmf{ do we have reference for this? So far this result is not published, maybe Boris paper?}.
% CFG: this is mentioned earlier in the paper and also implicit, but not actually calculated, in Boris's paper.  Let us not specify a correlation time here, so that we don't set the correlation time in multiple places.

An example of the distribution of the efficiency can be seen in the right half of  Fig. \ref{fig:varepsilon}. The efficiency quickly $\epsilon=0$ within the disk region while within the jet the efficiency reaches the defined base-efficiency. Thus (and as usual removing emission at $\sigma > \sigma_{\rm cut}=1$) the non-thermal particles are mainly located in jet sheath.

\begin{figure}[t!]
  \centering
  \includegraphics[width=\columnwidth]{./figures/GRMHDphiavera0.94sigmaeta.pdf}
  \caption{Time and azimuthal averaged distribution of the magnetization, $\sigma$ (left half) and the efficiency, $\epsilon(\varepsilon,\beta,\sigma)$ using $\varepsilon=0.2 $ for a \bhac MAD GRMHD simulation with $\abh=0.94$. The solid gray line corresponds to $\sigma=1$ and the dashed line indicates out-flowing plasma via the Bernoulli parameter ($-h u_{t}>1.02$).
  \ckc{Should we move this figure to the model section, as a way to demonstrate how the GRMHD and eDF look like?}}
  \label{fig:varepsilon}
\end{figure}

In the following we will elaborate on the impact of adding non-thermal particles via the kappa electron distribution with fixed kappa value ($\kappa=3.5$) and three different base efficiencies $\varepsilon=$0.05, 0.1 and 0.2 on the observational constraints listed in Section~\ref{subsec:thermal}.

%..............................................................................
\subsubsubsection{230\,GHz VLBI pre-image size}

The addition of non-thermal particles produces almost undetectable changes in the 230\,GHz VLBI pre-image sizes for the MAD models and a minor effect for the SANE models.

In the SANE case only models with $\Rh>40$ exhibit a minor increase in the source size where we see a monotonic increase of the source size with inclination. This effect can be understood if we consider that the bulk of the emission in all cases considered here is still produced by the thermal electron distribution, with temperature set by the $\Rh$ prescription.  An increase in $\Rh$ suppresses the emission from particles in the disk (by decreasing the electron temperature) and thus enhances emission from jet.  Since most the non-thermal particles are located in the sheath of the jet their impact on the source size is largest if the bulk of the thermal emission is also produced there.

% CFG: this is confusing.  is it needed?
%In addition the thermal synchrotron emissivity decreases at high frequency as $j_{\nu}\propto \exp{\left(-(\nu/\nu_c)^{1/3} \right)}$ while for the kappa distribution as $j_{\nu}\propto \nu^{-(\kappa -2)/2}$. This implies that for the same electron temperature the non-thermal flux is compared to a thermal one higher and thus leads to a more extended jet structure for the models including non-thermal particles.

In the MAD case, independent of the choice of $\Rh$ the emission is mostly produced in the disk region (see \citetalias{M87PaperV} and Fig.~8 in \citet{Wong_2022} for a 3-D rendering). Increasing $\Rh$ will not push the emission region into the jet where the non-thermal particles are located and thus their contribution to the total emission structure is negligible.

%..............................................................................
\subsubsubsection{86\,GHz flux}

The GMVA+ALMA observations at 86\,GHz \cite{2021ApJ...915...99I} \monika{we can skip reference here, there is only one 86 GHz we talk about in this paper} probe a larger field of view than the 230\,GHz EHT observations, so we increased the field of view for the 86\,GHz to 800\,$\rm{\mu as}$ during the radiative transfer calculations. Again the non-thermal particles are mainly located in the jet sheath and thus the increased field of view ensures that no extended flux is missing during the comparison with the 86\,GHz observations.

The 86\,GHz for both SANE and MAD models flux is not affected by the addition of non-thermal particles. In case of the SANE models the edges of the 86\,GHz flux distribution are slightly shifted in the case of $\Rh>40$. However, including non-thermal particles even with the highest base efficiency $\varepsilon=0.2$ does not change the scoring of a model, i.e., a thermal-only model which full-fills the 86\,GHz constrain is still accepted if non-thermal particles are included. This behaviour can be explained by the fact that the bulk of the emission in both accretion models is generated in with a few gravitational radii. Since the non-thermal particles are mainly located in the jet sheath and thei ratio between non-thermal to thermal particles is at most 20\% the contribution of the non-thermal particles to the 86\,GHz flux can be neglected.

%..............................................................................
\subsubsubsection{86\,GHz image size}

The behaviour of the 86\,GHz image size is very similar to the above described 230\,GHz image size: There is no change in image size for the MAD models and only a minor increase in the SANE models for $\Rh>40$. The physical reasons for this behaviour follows the same arguments as in the 230\,GHz VLBI pre-image size.

%..............................................................................
\subsubsubsection{NIR constraints}
As expected, the addition of non-thermal particles via the kappa electron distribution function with variable efficiency has a large influence on the NIR flux for all models independent of the accretion model and the $\Rh$ value. In Fig.~\ref{fig:NIR_kappaepsilon} we compare the distribution of the NIR fluxes for thermal and kappa eDF for a SANE accretion model.

\begin{figure*}
  \centering
  \includegraphics[width=0.8\textwidth]{./figures/SANE_NIR_standard.pdf}
  \caption{\monika{Impact of non-thermal electrons on NIR flux constraint.} NIR \sout{constraints}  \monika{fluxes are shown} for SANE accretion models \sout{for}\monika{with} thermal and non-thermal \sout{electron distribution function}\monika{eDF} with $\kappa=3.4$ \sout{fixed} and $\epsilon\left(\sigma,\beta,\varepsilon\right)$. The red violin plots correspond to a thermal eDF, blue, orange and green indicate kappa eDF with $\varepsilon=$0.05,0.1 and 0.2.\monika{add what is the horizontal dashed line? is that measurement?}}
  \label{fig:NIR_kappaepsilon}
\end{figure*}

As can be seen in the figure, including non-thermal particles changes the distribution of the NIR fluxes significantly. Except for the $\Rh=1$ models the addition of non-thermal particles even with the lowest base efficiency used in this analysis $\left( \varepsilon=0.05\right)$ leads to a over-production of NIR fluxes. In the case of the MAD models all models over-produce the NIR flux for $\varepsilon=0.05$.

The NIR fluxes are produced by particles in the tail of the distribution function and are therefore are sensitive to slope of the tail. The emissivity for a thermal distribution decreases exponentially $\left(j_{\nu}\propto\exp(-\nu^{1/3})\right)$ while for the kappa distribution it behaves like $j_{\nu}\propto \nu^{-(\kappa-2)/2}$. Thus even at low base efficiencies, $\varepsilon$, the the flux from the kappa distribution in the NIR $\left(\nu\sim 10^{14}\,\rm{Hz}\right)$ is larger than for the thermal eDF.

%..............................................................................
\subsubsubsection{ALMA Light curves}

For the comparison with the ALMA light curves we compute the modulation index for a 3\,hour time window and across the entire simulated time window of 28\,hours (5000\,GM/c$^3$). Similar to the 86\,GHz flux the 230\,GHz flux is mainly produced within a few gravitational radii and thus not affected by the addition of non-thermal particles using Eq. \ref{eq:kappaeff}. As in the previous constraints, the MAD accretion models are insensitive to the addition of non-thermal particles whereas the SANE models show some increased modulation index for $\Rh>40$. However, the distributions for thermal and non-thermal eDF are still largely overlapping.

%..............................................................................
\subsubsubsection{m-ring}

The m-ring fitting is applied to the 230\,GHz images. As mentioned earlier the flux and size of the 230\,GHz images are not affected by the inclusion of non-thermal particles with fixed kappa and variable efficiency. Thus, we do not expect any changes in the distribution of the m-ring parameters.  Applying m-ring fitting to non-thermal models and a detailed comparison with the thermal results confirmed our initial assumption. \cmf{based on the results of Kotaro, who run the m-ring on non-thermal models}

%------------------------------------------------------------------------------
\subsubsection{Variable kappa model $\kappa(\sigma,\beta)$}

%\cfg{could we avod the equation and simply provide a reference and enough detail that someone could reproduce it?}\aco{What about inline equations?}

An alternative procedure for assigning a nonthermal eDF uses the  kappa model \eqref{eq:nonthermaleDF}, and assigns the parameter $w$  and $\kappa$ parameters according to \cite{2018ApJ...862...80B}, who model the effects of particle acceleration in magnetic reconnection based on particle-in-cell simulations: $ \kappa=2.8 +0.7\sigma^{-1/2} + 3.7\sigma^{-0.19}\tanh{(23.4\sigma^{0.26} \beta)}\label{eq:kappa}$
and $w= \frac{ \kappa -3 }{\kappa} \Theta_{\rm e} + \frac{\varepsilon}{2}\left[1+\tanh(r-r_{\rm inj})\right]\, \frac{ \kappa -3 }{6 \kappa} \frac{m_{\rm p}}{m_{\rm e}} \sigma$.
Here $\sigma \equiv B^2/(8\pi\rho c^2)$, $\beta \equiv p_{gas} 8 \pi/B^2$, and $w$ contains a thermal and nonthermal contribution, with the nonthermal contribution defined by the magnetization and nonthermal particle acceleration efficiency $\varepsilon$  \citep{2019A&A...632A...2D, 2021NatAs.tmp..218C}. We set  $\varepsilon=0$ and injection radius $r_{\rm inj}=10r_{g}$ following previous studies.  Nonzero $\varepsilon$ increases the averaged NIR flux by two orders of magnitude as $\varepsilon$ goes from 0 to 1, causing all models to overshoot NIR constraint.  Setting $\varepsilon \ne 0$ also causes the 86GHz flux density to increase around two times as well as the image size \citep{2021arXiv211102518F}. We use the emissivity and absorptivity, computed numerically for the interval $3 < \kappa \le 8$, from  \cite{2016ApJ...822...34P}. For $\kappa > 8$ we substitute a thermal eDF.  We also turn off emission in the jet spine where $\sigma > 1$ (see Figure~\ref{fig:varepsilon}).



%..............................................................................
\subsubsubsection{86\,GHz image size}

The total flux and image size was computed using an extended field of view, $800 \mu as$ motivated in the 86GHz observations. We found that the image size is closely similar to the thermal and variable efficiency models, only different for non-spinning black holes, where we have low magnetized and small jet. As a consequence eDF with variable kappa power-law reduces to thermal distribution and the size of the image is a bit smaller than in thermal case. For $a_{\star}=0$ almost all SANE models pass the size constraint, $R_{\rm high}=1$ we have a bit larger image. For MAD $R_{\rm high}=40$ and $a_{\star}=0.94$, and cases for $R_{\rm high}=160$ and $a_{\star}=0.50$ we have a bit smaller size image than thermal case passing the constraint.

%..............................................................................
\subsubsubsection{86\,GHz flux}

The total flux on overall non-thermal models increases in comparison to thermal models, however, the general results for MAD models do not change. For SANE models with R$_{\rm high}=1$ slightly increases, all of them fail in contrast with thermal models. While for $R_{\rm high}=10,40,80$ and $R_{\rm high}=160$ the flux decreases under the lower bound $1.8 \rm {Jy}$.

%..............................................................................
\subsubsubsection{NIR constraints}

The contribution of non-thermal electrons at the NIR emission produce larger flux in all models, independent on black hole spin, inclination angle or electron temperature. For MAD accretion models with $a_{\star}=0$ and $R_{\rm high}=1$ only small inclination $i=10^{\circ}$ pass the constraint. While models with $R_{\rm high}=10,40$ overshooting GRAVITY mean flux, again for thermal case these models satisfy the constraint. On the other hand, since for non-spinning black hole the jet is no well developed and the magnetization is low, for only  $R_{\rm high}=80$, $a_{\star}=0.5$ and $30^{\circ} \leq i$ survive, similar behavior for $R_{\rm high}=160$ and spins $|a_{\star}|=0.5$.
Larger emission than thermal models excludes SANE models for $R_{\rm high}=10$ and $\abh < 0$ as well as  $R_{\rm high}=40,~80$ and $\abh \leq 0.0$. While even larger NIR flux the models for $R_{\rm high}=160$ are still under the threshold.

%..............................................................................
\subsubsubsection{ALMA Light curves}

We compute the the modulation index  of the light curves of variable kappa models by discretizing the entire window of GRMHD simulations, $\rm 5000\,\tg$ ($\rm 28\,hours$) every $\rm 3\,hour$ (see Table \ref{tab:GRMHDmodels}). Similar to variable efficiency models, the non-thermal particles with variable kappa has not big impact on the flux at 230GHz. The behavior of modulation index and total flux for MAD accretion models are same as thermal, and variable efficiency eDFs. We found sightly high modulation index for $R_{\rm high}=1$ and $30^{\circ} \leq i$ for SANE accretion models, the models for $R_{\rm high}=80$ has same trend as $R_{\rm high}=40$.


%..............................................................................
%\aco{First Draft ... please add missing references ... waiting for calculations with help from Michi, Ben, Ck and Kotaro}\\

\subsubsubsection{m-ring}

...

\subsubsubsection{Null Location}

...

\note{Razi, Angelo, Richard?}
\ckc{Write subsections only if they are different from the thermal models.}

%------------------------------------------------------------------------------
\subsubsection{Summary of Constraints on Non-thermal Models}

The effect of adding non-thermal electrons in a kappa eDF with  $\kappa=3.4$ and variable efficiency via Eq.~\ref{eq:efficiencybetasigma} can be summarised as follows:
\begin{itemize}
    \item The 86\,GHz and 230\,GHz constraints are hardly affected by the addition of non-thermal particles.
    \item Even the lowest base efficiency considered, $\varepsilon=0.05$, leads to an over-production of NIR flux. \cmf{indication that we need very localised regions of non-thermal particle productions and no "global" approach?}
\end{itemize}

\note{Tomohisa to write here about UWABAMI nonthermal power-law models}

\ckc{Write subsections only if they are different from the thermal models.}

%==============================================================================
\subsection{Tilted Models}

\note{Koushik to write here.  Describe the results and how they differ from aligned thermal results.}
\ckc{Write subsections only if they are different from the thermal models.}

In addition to the aligned models, we also consider misaligned, geometrically thick disks around black holes with a spin of $a_*=15/16$ from \citet{Liska2018} and \citet{Chatterjee2020}. While the other models in this paper produce either a SANE or MAD accretion flow, the tilted disk models assume a strongly magnetized non-MAD disk, otherwise known as an INSANE disk. Overall, we incorporate three GRMHD models of differing misalignment angles - $0^{\circ}$, $30^{\circ}$ and $60^{\circ}$. The models show a strongly warped disk and jets due to the Lense-Thirring torque imposed by the black hole frame dragging effect, and thus the disk/jet morphology is non-axisymmetric. Since the inner and the outer disk possess different orientations, it is necessary to specify the coordinate axis of the observer. We chose to align our observer to the outer disk such that at an inclination and azimuth of $90^{\circ}$ and $0^{\circ}$ respectively, we are observing the outer disk edge-on \citep[for more details, see][]{Chatterjee2020}. Note that the jets are always approximately perpendicular in orientation to the disk in these models.

The 230\,GHz pre-image size of edge-on large $R_{\rm high}$ images slightly increases for the tilt-$60^{\circ}$ as compared to the aligned case. This apparent increase in size occurs as the inner jet is warped and creates an extended image. This effect is also seen in the case of the 86\,GHz image size. On the other hand, the 86\,GHz flux constraint results are quite similar between the three tilt models despite the presence of a boosted jet component at large tilt angles. This suggests that, on average, most of the total flux at 230 and 86\,GHz comes from a few gravitational radii from the black hole where the relativistic Doppler boost have comparable values between the in-going accretion flow and the out-flowing jet. However, highly misaligned models are more variable that their aligned counterparts. In tilted disks, accretion occurs via thin plunging streams \citep[e.g.,][]{Fragile2007} where electrons in the shocked flow can be heated to relativistic temperatures \citep[e.g.][]{Dexter2013}, forming local hotspots more easily than in aligned disks, thereby increasing flux variability \citep{Chatterjee2020}.

The nIR flux exceeds the 1.0 mJy limit for all 3 models (except for a few cases, e.g. $R_{\rm high}=160$ models at $10^{\circ}$ inclination), which makes it difficult to favor the aligned case over the tilted one. Further, the presence of misalignment destroys the axisymmetric nature of the accretion flow. The current model set covers a small parameter space in inclination and $R_{\rm high}$ and a thorough exploration of the source azimuthal angle with respect to the observer is left to future studies. 

%==============================================================================
\subsection{Stellar Wind Fed Models}

The accretion models of \cite{2020ApJ...896L...6R, 2020MNRAS.492.3272R, 2018MNRAS.478.3544R} track plasma from  magnetized stellar winds down to the event horizon and provide a self-consistent picture of the origin of both gas and magnetic fields for the accreting plasma in \sgra.  The resulting inflow does not fully circularize, so the models also provide a distinct alternative to the standard models, which {\em assume} that the artificial SANE or MAD initial conditions relax to a astrophysically realizable state for the inner accretion flow.

We consider two versions of the model: one in which the stellar wind magnetization is low ($\beta = 10^6$) and another and a second in which the magnetization is higher ($\beta = 10^2$). $\Rh$ is adjusted until each model has the observed mean flux, with $\Rh = 13$ ($\beta = 10^6$) and $\Rh = 28$ ($\beta = 10^2$).

While both models pass many of the tests, they both fail the m-ring width test, producing a width that is too small.  In addition both fail the 86GHz flux test: the 86GHz flux is too large. Full results are given in Appendix~\ref{app:tables}, Table \ref{tab:resslerPF}.

Interestingly both non-EHT and EHT constraints have the power to test the wind-fed model.  We cannot draw a broad conclusion about the viability of wind-fed models, however, as the two models only consider a single spin and a single eDF model.

%==============================================================================
\subsection{Best Bet Models}

\begin{figure*}
  \centering
  \includegraphics[width=\textwidth]{figures/bestbet.pdf}
  \caption{Best bet models.  Each column corresponds to one best bet model; top row shows 86~GHz image, middle row 230~GHz and the bottom row shows the corresponding multiwavelength broadband SED.\monika{can we have here the golden SANE model with high spin?}} \label{fig:bestbet}
\end{figure*}

We now consider combined constraints, excluding variability, under the hypothesis that (1) there is a missing physical ingredient in the models that would lower variability, but (2) that missing ingredient would not vitiate the model comparison process entirely.  Models are eliminated if they fail any one constraint.

Figure showing combined constraints for thermal models.

Figure showing combined constraints for nonthermal models.

We then inspected the remaining models and selected four best-bet models that approximately span the space of successful models.  \note{Written characterization of remaining models; possibly 3 or 5}

\ckc{CK to work on sample plots, may add extra best bet models upon requests.}
