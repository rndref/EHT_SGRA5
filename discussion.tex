\section{Discussion}\label{sec:discussions}

First comprehensive analysis of Sgr A* including both resolved VLBI data and multiwavelength data.

Discussion organized according to physical parameters of the model.

\note{Avoid figures where possible.}

%==============================================================================
\subsection{MAD, SANE, and Self-Consistent Wind Feeding}

\note{Angelo to provide first draft}

There are clear differences between MAD and SANE models in VLBI data, as well as in the non-VLBI data.

Assessment of Ressler model.  Viable!

%==============================================================================
\subsection{Electron Distribution Function}

\note{Koushik to provide first draft}

Strong constraint on abundance of cold electrons from bremss.  A high density of cold electrons - which would be invisible in synchrotron - are ruled out.  This is in part because at $\Theta_e \equiv k T_e/(m_e c^2) \lesssim 1$, $j_\nu \propto \Theta_e^{-1/2}$ (emission in the x-ray band increases as temperature decreases).  In contrast, for $\Theta_e \gtrsim 1$ electron-electron bremsstrahlung becomes important and $j_\nu \propto \Theta_e^{+1/2}$.

Strong constraint on abundance of hot electrons from NIR.  In particular for

Strong constraint on $T_i/T_e$: models with ion temperature equal to electron temperature fail on several counts.

%==============================================================================
\subsection{Inclination}

\note{Michi to provide first draft}

Strong constraints on inclination from m-ring fitting.

%==============================================================================
\subsection{Position Angle}

Virtually no constraint on position angle [check m-ring fits]

%==============================================================================
\subsection{Black Hole Spin}

Still quite weak constraints on black hole spin.

%==============================================================================
\subsection{Accretion Rate and Outflow Power}

\note{Vedant to provide first draft of thermal section.}

We compute the outflow power in a fashion similar to that in \citet{M87PaperV},

\begin{equation}
    P_{out} = \int_{funnel}d\theta\frac{1}{\Delta t}\int dtd\phi\sqrt{-g}\big(-T^{r}_{t}-\rho u^{r}\big),
\end{equation}

evaluated at $r=100GM/c^{2}$, where $funnel\Rightarrow(\theta<1)\cup(\theta>\pi-1)$. We average the quantity in time $\Delta t$, where $\Delta t$ is the time interval we have considered for our analysis. We also consider only those regions where there is steady outflow, ie. the quantity in the parentheses is positive.

Figure showing accretion rate.

Accretion rate is consistent with earlier analyses.

Models at the highest accretion rate are ruled out by overproduction of x-ray emission.  If our models were in equilibrium over a larger range in radius, bremss from larger radius might increase the x-ray flux and rule out more models.

Figure showing outflow power.

Jet power is surprisingly large.  Where does the power come out in the galactic center?

Dependence on distribution function.

\note{Koushik, Alejandro, Razi}

%==============================================================================
\subsection{Caveats and Limitations}

...

%==============================================================================
\subsection{Future Constraints}

\monika{this section should be moved to the end of the paper}
\ckc{Agree}

integrated polarization,

resolved polarization

fits to more sophisticated models such as RIAF analytic models,

closure phase variability
