\section{Conclusions}\label{sec:conclusions}

\note{to be written last}
\ckc{ck's first pass}

We have carried out and presented the most extensive comparison
between numerical models and observations of \sgra to date.
By using multiple simulation pipelines (cite PATOKA, BHAC+BOSS, ...),
we have demostrated that, given a set of boundary conditions,
numerical simulations are remarkablely predictive in averaged image
properties.
This ability allows us to significantly reduce the uncertainty in the
plasma properties and infer the spacetime properties of \sgra.

EHT's VLBI data is most constraining on the ring size and width but
shows a supriringly low level of asymmetry in the image.
These constraints roughly reduce the dimension of the parameter by
two, carve out a region of the parameter space that prefers edge-on
retrogade or low spin SANE models, or face-on prograde models.
Together with multi-wavelength observations, we conclude that \sgra is
most likely a low spin prograde $\abh \sim 0.5$ MAD with mid
inclination $20\degree \lesssim i \lesssim 60\degree$.
For thermal models, \sgra is likely to have a low electron temperate
with $\Rh \gtrsim 40$.
\ckc{I think it will be useful to translate $\Rh$ to actual electron
  temperature.
  I think someone good at handling GRMHD needs to do it, maybe George?
  Hector?}
While there are also isolated pockets in the parameter space that pass
all constraints, however, even with this extensive study, it is
difficult to draw strong conclusion about these regions.
We expect the EHT's future results, especially on polarimetry, will
allow us to address these ambiguity and further narrow down \sgra's
parameter space.



Predictions:

- what would ngEHT see?

Future:

- polarization
