\section{Conclusions}\label{sec:conclusions}

\note{to be written last}
\ckc{ck's first pass}

We have carried out and presented the most extensive comparison
between numerical models and observations of \sgra to date.
By using multiple simulation pipelines (cite PATOKA, BHAC+BOSS, ...),
we have demostrated that, given a set of boundary conditions,
numerical simulations are remarkablely predictive in averaged image
properties.
EHT's VLBI data is most constraining on the ring size and width but
shows a supriringly low level of asymmetry in the image
(section~\ref{sec:mring}, see also \citealt{EHTpaperIII} and
\citealt{EHTpaperIV}), which prefers edge-on retrogade or low spin
SANE models, or face-on prograde models.
Together with multi-wavelength observations, especially the x-ray flux
limit, we conclude that \sgra is most likely a low spin prograde $\abh
\sim 0.5$ MAD with mid inclination $20\degree \lesssim i \lesssim
60\degree$ and low electron temperate $\Rh \gtrsim 40$.
There are also isolated pockets in the parameter space that also pass
constraints.
However, even with this extensive study, it is difficult to draw
strong conclusion about these regions.

Predictions:

- what would ngEHT see?

Future:

- polarization
