\section{Conclusions}
\label{sec:conclusions}

We have made a first comprehensive comparison of the EHT 2017 \sgra data to a state-of-the-art library of numerical models.  The models assume the mass and distance to \sgra are known and that the central object is a black hole described by the Kerr metric. We use multiple simulation pipelines and demonstrate that for a given set of initial and boundary conditions independent simulations are remarkably consistent.  

The model parameters are: horizon magnetic field is strong or weak (MAD or SANE, respectively); the black hole spin $\abh$; and the inclination angle $i$ between the line of sight and the black hole spin vector.  We also have one or more parameters for the electron distribution function (eDF).  In our ``standard'' model set, run with three independent codes, the eDF is determined using the so-called $\Rh$ prescription.  We have also considered non-standard models with alternate eDF prescriptions.   

We selected and applied 11 heterogeneous observational constraints.  Six derive directly from EHT data, two derive from 86GHz VLBI observations with the GMVA, one from variability of the 230 GHz light curve, and one each from the $2.2\mu$m flux density and the X-ray luminosity.  

Five structural constraints derive from EHT VLBI data.  When combined these constraints reject about 75\% of models.  In our ``standard'' model set the EHT cut favors $\abh \ge 0$ and avoids edge-on ($i = 90$deg) models and models with equal ion and electron temperatures ($\Rh = 1$).  EHT constraints are limited by sparse baseline coverage; new EHT imaging with more baselines will be far more constraining.  Nevertheless, the 2017  EHT-only observations are strongly constraining.

Four constraints derive from non-EHT data that is contemporaneous or near-contemporaneous.  Combined, the non-EHT constraints reject 88\% of the standard models.  The non-EHT cut favors strongly magnetized (MAD) models, eliminates all models with equal ion and electron temperatures, and eliminates most models at $i > 50$deg.  These results highlight the value of continued multiwavelength monitoring of \sgra.

Two constraints measure variability: one characterizes variability in the 230~GHz lightcurve (for example, from simultaneous ALMA data) and the other structural variability expressed through fluctuations in the visibility amplitudes.  Of these, the light curve variability is the tightest of all 11 constraints: it rejects $80\%$ of our standard models.  As a group, the models are more variable than the data.  We find that strongly magnetized (MAD) models are more variable than weakly magnetized (SANE) models.  The structural variability measures the slope and amplitude of the visibility amplitude power spectrum; the slope is consistent with the models, while the power spectrum amplitude is consistent for 66\% of standard models. 

No standard model survives the full gauntlet of 11 constraints.  What is remarkable is that some models come close to passing!  In our standard model set a cluster of strongly magnetized (MAD) models at intermediate inclination ($i = 30$deg), positive spin ($\abh = 0.94$), and large  ion-electron temperature ratios ($\Rh \ge 40$) survives 10/11 constraints.  We identify these as best-bet models and have discussed their common properties in Section \ref{sec:bestbets}.

In particular, the best-bet models have accretion rate ranging from $\dot{M} = 1.5 \times 10^{-7}$ to $3 \times 10^{-9} \msun \yr^{-1}$.  These accretion rates overlap with estimates of the accretion rate for wind-fed models, with are of order $10^{-8} \msun\yr^{-1}$.  

The models produce bipolar outflows, and we have measured the corresponding outflow power $\P_{out}$ (defined in Section \ref{sec:discussions}).  For the best-bet models $2 \times 10^{36} \le P_{out} \le 3 \times 10^{38} \erg \sec^{-1}$, corresponding to an outflow efficiency of $$ to $$.  

We also have full SEDs for many of the models in our model set, so we can measure the bolometric luminosity.  Typically this is dominated by the synchrotron bump in the submillimeter and for the best-bet models ranges from $$ to $$.  From this we can directly calcuate a bolometric efficiency that ranges from $$ to $$.

All the best-bet models have $\abh > 0$, so the accretion flow is prograde.   

The constraint that rejects the most models is the rms\% of the $230$GHz light curve on a 3 hour timescale, also known as the 3-hour modulation index $\mi{3}$.


between numerical models and observations of \sgra to date.
By using multiple simulation pipelines (cite \patoka, $\bhac+\bhoss$,
...), we have demonstrated that, given a set of initial and boundary
conditions, numerical simulations are remarkably predictive in the
averaged image properties.
This ability significantly narrows down the range of possible model
parameters and allows us to infer the spacetime properties of \sgra
\citepalias{PaperVI}.

EHT's VLBI data is most constraining on the ring size and width.
However, its surprisingly low-level of asymmetry makes the position
angle non-constraining.
The EHT constraints roughly reduce the dimension of the parameter
space by two, carving out a region of the parameter space that prefers
intermediate inclination retrograde or low spin prograde SANE models,
or face-on prograde models.
Together with non-EHT observations in other wavelengths, we conclude
that \sgra is most likely a low spin prograde $\abh \sim 0.5$ MAD with
intermediate inclination $20\degree \lesssim i \lesssim 60\degree$.
Using a thermal model, the accretion flow around \sgra is two
temperature, and likely to have ion-electron temperate $\Rh \gtrsim
40$.
\ckc{I think it will be useful to translate $\Rh$ to actual electron
  temperature.
  I think someone good at handling GRMHD needs to do it, maybe George?
  Hector?}
\gw{Not sure what is being looked for here. Actual electron
  temperatures in different parts of the flow? Characteristic electron
  temperatures?
  The relationship between $\Rh$ and $T_e$ is given in (what is now)
  Section 2.3.2+.}
\ckc{But I think we need a typical fluid temperature in order to do
  that conversion?
  So we need to pull out a typical fluid temperature of the flow from
  the GRMHD data.}
For the $\kappa$ model, ...
\ckc{Fill in what the likely parameters for the nonthermal models.}

% Future: polarization
% Predictions: what would ngEHT see?
While there are also isolated pockets in the parameter space that pass
many constraints, however, even with an extensive study like this, it
is difficult to draw strong conclusion about these isolated regions.
Do these models pass all constraint by ``luck'' and are simply noise?
Or do we not have fine enough samples in the parameter space to
identify the narrow region of the parameter space?
While further expanding the simulation library would reduce model
uncertainty and help addressing these questions, we expect using
future EHT results, especially polarimetry information and
observations with larger arrays and better $(u, v)$ coverage, will
reduce observation uncertainty and resolve these ambiguity.

Although we are successful in predicting average image properties, the
variability is more challenging.
Since \sgra is radiative inefficient, the radiation energy budget is
insignificant to the total energy budget.
While this simplify modeling because we can solve the radiative
transfer equation as postprocess, it also leaves very little control
by the conservation laws over the radiation.
In fact, from our large simulation library, it is clear that radiation
can be sensitive to the small details in the chaotic turbulence flow.
We also provide a few possible physical origins in
section~\ref{sec:variability}.
Dissipation mechanisms and their typical timescales that can directly
modulate variability, like turbulence, reconnection, and their
interplay are governed by non-ideal effects like viscosity,
resistivity, heat conduction, and pressure anisotropy that we
currently do not model.
To capture these effects, higher-resolution GRMHD simulations are
essential.
Typical dissipative time-scales, governing variability, may be very
different in collisionless plasma, than in the collisional fluid as
described by (ideal or non-ideal) GRMHD.
To probe dissipative timescales in the collisionless plasma, a kinetic
approach is required and GRMHD is not sufficient as a model.
However, these hypotheses can only be tested once models that
accurately predicted variability are available.

The EHT observations, combined with multiwavelength information, are
now able to provide quantitative measurements to challenging numerical
models.
\emph{We have arrived a new era of precision black hole astrophysics.}
On the theory side, clearly one of the next major breakthroughs will
come from more accurate modeling of the kinetic collisionless plasma
aspects, as it will provide the electron distribution functions from
first principles.
Non-ideal MHD is also a promising intermediate step to improve our
predictability of the variability.
