\section{Conclusions}
\label{sec:conclusions}

\note{to be written last}
\ckc{ck's first pass}

We have carried out and presented the most extensive comparison
between numerical models and observations of \sgra to date.
By using multiple simulation pipelines (cite \patoka, $\bhac+\bhoss$,
...), we have demonstrated that, given a set of initial and boundary
conditions, numerical simulations are remarkably predictive in the
averaged image properties.
This ability significantly narrows down the range of possible model
parameter and allows us to infer the spacetime properties of \sgra
\citepalias{PaperVI}.

EHT's VLBI data is most constraining on the ring size and width.
However, its surprisingly low-level of asymmetry makes the position
angle non-constraining.
Overview, the EHT constraints roughly reduce the dimension of the
parameter space by two, carving out a region of the parameter space
that prefers edge-on retrograde or low spin prograde SANE models, or
face-on prograde models.
Together with non-EHT observations in other wavelengths, we conclude
that \sgra is most likely a low spin prograde $\abh \sim 0.5$ MAD with
intermediate inclination $20\degree \lesssim i \lesssim 60\degree$.
Using a thermal model, the accretion flow around \sgra is two
temperature, and likely to have ion-electron temperate $\Rh \gtrsim
40$.
\ckc{I think it will be useful to translate $\Rh$ to actual electron
  temperature.
  I think someone good at handling GRMHD needs to do it, maybe George?
  Hector?}
\gw{Not sure what is being looked for here. Actual electron
  temperatures in different parts of the flow? Characteristic electron
  temperatures?
  The relationship between $\Rh$ and $T_e$ is given in (what is now)
  Section 2.3.2+.}
\ckc{But I think we need a typical fluid temperature in order to do
  that conversion?
  So we need to pull out a typical fluid temperature of the flow from
  the GRMHD data.}
For the $\kappa$ model, ...
\ckc{Fill in what the likely parameters for the nonthermal models.}

% Future: polarization
% Predictions: what would ngEHT see?
While there are also isolated pockets in the parameter space that pass
all constraints, however, even with this extensive study, it is
difficult to draw strong conclusion about these isolated regions.
Do these models pass all constraint by ``luck'' and are simply noise?
Or do we not have fine enough samples in the parameter space to
identify the narrow region of the parameter space?
While further expanding the simulation library would reduce model
uncertainty and help addressing these questions, we expect use the
EHT's future results, especially polarimetry information and
observations with larger arrays and better $uv$ coverage, will reduce
observation uncertainty and resolve these ambiguity.

Although we are successful in predicting average image properties, the
variability is more challenging.
Since \sgra is radiative inefficient, the radiation energy budget is
insignificant to the total energy budget.
This leaves very little control by the conservation laws over the
radiation.
In fact, from our large simulation library, it is clear that radiation
can be sensitive to the small details in the chaotic turbulence flow.
We also provide a few possible physical origins in
section~\ref{sec:variability}.
Dissipation mechanisms and their typical timescales that can directly
modulate variability, like turbulence, reconnection, and their
interplay are governed by non-ideal effects like viscosity,
resistivity, heat conduction, and pressure anisotropy that we
currently do not model.
To capture these effects, higher-resolution GRMHD simulations are
essential.
Typical dissipative time-scales, governing variability, may be very
different in collisionless plasma, than in the collisional fluid as
described by (ideal or non-ideal) GRMHD.
To probe dissipative timescales in the collisionless plasma, a kinetic
approach is required and GRMHD is not sufficient as a model.
However, these hypotheses can only be tested once models that
accurately predicted variability are available.

The EHT observations, combined with multiwavelength information, are
now able to provide quantitative measurements to challenging numerical
models.
\emph{We have arrived a new era of precision black hole astrophysics.}
On the theory side, clearly one of the next major breakthroughs will
come from more accurate modeling of the kinetic collisionless plasma
aspects, as it will provide the electron distribution functions from
first principles.
Non-ideal MHD is also a promising intermediate step to improve our
predictability of the variability.
