\section{Conclusions}
\label{sec:conclusions}

We have made a first comparison of the EHT 2017 \sgra data to a state-of-the-art library of ideal general relativistic magnetohydrodynamics (GRMHD) models.  The models assume that the mass and distance to \sgra are known and that the central object is a black hole described by the Kerr metric. We use multiple simulation pipelines and demonstrate that for a given set of initial and boundary conditions independent simulations are remarkably consistent.  

The model parameters are: horizon magnetic field is strong or weak (MAD or SANE, respectively); the black hole spin $\abh$; and the inclination angle $i$ between the line of sight and the black hole spin vector.  We also have one or more parameters for the electron distribution function (eDF).  In our ``standard'' model set, run with three independent codes, the eDF is determined using the so-called $\Rh$ prescription.  We have also considered non-standard models with alternate eDF prescriptions.   

We selected and applied 11 heterogeneous observational constraints.  Six derive directly from EHT data, two derive from 86GHz VLBI observations with the GMVA, one from variability of the 230 GHz light curve, and one each from the $2.2\mu$m flux density and the X-ray luminosity.  

Five structural constraints derive from EHT VLBI data.  When combined these constraints reject about 75\% of models.  In our ``standard'' model set the EHT cut favors $\abh \ge 0$ and avoids edge-on ($i = 90$deg) models and models with equal ion and electron temperatures ($\Rh = 1$).  We are {\em not} able to constrain the source position angle due to sparse baseline coverage.  The 2017 EHT observations are, nevertheless, strongly constraining and we expect new EHT observations with more baselines to be even more so.   

Four constraints derive from non-EHT data that is contemporaneous or near-contemporaneous.  Combined, the non-EHT constraints reject 88\% of the standard models.  The non-EHT cut favors strongly magnetized (MAD) models, eliminates all models with equal ion and electron temperatures, and eliminates most models at $i > 50$deg.  These results highlight the value of continued multiwavelength monitoring of \sgra.

\sgra is variable.  We have used two tests to compare the variability of models and data. One characterizes variability in the 230~GHz lightcurve (including simultaneous ALMA data) and the other structural variability expressed through fluctuations in the visibility amplitudes.  The light curve variability is the tightest of all 11 constraints: it rejects $80\%$ of our standard models.  We find that strongly magnetized (MAD) models are more variable than weakly magnetized (SANE) models.  As a group, the models are more variable than the data.  The structural variability measures the slope and amplitude of the visibility amplitude power spectrum.  Remarkably, we find that the power spectrum slope is consistent for all models, while the power spectrum amplitude is consistent for 66\% of standard models.  The failure of nearly all standard models to match the light curve variability is interesting.  It may signal the presence of extended, slowly varying structure that is resolved out by EHT, or it may signal that future models need to incorporate collisionless effects or a more sophisticated treatment of electron thermodynamics.   

None of the standard models survive the full gauntlet of 11 constraints.  What is remarkable is that some models come close to passing!  In our standard model set a cluster of strongly magnetized (MAD) models at intermediate inclination ($i = 30$deg), positive spin ($\abh = 0.94$), and large ion-electron temperature ratios ($\Rh \ge 40$) survives 10/11 constraints, failing only the light curve variability test.  We identify these as best-bet models and have discussed their common properties in Section \ref{sec:bestbets}.

The best-bet models have accretion rate ranging from $\dot{M} = 1.5 \times 10^{-7}$ to $3 \times 10^{-9} \msun \yr^{-1}$.  These accretion rates are consistent with earlier estimates and overlap with accretion rates in wind-fed models, $\sim 10^{-8} \msun\yr^{-1}$ \citep{2020ApJ...896L...6R}.  

We produced synthetic SEDs, and therefore bolometric luminosities $L_{bol}$, for all  standard models.  Typically $L_{bol}$ is dominated by a synchrotron bump in the submillimeter and for the best-bet models ranges from $$ to $$; the radiative  efficiency $L_{bol}/(\dot{M} c^2)$ ranges from $$ to $$.  The maximum radiative efficiency over the entire standard model set is $0.08$ (for a MAD, $\abh = 0.94$, $\Rh = 1$ model), consistent with our neglect of radiative cooling in the GRMHD evolution.  

All our models produce bipolar outflows, and for each we measured the outflow power $P_{out}$, defined in Section \ref{sec:discussions}. Consistent with earlier work we find that outflow power is higher for strongly magnetized (MAD) models than for comparable weakly magnetized (SANE) models, and increases by more than an order of magnitude from $\abh = 0$ to $|\abh| = 0.94$.  For the best-bet models $2 \times 10^{36} \le P_{out} \le 3 \times 10^{38} \erg \sec^{-1}$, corresponding to an outflow efficiency $P_{out}/(\dot{M} c^2)$ of $$ to $$.  $P_{out}$ is surprisingly large, with $P_{out}/L_{bol}$ ranging from $$ to $$.  It is unclear how outflows might interact with incoming gas in a self-consistent accretion model that follows plasma over a larger range in radius than our standard models.  It is also an open question whether the outflow power could be detected in the dense but crowded galactic center environment.

Our standard models assume a particular parameterization for the electron distribution function (the $\Rh$ prescription), use a common initial setup (a magnetized torus), and assume the black hole spin vector and torus angular momentum are aligned or anti-aligned.  To partially control for the errors introduced by these assumptions we have explored a set of nonstandard models.  The nonstandard models include several eDF prescriptions, a wind-fed model that tracks accretion from stellar winds down to the scale of the horizon, and tilted disk models in which the black hole spin and torus angular momentum are misaligned.

The nonthermal models differ remarkably little from their thermal counterparts.  For the limited set of nonthermal eDF prescriptions we consider here the 230 GHz image structure differs very little.  The 230 GHz variability is not detectably different than corresponding thermal models.\footnote{The nonthermal models are imaged over $5\times 10^3\tg$, so constraints on $\mi{3}$ are weaker than for the standard models, which are imaged for 3 times as long.} The 86GHz size and flux density, which are the most restrictive non-EHT constraints, are not detectably affected by the addition of nonthermal electrons.  Nonthermal electrons consistently increase the $2.2\mu$m flux density over similar thermal models, however.  Accelerating even a small fraction of the electron population into a nonthermal tail risks overproducing $2.2\mu$m emission.  The $2.2\mu$m (and submm through mid-IR) flux density therefore provides the strongest eDF constraints.  Future EHT analyses would benefit from incorporating submillimeter constraints \citep[e.g.]{2019ApJ...881L...2B} and, because model submillimeter SEDs are highly variable, the submillimeter and $2.2\mu$m observations should be as close to simultaneous as possible.   

The stellar wind fed models of \cite{2020ApJ...896L...6R} feature the best-motivated treatment of boundary and initial conditions for \sgra models.  They are distinct from our torus-initialized standard models in that they follow plasma from its ejection from stars on known orbits down to the event horizon.  We have imaged these models using an $\Rh$ prescription for the electron temperature, with $\Rh$ adjusted in the otherwise parameter-free models to produce the correct time-averaged 230 GHz flux density.  The two models considered here, both with $\abh = 0$, fail the $86$ GHz flux, m-ring width, and $\mi{3}$ constraints.  This does {\em not} imply that the wind fed models are ruled out; they clearly merit further investigation with longer integrations over a broader range of eDFs and $\abh$.  

In general black hole accretion flows are tilted in the sense that the orbital angular momentum of the disk and the spin angular momentum of the hole are misaligned.  Tilted disks have not been included in EHT analyses because (1) it is conceivable that accretion flows align either by consistently oriented long term accretion or by some analog of the Bardeen-Petterson effect \citep{1975ApJ...195L..65B}, and (2) the tilted disk parameter space is larger than the aligned disk parameter space by two dimensions: the tilt angle and the longitude of the observer.  We considered models with tilt $30$deg and $60$deg, observed at a single longitude.  The integrations were too short ($3000 \tg$) to provide strong constraints on tilt, but we find that m-ring width test is particularly sensitive to tilt and rejects a progressively larger fraction of the models as tilt increases.  Tilted models clearly merit further investigation.  

Our standard models and variable kappa nonthermal models have been run with independent GRMHD codes and imaged with independent radiative transfer codes.  The outcomes are largely consistent (see Appendix \ref{app:numerical} for details).  The code comparisons were valuable and helped us identify, for example, the importance of a large field of view in 86~GHz imaging.  The consistency between codes is remarkable given the complexity of the modeling process and the scope for error.  Tracking down the remaining discrepancies (for example, in the $2.2\mu$m flux density) and developing a quantitative error budget is an essential but difficult task for the future.   

Our numerical models predict the time-, space-, and polarization-dependent radiative properties of \sgra over 10 decades in frequency.  

Future: polarization.

Future: rotation

Future: better baseline coverage. 

Future: denser coverage of parameter space.  

Given the remarkable successes and interesting failures in the models, this work signals the arrival of a new era of precision black hole astrophysics in which the interaction of  theoretical models with EHT and multiwavelength data have the potential to unveil the deepest secrets of the galactic center.  

%EHT's VLBI data is most constraining on the ring size and width.
%However, its surprisingly low-level of asymmetry makes the position
%angle non-constraining.
%The EHT constraints roughly reduce the dimension of the parameter
%space by two, carving out a region of the parameter space that prefers
%intermediate inclination retrograde or low spin prograde SANE models,
%or face-on prograde models.
%Together with non-EHT observations in other wavelengths, we conclude
%that \sgra is most likely a low spin prograde $\abh \sim 0.5$ MAD with
%intermediate inclination $20\degree \lesssim i \lesssim 60\degree$.
%Using a thermal model, the accretion flow around \sgra is two
%temperature, and likely to have ion-electron temperate $\Rh \gtrsim
%40$.
%\ckc{I think it will be useful to translate $\Rh$ to actual electron
%  temperature.
%  I think someone good at handling GRMHD needs to do it, maybe George?
%  Hector?}
%\gw{Not sure what is being looked for here. Actual electron
%  temperatures in different parts of the flow? Characteristic electron
%  temperatures?
%  The relationship between $\Rh$ and $T_e$ is given in (what is now)
%  Section 2.3.2+.}
%\ckc{But I think we need a typical fluid temperature in order to do
%  that conversion?
%  So we need to pull out a typical fluid temperature of the flow from
%  the GRMHD data.}
%For the $\kappa$ model, ...
%\ckc{Fill in what the likely parameters for the nonthermal models.}

% Future: polarization
% Predictions: what would ngEHT see?
%While there are also isolated pockets in the parameter space that pass
%many constraints, however, even with an extensive study like this, it
%is difficult to draw strong conclusion about these isolated regions.
%Do these models pass all constraint by ``luck'' and are simply noise?
%Or do we not have fine enough samples in the parameter space to
%identify the narrow region of the parameter space?
%While further expanding the simulation library would reduce model
%uncertainty and help addressing these questions, we expect using
%future EHT results, especially polarimetry information and
%observations with larger arrays and better $(u, v)$ coverage, will
%%reduce observation uncertainty and resolve these ambiguity.

%Although we are successful in predicting average image properties, the
%variability is more challenging.
%Since \sgra is radiative inefficient, the radiation energy budget is
%insignificant to the total energy budget.
%While this simplify modeling because we can solve the radiative
%transfer equation as postprocess, it also leaves very little control
%by the conservation laws over the radiation.
%In fact, from our large simulation library, it is clear that radiation
%can be sensitive to the small details in the chaotic turbulence flow.
%We also provide a few possible physical origins in
%section~\ref{sec:variability}.
%Dissipation mechanisms and their typical timescales that can directly
%modulate variability, like turbulence, reconnection, and their
%interplay are governed by non-ideal effects like viscosity,
%resistivity, heat conduction, and pressure anisotropy that we
%currently do not model.
%To capture these effects, higher-resolution GRMHD simulations are
%essential.
%Typical dissipative time-scales, governing variability, may be very
%different in collisionless plasma, than in the collisional fluid as
%described by (ideal or non-ideal) GRMHD.
%To probe dissipative timescales in the collisionless plasma, a kinetic
%approach is required and GRMHD is not sufficient as a model.
%However, these hypotheses can only be tested once models that
%accurately predicted variability are available.

%The EHT observations, combined with multiwavelength information, are
%now able to provide quantitative measurements to challenging numerical
%models.
%\emph{We have arrived a new era of precision black hole astrophysics.}
%On the theory side, clearly one of the next major breakthroughs will
%come from more accurate modeling of the kinetic collisionless plasma
%aspects, as it will provide the electron distribution functions from
%first principles.
%Non-ideal MHD is also a promising intermediate step to improve our
%predictability of the variability.
