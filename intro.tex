\section{Introduction}\label{sec:intro}

\color{red}
[{\bf Monika's first pass, nobody touches this now!!}]
\color{green}
[Reference list is incomplete and will be completed soon]
\color{black}

Long term monitoring of the central region of the Milky Way strongly suggest the presence of a massive and compact object in it's very center, most likely a supermassive black hole \citep{2019A&A...625L..10G,2019Sci...365..664D}. The supermassvie object is surrounded by an accretion flow visible across 17 decades in frequency of electromagnetic spectrum. Hereafter we will use Sgr~A* to refer to the supermassive black hole candidate, accretion flow and it's emission.

Sgr~A* is one of the most frequently observationally and theoretically studied systems.  We refer reader to \citetalias{PaperII} and \citetalias{PaperIII} of this series for a comprehensive overview of the historical and more recent observations of the source. The main characteristics of Sgr~A* system is that it has overall extremely low luminosity with respect to it's Eddington limit (ref). This suggest that matter falls onto Sgr~A* central object in form of a radiativelly inefficient/advection dominated accretion flow (RIAF/ADAF, \citealt{1977ApJ...214..840I,1994ApJ...428L..13N, 1995ApJ...444..231N,
  1995ApJ...452..710N, 1996A&AS..120C.287N, 1998ApJ...492..554N}) rather than in form of a thin disk \citep{1973A&A....24..337S}. Models of RIAF/ADAF have been realized using semi-analytic prescriptions (e.g ref ,Broderick et al. 2016) and using the time-dependent numerical General Relativistic Magnetohydrodynamics (GRMHD) simulations \citep[e.g.,][]{2000ApJ...528..462H, 2003ApJ...589..458D,
  2003ApJ...589..444G, 2007CQGra..24S.235G, 2012ApJS..201....9F,
  2014ApJ...796...22F, 2016ApJS..225...22W, 2017ApJS..231...17A,
  2018JPhCS1031a2008O, 2019A&A...629A..61O, 2019ApJS..243...26P}, with the difference of outflows and
relativistic jets being naturally produced in the latter models. In fact, it has been suggested in the past that majority of the Sgr~A* emission could be produced by a jet launched by an accreting black hole rather then only matter falling onto the black hole event horizon (\citealt{2000A&A...362..113F,2004A&A...414..895F, 2005ApJ...635.1203M, 2013A&A...559L...3M}). While jet scenario is further supported by time-lags observed between emission measured at multiple radio frequencies \citep{2015A&A...576A..41B,2021arXiv210713402B}, the major difficulty in attempts to determine the exact nature of Sgr~A* emission is caused by an interstellar scattering screen that distorts
our view of the Galactic Center up to about millimeter wavelengths \citep{2016ApJ...824...40O,2019A&A...621A.119B}. The daily flaring activity of Sgr~A*, observed at higher energies in near infra-red and X-rays \citep{2009ApJ...698..676D,2019ApJ...886...96H}, suggest that the conditions around the supermassive black hole occasionally favor electron acceleration however these observations do not discriminate between exact acceleration scenarios because they do not have sufficient resolution \citep{2000ApJ...541..234O}. Hence the existing models for Sgr~A* emission remain weakly constrained \citep[e.g.,][]{2005ApJ...621..785G,2006MNRAS.370..219M,
  2007A&A...474....1M, 2007MNRAS.379.1519M, 2009A&A...508L..13M,
  2009ApJ...701..521C, 2009ApJ...706..497M, 2012ApJ...746L..10D,
  2012MNRAS.421.1315Z, 2013A&A...559L...3M, 2014A&A...570A...7M,
  2014ApJ...790....1B, 2015ApJ...799....1C, 2015ApJ...802...69B,
  2015ApJ...812..103C, 2015Sci...350.1242J, 2016A&A...588A..57F,
  2016ApJ...826...77B, 2016ApJ...831....4P, 2016MNRAS.455.2187M,
  2017ApJ...837..180G, 2017ApJ...844...35M, 2017ApJ...851..148M,
  2017MNRAS.467.3604R, 2018A&A...612A..34D, 2018ApJ...856..163M,
  2018ApJ...859...60L, 2018ApJ...863..148P, 2018ApJ...865..104J,
  2018ApJ...868..101B, 2018JCAP...07..015H, 2018MNRAS.478.1875J,
  2018MNRAS.478.5209C, 2019ApJ...881L...2B, 2019ApJ...884..148B,
  2020ApJ...896L...6R, 2020ApJ...897...99T, 2020MNRAS.492.3272R,
  2020MNRAS.493.1404A, 2020MNRAS.494.4168D, 2020MNRAS.494.5923P,
  2020MNRAS.497.4999D, 2020ApJ...896L...6R, 2021ApJ...917....8B,
  2021MNRAS.502.2023P}.\monika{these references will be moved around, not all match here}

Event Horizon Telescope (EHT) is a Very Long Baseline Interferometric (VLBI) experiment operating at $\lambda=1.3$mm waves with a resolution power sufficient to directly image the Sgr~A* structures on event horizon scales \citep{x}.

In April 2017, among other objects (including the core of the M87 galaxy, \citetalias{M87PaperIV}) EHT has observed Sgr~A* which resulted in the first ever horizon scale images of Sgr~A* at wavelength where the interstellar scattering effects become subdominant. We report the results of these observations in \citetalias{PaperIII} and characterise the basic properties of the 1.3mm emission visible in the EHT images in \citetalias{PaperIV}. The main goal of the present paper \citepalias{PaperV} is to provide the first theoretical interpretation of the EHT~2017 Sgr~A* dataset.

The paper is structured as follows. In Section~\ref{sec:observations} we gather standard observation data set that are used in the present work to test theoretical models. These data sets are comprised of a subset of EHT~2017 observations and other non-EHT historical or other data. All future theoretical studies can use this information to test their theories. In Section~\ref{sec:models} we describe one-zone source models and a standard simulation library used to model near horizon emission from Sgr~A* system. Our library of theoretical models assumes that general relativity is valid (the full discussion of Sgr~A* EHT~2017 observations in context of alternative theories of gravity can be found in accompanying paper, \citetalias{PaperVI}).
%e.g., \citealt{2010ApJ...718..446J, 2014ApJ...784....7B,
%  2015ApJ...802...63B, 2015ApJ...814..115P, 2016ApJ...818..121P,
%  2016PhRvL.117i1101J, 2019GReGr..51..137P}).
Our library is based on time-varying GRMHD simulations combined with general relativistic light transfer models that result in images and broadband spectra of the models. The library's simulated images have been used in \citetalias{PaperIII} and \citetalias{PaperIV} to validate the Sgr~A* EHT imaging and parameter estimation algorithms.
In Section~\ref{sec:comparisons} we describe model scoring procedures and use a large number of the source models to infer physical properties of Sgr~A* system. We discuss the model limitations, results in context of previous studies and outlook into future Sgr~A* theoretical research directions in Section~\ref{sec:discussions}. We conclude in Section~\ref{sec:conclusions}.
The manuscript is supplemented with two appendices: in Appendix~\ref{app:tables} we summarise theoretical models used in this work in form of a table; in Appendix~\ref{app:numerical} we discuss important numerical details of our simulations.



% The scientific goals and legacies of this paper may include:
% \begin{itemize}
% \item Description of a standard simulation library that is used in EHT's
%   first \sgra papers (the ``GRMHD'' test set for imaging and calibration sets for
%   MCFE and gravity) and subsequent papers (e.g., future papers that dive in the origin of
%   variabilities).
% \item Providing the first estimate on uncertainties in theoretical
%   modelings by comparing models from different realizations/groups in
%   similar parameter space.
% \item Using the large number of models to extract theory-backed and
%   statistical significant trends.
%   E.g., edge-on images are more asymmetric.
% \item Gathering a standard observation data set that \emph{all future}
%   theoretical studies can use to test their models.
% \item Providing the first theoretical/numerical interpretation of the
%   EHT \sgra VLBI observation, with helps from historical data.
%   This should include ``scores'' of the standard model sets and a few
%   best bet models.
% \item Summary of what models do not work and explain why.
% \item Summary of what models work and their implications and
%   limitations.
% \item Outlook into future theoretical research directions, i.e.,
%   importance of variability and viscosity, uncertainty in plasma
%   models, etc.
% \end{itemize}
