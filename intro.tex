\section{Introduction}\label{sec:intro}

\note{Monika's first pass, under re-revision}

Long term monitoring of the central region of the Milky Way strongly suggests the presence of a massive and compact object in it's very center, most likely a supermassive black hole \citep{2019Sci...365..664D,2019A&A...625L..10G}. The supermassvie object is surrounded by an accretion flow visible across 17 decades in frequency of electromagnetic spectrum. Hereafter we will use \sgra to refer to the supermassive black hole candidate, accretion flow and it's emission.

Sgr~A* is one of the most frequently observationally and theoretically studied systems.  We refer reader to \citetalias{PaperII} and \citetalias{PaperIII} of this series for a comprehensive overview of the historical and more recent observations of the source. The main characteristics of \sgra system is it's overall extremely low luminosity with respect to the Eddington limit. This suggests that matter falls onto \sgra central object in form of a radiativelly inefficient/advection dominated accretion flow (RIAF/ADAF, \citealt{1977ApJ...214..840I,1994ApJ...428L..13N, 1995ApJ...444..231N,
  1995ApJ...452..710N, 1996A&AS..120C.287N, 1998ApJ...492..554N,2014ARA&A..52..529Y}) rather than in form of a thin disk \citep{1973A&A....24..337S}. Models of RIAF/ADAF have been realized using semi-analytic prescriptions \citep[e.g.,][]{1995Natur.374..623N,2000ApJ...541..234O, 2009ApJ...697...45B,2011ApJ...735..110B,2018ApJ...863..148P} and using the time-dependent numerical General Relativistic Magnetohydrodynamics (GRMHD) simulations \citep[e.g.,][]{2000ApJ...528..462H, 2003ApJ...589..458D,
  2003ApJ...589..444G, 2007CQGra..24S.235G, 2012ApJS..201....9F,
  2014ApJ...796...22F, 2016ApJS..225...22W, 2017ApJS..231...17A,
  2018JPhCS1031a2008O, 2019A&A...629A..61O, 2019ApJS..243...26P}, with the difference of outflows and
relativistic jets being naturally produced in the latter. Since nearly-flat spectrum emitted by \sgra in radio is alike radio spectra observed in jets from Active Galactic Nuclei, it has been suggested that majority of the \sgra emission could be in fact produced by a jet launched by an accreting black hole rather then only matter falling onto the black hole event horizon (\citealt{2000A&A...362..113F,2004A&A...414..895F, 2005ApJ...635.1203M, 2013A&A...559L...3M}).  While the jet scenario in \sgra is further supported by time-lags observed between emission measured at multiple radio frequencies \citep{2015A&A...576A..41B,2021arXiv210713402B}, the major difficulty in attempts to determine the exact nature of \sgra radio emission is caused by an interstellar scattering screen that distorts
our view of the Galactic Center up to $\lambda \sim 1$\,mm wavelengths \citep{2016ApJ...824...40O, 2017MNRAS.471.3563D, 2018ApJ...865..104J,2019A&A...621A.119B}. The daily flaring activity of Sgr~A*, observed at higher energies in near infra-red and X-rays \citep{2009ApJ...698..676D,2019ApJ...886...96H}, suggests that the conditions around the supermassive black hole occasionally favor electron acceleration \citep{2000ApJ...541..234O,2020MNRAS.494.5923P} \br{here we are missing many citations, at least to Ponti+2019 and a few papers by Dodds-Eden, but likely more, including Ripperda+2020 and Ripperda+2021}\monika{we dont want to overload this paper with observational papers since this is all better described in paper II and III, will cite theoretical work, no problem} however these observations also do not discriminate between source models because they do not have sufficient resolution. Hence the existing models for \sgra emission remain weakly constrained \citep[see e.g.,][]{2005ApJ...621..785G,2006MNRAS.370..219M,
  2007A&A...474....1M, 2007MNRAS.379.1519M,2007ApJ...671.1696S, 2009A&A...508L..13M,
  2009ApJ...701..521C, 2009ApJ...706..497M, 2012ApJ...746L..10D,
  2012MNRAS.421.1315Z, 2012ApJ...755..133S, 2013A&A...559L...3M, 2014A&A...570A...7M,  2015ApJ...812..103C,
  2015ApJ...799....1C, 
  2016A&A...588A..57F,
  2016ApJ...826...77B, 2016ApJ...831....4P, 2016MNRAS.455.2187M,
  2017ApJ...837..180G, 2017ApJ...844...35M, 2017ApJ...851..148M,
  2017MNRAS.467.3604R, 2018A&A...612A..34D, 2018ApJ...856..163M,
 2018ApJ...863..148P,
 2018JCAP...07..015H, 2018MNRAS.478.1875J,
  2018MNRAS.478.5209C,  2019ApJ...884..148B,
  2020ApJ...896L...6R, 2020ApJ...897...99T, 2020MNRAS.492.3272R,
  2020MNRAS.493.1404A, 2020MNRAS.494.4168D, 
  2020MNRAS.497.4999D, 2020ApJ...896L...6R, 2021ApJ...917....8B,
  2021MNRAS.502.2023P,2021arXiv210105327E}.
\br{also here we are missing Chatterjee+2021; Scepi+2021; Ripperda+2020; Ripperda+2021 and also Nathanail+2020 and Nathanail+2021 and Chashkina+2021}
%alma light curves papers \citep{2014ApJ...790....1B,2015ApJ...802...69B,  2018ApJ...868..101B,2019ApJ...881L...2B}

Event Horizon Telescope (EHT) is a Very Long Baseline Interferometric (VLBI) experiment 
operating at $\lambda1.3$\,mm (or frequency $\nu=230$\,GHz)
with a resolution power sufficient to directly image the intrinsic \sgra structures in the immediate vicinity of the black hole event horizon (the first proto-EHT, non-imaging detections of \sgra are reported by \citealt{2008Natur.455...78D, 2015Sci...350.1242J} and \citealt{2018ApJ...859...60L}). In April 2017, among other objects (including the core of the M87 galaxy, \citetalias{M87PaperI}) EHT has observed \sgra which resulted in the first ever horizon scale images of the source. We report the results of these observations in \citetalias{PaperII} and \citetalias{PaperIII} and we characterise the basic properties of the emission visible in the EHT images in \citetalias{PaperIV}. The main goal of the present paper \citepalias{PaperV} is to provide the first comprehensive physical interpretation of the EHT~2017 \sgra data sets. 

The paper is structured as follows. In Section~\ref{sec:observations} we gather standard observation data sets that are used in the present work to test theoretical models of \sgra. These data sets are comprised of a subset of EHT~2017 observations and other non-EHT historical or other data. All future theoretical studies can use this information to test their theories. In Section~\ref{sec:models} we describe one-zone source models and a standard simulation library used to model near horizon emission from \sgra system. Our library of theoretical models assumes that general relativity is valid and the spacetime around Sgr~A* is described by Kerr metric \citep{1963PhRvL..11..237K}. The full discussion of \sgra observations in context of alternative theories of gravity can be found in accompanying paper, \citetalias{PaperVI}.
Our model library is based on time-varying GRMHD simulations combined with general relativistic light transfer models that result in images and broadband spectra of the models. The library's simulated images have been used in \citetalias{PaperIII} and \citetalias{PaperIV} to validate the \sgra EHT imaging and parameter estimation algorithms.
In Section~\ref{sec:comparisons} we describe model scoring procedures and use a large number of the source models to infer physical properties of \sgra system. We discuss the model limitations, results in context of previous studies and outlook into future \sgra theoretical research directions in Section~\ref{sec:discussions}. We conclude in Section~\ref{sec:conclusions}.
The manuscript is supplemented with several appendices: in Appendix~\ref{app:tables} we summarise theoretical models used in this work in form of a table; in Appendix~\ref{app:numerical} we discuss important numerical details of our simulations.\monika{add more here once all appendices are in place.}


