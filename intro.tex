\section{Introduction}
\label{sec:intro}

The center of the Milky Way contains a massive, compact object that is likely to be a supermassive black hole \citep{2019Sci...365..664D, 2019A&A...625L..10G}.
The putative black hole is surrounded by hot plasma that is visible across 17 decades in electromagnetic frequency.
Hereafter, we will use \sgra to refer to the supermassive black hole candidate and the hot plasma.

\sgra is one of the most commonly studied objects on the sky, both observationally and theoretically.  A key  characteristic of the \sgra system is its extremely low overall luminosity with respect to the Eddington limit.
The low luminosity suggests that matter falls onto \sgra's central object in the form of a radiatively inefficient/advection dominated accretion flow (RIAF/ADAF, as proposed by \citealt{1977ApJ...214..840I,1994ApJ...428L..13N, 1995ApJ...444..231N, 1995ApJ...452..710N, 1996A&AS..120C.287N, 1998ApJ...492..554N,2014ARA&A..52..529Y}) rather than in form of a radiatively efficient thin disk \citep{1973A&A....24..337S}.
Since the nearly flat spectrum emitted by \sgra in radio is similar to radio spectra observed in jets from Active Galactic Nuclei, it has been also suggested that the majority of the \sgra emission could be in fact produced by a jet launched by an accreting black hole rather than matter falling onto the black hole event horizon \citep{1993A&A...278L...1F, 2000A&A...362..113F}.
The jet scenario in \sgra is further supported by time lags observed between emission measured at multiple radio frequencies \citep{2021arXiv210713402B}.

\sgra's millimeter wavelength emission is produced by the  synchrotron process so source models require a description of the plasma and magnetic field configuration.  Models of magnetized RIAFs/ADAFs have been constructed using semi-analytic prescriptions \citep[e.g.,][]{1995Natur.374..623N,2000ApJ...541..234O, 2009ApJ...697...45B,2011ApJ...735..110B} and using time-dependent General Relativistic Magnetohydrodynamics (GRMHD) simulations \citep[e.g.,][]{2000ApJ...528..462H, 2003ApJ...589..458D, 2003ApJ...589..444G, 2007CQGra..24S.235G, 2012ApJS..201....9F, 2014ApJ...796...22F, 2016ApJS..225...22W, 2017ApJS..231...17A, 2018JPhCS1031a2008O, Olivares2019, 2019ApJS..243...26P}.
Semi-analytic RIAF/ADAF models typically do not include relativistic jets or outflows, but those are naturally produced in GRMHD simulation and contribute to the observed emission. GRMHD simulations also naturally produce variability that is observed in \sgra at multiple wavelengths.

GRMHD simulations of ADAFs have revealed that ADAF-like inflows are not unique. In particular two dramatically different modes are observed, depending on the magnetic flux interior to the black hole equator: the standard and normal evolution (SANE) mode, in which the midplane magnetic field pressure is less than gas pressure and magnetic fields are turbulent; and the magnetically arrested disk (MADs) mode, in which magnetic fields are strong and organized and can even disrupt accretion.  One outstanding question about \sgra is whether the flow is in MAD or SANE mode, or possibly in a third mode that results from wind-driven accretion \citep{2020ApJ...896L...6R}.

In addition to the unknown accretion mode, the energy distribution of electrons in the emitting plasma is not known.  Because emission is driven by the synchrotron process this is critical in determining the observational appearance of the source.  In particular the energy per electron may increase with latitude in the flow, leading to a jet or outflow that outshines an equatorial inflow.

The question of whether emission is dominated by an inflow or outflow is intimately tied to the problem of what drives an outflow, if there is one.  In GRMHD simulations of black hole accretion the strength of the outflow depends sensitively on the black hole spin (e.g., \citetalias{M87PaperV}).  At large spin GRMHD simulations show powerful jets driven by extraction of black hole spin energy via the \cite{1977MNRAS.179..433B} process, or by a closely related process.  A resolved study of \sgra may thus also contain information about the black hole spin and potentially provide direct evidence for black hole energy extraction.

\note{References need fixing.}

Previously published GRMHD models of \sgra generically predict source sizes at millimeter wavelengths consistent with observational data (e.g., \citealt{2008Natur.455...78D}, Moscibrodzka+2009, Dexter+2009);
the radio spectral shape is best described by jet emission (e.g., Moscibrodzka \& Falcke 2013, Ressler+2017), and the source linear polarization requires strongly magnetized flow or non-thermal electrons (Johnson+2017, Gold +2017, Dexter+2020). Near event horizon variability can be produced by a turbulence (Dexter+2010), lensing events (Chan+2015), or magnetic flux eruptions associated with MAD flow (e.g., Tchekhovskoy+2011). The 86 GHz data seem to require non-thermal electrons (Chael+2018, Issaoun+2019).

A major difficulty in determining the nature of \sgra radio emission is caused by the interstellar scattering screen that distorts our view of the Galactic Center up to $\lambda \sim 1\mm$ wavelengths
\citep[see][and references therein]{2018arXiv180501242P, 2018ApJ...865..104J,2019ApJ...871...30I}.
The Event Horizon Telescope (EHT) is a very-long-baseline interferometric (VLBI) experiment operating at $230\GHz$ or wavelength $\lambda = 1.3\mm$
(see \citealt{M87PaperII}, hereafter \citetalias{M87PaperII}, for an introduction to the instrument).  EHT operates at high enough frequency to penetrate the scattering screen, with angular resolution sufficient to directly image structures in the immediate vicinity of the black hole event horizon.
In April~2017, the EHT observed \sgra (among other sources, including the core of the M87 galaxy, see \citealt{M87PaperI}, hereafter \citetalias{M87PaperI}) and produced the first ever horizon scale images of the source.
We report the results of these observations in \citetalias{PaperII} and \citet{PaperIII}, hereafter \citetalias{PaperIII}, and characterize the basic properties of the emission visible in the EHT images in \citetalias{PaperIV}.
The main goal of this paper \citepalias{PaperV} is to provide the first comprehensive physical interpretation of the EHT~2017 \sgra datasets.

The paper is structured as follows. Section~\ref{sec:models} describes our main assumptions, a one-zone source models, and a standard simulation and synthetic image library used to model near horizon emission from \sgra. Our model library assumes that general relativity is valid and the spacetime around \sgra is described by a Kerr spacetime \citep{1963PhRvL..11..237K}.
A discussion of \sgra observations in the context of alternative theories of gravity can be found in  \citet{PaperVI}, hereafter \citetalias{PaperVI}.
Our model library is based on time-dependent GRMHD simulations that, combined with general relativistic radiative transfer models, result in images and broadband spectra of the models.
The library of simulated images was used in \citetalias{PaperIII} and \citet{PaperIV}, hereafter \citetalias{PaperIV}, to validate the \sgra EHT imaging and parameter estimation algorithms.
In Section~\ref{sec:observations}, we describe the  observational constraints that are used in the present work to test theoretical models of \sgra.
These data comprise a subset of EHT~2017 observations and other non-EHT historical or other data.
In Section~\ref{sec:comparisons}, we describe model scoring procedures and use our model library to infer physical properties of \sgra system.
We discuss model limitations, results in the context of previous studies and outlook for future \sgra theoretical research directions in Section~\ref{sec:discussions}.
Finally, we conclude in Section~\ref{sec:conclusions}.

This paper is supplemented with several appendices.
In Appendix~\ref{app:tables},      we summarize theoretical models used in this work in the form of tables and figures.
In Appendix~\ref{app:numerical},   we discuss important numerical details of our simulations.
In Appendix~\ref{app:variability}, we discuss the impact of physical effects on the variability of the models.
