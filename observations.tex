\section{Observational Constraints}\label{sec:observations}

\sgra is one of the most frequently observed objects on the sky: it has been observed with a slew of telescopes over 5 decades in time and 17 decades in electromagnetic frequency.
We must select a manageable subset of this data to constrain our models.
In doing so we have attempted to identify constraints
\emph{i})~that are believed to be uncorrelated, so that each tests a distinct aspect of the model;
\emph{ii})~that use data that can be simulated with the models; and
\emph{iii})~that are based on EHT 2017 $230\GHz$ VLBI data or that are based on emission produced within or close to the $230\GHz$ emitting region and \emph{iv})
that are observed contemporaneously or near-contemporaneously with the EHT 2017 campaign.

The selected constraints are described in detail below.
In brief, the 11 constraints can be divided into 3 classes.
The first class uses EHT data and compares estimates of source size, morphology of the visibility amplitude distribution, and three parameters of the best-fit \mring image model (5 constraints).
The second class uses non-EHT data, including 86\GHz flux density and source size, the median $2.2\um$ flux density, and the X-ray luminosity (4 constraints).
The third class considers variability, including the 230\GHz source-integrated variability and the visibility amplitude (VA) variability based on EHT data (2 constraints).

The selected constraints are heterogeneous and it is not yet possible to combine them in a consistent, fully satisfactory way.
Indeed, uncertainties in the data and the models are not well enough understood to make that possible.
In this first analysis we set a pass/fail criterion for each constraint and consider the implications of various combinations of constraints.

As the number of constraints increases so does the probability of wrongly rejecting a model.
Consider a set of $N$ constraints, and for each assigns a probability $p$ that the model is consistent with the data.
The model is rejected if $p < p_c$.
Then the probability that the model is wrongly rejected by a single constraint is $p_c$.
Applying all $N$ constraints, the probability that the model is wrongly rejected is $1 - (1 - p_c)^N$; for $N = 11$ and $p_c = 0.01$, this is $\approx N p_c \simeq 10\%$.
Each of $N$ constraints must therefore be able to reject a model with probability $\ll 1/N$ or the model scoring is meaningless.

The confidence with which a model can be evaluated is limited by sampling noise.
Many constraints (e.g. 86\GHz flux density) compare an observation to a distribution of synthetic observations from a model.
Time series of synthetic observations are not yet well characterized, but most have a correlation time $\tau\sim$ few $\times 100\,\tg$.
If the model decorrelates on timescales longer than $\tau$ then a model of duration $T$ yields $\sim T/\tau$ independent samples,\footnote{In what follows we must sometimes estimate how many independent samples are available in a time series.
Rather than estimating $\tau$  model-by-model we uniformly assume $\tau \simeq 500 \tg$.
The analysis is insensitive to this choice.} and thus a fractional error in moments of the distribution $\sim (T/\tau)^{-1/2} = 0.18 (T/\no{15000})^{-1/2}(\tau/500)^{1/2}$.
Increasing the number of constraints, then, requires increasing the duration of the GRMHD simulations.

Evidently the models have significant sampling noise, which we control for in part by using three redundant fiducial models.
Nevertheless one should not attach too much significance to the success or failure of individual models.
%The success or failure of regions in the parameter space is more significant.

%==============================================================================
\subsection{EHT Observational Constraints}

We test the models against EHT interferometric data
%, which constrains the spatial structure of the source,
in three ways.
First, we compare an estimate of the source size (``second moment'')
against an estimate based on short baseline visibility amplitudes
(VAs).
Second, we check the location of the first minimum and the long
baselines values of the VAs (``\vam'').
Finally, using a variant of a procedure from \citetalias{PaperIV}, we
compare fits for the diameter, width, and asymmetry of an \mring (a
parameterized image-plane model, ``\mring constraints'') to
distributions based on synthetic data generated from the model library.

%------------------------------------------------------------------------------
\subsubsection{230\GHz VLBI Pre-Image Size}
\label{sec:sz}

The source size can be characterized using the second moments of the
source image on the sky.
The second moments in the image domain map to second derivatives of
the visibilities near zero baseline in the \uv domain, so short
baseline VAs can be used to directly estimate the source size.

This procedure is used in \citetalias{PaperII} to set an upper limit
of 95\uas full width at half maximum (FWHM) and lower limit of 38\uas FWHM for the second moment along a direction
through the source corresponding to the orientation of the short
baselines (SMT-LMT and ALMA-LMT). This is done without any assumption about the structure of the source and is therefore quite permissive.

These limits do not include scattering.  The scattering kernel is estimated to have 16.2\uas FWHM along the relevant EHT baselines. To descatter the sky image size, we subtract this value in quadrature, which produces a scattering-corrected 93.6\uas FWHM upper limit and 34.4\uas FWHM lower limit.

To score a model we evaluate the second moment tensor for each
simulated 230\GHz image and find its eigenvalues
$\lambda_\mathrm{maj}^2/(8\ln 2)$ and $\lambda_\mathrm{min}^2/(8\ln
2)$, where $\lambda_\mathrm{maj}$ and $\lambda_\mathrm{min}$ are the
major and minor axis FWHM.
The image is deemed compliant if there exists any position angle for which the second moment would satisfy the size constraints, i.e. it is compliant if for any $\lambda$ such that $\lambda_\mathrm{min} \le \lambda \le \lambda_\mathrm{maj}$, $\lambda$ lies between the scattering-corrected upper and lower limits.
We reject models with compliance fraction $< 0.01$.

%------------------------------------------------------------------------------
\subsubsection{230\GHz VLBI Visibility Amplitude Morphology}

The second constraint provides a morphological check on the VAs.
We ask two questions of each model snapshot:
\emph{i})~is the first minimum in the visibilities---``the null''---at about the right place, and
\emph{ii})~are the long-baseline VAs comparable to the data?
The null locations and long-baseline amplitudes are sensitive to
the source structure.
For example, if the source is a simple, circularly symmetric ring of
finite width then the location of the first minimum depends only on
the ring diameter, while the VAs on long baselines depend mainly on
ring width.
GRMHD models are more complicated, with fluctuations in the null locations and long-baseline amplitudes (e.g., \citealt{2018ApJ...856..163M}, \citetalias{M87PaperV}).

We compare with data from \aprilvii, which has the best \uv
coverage near the minima in the VAs.
The first visibility minimum in both the N-S and E-W directions in the data always occurs between 2.5--3.5\,$\mathrm{G}\lambda$ \citepalias[see][for details]{PaperII}.
For the long-baseline interval between 6--8\,$\mathrm{G}\lambda$ in the data,
the VAs have $\lesssim 4\%$ of the zero-baseline flux.
One complication when comparing models to data on long baselines is
the effect of interstellar scattering.
Diffractive scattering effectively convolves the image with a smooth
kernel and can reduce the amplitudes to about $\sim 50\%$ of their
descattered values in the 6--8\,$\mathrm{G}\lambda$ range; refractive
scattering, on the other hand, introduces noise at all baselines of
order 0.5--3\%, depending on the characteristics of the
scattering screen \citep{2018arXiv180501242P, 2018ApJ...865..104J}.

To apply this constraint, we compute the VA of each model snapshot
along position angles (PAs) $0\degree$, $45\degree$, $90\degree$,
$135\degree$ (because of Hermitian symmetry we need only consider PAs in the 0--$180\degree$ range).
We find the first minimum numerically and compute the median VAs
between 6 and 8\,$\mathrm{G}\lambda$.
We classify a snapshot as compliant if
\emph{i})~for at least one position angle the first minimum falls
between 2.5 and 3.5\,$\mathrm{G}\lambda$ and
\emph{ii})~at no position angle do the median VAs exceed $4\% / 50\% = 0.08$ of the zero-baseline flux.
We reject models with compliance fraction $< 1\%$.

%------------------------------------------------------------------------------
\subsubsection{230\GHz \Mring Fitting}

Following \citetalias{PaperIV}, we fit an \mring image plane model to snapshots from EHT data and from simulated data and then compare the distributions of fit parameters.

The \mring is a $\delta$ function in radius with diameter $d$ multiplied by a truncated (up to $m = 3$; notice that Paper IV truncates at $m = 4$) Fourier series, convolved with a Gaussian of width $w$.
The model also contains a centered Gaussian component, with amplitude and width as free parameters, to absorb large scale emission and emission interior to the ring.\footnote{In \citetalias{PaperIV} this is called an mG-ring.}
The \mring model has 10 parameters.\footnote{The 10 parameters are: ring diameter, ring width, fraction of the flux in the Gaussian component, width of the Gaussian, and six parameters describing the amplitude and phase of the three Fourier components} We use 3 of the parameters that are well constrained and physically interpretable: the \mring diameter $d$, the \mring width $w$ (FWHM of the convolving Gaussian), and the $m=1$ relative amplitude $\beta_1$ (the ``asymmetry'').  For more details about the \mring model see Section 4.3 of \citetalias{PaperIV}.

We fit the \mring independently to snapshots consisting of 2-minute intervals of EHT data (this averaging interval is consistent with that used in \citetalias{PaperIV}).
Over these short intervals, we approximate the source as static.
Uncertainties in the fitted \mring parameters are dominated by the limited baseline coverage during these snapshots rather than by calibration uncertainties or thermal noise.
Because snapshots that are close in time sample nearly identical baselines, they do not provide additional model constraints.

To compare fitted \mring parameters from EHT data to those from  synthetic data, we select a subset of ten 120-second scans  that have detections on more than 10 baselines and integration times at all stations $> 40\sec$.
The selected scans are as widely separated in time as possible so that they sample distinct baseline coverage, with an average separation of $\simeq \no{1240}\sec \simeq 60 \,\tg$, which is small compared to the VA correlation time in the models \citep{Georgiev_2022}.
Note the selected scans overlap with those found in \cite{Farah_2022}.
Only small changes in model selection were observed if any one scan was removed from the comparison.
The data were descattered before fitting, that is, the VA were divided by the scattering kernel.

The maximum likelihood \mring parameters for the ten selected EHT scans are listed in Table~\ref{tab:mringfits}.
Evidently the fit parameters are noisy.
The fits for $d$ range from 39\uas to 84\uas, for $w$ from 9\uas to 21\uas, and for $\beta_1$ from $0.04$ to $0.48$
The variation in fit parameters could be caused by source variability, thermal noise, or gain variations.
In the models the main driver of fit variations is source variability.

\begin{deluxetable}{ccccc}  \label{tab:mringfits}
  %
  \tablecaption{\Mring Fits to EHT Observations}
  \tablehead{ %
    \colhead{Scan \#} & %
    \colhead{$t$ [UTC hrs]} & %
    \colhead{$d$ [$\mu$as]} & %
    \colhead{$w$ [$\mu$as] } & %
    \colhead{$\beta_1$} %
  }
  \startdata
  111 & 11.28 & 83.87 & 8.87  & 0.122 \\
  121 & 11.78 & 57.09 & 13.98 & 0.220 \\
  125 & 11.92 & 55.63 & 16.46 & 0.132 \\
  130 & 12.35 & 40.68 & 19.08 & 0.039 \\
  134 & 12.62 & 57.22 & 17.22 & 0.368 \\
  142 & 12.92 & 58.80 & 17.55 & 0.208 \\
  149 & 13.28 & 52.31 & 21.16 & 0.278 \\
  155 & 13.75 & 38.94 & 18.17 & 0.482 \\
  163 & 14.05 & 56.22 & 19.86 & 0.470 \\
  171 & 14.38 & 39.48 & 17.71 & 0.408 \\
  \enddata
  %
  \tablecomments{\mring fits to selected 120s scans from \aprilvii.
    Column 2 gives UTC in hours for the observation.
    Columns 3-5 give best-fit parameters for the \mring diameter, width, and asymmetry parameter respectively.}
\end{deluxetable}

For the models, we read in a series of model images, generate synthetic data for each image for each scan at four position angles (0\degree, 45\degree, 90\degree, 135\degree), and fit \mrings to the synthetic data.
This produces a distribution of \mring parameters for each model.

The synthetic data is generated as follows.
A model image $I(x,y)$ is Fourier transformed to complex visibilities $V(u,v)$ with an assumed position angle then sampled on baselines $i$ drawn from the comparison scan, $V_i \equiv V(u_i,v_i)$.
Normally distributed thermal noise $\delta V_{\mathrm{th},i}$ with amplitude based on telescope performance during the scan is added, and multiplicative, normally distributed noise with unit variance $N$ is added to crudely model gain corrections: $\tilde{V}_i = V_i (1 + \epsilon N) + \delta V_{\mathrm{th},i}$.
We set $\epsilon = 0.05$, but no substantial changes in fit parameters were observed for $\epsilon = 0.02$.
We then fit to the VAs $|\tilde{V}_i|$ and closure phases.\footnote{Maximum likelihood \mring parameters were found for each scan using the Julia package \texttt{Comrade.jl} \citep{comrade} in combination with a differential evolution-based optimizer
found in the Julia package \texttt{Metaheuristics.jl}.  The set of scripts used for the fits can be found in the GitHub repository \url{https://github.com/ptiede/EHTGRMHDCal}.}

We sample the model images once per $500\,\tg$, which is comparable to a correlation time.
A model with a $\no{15000}\,\tg$ imaging window thus produces $30$ fits per scan per position angle.

In comparing the models to the data we
\emph{i}) generate the distribution of fit parameters at each position angle;
\emph{ii}) use a Kolmogorov-Smirnov (KS) test to compare the distribution of $\sim 300$ synthetic data fits with the distribution of $10$ observational fits, and obtain a p-value (what is the probability they are drawn from the same underlying distribution?);
\emph{iii}) average the p-values over the four sampled position angles (i.e., marginalize over position angle; the models do not show a significant position angle preference); and
\emph{iv}) reject the model if $p < 0.01$.

%==============================================================================
\subsection{Non-EHT Constraints}

In addition to the EHT data, the SED of \sgra is well constrained in \citetalias{PaperII} and thus potentially useful for model selection.
We limit comparison to three bands: 86\GHz, 2.2\um, and X-ray.

%------------------------------------------------------------------------------
\subsubsection{86\GHz Flux}

The Global Millimeter VLBI Array (GMVA) observed \sgra on April 3, 2017, just 3 days ($\simeq \no{13000}\,\tg$) before the EHT campaign.
\citet{2019ApJ...871...30I} estimate that the compact flux during this observation was $F_{86} = 2.0 \pm 0.2\,\mathrm{Jy}$ ($2\sigma$ errors).

To test the models we compute a library of 86\GHz images for all GRMHD snapshots for all models, and integrate over them to obtain  $F_{86}$.
We assume normally distributed measurement errors with $\sigma = 0.1\,\mathrm{Jy}$ and convolve the $F_{86}$ distribution for each model with the resulting Gaussian.
We reject models where the value of the error-broadened cumulative distribution function (CDF) at 2.0\,Jy is <1\% or >99\%.

%------------------------------------------------------------------------------
\subsubsection{86\GHz Image Size}\label{sec:86size}

The GMVA observations from April 3, 2017 constrain the FWHM of the source major axis.
Notice that two different values for the major axis FWHM have been published in the literature: $120 \pm 34\uas$ \citep{2019ApJ...871...30I}
${\rm FWHM}_{maj} = 146^{+11}_{-12}\uas$ \citep[95\% confidence][]{2021ApJ...915...99I}.
We adopt the later analysis.

We compute the major axis FWHM for each image in the 86\GHz image library.
We assume normally distributed errors with $\sigma = 6\uas$ and convolve the model major axis distribution with the normal distribution.
We reject models with error-broadened CDF $< 1\%$ or $> 99\%$ at $146\uas$.

Our synthetic 86\GHz images have a 800\uas field of view.
A 200\uas field of view cuts off enough emission that the major axis is biased downward in many models by $\sim 20\%$.
Increasing the field of view beyond 800\uas has negligible effect.

%------------------------------------------------------------------------------
\subsubsection{\texorpdfstring{$2.2\um$}{2um} Median Flux Density Constraint}\label{subsec:nir}

\sgra has a quiescent and a flaring component in the near-infrared (NIR), with flares occurring a few times per day
(1 day $\simeq \no{4200}\,\tg$) \citep{2018ApJ...863...15W}.
Since there is as yet no generally accepted model for NIR flares, we accept models that do not produce flares (indeed none of our models reliably produce flares, even those with non-thermal eDFs).
Our working hypothesis is that the models can be made to produce flares by introducing a process that accelerates a small fraction of electrons into a  NIR-bright tail of the eDF.  If the model overproduces quiescent 2.2\um emission, however, then we reject it.

\sgra had a median 2.2\um flux $= 0.8 \pm 0.3\,\mathrm{mJy}$ in 2017  \citep[][see Table 1]{2020A&A...638A...2G}.
The median flux density likely overestimates the median quiescent flux density since it includes flares.

We compute the model median 2.2\um flux density using one of two procedures.
If a full SED\textemdash which includes Compton scattering\textemdash is available then we use it.
The SEDs are generated by the \grmonty Monte Carlo code \citep{2009ApJS..184..387D, Wong_2022}.
If a full SED is not available (see Table \ref{tab:radiativemodels}) then we compute a $2.2\um$ image that includes only synchrotron emission (synchrotron absorption is negligible at $2.2\um$ for \sgra).

A rigorous model evaluation procedure would correct for the upward bias in median quiescent flux density from flares and allow for errors in the model and observed median flux density, but these refinements are sufficiently uncertain that, instead, we set a conservative threshold of $1.0$\,mJy and reject the model if its
median $2.2\um$ flux density exceeds threshold.

%------------------------------------------------------------------------------
\subsubsection{X-ray Luminosity Constraints}

\sgra flares in the X-ray less than about once per day \citep[see][and references therein]{2018MNRAS.473..306Y}.
\emph{Chandra} observations during the 2017 campaign suggest a conservative upper limit on the median (quiescent) $\nu L_\nu$ at $6\,\mathrm{keV}$ of $10^{33}\ergps$ (\citetalias{PaperII}).

As for the model $2.2\um$ flux density, we estimate $\left.\nu L_\nu\right|_{h\nu=6\,\mathrm{keV}}$ in two ways.
The SED, which incorporates Compton scattering and bremsstrahlung, is used if it is available.
If the SED is not available then we compute an X-ray image that includes only bremsstrahlung (which dominates the X-ray emission in thermal SANE models with $\Rh = 40$, $160$) enabling us to eliminate a few additional models.

We reject the model if its median $\left.\nu L_\nu\right|_{h\nu=6\,\mathrm{keV}} > 10^{33}\ergps$.

%==============================================================================
\subsection{Variability}

\sgra shows variability on a wide range of timescales.
This is expected: fluctuations in stellar wind feeding at the scale of the S-stars plausibly introduce long timescale variations, while turbulence at smaller radii, down to the scale of the event horizon, introduces a spectrum of shorter timescale variations.
Quantitative comparison of observed variability to the models is therefore a potentially powerful tool for model selection.

We consider two variability measures: one characterizes variability in the 230\GHz light curves \citep{Wielgus2022} and a second  characterizes variability of VAs in EHT data (\citetalias{PaperIV}; \citealt{NoiseModeling}).

%------------------------------------------------------------------------------
\subsubsection{230 \GHz light curves}

We compare variability in the models to light curve observations of \sgra from 2005--2017 using the 3-hour {\em modulation index} $\mi{3}$, where $\mi{\Delta T} \equiv \sigma_{\Delta T}/\mu_{\Delta T}$, $\sigma_{\Delta T}$ is the standard deviation measured over an interval $\Delta T$ (in hours), and $\mu_{\Delta T}$ is the mean measured over the same interval.

Following \citet{2015ApJ...812..103C} we use $\mi{\Delta T}$ because it is easy to describe, easy to compute, commonly used in the literature (in the X-ray astronomy literature it is ``rms \%''), and closely related to the structure function, since the expectation value for $\sigma_T^2$ is given by an integral over the structure function \citep[see][]{Lee_2022}.

We use $\Delta T = 3$\,hours ($\sim 530\,\tg$) because it is long enough to be comparable to the characteristic timescale measured in damped random walk fits to the ALMA light curve \citep[see Table 10 of][]{Wielgus2022} but short enough that the model light curves provide a sample that is large enough to be constraining.
In extracting a sample of $\mi{3}$ from the light curves we use as many 3-hour segments as possible, equally spaced away from the light curve endpoints and each other, and calculate $\mi{3}$ on each segment.
We treat consecutive measurements of $\mi{3}$ as independent, consistent with the minimal correlation expected for a damped random walk \citep{Lee_2022}.

We must select an observed distribution of $\mi{3}$.
The April 7 data alone provide only a weak constraint because there are only 3 samples.
The $\mi{3}$ measured from EHT 2017 observations on April 5--11 provide 7 samples, while the $\mi{3}$ measured from all available light curves longer than 3\,hours, including earlier SMA and CARMA data (the ``historical distribution''; see \citealt{Wielgus2022}) yields 42 samples.
The 2017 distribution is consistent with being drawn from the historical distribution, although April 6 has one of the quietest segments on record, and April 11 one of the most variable.
We selected the historical distribution and note that the 2017 distribution rejects slightly {\em more} models but leads to identical conclusions.

For each model we use a two-sample KS test to estimate the probability $p$ that the model and observed $\mi{3}$s are drawn from the same underlying distribution.
We reject the model if $p < 0.01$.

Through the KS test, the strength of the $\mi{3}$ constraint depends on the number of data and model samples.
The fiducial models have duration $10^4$ or $1.5\times 10^4\,\tg$ (18 or 28 samples), whereas most exploratory models have duration $5 \times 10^3\,\tg$ (9 samples).
The $\mi{3}$ constraint is therefore weaker for the exploratory models: an exploratory model that passes the constraint may be more variable than a fiducial model that fails.

%------------------------------------------------------------------------------
\subsubsection{EHT Structural Variability}

Fluctuations in the spatial structure of the source produce fluctuations in the VAs.
Here we compare  the power spectrum of structural variability from EHT observations with predictions from GRMHD models.

A nonparametric technique to measure the variance of the spatially-detrended VAs at a location in the $(u,v)$-plane is described in \citet{NoiseModeling} and briefly summarized here.
We use EHT observations of \sgra from April 5, 6, 7, and 10 (April 11 was excluded).
To remove correlations associated with variations in the total flux, we normalize the VA data with the contemporaneous intrasite light curve \citep{Georgiev_2022}.
The light curve-normalized visibility amplitudes are then linearly detrended, and variances are computed and azimuthally averaged \citep{NoiseModeling}.
The resulting $\sigma_\text{var}^2 (|u|)$ is a measure of the fractional structural variability as a function of baseline length $|u|$.
The $\sigma_\text{var}^2 (|u|)$ is included in an inflated error budget when making images of and fitting models to the 2017 EHT observations of \sgra \citepalias{PaperIII, PaperIV}.

We measured this quantity from the GRMHD simulations \citep[see][for details]{Georgiev_2022}.
For all simulations reported here, $\sigma_\text{var}^2$ is well-approximated by a broken power law with parameters that are nearly universal among simulations.
The $\sigma_\text{var}^2$ is measured over a four-day period, which is longer than the typical model duration.
We therefore expect that model values will be biased downward compared to the data. Furthermore, each GRMHD simulation can only give one draw from a distribution that is broader than if the simulation spanned 4 days. This secondary effect negates the downward bias, which is further unimportant as we do not exclude models for being not variable enough. To measure the larger broadness of the distribution, we use multiple simulations with the same parameters and subdivide the analysis of long simulations into windows.
The uncertainties in the measurement from the GRMHD simulations due to simulation resolution, the fast-light approximation, and code differences are small compared to the uncertainty due to the variability of $\sigma_\text{var}^2$ due to short simulations \citep{Georgiev_2022}.

The measured $\sigma_\text{var}^2$ is well characterized by a power law for $2~{\rm G}\lambda < |u| < 6~{\rm G}\lambda$ \citep{Georgiev_2022}.
For comparison with the models presented here, we distill the $\sigma_{\text{var}}^2$ to two numbers: the amplitude $\afour^2$ at $4~{\rm G}\lambda$ and a power law index $b$.
Because the normalization is done in the center of the fit range the estimated $\log_{10}(\afour^2) = -3.4 \pm 0.1$ and $b=2.4\pm0.8$ are essentially uncorrelated.

Model predictions for $\afour^2$ and $b$ are computed using the power spectral densities from \citet{Georgiev_2022}\footnote{\citet{Georgiev_2022} gives the power spectral density of the complex visibility, $\langle\hat{P}(|u|)\rangle$, rather than the VA, and thus $\sigma_\text{var}^2=\langle \hat{P}\rangle/2$.}.
The anisotropic diffractive scattering kernel from \citet{Johnson_2018} is applied to $\sigma_\text{var}^2(|u|)$ and averaged over relative orientations of the major axis of the scattering kernel and the black hole spin.
These estimates are then azimuthally averaged, and the parameters $\afour^2$ and $b$ are determined from a least-squares linear fit to $\sigma_\text{var}^2(|u|)$ in $2~{\rm G}\lambda < |u| < 6~{\rm G}\lambda$.

For each model the fits for $\afour^2$ and $b$ are done separately on each window of length $5 \times 10^3\,\tg$, giving at most three measurements for most models.
This makes a direct comparison with the measured value difficult, as the model distribution is poorly constrained.

\citet{Georgiev_2022} estimates the typical width of a model distribution is $\log_{10}(\afour^2) \pm 0.1$.
We can obtain a rough estimate for how the models fare compared to the measurement by taking the mean across windows, assuming the width of the distribution is $\sigma = 0.1$, and comparing this with the observed distribution under the assumption that both are distributed normally.
We reject models with error-broadened CDF <1\% or >99\% at $\log_{10}(\afour^2) = -3.4$.
