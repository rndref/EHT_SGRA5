\section{Observational Constraints}\label{sec:observations}

\sgra is one of the most observed objects in the sky.
We have data obtained with a slew of telescopes, across 5 decades in time and more than 17 decades in frequency. We need to select a manageable subset of this data for comparison with the models. In doing so we have attempted to select (1) approximately uncorrelated constraints, so that each can test a distinct aspect of the model; (2) constraints based on data that can be simulated with the models; (3) constraints based on EHT 2017 1.3mm VLBI data or based on photons produced within or close to the 1.3mm emission region that are contemporaneous or near-contemporaneous.

%==============================================================================
% \subsubsection{Scattering Models}

%==============================================================================
\subsection{EHT Observational Constraints}

\ckc{ck's first pass}
\mw{I think it would be good to mention the EHT array composition (name the telescopes)}
\cfg{much of this material could be incorporated by reference to paperII}

\begin{figure*}
  \centering
  %\includegraphics{}
  [altex: (left) visibility amplitude vs baseline for the day(s) that
    this study use, overplotted by the visibility amplitude from a
    fiducial model.
    Similar to paper~II, figure~7.
    Visually mark null location constraint, pre-imaging size (i.e.,
    second moment).
    (right) SED from Gurther overplotted by the SEDs from a fiducial
    model.
    Visually mark the SED constraints.]
  \caption{(\emph{left}) Measured correlated flux densities of \sgra
    on [DAY X] from the HOPS pipeline overplotted with a fiducial
    GRMHD+GRRT model.
    Details on the data can be found in paper~II, section~5.
    A description of the fiducial model is in
    section~\ref{sec:models}.
    (\emph{right}) \ckc{Q: do we want to show only EHT observation and
      referencing to other papers for non-EHT constraints?
      Or should we have representative figures for all measurements?}}
  \label{fig:visibility}
\end{figure*}

The EHT observed \sgra at 1.3 mm during the 2017 April 5--11 observing campaign and obtained horizon scale complex visibilities.  Figure~\ref{fig:visibility} shows the visibility amplitudes on April 6 and 7 from  the HOPS pipeline, overplotted with visibility amplitudes derived from a model.  Evidently there are nulls near $\sim 5\mathrm{G}\lambda$ and $7\mathrm{G}\lambda$, suggesting a ringlike structure.

We test the models using interferometric data in three ways.  First, we compare the location of the first null and the visibility amplitude of long baselines to the model (``null constraint'').  Second, we compare an estimate of the source size (``second moment constraint'') from short baselines to the model.  Finally, we follow a variant of the procedure used in Paper~\ref{PaperIV} and compare fits for the diameter, width, and asymmetry of an m-ring to the model.  

%------------------------------------------------------------------------------
\subsubsection{230\,GHz VLBI Null Locations}

\ckc{ck's first pass}

One way we characterize the GRMHD simulations carried out for
different black hole and flow parameters and assess their
compatibility with the \sgra\ data involves an analysis in the Fourier
domain.
To this end, we first calculate the complex visibilities by performing
a two-dimensional Fourier transform on each snapshot of each
simulation
\begin{equation}
  V(u,v) = \iint I(\alpha,\beta) e^{-2\pi i(u\alpha+v\beta)}d\alpha d\beta,
\end{equation}
where we defined $\alpha \equiv X/D$ and $\beta \equiv Y/D$, with D
being the distance to \sgra.
Figure~\ref{fig:cmp_VA} shows two example snapshots and the
corresponding visibility amplitudes (VA) as a function of baseline
length for a vertical and horizontal image cross sections from a
simulation.
Even though the locations and depths of the minima in the visibility
amplitudes are primarily set by the image size, which is set by the
black hole mass, this figure shows that they can exhibit significant
variability from snapshot to snapshot because of the multitude of
structures that originate in the turbulent flow (Medeiros et
al. 2018).
In particular, the minima tend to move to larger baselines as the ring
thickness temporarily changes in response to, e.g., the appearance of
a flux tube or a similar structure.
In addition, images that have a higher degree of azimuthal asymmetry
show pronounced differences in their vertical and horizontal cross
sections.

The degree of VA variability is different from model to model and also
depends on some of the global characteristics of the flow.
For example, the overall electron density in the disk plays a role by
its effect on the ring thickness: thicker rings show more change in
the visibility minima between snapshots than thinner ones (see also
Satapathy et al. 2021 for the effect on closure phases).
Because of this, while no model is expected to resemble the observed
visibility amplitude data 100\% of the time, it is nevertheless
possible as well as discriminating to require an agreement between the
locations of the minima in each simulation snapshot and the visibility
amplitude data from \sgra\ a reasonable fraction of the time.

To carry out the comparison with the data, we focus on the VA observed
on 2017 April 7, \#3599 because that night has the best u-v coverage
near the minima.
The first visibility minima in both the N-S and E-W directions occur
between $2.5-3.5$\;G$\lambda$, as we show in
Figure~\ref{fig:cmp_null}.
We define compliance for a snapshot by requiring that the VA obtained
for {\it either} the horizontal {\it or} the vertical cross section of
a snapshot have a minimum in this range of baseline lengths.

A second feature of the visibility amplitudes that can help discern
between models is the behavior of the VA at long baselines.
April 7 data also show that the amplitudes have declined to $<6\%$ of
the zero baseline flux at baselines between $6-8$\;G$\lambda$ along
all orientations.
This is characteristic of ring-like images with a relatively high
degree of azimuthal symmetry.
Images that are more asymmetric, on the other hand, lead to
significantly higher amplitudes at baselines much larger than the
first minimum, or no minima at all, as the lower panels in
Figure~\ref{fig:cmp_VA} illustrate.
As a result, selecting models based on how frequently they produce
snapshots with such large-baseline power helps identify models that
are in accordance with the data.

One consideration when comparing models to data at long baselines is
the effect of interstellar scattering.
Diffractive scattering has the effect of convolving the image with a
smooth kernel and can reduce the amplitudes to $\sim 70\%$ of their
intrinsic value in the $6-8$\;G$\lambda$ range (REF).
Refractive scattering, on the other hand, introduces a noise at these
baselines of the order of $0.5-3\%$, depending on the particular
characteristics of the scattering screen toward \sgra\ (REF).
To account for both of these effects, we choose a $6\%$ upper limit to
visibility amplitudes from a model snapshot when defining compliance,
as we show in Figure~\ref{fig:cmp_null}.

We assign a compliance fraction to a simulation based on the fraction
of snapshots that pass both criteria we described above.
We will discuss the results of this comparison in the next section,
along with the other model scoring criteria.
[I will add to the discussion of the scoring table when that part is
  written.]

%------------------------------------------------------------------------------
\subsubsection{230\,GHz VLBI Pre-Image Size}

\ckc{ck's first pass}

The second moment of an EHT source corresponds, in the $uv$ domain, to
the curvature of the visibility amplitude near zero baseline length.
That is, the 2nd moment tensor
\begin{equation}
    \sigma_{ij} \equiv \int \, d^2x\, x_i x_j I/\int \, d^2x \, I = (2\pi)^2 \left(\partial_i \partial_j \tilde{I}\right)/\tilde{I}
\end{equation}
where $I$ is the intensity, $\tilde{I}$ its Fourier transform, $i =
x,y$ on the left and $i = u,v$ on the right, and the terms on the
right are evaluated at $u = v = 0$.
A pair of visibility amplitudes ($|\tilde{I}|$) on short baselines or
zero can therefore be used to estimate the second moments of the
image.

This procedure is used in EHT3 to set an upper limit of $95\mu$as on
the scattered source size and lower limit of $38\mu$as.

NOTE: data is descattered.

%------------------------------------------------------------------------------
\subsubsection{230\,GHz M-Ring Fitting}

EHT4 fits an ``m-ring'' source model to the 7 April data.
The simplified m-ring model we use here takes $I$ to be a $\delta$
function in radius multiplied by a truncated Fourier series, and
convolves it with a Gaussian.
The resulting model is
\begin{equation}
    I(r,\theta) =
\end{equation}

NOTE: data is descattered.

%------------------------------------------------------------------------------
\subsubsection{230\,GHz VLBI Ring Thickness}

%------------------------------------------------------------------------------
\subsubsection{230\,GHz VLBI Ring Asymmetry}

%==============================================================================
\subsection{Non-EHT Observations Constraints}

%------------------------------------------------------------------------------
\subsubsection{86\,GHz Flux}

%------------------------------------------------------------------------------
\subsubsection{86\,GHz Image Size}

%------------------------------------------------------------------------------
\subsubsection{NIR (Non-Overproduction) Constraints}

%------------------------------------------------------------------------------
\subsubsection{X-ray (Non-Overproduction) Constraints}

%==============================================================================
\subsection{Time Dependence}

\ckc{we can model "long-term climate" well but not "long-term weather".  Hence we only compare the slowly varying quantities such as mean 2nd moments (climate) but not modulation index (depends on weather).  TODO: look up correct statistical terminology in weather/climate modeling and build that this into this paper.}

%------------------------------------------------------------------------------
\subsubsection{EHT (Baseline variability)}

\ckc{I think Boris needs to write this...}

%------------------------------------------------------------------------------
\subsubsection{ALMA Light curves}
\mw{I edited this a bit}
\\
ALMA and SMA produced \sgra light curves at 230GHz as a byproduct of the
2017 EHT VLBI observing campaign. The complete set of light curves is presented and analyzed in \cite{wielgus2021}. We have chosen to compare the models to the 7 April 2017 ALMA light using
the 3 hour {\em modulation index} $M_3$, where $M_T \equiv
\sigma/\mu$, $\sigma$ is the standard deviation measured over some
interval $T$, and $\mu$ is the mean measured over the same interval.

We use $M_T$ because it is: easy to describe; easy to compute;
commonly used in the literature (in the X-ray astronomy literature it
is ``rms \%''); and closely related to the structure function, since
the expectation value for $\sigma^2$ is given by an integral over the
structure function (see Lee et al. 2021).
We use $T = 3$ hours because: 3 hours is similar to the correlation
time for $F_{230}$ in most of the models; 3 hours is similar to the
characteristic timescales measured in damped random walk fits to the
ALMA lightcurve \citep[see Table 10 of][]{wielgus2021}; 3 hours is the
longest timescale for which we can consistently estimate the mean and
variance of the distribution of M$_3$ from the models.
For a damped random walk process one can show that $M_3$ is very weakly
correlated over successive 3 hour intervals (Lee et al. 2021).

The constraint comes from $M_T$ measured over 3 maximally spaced intervals
in the 7 April 2017 ALMA light curve, where $M_3 = 0.024, 0.051,
0.047$ (Wielgus et al. 2021). These values are consistent with being drawn from the distribution estimated from historical non-EHT 2005-2017 light curves.
%curves. \mw{maybe sth like "from the estimated distribution of the historical 2005-2017 lightcurves"?}

%==============================================================================
\subsection{Future Constraints}

\monika{this section should be moved to the end of the paper}
\ckc{Agree}
integrated polarization,

resolved polarization

fits to more sophisticated models such as RIAF analytic models,

closure phase variability
