\section{Variability of GRMHD models}\label{app:variability}

%%%%%%%%%%%%%%%%%%%%%%%%%%%%%%%%%%%%%%%%%%%%%%%%%%
%% notes from people who drafted the section

% \bp{Seems like (1) and (2) here are (or will be) discussed more thoroughly in the Numerical Methods appendix.  Given both seem to conclude that all models are consistent, and thus that changing these parameters would not reduce variability, maybe we can refer there instead of re-writing?}
%\monika{11Dec:addressed below}
%==============================================================================
% 1. GRMHD resolution.

% discussion of varying resolution [Ben to spelunk to find a 448 down set of images]

% Discussion of high resolution HAMR model.

% mapping between 230 GHz variability and accretion rate variability
%\monika{11Dec:addressed in B1}

%==============================================================================
%2. Image resolution.
%\monika{11Dec:addressed in B1}
%==============================================================================
%3. Model duration is not sufficiently long.
%KORAL model analysis.
% MM 11 Dec: now in C1
%==============================================================================
%4. $\sigma_{cut}$ is overproducing variability.
% MM 11 Dec: when sigma_cut>sigma_crit  physics is unreliable and this is mentioned in the model section now. maybe not appropriate to discuss here
%==============================================================================
%5. Initial conditions are not a good model.
%Comparison of Monika model initial conditions
% there was nothing added on this to the draft yet
%==============================================================================
% 6. Cooling filters the lightcurve. [Ben]
% \monika{11Dec:moved at the end: merged into subsection C.2}
%==============================================================================
%7. Electron distribution function model is inadequate.

%a.  Discussion of models with varying electron heating models.  Discussion of Jason's eheating models.   Forward reference to Diaz et al.  [Vedant]

%b. Discussion of models with $\Rl = 10$. [Vedant]
% \monika{11Dec: now in C2}

%%%%%%%%%%%%%%%%%%%%%%%%%%%%%%%%%%%%%%%%%%%%%%%%%%

The majority of our simulations fail to recover the \sgra observed variability in total flux.  In this Appendix we discuss and dismiss a few possible causes for mismatch of $\mi{3}$ metric in models and ALMA data.

%==============================================================================
\subsection{Simulations Duration}\label{app:narayan}

\begin{figure}
  \centering
  \includegraphics[width=\columnwidth]{./figures/Koral_vs_IL_MI.png}
  \caption{Distribution of \mi{3} from the full \koral model set, from the first half, from the second half, and from the comparable \kharma thermal models.}
  \label{fig:koral_MI}
\end{figure}

Our standard models of accretion are typically evolved for $\sim 30,000 \tg$. In Figure~\ref{fig:koral_MI} we show comparison of $\mi{3}$ distributions from typical duration simulation to the one that was evolved approximately three times longer. The distribution functions are in good agreement in terms of mean and standard deviation regardless of time interval chosen for comparison. The two models that are displayed in the figure are \kharma and \koral simulations and differ in resolution of the numerical grid and initial torus size, meaning that the variability mismatch is unlikely due to numerical resolution issues or initial conditions of the GRMHD simulations.

%==============================================================================
%
% some questions about cooling work; commented out until/unless we are sure we understand what is going on.
\subsection{Effects of \texorpdfstring{$\Rl$}{Rl} } %and radiative cooling}

For systems with sub-Eddington accretion rates, $\Dot{M}\ll\Dot{M}_\mathrm{Edd}$, such as \sgra, radiative processes can be neglected during fluid evolution (\citealt{2012MNRAS.426.1928D, 10.1093/mnras/stw3116, Ryan_2017}).
However, the uncertainties with electron heating and advection, and a limited understanding of the funnel region warrant a discussion of electron cooling in regions of high magnetization, and in particular, its effect on the 230\,GHz lightcurve variability.

The $\Rh$ prescription (Equation \ref{eq:rhigh_prescription}) has three free parameters: $\Rh$, $\Rl$ and $\beta_\mathrm{crit}$. During post-processing, the $\Rh$ parameter is generally varied while $\Rl$ and $\beta_\mathrm{crit}$ is set to unity because the Particle-in-cell (PIC) simulations modeling turbulent dissipation or dissipation associated with magnetic reconnection suggest preferential heating of the electrons in regions of low plasma $\beta$ (\citealt{2010MNRAS.409L.104H, Rowan_2017, 10.1093/mnras/stx2530, Rowan_2019, Kawazura771, PhysRevX.10.041050, kawazura2021energy}).

We first investigate the effect of varying $\Rl$ on the 3 hour modulation index, $\mi{3}$. The $\Rl$ parameter dictates the electron temperature in these regions, that is, in the funnel. Increasing $\Rl$ mimics rapid electron cooling along the funnel.
We vary $\Rl$ for various set of the thermal (\kharma data sets) and calculate $\mi{3}$. In Figure~\ref{fig:mi_rlow} we show that the modulation index is weakly dependent on $\Rl$ parameter.
The $\mi{3}$ distributions do not match the observational values.

\begin{figure*}
  \centering
  \includegraphics[width=0.95\textwidth]{figures/mi_rlow_select_models.png}
  \caption{Modulation index computed over 3 hour intervals $\mi{3}$ for a subset of the thermal models (\kharma data sets). For this analysis, we considered the 25,000--30,000$\tg$ time interval.}
  \label{fig:mi_rlow}
\end{figure*}


%==============================================================================
%8. The flow is actually made of helium. [George]

%brief discussion, forward reference to Wong+.

% now C4

%==============================================================================
%9. $\gamma = 5/3$ rather than $4/3$.   [Vedant]  When $\gamma = 4/3$ the compression ratio across the shock is larger.

\subsection{Effect of self-consistent electron heating}


\citealt{10.1093/mnras/stv2084} provided a formulation to model electron thermodynamics during the fluid evolution. Numerical dissipation at the grid scale sources entropy generation and is used to heat the electrons based on a microphysical, sub-grid heating prescription. Local fluid and electromagnetic variables are used to compute the electron entropy which, along with the ideal gas equation of state, can be converted into a temperature $\Theta_{e}$. This approach allows computing the electron temperature at each timestep of the simulation rather than post-processing, as it is done in the $\Rh$ and Critical-$\beta$ prescriptions.

We consider three sub-grid heating models that prescribe the partition of dissipated energy into electrons and ions.
\citep{2010MNRAS.409L.104H} computed the ratio of ion-to-electron heating due to dissipation of Alfv\'enic turbulent cascade, while \citep{10.1093/mnras/stx2530} and \citep{Rowan_2017} considered magnetic reconnection as the source of energy dissipation at sub-grid scales. These studies provide an approximate fitting formula for the ion-to-electron heating rate $Q_{i}/Q_{e}$ based on local ion-to-electron temperature ratio $T_{i}/T_{e}$ and local magnetic field strength---parameterized by $\sigma$ or plasma $\beta$.

The GRMHD simulations considered here are a subset of the simulations analyzed in \citealt{2020MNRAS.494.4168D}. These include MAD and SANE accretion flows at spins, $\abh = 0,+1/2,+15/16$. We compute the 3 hour modulation index $\mi{3}$, over the time interval 5,000--10,000$GM/c^{3}$. The average $\mi{3}$ values are comparable to similar $\Rh$ models, with SANE reconnection models exhibiting a reduced variability as compared to the corresponding turbulent heating models. However, the average $\mi{3}$ for all the models is greater than the $\mi{3}$ measured from the ALMA lightcurve on three days.

%==============================================================================
\subsection{Effects of fluid adiabatic index, \texorpdfstring{$\Gamma_\mathrm{ad}$}{Gad}}

We expect the ions and electrons in hot accretion flows to be thermally decoupled and the resulting plasma to be two-temperature \citep{1976ApJ...204..187S, Quataert_1998, 10.1093/mnras/stw3116, Ryan_2018, Chael2018}. The electrons in such flows are relativistic and can be modeled as a fluid with an adiabatic index $\Gamma_{e}=4/3$, while the ions are nonrelativistic and possess an adiabatic index $\Gamma_{i}=5/3$.

The adiabatic index of the fluid assumes a value between $\Gamma_{e}$ and $\Gamma_{i}$ dictated by the thermodynamics of the ions and electrons (cf. Figure 4 in \citealt{10.1093/mnras/stw3116}). Since we do not model electron thermodynamics and ignore radiative effects during our fluid simulations, we consider a constant value $\Gamma_\mathrm{ad}$ and set it to 4/3 for the \textit{standard} set of simulations, i.e., $\Gamma_\mathrm{ad}=\Gamma_{e}$. While this may be the case in the funnel, where the electrons are the hottest and highly relativistic; the fluid adiabatic index value away from the poles can be higher than the relativistic value.
Two-temperature simulations can self-consistently evolve adiabatic indices of electrons and ions and compute the net fluid adiabatic index with contributions from both species \cite{10.1093/mnras/stw3116}. These two-temperature simulations often show variation of the adiabatic index with polar angle, with the fluid energy dominated by hot electrons near the poles ($\Gamma = 4/3$) and by cooler ions and  electrons in the midplane ($\Gamma=5/3$).

We look at the interplay between fluid adiabatic index and lightcurve variability by evaluating $\mi{3}$ for thermal, GRMHD simulations with a higher fluid adiabatic index. This includes MAD models with $\Gamma_\mathrm{ad}=13/9$ (see Section~\ref{app:narayan} and  \citealt{2021arXiv210812380N}) and SANE models with $\Gamma_\mathrm{ad}=5/3$. The models exhibit lightcurve variability similar to the \textit{standard} library and have an average $\mi{3}$ that is greater than the value obtained from the ALMA lightcurve.

